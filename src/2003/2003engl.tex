\documentclass[fleqn]{jbook}
\usepackage{physpub}
\usepackage{txfonts}

\begin{document}

%%%%%【英語1 問題】%%%%%%%%%%%%%%%%%%%%%%%%%%%%%%%%%%%%%%%%%%%%%%%%%%%%%%%
\begin{question}{問題1}{茂木康平}
次の英文はある国際研究集会の案内文である。これを読み、設問1〜4に答えよ。

\begin{center}
    \bf
    2nd Announcement - 6th International Interferometry Workshop

    October 1-3, 2003, US Atomics, San Diego, USA
\end{center}

{\parindent 0zw
{\bf Venue, Dates and Organizers}

The 6th International Interferometry Workshop will be held at the main campus of US Atomics, in San Diego, California. The meeting duration is three days, October 1-3. The local scientific organizers are E.Kepler and T.Planck of UCLA, while US Atomics is providing meeting facilities and administrative support by Alice Newton.

{\bf Meeting Format}

As at previous meetings, we wish to have a true `workshop' format with opportunities for extensive interaction and discussion amongst participants. To this end, the meeting will consist of oral presentations (10-20 min duration) with scheduled periods for \framebox[1cm]{(a)}.

{\bf Workshop Dinner, Facility Visit}

A dinner will be organized for the workshop participants, at participants' own cost - details to be announced later. A visit to the US Atomics facility will be arranged for those interested (please \framebox[1cm]{(b)} if you are interested when responding).

{\bf Registration and Abstracts}

All potential attendees should register by completing and returning the attached Foreign Visitor Form to Alice Newton by August 17, 2003 (see security requirements, below). US citizens should complete only the name and contact details, and indicate their citizen status. In order to prepare the scientific program an \framebox[1cm]{(c)} should be sent to E.Kepler by August 22. Please submit electronically as either a Word or PDF file.

{\bf Proceedings}

As at previous meetings, participants should submit copies of their \framebox[1cm]{(d)} for inclusion in the CD-ROM which will be distributed to all the participants after the workshop. Participants are also encouraged to submit a written \framebox[1cm]{(e)} for the Proceedings as a more complete record of their work, with a maximum length of 10 pages.

{\bf Security Requirements}

As a DOE facility, access to the US Atomics site is restricted. All non-US citizens MUST \framebox[1cm]{(f)} and return the accompanying ``Foreign Visitor Form'' (attached below). All prospective foreign attendees should return this form as soon as possible, no later than 45 days before the meeting (August 17, 2003).

{\bf Hotel Accommodations and Transport}

Accommodation at a discounted rate is available in a number of local hotels. A hotel reservation form follows below, which includes web links. All participants are requested to return it to Alice Newton. With regard to transport, you will find it more flexible if you can \framebox[1cm]{(g)} a car and drive yourself. For those not renting, details of shuttle vans and public transport will be provided later.

{\bf Visitor Information}

Information for visitors to US Atomics is available at http://web.usat.com/visitors/ which provides links to maps, information on things to \framebox[1cm]{(h)} and see in the vicinity, airport information, etc. The online map shows the location of a number of the local hotels.
}

\vspace{3mm}
{\parindent 8mm
Interferometry: 干渉計測 \qquad DOE: Department of Energy

UCLA: University of California, Los Angleles
}
\vspace{5mm}

\begin{enumerate}
%%%%%%【1】%%%%%%
    \item 研究集会参加者が文中の人物Alice Newtonに送らなければならない書類を全て列挙せよ。ただし、英語、日本語どちらで答えてもよい。

%%%%%%【2】%%%%%%
    \item この案内文中で、「後日案内する」とされている項目を全て列挙せよ。ただし、英語、日本語どちらで答えてもよい。

%%%%%%【3】%%%%%%
    \item 文中の\framebox[1cm]{(a)}〜\framebox[1cm]{(h)}に当てはまる語を以下のリストから選択せよ。ただし、同じ語を二回以上選択してはならない。

abstract, \quad complete, \quad discussion, \quad do, \quad indicate, \quad paper, \quad presentations, \quad rent

%%%%%%【4】%%%%%%
    \item 下記の状況設定のもとで、Dr. E. Kepler宛の英文手紙をかけ。必要ならば適当に情報を追加してよい。

\begin{quotation}
    差出人の名前は夏目三四郎で、本郷大学の学生である。ごく最近、この研究集会の主題に関連した画期的な実験結果を得た。そこで、この研究集会に出席し、発表したいと思うが、参加申し込みの締め切り日はすでに過ぎている。
\end{quotation}
\end{enumerate}
\end{question}

%%%%%【英語1 解答】%%%%%%%%%%%%%%%%%%%%%%%%%%%%%%%%%%%%%%%%%%%%%%%%%%%%%%%

\begin{answer}{問題1}{茂木康平}
\setcounter{equation}{0}

{\bf \noindent 全訳}

\begin{center}
    \bf 2度目の告知 \, 第6回国際干渉計測研究集会

    2003年10月 1-3日  アメリカ原子力機関, San Diego, USA
\end{center}

{\parindent 0zw
{\bf 開催地 日程 主催者}

第6回国際干渉計測研究集会がSan Diego CaliforniaのUS Atomicsのメーンキャンパスで開かれます。集会の期間は10月1-3~日の3日間です。地元の主催科学者はUCLAのE.KeplerとT.Planckです。一方、US AtomicsはAlice Newtonにより集会施設と行政上の支援を提供します。

{\bf 集会構成}

前の集会のときのように私たちは参加者間の広範囲にわたる交流、議論の機会のための真の「研究集会」の構成したい。この目的のために集会は議論のための定められた時間を伴う口頭発表(10-20分)からなる。

{\bf 集会での夕食 施設への訪問}

夕食会が集会の参加者のために、参加者負担でおこなわれます。詳細は後で発表されます。US Atomics施設への訪問は興味のある人のためにとりはかわられます。(興味があれば返答する際に示唆してください。)

{\bf 登録とアブストラクト}

参加する可能性のあるすべての人は2003年8月17日までにAlice Newtonに、取り付けられた訪問外国者用紙を仕上げて返信することによって登録するべきです(下の(安全保障上、必要なこと)を見よ)。アメリカ国民は名前と連絡の詳細を仕上げ国民の身分を示すだけで十分です。科学プログラムを準備するため、8月22日までにE.Keplerに要約を送るべきです。WordかPDFファイルにして電子メールで提出してください。

{\bf 議事}

集会のあとで参加者に配布されるCD-ROMに含むため、参加者は発表のコピーを提出するべきです。参加者はより完全な発表の記録として最大10ページで議事のために論文を提出することがすすめられます。

{\bf 安全保障上、必要なこと}

エネルギー省の施設なのでUS Atomics施設へのアクセスは制限されます。アメリカ国民でないすべての人は外国訪問フォーム(下に取り付けられてる)を仕上げて返信しなければなりません。参加する見込みのあるすべての外国人は集会(2003年10月17日)の45日前よりも前にできるだけ早く返信するべきです。

{\bf 宿泊設備と移動}

多くの地域ホテルで値引きされた値段で宿泊設備が利用可能です。ホテルの予約フォームは以下にあってウェブのリンクを含みます。すべての参加者はAlice Newtonに返信することが求められます。移動に関しては車を借りて自分自身で運転することができればかなり融通がきくとわかるでしょう。借りない人たちにはシャトルバスや公共の移動手段の詳細があとで提供されます。

{\bf 訪問者情報}

US Atomicsへの訪問者の情報はhttp://web.usct.com/visitors/で利用可能でそれは地図へのリンク、近隣の観光情報、空港情報等を提供します。オンライン地図はたくさんの地域ホテルの位置を示します。
}

\begin{enumerate}
%%%%%%【1】%%%%%%
    \item Foreign Visitor Form, \, hotel reservation form 

%%%%%%【2】%%%%%%
    \item  dinner, \, details of shuttle vans and public transport

%%%%%%【3】%%%%%%
    \item (a) discussion \, (b) indicate \, (c) abstract \, (d) presentations \, (e) paper \, (f) complete \, (g) rent \, (h) do


%%%%%%【4】%%%%%%
    \item
\begin{verbatim}
                                                       University of Hongo.
                                                        Hongo Bunkyoku, Tokyo
                                                        August 30, 2003  
Mr. E.Kepler
University of California Los Angeles.
San Diego
U.S.A.

      Dear Dr. E. Kepler,               
            My name is Sanshiro Natsume, a student of University of Hongo.
            Recently I got an amazing experimental result related to the subject of
            this workshop organized by you. So I wish I could attend the workshop 
            to present the results, but the deadline for application to attend is
            over. Would you please make it so that I can attend the workshop?
            I will be waiting for your reply.

                                                        Yours truly,
                                                           Sanshiro Natsume
	 \end{verbatim}




\end{enumerate}

\end{answer}


%%%%%【英語2 問題】%%%%%%%%%%%%%%%%%%%%%%%%%%%%%%%%%%%%%%%%%%%%%%%%%%%%%%%
\begin{question}{問題2}{茂木康平}

次の英文を読んで、設問1, 2, 3に答えよ。

{\parindent=10pt
The second law of thermodynamics is the law of decay. It says that isolated systems tend to become more disordered as time passes: that iced and hot tea soon become indistinguishable, that all living creatures must sicken and die, and that all machines must someday wear out.

Consider a room which is divided into two sections, A and B, by an impermeable partition where there is a pinhole. The air on both sides is at the same pressure and temperature. Randomly moving air molecules, some fast and some slow, pass through the pinhole in either direction. It may be unlikely, but surely it is possible, that most of the molecules passing from A to B happen to be fast, while most of those passing from B to A happen to be slow. If the process continues long enough, region B becomes unbearably hot while region A becomes frigid. The second law of thermodynamics is flouted! However, the probability of this occurrence is unimaginably tiny. It is as unlikely as flipping a fair coin a billion billion times and coming up heads every time. It is so unlikely as to be, for any practical purpose or otherwise, impossible.

Enter James Clerk Maxwell, the nineteenth-century physicist. \underlineeng[]{He proposed a scheme to evade the second law that does not rely on chance. ``Now consider a creature,'' he wrote, ``who knows the paths of all the molecules, but who can do no work but to open and close the hole in the diaphragm.'' This hypothetical being, known as {\it Maxwell's demon}, allows fast molecules to pass from A to B and slow molecules to pass from B to A.} All others find the door closed. In this case, a temperature difference between the two regions will certainly develop. The second law is undone ``by the intelligence of a very observant and neat-fingered being.''

There is a loophole to this argument; you can't fool Mother Nature. The system consisting of room, diaphragm, door, and demon is not isolated. The metabolism of the imaginary being must be taken into account. Whether it be microbe, machine, or genetically engineered monster, it must be connected to the outside world to take its sustenance and eject its waste products. When these are taken into account, the order created in the room is necessarily compensated, or more than compensated, by the disorder produced outside its walls.
}

\begin{table}[h]
    \hspace{8mm}
    \begin{tabular}{ll}
	flout :  (規則を)無視する、侮辱する & evade : (巧みに)避ける、逃れる \\
	diaphragm : 障壁 & metabolism : (新陳)代謝 \\
	microbe : 微生物 & sustenance : 食物、栄養物
    \end{tabular}
\end{table}

\begin{enumerate}
    \item 文中の下線部を和訳せよ。

    \item この英文では、「マックスウェルの悪魔」が存在して熱力学の第2法則が破れることがあり得ると言っているのか、あり得ないと言っているのか、理由も含めて200字程度の日本語で答えよ。

    \item 熱力学の第2法則を表す具体的な例を50 words程度の英語で示せ。ただし、この英文とは別の例を挙げること。
\end{enumerate}

\end{question}

%%%%%【英語2 解答】%%%%%%%%%%%%%%%%%%%%%%%%%%%%%%%%%%%%%%%%%%%%%%%%%%%%%%%

\begin{answer}{問題2}{茂木康平}
\setcounter{equation}{0}

{\bf \noindent 全訳}

{\parindent=10pt
熱力学第二法則は崩壊の法則である。それによると、時間が経過するに従い孤立系はより無秩序になりやすい。すなわち、氷で冷やしたお茶と熱いお茶はいつか見分けがつかなくなる。すべての生物は病気になり死ぬ。すべての機械はいつか使えなくなる、ということである。

ピンホールのある浸透性のない仕切りで二つの部分AとBに分割された部屋を考えよう。両方での空気は等圧、等温である。中には速く、中には遅いランダムに動く空気の分子はピンホールをいずれかの方向に通り抜ける。AからBへ通るほとんどの分子がたまたま速い一方、BからAへ通るほとんどの分子がたまたま遅いことはありえなさそうだが、もちろん可能性はある。もしその過程が十分長く続けば領域Bは耐え難いほど熱くなる一方、領域Aは極寒になる。熱力学第二法則は無視されている!しかしながらこれが起こる確率は信じられないくらい小さい。それは公正なコインを何十億の何十億回投げて毎回表が出るくらいありえない。いかなる実用的な目的やそうでなくてもそれは不可能なくらいありえない。

19世紀の物理学者、ジェームズ クラーク マックスウェルが現れる。彼は確率に依らない、第二法則を避ける案を提案した。「さあ、ある生き物を考えよう」彼は書いた、「それはすべての分子の経路を知っているが、障壁の穴を開けたり、閉じたりする仕事しかできない。」マックスウェルの悪魔として知られる仮説上の存在は速い分子がAからBへ通り、遅い分子がBからAへ通ることを許容する。他のすべての人はドアが閉じているとわかる。この場合、二つの領域の温度差は確実に上昇する。第二法則は「非常に注意深く、適切な指をもつ存在の知性」によって破れる。

この議論には抜け穴がある。大自然をだますことはできない。部屋、障壁、ドア、悪魔からなる系は孤立していない。想像上の存在の代謝を考慮にいれないといけない。微生物、機械や遺伝的に設計された怪物であれ、栄養物をとり、廃棄物を排出するため、外部の世界とつながっていなければならない。これらが考慮されたとき、壁の外部で生み出された無秩序により、部屋でつくりだされた秩序は必然的に補われ、あるいは補われる以上のものがある。
}

\begin{enumerate}
%%%%%%【1】%%%%%%
    \item マックスウェルは確率に依らない第二法則を避ける案を提案した。「さあ、ある生物を考えよう。」彼は書いた。「それは全ての分子の経路をしっているが、隔壁の穴を開けたり閉じたりする仕事しかできない。」マックスウェルの悪魔として知られるこの仮説上の生物は速い分子がAからBへ通り、遅い分子がBからAへ通ることを許容する。

%%%%%%【2】%%%%%%
    \item 部屋、隔壁、ドア、悪魔から成り立ってる系は孤立系ではない。またマックスウェルの悪魔の代謝も考える必要がある。それがいかなるものであれ、栄養物を摂取し、不要物を排出するために外部世界とつながっていなければならない。これらを考慮に入れると、部屋で作り出した秩序は部屋の外部の無秩序によって補われ、熱力学第二法則が破れることはない。

%%%%%%【3】%%%%%%
    \item Consider a system made of atoms whose spins are at first in a specific direction. As time passes by the direction of spins become isotropic. Thus the ferromagnetic system will turn to paramagnetic system.

\end{enumerate}

\end{answer}

\end{document}

