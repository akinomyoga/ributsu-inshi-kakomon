%% -*- coding:sjis -*-
%%
%% 2013-07-16, KM, 入力
%%
\begin{question}{第2問}{村瀬}
二つの実変数 $x,y$ の実関数$u(x,y),\, v(x,y)$ は至るところで有限な値を持ち、任意回微分可
能で、さらに関係式
\begin{align}
  \frac{\partial u(x,y)}{\partial x} = \frac{\partial v(x,y)}{\partial y}, \quad
  \frac{\partial u(x,y)}{\partial y} = -\frac{\partial v(x,y)}{\partial x} \ilabel{2006mathQ2.1}
\end{align}
を満たしているものとする。
\begin{enumerate}
\item
  以下の等式が成り立つことを示せ。
  \begin{align}
    \frac{\partial^2 u(x,y)}{\partial x^2} + \frac{\partial^2 u(x,y)}{\partial y^2} =0, \quad
    \frac{\partial^2 v(x,y)}{\partial x^2} + \frac{\partial^2 v(x,y)}{\partial y^2} =0. \ilabel{2006mathQ2.2}
  \end{align}

\item
  ストークスの定理は、微分可能なベクトル場 $\bm{A}(\bm{x})$ の、閉曲線$\mathrm{C}$を一周する線積分と、$\mathrm{C}$
  を縁とする面$\mathrm{S}$上の面積分との関係を表し、次式で与えられる。
  \begin{align}
    \oint_{\mathrm{C}} \bm{A}(\bm{x})
    &= \iint_{\mathrm{S}} [\bm{\nabla}\times\bm{A}(\bm{x})]\cdot\bm{n}(\bm{x})dS. \ilabel{2006mathQ2.3}
  \end{align}
  ただし、ベクトル$\bm{n}(\bm{x})$は、面$\mathrm{S}$上の点$\bm{x}$におけるこの面の法線ベクトルである。この定
  理を用いて、$xy$平面上の任意の閉じた経路$\mathrm{C}$に関する線積分に対して、以下の二式が成
  り立つことを示せ。
  \begin{align}
    \oint_{\mathrm{C}}[u(x,y)dx - v(x,y)dy] = 0, \quad
    \oint_{\mathrm{C}}[u(x,y)dy + v(x,y)dx] = 0. \ilabel{2006mathQ2.4}
  \end{align}

\item
  $i$を虚数単位とし、複素数$z$を$z=x+iy$によって定義する。また、複素関数$f(z)$は実
  関数 $u(x,y),\, v(x,y)$を用いて、$f(z)=u(x,y)+iv(x,y)$のように表されるものとする。
  式\ieqref{2006mathQ2.4}が成り立っているとき、複素平面上の任意の閉じた経路$\mathrm{C}$に関する線積分に対し
  て、以下の等式が成り立つことを示せ。
  \begin{align}
    \oint_{\mathrm{C}}f(z)dz &= 0. \ilabel{2006mathQ2.5}
  \end{align}

\item
  複素関数 $f(z)$ の例として、$f(z)=e^{-x^2}$ を考える。\\
  $f(x+iy)$ の実部を $u(x,y)$, 虚部を$v(x,y)$としたとき、式\ieqref{2006mathQ2.1}が成り立つことを示せ。

\item
  以上の結果を用いて、実数$p$に対して、定積分
  \begin{align}
    \int_{-\infty}^{\infty} e^{-x^2} \cos(2px) dx \ilabel{2006mathQ2.6}
  \end{align}
  を求めよ。ただし、
  \begin{align}
    \int_{-\infty}^{\infty} e^{-x^2} dx &= \sqrt{\pi} \ilabel{2006mathQ2.7}
  \end{align}
  となることは既知としてよい。
\end{enumerate}
\end{question}
