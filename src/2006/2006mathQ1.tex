%% -*- coding:sjis -*-
%%
%% 2013-07-16, KM, 入力
%%
\begin{question}{第1問}{村瀬}
正方行列$A$に対して、$U^{-1}AU$が対角行列になるようなユニタリ行列$U$が存在するためには、
$A$ が正規行列であることが必要十分である。正規行列とは、$A^*A=AA^*$ を満たす行列であり、
$A^*$ は行列$A$を転置し複素共役をとった行列である。実ユニタリ行列を直交行列と呼ぶ。以下
の設問に答えよ。
\begin{enumerate}
\item
  行列
  \begin{align*}
    B &= \begin{pmatrix}
      \cos\theta & -\sin\theta \\
      \sin\theta & \cos\theta
    \end{pmatrix}
  \end{align*}
  は直交行列であることを示せ。ここで、$\theta$ は実数。

\item
  任意の正方行列$A$について、
  \begin{align*}
    \exp A &\equiv \sum_{m=0}^\infty \frac{A^m}{m!}
  \end{align*}
  は収束し、$\exp A$ を定義することができる。ただし、$A^0$ は単位行列とする。
  $A=\begin{pmatrix} 0 & -\theta \\ \theta & 0 \end{pmatrix}$ のとき、
  \begin{align*}
    \exp A &= \begin{pmatrix}
      \cos\theta & -\sin\theta\\
      \sin\theta & \cos\theta
    \end{pmatrix} = B
  \end{align*}
  を示せ。

\item
  行列$B$を対角化せよ。
\end{enumerate}
次に、一般の次元をもつ行列の例として、変数の組$\{x_1,\ldots,x_n\}$ について
\begin{align*}
  C &= \begin{pmatrix}
    1 & 1 & \ldots & 1\\
    x_1 & x_2 & \ldots & x_n\\
    \vdots & \vdots & \vdots & \vdots\\
    x_1^{n-1} & x_2^{n-1} & \ldots & x_n^{n-1}
  \end{pmatrix}
\end{align*}
を考えよう。

\begin{enumerate}\setcounter{enumi}{3}
\item
  $|C|\propto\prod_{i<j}(x_j-x_i)$ を示せ。ここで $|C|$ は行列$C$の行列式。
\item
  この結果を用いて、上記の行列$C$が正則である (逆行列をもつ) のはどのような場合か
  を述べよ。
\end{enumerate}
\end{question}
