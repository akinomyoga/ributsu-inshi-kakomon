%% -*- coding:sjis -*-
\newcommand{\pdif}[2]{\frac{\partial {#1}}{\partial {#2}}}
\newcommand{\odif}[2]{\frac{d{#1}}{d{#2}}}
\begin{answer}{第3問}{見上敬洋}
\begin{enumerate}
\item
  ベクトル$(\xi,\eta)$は,ベクトル$(x, y)$を$+\omega t$回転させたものなので
  \begin{equation}
  \begin{pmatrix} \xi \\ \eta \end{pmatrix}
  = \begin{pmatrix} \cos \omega t & -\sin\omega t \\ \sin\omega t & \cos \omega t \\\end{pmatrix}
  \begin{pmatrix} x \\ y \\\end{pmatrix}. \notag
  \end{equation}

\item
  慣性系$S$でのラグランジアン$\displaystyle \mathcal{L} = \frac m2 (\dot{\xi}^2+\dot{\eta}^2) + Gm \frac{m_1}{|\mathrm{PP_1}|} + Gm\frac{m_2}{|\mathrm{PP_2}|} $を,前問の関係式を使って書き直すだけです.

\item
  Lagrange 方程式$\displaystyle \pdif{\mathcal{L}}{q} - \odif{}{t}\pdif{\mathcal{L}}{\dot{q}}$を$q = x,y$について立てます.結果は次のようになります.
  \begin{align}
  0 &= m\ddot{x} -2m\omega \dot{y} -m\omega^2 x + Gmm_1 \frac{x+\mu_2a}{r_1^3} + Gmm_2 \frac{x-\mu_1a}{r^3_2}, \ilabel{3eqx}\\
  0 &= m\ddot{y} +2m\omega \dot{x} -m\omega^2 y + Gmm_1 \frac{y}{r_1^3} + Gmm_2 \frac{y}{r^3_2}. \ilabel{3eqy}
  \end{align}

\item
  式\ieqref{3eqy}で,時間微分をゼロと置けば
  \begin{equation}
  y\left( -\omega^2 + \frac{Gm_1}{r_1^3} + \frac{Gm_2}{r_2^3}\right) = 0 \notag
  \end{equation}
  という式を得ますが,$m_1 G = \mu_1 \omega^2 a^3,\> m_2 G = \mu_2 \omega^2 a^3$に注意すれば
  \begin{equation}
  y\left(-1 + \mu_1\frac{a^3}{r_1^3}+ \mu_2\frac{a^3}{r_2^3}\right) = 0, \quad
  \therefore y = 0 \> \text{  または  } \> \mu_1\frac{a^3}{r_1^3}+ \mu_2\frac{a^3}{r_2^3} = 1 .\notag
  \end{equation}

\item
  式\ieqref{3eqx}で,時間微分と$y$をゼロとおくと
  \begin{equation}
  -\omega^2 x + Gm_1 \frac{x+\mu_2a}{r_1^3} + Gm_2 \frac{x-\mu_1a}{r_2^3}  = 0 \notag
  \end{equation}
  となります.位置関係から$r_1 = x + \mu_2 a ,\> r_2 = x - \mu_1a$が分かるので,$x=\mu_1a + r_2, \> r_1 = r_2 + a$ として$x,r_1$を消去します.さらに$\> m_1 G = \mu_1 \omega^2 a^3,\> m_2 G = \mu_2 \omega^2 a^3$を使うと
  \begin{equation}
  -(\mu_1 a + r_2) + \frac{a^3\mu_1}{(r_2+a)^2} + \frac{a^3\mu_2}{r_2^2}  = 0 \notag
  \end{equation}
  が得られます.$a$で割り,$1 = \mu_1+\mu_2 $を適当に割り込むと
  \begin{equation}
  -\mu_1 - u(\mu_1+\mu_2) + \frac{\mu_1}{(1+u)^2}+\frac{\mu_2}{u^2} = 0 \notag
  \end{equation}
  となります.これを$\mu_2/\mu_1$について解けば次が得られます.
  \begin{equation}
  \frac{\mu_2}{\mu_1} = \frac{3u^3}{(1+u)^2} \frac{1+u+u^2/3}{1-u^3}. \ilabel{3eq}
  \end{equation}

\item
  式\ieqref{3eq}の$u\to 0$の極限を取ると,$\mu_2/\mu_1 \simeq 3u^3$です.よって
  $\displaystyle u \simeq \left[\frac 13 \frac{\mu_1}{\mu_2} \right]^{1/3} = 1\times 10^{-2}$となり,$r_2 = au = 1.5\times 10^9 \mathrm{m}$となります.
\end{enumerate}
\end{answer}
