%% -*- coding:sjis -*-
\newcommand{\pdif}[2]{\frac{\partial {#1}}{\partial {#2}}}
\newcommand{\ket}[1]{\left| {#1} \right\rangle}
\newcommand{\kett}[1]{\left. \left| {#1} \right\rangle \!\! \right\rangle}
\begin{answer}{第2問}{見上敬洋}
\begin{enumerate}
\item
  相互作用が無い$N$粒子系なので,分配関数 $Z$は次のように計算されます.
  \begin{equation}
  Z = \sum_{i:\text{all state}} e^{-\beta E_i} = \left( \sum_{i:\text{1 particle state}} e^{-\beta \epsilon_i} \right)^N = (2 \cosh \beta \mu H)^N . \notag
  \end{equation}

  したがって自由エネルギーは$\displaystyle F = -\frac 1\beta \ln Z = -\frac N\beta \ln(2\cosh \beta \mu H) .$

\item
  計算だけなので結果だけ記すと
  \begin{align}
  M &= -\pdif{F}{H} = \mu N \tanh \beta \mu H, \notag \\
  \chi &= \pdif{M}{H} = \beta \mu^2 N \frac{1}{\cosh ^2 \beta \mu H} \stackrel{H\to 0}{\longrightarrow} \beta \mu^2 H .\notag
  \end{align}

\item
  $\bm{S} = \bm{s}_1+\bm{s}_2$と置くと,ハミルトニアンは
  \begin{equation}
  \mathcal{H} = V \frac{\bm{S}^2-\bm{s}_1^2 -\bm{s}_2^2}{2} -2\mu S^z H \notag 
  \end{equation}
  とかけます.2粒子の$s_1^z, s_2^z$に関する固有状態$\ket{s_1^z}\ket{s_2^z}$を,合成角運動量$\bm{S}^2, S^z$の固有状態$\kett{S,S^z}$に取り直せば,$\mathcal{H}$の固有状態になることが分かります.結果は次の通りです.
  \begin{table}[ht!]
  \begin{center}
    \caption{ハミルトニアン$\mathcal{H}$の固有状態と固有値}\ilabel{2table}
      \begin{tabular}{|c|cccl|} \hline
      & $S(S+1)$ & $S^z$ & $s_i(s_i+1)$ & $\mathcal{H}$ \\\hline
      $\kett{1,1}$ & 2 & 1 & 3/4 & $\frac V4 -2\mu H $ \\
      $\kett{1,0}$ & 2 & 0 & 3/4 & $\frac V4 $ \\
      $\kett{1,-1}$ & 2 & -1 & 3/4 & $\frac V4 +2\mu H $ \\
      $\kett{0,0}$ & 0 & 0 & 3/4 & $-\frac {3V}4$\\\hline
      \end{tabular}
    \end{center}
  \end{table}

\item[4,5.]
  ペア固有エネルギーが求まったので,分配関数は
  \begin{equation}
  Z = \left( \sum_{i:\text{pair state}} e^{-\beta \epsilon_i} \right)^{N/2}
  =\cdots =  \left( 2e^{-\beta V/4} \cosh 2\beta\mu H + 2e^{\beta V/4} \cosh \frac{\beta V}{2} \right)^{N/2}\notag
  \end{equation}
  と求まり,自由エネルギー$F$,磁化$M$,磁化率$\chi$は次のように計算されます.
  \begin{align}
  F &= -\frac 1\beta \ln Z = -\frac {N}{2\beta}\ln\left( 2e^{-\beta V/4} \cosh 2\beta\mu H + 2e^{\beta V/4} \cosh \frac{\beta V}{2}\right), \notag\\
  M &= -\pdif{F}{H} = \mu N \frac{\sinh 2\beta \mu H}{ \cosh 2\beta \mu H + \dfrac 12 (1+e^{\beta V}) }, \notag \\
  \chi &=\pdif{M}{H}  \stackrel{H\to 0}{\longrightarrow} = \frac{4\beta\mu^2N}{3+e^{\beta V}}. \notag
  \end{align}

\setcounter{enumi}{5}
\item
  $V$の正負で$\chi$の低温極限を見てみると
  \begin{equation}
    \chi \stackrel{\beta \to \infty}{\longrightarrow}
    \begin{cases}
      0 & (V>0) \\
      \beta \mu^2 N & (V=0) \\
      \dfrac 43 \beta \mu^2 N & (V<0) \\
    \end{cases} \notag
  \end{equation}
  となります.$V=0$はペア間の結合がない,問題{\bf 2. }の再現です.

  $T=0$における基底状態をみるために,$H=0$でのペア状態のエネルギー固有値の表\iref{2table}を見てみます.

  $V>0$の時は,1重項$\kett{0,0}$が最もエネルギーが低いので,$T=0$ではこの状態が選ばれます.こうしてスピン0の2粒子分子が並ぶので,磁場に対する応答は生じず$\chi = 0$です.

  $V<0$のときは,3重項$\kett{1,S^z}$が基底状態に選ばれます.今度はスピン1の2粒子分子が並ぶことになります.この3つの準位は外部磁場によってエネルギー分岐を起して常磁性を生じます.

  \medskip

  \paragraph{たしかめ} 
  {\small 上で導いた$\chi$の係数\footnote{最初にupしたものが計算ミスの指摘をうけたので,別の方法で確かめました.川瀬君 thanks.}は,実際にスピン$S=1$の粒子を$N' = N/2$個並べてみれば分かります.
  \[
  \mathcal{H} = \sum_{n=1}^{N'} \mu' \hat{s}_n^z\cdot H 
  \]
  なるハミルトニアンで,$\hat{s}_n^z$の固有値$-S,-S+1,\cdots, S-1,S$を考えるよくあるモデルを考えると,単純な計算で次が分かります.
  \begin{gather}
  Z = \left[\frac{\sinh\dfrac{2S+1}{2}\beta\mu' H}{\sinh\dfrac 12 \beta \mu' H }\right]^{N'}, \notag\\
  F = -\frac{N'}{\beta}\ln \frac{\sinh\dfrac{2S+1}{2}\beta\mu' H}{\sinh\dfrac 12 \beta \mu' H }, \notag\\
  M = \mu' N'\left[\frac{2S+1}{2}\coth\left(\frac{2S+1}{2}\beta\mu' H\right) - \frac 12 \coth\left(\frac 12 \beta\mu' H\right)\right], \notag \\
  \chi \stackrel{H\to 0}{\longrightarrow} \frac{S(S+1)}{3}\beta\mu'^2 N' . \notag
  \end{gather}
  問題\textbf{6.}の$V<0$を再現するには,$S=1,\mu' = 2\mu, N' = N/2$とすれば$\chi \to (4/3)\beta\mu^2N$が出ます.

  $V=0$は$S=1/2, \mu' =2\mu, N' = N$とすれば再現されます.
  }

\end{enumerate}
\end{answer}
