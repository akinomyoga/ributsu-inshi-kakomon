%% -*- coding:sjis -*-
%%
%% 2013-07-16, KM, math-19-ishikawa.tex から作成
%%
\newcommand{\tr}[1]{{}^t\hspace{-1mm}#1}
\newcommand{\pdif}[2]{\frac{\partial #1}{\partial #2}}

\begin{answer}{第1問}{石川}
\begin{enumerate}
\item
  \begin{eqnarray}
  B\tr{B} &=& 
  \left(
    \begin{array}{cc}
     \cos\theta & -\sin\theta \\
     \sin\theta  & \cos\theta 
    \end{array}
   \right)
   \left(
    \begin{array}{cc}
     \cos\theta & \sin\theta \\
     -\sin\theta  & \cos\theta 
    \end{array}
   \right)\\
   &=&
   \left(
    \begin{array}{cc}
     1 & 0 \\
     0  & 1 
    \end{array}
   \right).
  \end{eqnarray}

\item
  \begin{eqnarray}
    A^2 = -\theta^2 I
  \end{eqnarray}
  なので
  \begin{eqnarray}
    \exp{A}
      &=& \sum^{\infty}_{m=0}\frac{A^m}{m!}\\
      &=& \sum^{\infty}_{m=0}\frac{A^{2m}}{2m!}+\sum^{\infty}_{m=0}\frac{A^{2m+1}}{(2m+1)!}\\
      &=& \sum^{\infty}_{m=0}\frac{(-\theta^2)^{m}}{2m!}I+\sum^{\infty}_{m=0}\frac{(-\theta^2)^m}{(2m+1)!}A\\
      &=& \sum^{\infty}_{m=0}\frac{(-1)^m\theta^{2m}}{2m!}
        \left(
          \begin{array}{cc}
           1 & 0 \\
           0 & 1 
          \end{array}
        \right)
        +\sum^{\infty}_{m=0}\frac{(-1)^m\theta^{2m+1}}{(2m+1)!}
        \left(
          \begin{array}{cc}
           0 & -1 \\
           1 & 0 
          \end{array}
        \right)\\
      &=& 
        \left(
          \begin{array}{cc}
           \cos\theta & -\sin\theta \\
           \sin\theta  & \cos\theta 
          \end{array}
        \right) = B.
  \end{eqnarray}

\item
  \begin{eqnarray}
  \left|
    \begin{array}{cc}
     \cos\theta-\lambda & -\sin\theta \\
     \sin\theta  & \cos\theta-\lambda 
    \end{array}
   \right| &=& 0,\\
   \lambda = e^{\pm i \theta}
  \end{eqnarray}
  となるので固有ベクトルとして\\
  \begin{equation}
    \bm{x}_{\pm}=\frac{1}{\sqrt{2}}\left(
      \begin{array}{c}
        1 \\
        \mp i  
      \end{array}
    \right)
  \end{equation}
  がとれる。\\
  \begin{align}
    U \equiv \left(
      \begin{array}{cc}
        \bm{x}_{+}  & \bm{x}_{-} 
      \end{array}
    \right) &= \frac{1}{\sqrt{2}}\left(
      \begin{array}{cc}
       1 & 1 \\
       -i  & i 
      \end{array}
    \right),\\
    U^{-1}BU &= \left(
      \begin{array}{cc}
        e^{i\theta} & 0 \\
        0 & e^{-i\theta} 
      \end{array}
    \right).
  \end{align}

\item
  $n$に関する数学的帰納法で示す:\\
  \begin{equation}
    |C|= \prod _{i<j}(x_j-x_i) \ilabel{2006mathA1.1}
  \end{equation}
  と仮定する. $n=2$のときは成立.\\
  式\ieqref{2006mathA1.1}が$n-1$のとき正しいと仮定する.\\
  第$i$行から第$i-1$行の$x_1$倍をひく、という操作を$i=n,n-1,\ldots,3,2$の順に行うと\\
  \begin{eqnarray}
    |C|
      &=& \begin{vmatrix}
         1 & 1 & 1 & \cdots  & x_n\\
         0 & x_2-x_1 & x_3-x_1 & \cdots  & x_n-x_1 \\
         \vdots & \vdots & \vdots & & \vdots\\
         0 & x_2^{n-2}(x_2-x_1) & x_3^{n-2}(x_3-x_1) & \cdots  & x_n^{n-2}(x_n-x_1) 
        \end{vmatrix}\\
      &=&(x_2-x_1)(x_3-x_1)\cdots (x_n-x_1)
        \begin{vmatrix}
          1 & 1 & \cdots  & 1\\
          x_2 & x_3 & \cdots  & x_n \\
          \vdots & \vdots & & \vdots\\
          x_2^{n-2} & x_3^{n-2} & \cdots  & x_n^{n-2} 
        \end{vmatrix}\\
      &=&\prod_{j=2}^n (x_j-x_1) \prod_{i,j\geq 2 , i<j}(x_j-x_i) = \prod_{i<j}(x_j-x_i).
  \end{eqnarray}
  よって$n\geq 2$のとき式\ieqref{2006mathA1.1}が成り立つ.
\item
  $|C|\neq 0$であればよいので、任意の互いに異なる$i,j$について$x_i\neq x_j$であること.
\end{enumerate}
\end{answer}

\begin{answer}{第2問}{石川}
\begin{enumerate}
\item
  \begin{eqnarray}
    \pdif{^2u(x,y)}{x^2}+\pdif{^2u(x,y)}{y^2} &=& \pdif{^2v(x,y)}{x\partial y} - \pdif{^2v(x,y)}{y\partial x}
      = 0,\\
    \pdif{^2v(x,y)}{x^2}+\pdif{^2v(x,y)}{y^2} &=& -\pdif{^2u(x,y)}{x\partial y} + \pdif{^2u(x,y)}{y\partial x}
      = 0.
  \end{eqnarray}

\item
  \begin{eqnarray}
    \nabla \times \left(
      \begin{array}{c}
        u(x,y) \\
        -v(x,y)\\
        0
      \end{array}
    \right)
      &=&
      \left(
        \begin{array}{c}
          0\\
          0\\
          -\pdif{v(x,y)}{x} -\pdif{u(x,y)}{y}
        \end{array}
      \right) = \bm{0},\\
  \nabla \times \left(
    \begin{array}{c}
      v(x,y) \\
      u(x,y)\\
      0
    \end{array}
  \right)
    &=&
    \left(
      \begin{array}{c}
        0\\
        0\\
        \pdif{u(x,y)}{x} -\pdif{v(x,y)}{y}
      \end{array}
    \right) = \bm{0}.
  \end{eqnarray}

\item
  \begin{eqnarray}
    \oint _{C}f(z)dz &=& \oint_{C}\{u(x,y)+iv(x,y)\}d(x+iy)\\
      &=& \oint_{C}u(x,y)dx-v(x,y)dy + i\oint_{C}u(x,y)dy+v(x,y)dx\\
      &=& 0.
  \end{eqnarray}

\item
  \begin{eqnarray}
    u(x,y) &=& e^{y^2-x^2}\cos(2xy),\\
    v(x,y) &=& -e^{y^2-x^2}\sin(2xy)
  \end{eqnarray}
  となるので\\
  \begin{eqnarray}
  \pdif{u(x,y)}{x} = -2xe^{y^2-x^2}\cos(2xy)-2ye^{y^2-x^2}\sin(2xy) = \pdif{v(x,y)}{y},\\
  \pdif{u(x,y)}{y} = 2ye^{y^2-x^2}\cos(2xy)-2xe^{y^2-x^2}\sin(2xy) = -\pdif{v(x,y)}{x}.
  \end{eqnarray}

\item
  $f(z)=e^{-z^2}$を、$p,R$を実数として経路$C_1→C_2→C_3→C_4$で複素積分する.\\
  ただし\\
  $C_1:-R→R$,\\
  $C_2:R→R+ip$,\\
  $C_3:R+ip→-R+ip$,\\
  $C_4:-R+ip→-R$.
  \begin{eqnarray}
    \lim_{R\rightarrow \infty}\oint_{C}f(z)dz = \lim_{R\rightarrow \infty}\int_{C_1}f(z)dz+\oint_{C_2}f(z)dz+\oint_{C_3}f(z)dz+\oint_{C_4}f(z)dz.
  \end{eqnarray}
  $R$が無限大の極限で経路$C_2,C_4$は無視できるので\\
  \begin{eqnarray}
    \lim_{R\rightarrow \infty}\oint_{C}f(z)dz &=& \int_{C_1}f(z)dz+\oint_{C_3}f(z)dz\\
      &=&\int_{-\infty}^{\infty}f(z)dz+\oint_{\infty + ip}^{-\infty + ip}f(z)dz.
  \end{eqnarray}
  問題文\ieqref{2006mathQ2.5}より
  \begin{eqnarray}
    \lim_{R\rightarrow \infty}\oint_{C}f(z)dz &=& 0,\\
    \int_{-\infty+ip}^{\infty+ip}e^{-z^2}dz&=&\int_{-\infty}^{\infty}e^{-z^2}dz,\\
    \int_{-\infty}^{\infty}e^{-(x+ip)^2}dx&=&\sqrt{\pi}. \ilabel{2006mathA2.1}
  \end{eqnarray}
  となるので\\
  \ieqref{2006mathA2.1}をもちいて問題文\ieqref{2006mathQ2.6}の定積分を計算すればよい\\
  \begin{eqnarray}
    \int_{-\infty}^{\infty}e^{-x^2}\cos(2px)dx &=& \frac{1}{2}\left(\int_{-\infty}^{\infty}e^{-x^2+i2px}dx+\int_{-\infty}^{\infty}e^{-x^2-i2px}dx\right)\\
      &=& \frac{e^{-p^2}}{2}\left(\int_{-\infty}^{\infty}e^{-(x-ip)^2}dx+\int_{-\infty}^{\infty}e^{-(x+ip)^2}dx\right)\\
      &=& \sqrt{\pi}e^{-p^2}.
  \end{eqnarray}
\end{enumerate}
\end{answer}
