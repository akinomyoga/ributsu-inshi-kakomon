%% -*- coding:sjis -*-
\newcommand{\pdif}[2]{\frac{\partial {#1}}{\partial {#2}}}
\begin{answer}{第5問}{見上敬洋}
\begin{enumerate}
\item
  すっきりした答えが出来ないので,思いつくだけ書いておきます.
  \begin{itemize}
  \item 温度計の位置は試料に近い所に取る.(温度計温度=試料温度,とするため)
  \item 試料の不均一を抑えるために,なるべく小さな試料を使う.
  \item 試料付近は,外界からの輻射や,寒剤の対流を抑えるために,シールドをつける.
  \end{itemize}

  \paragraph{コメント1} 
  {\small 以下の問題は,超伝導体の熱力学的な取り扱いが題材になっています.熱力学に関しては,理想気体,van der Waars気体,磁性体などに適用してきたことがあるかと思いますが,超伝導体に対しても,磁化$\bm{M}$もしくは磁束密度$ \bm{B} = \bm{M} + \mu_0\bm{H}$に関する状態方程式,そして内部エネルギーを取り入れれば,たとえば自由エネルギー$G$と,その自然な変数$T,H$によって同様に熱力学が組めます.

  超伝導体の状態方程式は,Meissner効果から出てきます.これはミクロに見れば「超伝導状態は抵抗ゼロで電流が流れるので,レンツ則が完全に働いて($\bm{M} = -\mu_0 \bm{H}$)磁場の進入を許さない($\bm{B} = 0$)」ということです.
  ただし超伝導体には臨界磁場$H_c(T)$なるものがあって,これより大きな磁場がかかると常伝導に転移します.

  以上のミクロな事情から$(H,T)$の全域で$\bm{B}$の状態方程式が得られ,後は比熱$C(T)$を調達すれば内部エネルギーが決まって熱力学の話になるわけです.
  }

  \paragraph{コメント2}
  {\small 
  $H,T$で系の状態が指定されるわけなので,$G_n, G_s$という2つの自由エネルギーがあるのはおかしい,と個人的に思うので,私見を述べてしまいます.コメントお待ちしています.

  $H < H_c(T)$での「常伝導相のエネルギー」$G_n(H,T)$は「仮に相転移が起こらない系(つまり,測定された比熱転移を無視して外挿\footnote{外挿にはもちろん任意性があって,それゆえ$S_{n}(T_c) = \int_0^{T_c}C(T')/T' \, dT'$の値は決まりません.ただ2次相転移で繋がる事情を考慮して,問題文の図3で面積を等しく取ります(問題5-4).(・・・と僕が考えていますが議論中です.thanks 立本君) }し,$(H,T)$全域で$B = \mu_0 H$であるとした系)だった場合の自由エネルギー」であると考えます.当然,相転移が起きる系と比較すると,常伝導相の自由エネルギーは一致し,超伝導領域では$G_s < G_n$を満たすものになるはずです..
  }

\item
  熱力学関係式$dU = T\,dS = C(T)\,dT$を積分すればよいです.
  \begin{equation}
  S(T) = \int_0^T \frac{C(T)}{T} dT , \quad 
  U(T) = U(0) + \int_0^T C(T) \,dT  \notag
  \end{equation}

\item
  $C/T$直線と縦軸との交点を$\alpha = 1.35 \mathrm{mJ/mol\cdot K^2}$と読んで,$\dfrac CT = \alpha + \gamma T$すなわち$C = \alpha T + \gamma T^3$となります.$\alpha T$は電子比熱による線形項です.

\item
  2次相転移なので,エントロピー$S = -\dfrac{dG}{dT}$は連続.したがって
  \begin{equation}
  0 = S_s(T_c) - S_n(T_c) = \int_0^T \frac{C_s -C_n}{T} dT \notag
  \end{equation}
  これは$C_s/T$と$C_n/T$曲線で挟まれる面積が等しく分割されていることを示しています.

  \begin{figure}[h]
    \begin{center}
      %% \begin{wrapfigure}[10]{r}{5cm}
      %% -*- coding:sjis -*-
%WinTpicVersion3.08
\unitlength 0.1in
\begin{picture}( 16.0500, 14.4000)(  7.9500,-20.4000)
% VECTOR 3 0 3 0
% 2 800 2000 2400 2000
% 
\special{pn 4}%
\special{pa 800 2000}%
\special{pa 2400 2000}%
\special{fp}%
\special{sh 1}%
\special{pa 2400 2000}%
\special{pa 2334 1980}%
\special{pa 2348 2000}%
\special{pa 2334 2020}%
\special{pa 2400 2000}%
\special{fp}%
% VECTOR 3 0 3 0
% 2 800 2000 800 600
% 
\special{pn 4}%
\special{pa 800 2000}%
\special{pa 800 600}%
\special{fp}%
\special{sh 1}%
\special{pa 800 600}%
\special{pa 780 668}%
\special{pa 800 654}%
\special{pa 820 668}%
\special{pa 800 600}%
\special{fp}%
% STR 2 0 3 0
% 3 2280 1840 2280 1940 2 0
% $T$
\put(22.8000,-19.4000){\makebox(0,0)[lb]{$T$}}%
% STR 2 0 3 0
% 3 870 680 870 780 2 0
% $H$
\put(8.7000,-7.8000){\makebox(0,0)[lb]{$H$}}%
% SPLINE 1 0 3 0
% 3 800 1000 1600 1200 1800 2000
% 
\special{pn 13}%
\special{pa 800 1000}%
\special{pa 838 1002}%
\special{pa 874 1004}%
\special{pa 910 1006}%
\special{pa 948 1008}%
\special{pa 984 1012}%
\special{pa 1020 1014}%
\special{pa 1056 1018}%
\special{pa 1090 1020}%
\special{pa 1126 1024}%
\special{pa 1160 1028}%
\special{pa 1194 1034}%
\special{pa 1228 1038}%
\special{pa 1260 1044}%
\special{pa 1292 1052}%
\special{pa 1324 1060}%
\special{pa 1354 1068}%
\special{pa 1384 1076}%
\special{pa 1412 1086}%
\special{pa 1440 1098}%
\special{pa 1468 1110}%
\special{pa 1494 1122}%
\special{pa 1518 1136}%
\special{pa 1542 1152}%
\special{pa 1564 1168}%
\special{pa 1586 1186}%
\special{pa 1604 1206}%
\special{pa 1624 1226}%
\special{pa 1640 1248}%
\special{pa 1656 1270}%
\special{pa 1672 1294}%
\special{pa 1684 1320}%
\special{pa 1698 1346}%
\special{pa 1708 1372}%
\special{pa 1720 1402}%
\special{pa 1728 1430}%
\special{pa 1738 1460}%
\special{pa 1746 1492}%
\special{pa 1752 1524}%
\special{pa 1758 1556}%
\special{pa 1764 1588}%
\special{pa 1770 1622}%
\special{pa 1774 1656}%
\special{pa 1778 1692}%
\special{pa 1782 1726}%
\special{pa 1786 1762}%
\special{pa 1788 1798}%
\special{pa 1790 1834}%
\special{pa 1794 1870}%
\special{pa 1796 1908}%
\special{pa 1798 1944}%
\special{pa 1800 1980}%
\special{pa 1800 2000}%
\special{sp}%
% STR 2 0 3 0
% 3 840 1770 840 1870 2 0
% s相 $B=0$
\put(8.4000,-18.7000){\makebox(0,0)[lb]{s相 $B=0$}}%
% STR 2 0 3 0
% 3 1680 730 1680 830 2 0
% n相 $B=\mu_0 H$
\put(16.8000,-8.3000){\makebox(0,0)[lb]{n相 $B=\mu_0 H$}}%
% VECTOR 2 0 3 0
% 4 2000 2000 2000 1600 2000 1600 2000 1200
% 
\special{pn 8}%
\special{pa 2000 2000}%
\special{pa 2000 1600}%
\special{fp}%
\special{sh 1}%
\special{pa 2000 1600}%
\special{pa 1980 1668}%
\special{pa 2000 1654}%
\special{pa 2020 1668}%
\special{pa 2000 1600}%
\special{fp}%
\special{pa 2000 1600}%
\special{pa 2000 1200}%
\special{fp}%
\special{sh 1}%
\special{pa 2000 1200}%
\special{pa 1980 1268}%
\special{pa 2000 1254}%
\special{pa 2020 1268}%
\special{pa 2000 1200}%
\special{fp}%
% VECTOR 2 0 3 0
% 4 1600 2000 1600 1600 1600 1600 1600 1000
% 
\special{pn 8}%
\special{pa 1600 2000}%
\special{pa 1600 1600}%
\special{fp}%
\special{sh 1}%
\special{pa 1600 1600}%
\special{pa 1580 1668}%
\special{pa 1600 1654}%
\special{pa 1620 1668}%
\special{pa 1600 1600}%
\special{fp}%
\special{pa 1600 1600}%
\special{pa 1600 1000}%
\special{fp}%
\special{sh 1}%
\special{pa 1600 1000}%
\special{pa 1580 1068}%
\special{pa 1600 1054}%
\special{pa 1620 1068}%
\special{pa 1600 1000}%
\special{fp}%
% STR 2 0 3 0
% 3 2090 1320 2090 1420 2 0
% path 1
\put(20.9000,-14.2000){\makebox(0,0)[lb]{path 1}}%
% STR 2 0 3 0
% 3 1640 1030 1640 1130 2 0
% path 2
\put(16.4000,-11.3000){\makebox(0,0)[lb]{path 2}}%
% STR 2 0 3 0
% 3 1720 2110 1720 2210 2 0
% $T_c$
\put(17.2000,-22.1000){\makebox(0,0)[lb]{$T_c$}}%
\end{picture}%

    \end{center}
    \caption{超伝導体の相図 } \ilabel{5phase}
  \end{figure}

\item
  右の図は,$HT$に関する相図です.

  正常状態については,$B = \mu_0 H$ですので,path 1での積分を実行して
  \begin{equation}
  G_n(T,H) = G_n(T,0) - V \frac 12 \mu_0 H^2 \quad ( T> T_c). \notag
  \end{equation}
  $T>T_c$としましたが,コメントしたとおり,これを常伝導状態での$(H,T)$全域での自由エネルギーと考えます.

  一方,超伝導状態$H<H_c(T)$では$B=0$,$H>H_c(T)$のとき$B = \mu_0 H$になることに注意して,path 2で積分をすると,
  \begin{align}
  G_s(T,H) &= G_s(T,0) & (T<T_c, H< H_c(T) ) \notag \\
  G_n(T,H) &= G_s(T,0) -V\dfrac 12 \mu_0 \big( H^2 -H_c(T)^2 \big) & (T<T_c,  H> H_c(T) ) \notag
  \end{align}
  となります.$G_n(T,H)$に関して二つの表式が得られたので,これを等しいと置くと, 
  \begin{equation}
  G_n(T,0) = G_s(T,0) + V \frac 12 \mu_0 H_c(T)^2 \ilabel{5fe}
  \end{equation}
  が得られます.

  さて,path 2に沿って$H$を徐々に大きくしていくとき,$\displaystyle \pdif GH$の変化はどうなるかを考えると,
  \begin{equation}
  \pdif GH =
  \begin{cases}
  0 & H < H_c(T) \\
  -V\mu_0 H & H> H_c(T) \\
  \end{cases} \notag
  \end{equation}
  となって,臨界磁場付近で不連続.こうして1次相転移であることがわかります.

\item
  $(T,H) = (0,H_c(0))$で,$G_s$と$G_n$の値は等しくなければならないので
  \begin{equation}
  0= G_n(0,H_c(0))-G_s(0,H_c(0))= G_n(0,0) -G_s(0,0) - V\frac 12 H_c(0)^2 \notag
  \end{equation}
  であって,$G_n(0,0)-G_s(0,0) = 0.42\times 0.1 \cdot V[\mathrm{cm}^3] \cdot \mathrm{mJ}$を代入して$\mu_0H_c(0) = 1.0\times 10^{-5} \mathrm{T}$.

\item
  $T \sim 0$での比熱の様子を実験結果から見ると,$C_n \sim \gamma T$, $C_s \sim 0$と分かります.

  ここで$C = -T\dfrac{\partial^2 G}{\partial T^2}$,$S = -\dfrac{\partial G}{\partial T}$により,
  \begin{align}
  G_n(T,0) &= G_n(0,0) - TS_n(0,0) - \frac 12 \gamma T^2, \notag \\
  G_s(T,0) &= G_s(0,0) - TS_s(0,0)  \notag
  \end{align}
  と$T$で展開することが出来ます.あえて書きましたが第3法則より$S(0,0) =0$です.この式と,式\ieqref{5fe}を使えば
  \begin{equation}
  H_c(T) = \sqrt{\left(\delta - \dfrac 12 \gamma T^2\right)\left(\dfrac{2}{\mu_0V}\right)} \simeq H_c(0)\left(1 - \dfrac{\gamma}{4\delta} T^2 \right) \notag
  \end{equation}

\end{enumerate}
\end{answer}
