%% -*- coding:sjis -*-
\newcommand{\odif}[2]{\frac{d{#1}}{d{#2}}}

\begin{answer}{第4問}{見上敬洋}
\begin{enumerate}
\item
  中性子が$180^\circ$の角度で散乱されたときが反跳エネルギー最大です.
  このとき,散乱前の中性子の運動エネルギー$T$,散乱後の中性子,原子核の運動エネルギー$T', E_\mathrm{max}$とすると,
  エネルギー保存則と運動量保存則から
  \begin{gather}
    T = T' + E_\mathrm{max} , \notag\\
    \sqrt{2mT} = -\sqrt{2mT'} + \sqrt{2mAE_\mathrm{max}} \notag
  \end{gather}
  が成り立ちます.核子の質量$m$としました.
  これを解けば$\displaystyle E_\mathrm{max} = \frac{4A}{(A+1)^2} T$となって,
  $A=1$で最大になることも分かります.

  \begin{figure}[h]
    \begin{center}
      %% \begin{wrapfigure}[12]{r}{6cm}
      %% -*- coding:sjis -*-
%WinTpicVersion3.08
\unitlength 0.1in
\begin{picture}( 20.3000, 16.5000)(  4.0000,-20.0000)
% LINE 3 0 3 0
% 4 400 1400 2200 1400 1200 2000 1200 400
% 
\special{pn 4}%
\special{pa 400 1400}%
\special{pa 2200 1400}%
\special{fp}%
\special{pa 1200 2000}%
\special{pa 1200 400}%
\special{fp}%
% VECTOR 3 0 3 0
% 4 2000 800 2400 800 2000 800 2000 400
% 
\special{pn 4}%
\special{pa 2000 800}%
\special{pa 2400 800}%
\special{fp}%
\special{sh 1}%
\special{pa 2400 800}%
\special{pa 2334 780}%
\special{pa 2348 800}%
\special{pa 2334 820}%
\special{pa 2400 800}%
\special{fp}%
\special{pa 2000 800}%
\special{pa 2000 400}%
\special{fp}%
\special{sh 1}%
\special{pa 2000 400}%
\special{pa 1980 468}%
\special{pa 2000 454}%
\special{pa 2020 468}%
\special{pa 2000 400}%
\special{fp}%
% STR 2 0 3 0
% 3 2430 810 2430 910 2 0
% $x$
\put(24.3000,-9.1000){\makebox(0,0)[lb]{$x$}}%
% STR 2 0 3 0
% 3 2030 420 2030 520 2 0
% $y$
\put(20.3000,-5.2000){\makebox(0,0)[lb]{$y$}}%
% VECTOR 1 0 3 0
% 4 400 1400 1000 1400 1400 1200 1800 800
% 
\special{pn 13}%
\special{pa 400 1400}%
\special{pa 1000 1400}%
\special{fp}%
\special{sh 1}%
\special{pa 1000 1400}%
\special{pa 934 1380}%
\special{pa 948 1400}%
\special{pa 934 1420}%
\special{pa 1000 1400}%
\special{fp}%
\special{pa 1400 1200}%
\special{pa 1800 800}%
\special{fp}%
\special{sh 1}%
\special{pa 1800 800}%
\special{pa 1740 834}%
\special{pa 1762 838}%
\special{pa 1768 862}%
\special{pa 1800 800}%
\special{fp}%
% VECTOR 1 0 3 0
% 4 2000 1400 1400 1400 1400 1400 1400 1400
% 
\special{pn 13}%
\special{pa 2000 1400}%
\special{pa 1400 1400}%
\special{fp}%
\special{sh 1}%
\special{pa 1400 1400}%
\special{pa 1468 1420}%
\special{pa 1454 1400}%
\special{pa 1468 1380}%
\special{pa 1400 1400}%
\special{fp}%
\special{pa 1400 1400}%
\special{pa 1400 1400}%
\special{fp}%
% VECTOR 1 0 3 0
% 2 1000 1600 600 2000
% 
\special{pn 13}%
\special{pa 1000 1600}%
\special{pa 600 2000}%
\special{fp}%
\special{sh 1}%
\special{pa 600 2000}%
\special{pa 662 1968}%
\special{pa 638 1962}%
\special{pa 634 1940}%
\special{pa 600 2000}%
\special{fp}%
% SPLINE 2 2 3 0
% 3 1000 1400 1200 1360 1400 1200
% 
\special{pn 8}%
\special{pa 1000 1400}%
\special{pa 1034 1398}%
\special{pa 1066 1394}%
\special{pa 1098 1388}%
\special{pa 1130 1382}%
\special{pa 1160 1374}%
\special{pa 1190 1364}%
\special{pa 1218 1352}%
\special{pa 1246 1336}%
\special{pa 1272 1320}%
\special{pa 1296 1300}%
\special{pa 1322 1278}%
\special{pa 1344 1256}%
\special{pa 1368 1234}%
\special{pa 1392 1210}%
\special{pa 1400 1200}%
\special{sp -0.045}%
% SPLINE 2 2 3 0
% 3 1400 1400 1200 1450 1000 1600
% 
\special{pn 8}%
\special{pa 1400 1400}%
\special{pa 1368 1406}%
\special{pa 1336 1412}%
\special{pa 1304 1418}%
\special{pa 1274 1426}%
\special{pa 1242 1434}%
\special{pa 1214 1446}%
\special{pa 1184 1458}%
\special{pa 1158 1474}%
\special{pa 1130 1490}%
\special{pa 1104 1510}%
\special{pa 1080 1530}%
\special{pa 1056 1550}%
\special{pa 1030 1572}%
\special{pa 1006 1594}%
\special{pa 1000 1600}%
\special{sp -0.045}%
% CIRCLE 3 0 3 0
% 4 1200 1400 1600 1400 1600 1400 1600 1000
% 
\special{pn 4}%
\special{ar 1200 1400 400 400  5.4977871 6.2831853}%
% CIRCLE 3 0 3 0
% 4 1200 1400 800 1400 800 1400 800 1800
% 
\special{pn 4}%
\special{ar 1200 1400 400 400  2.3561945 3.1415927}%
% STR 2 0 3 0
% 3 1630 1170 1630 1270 2 0
% $\theta^*$
\put(16.3000,-12.7000){\makebox(0,0)[lb]{$\theta^*$}}%
% STR 2 0 3 0
% 3 620 1590 620 1690 2 0
% $\theta^*$
\put(6.2000,-16.9000){\makebox(0,0)[lb]{$\theta^*$}}%
% STR 2 0 3 0
% 3 570 1250 570 1350 2 0
% $E^*$
\put(5.7000,-13.5000){\makebox(0,0)[lb]{$E^*$}}%
% STR 2 0 3 0
% 3 1790 1510 1790 1610 2 0
% $E^*$
\put(17.9000,-16.1000){\makebox(0,0)[lb]{$E^*$}}%
% STR 2 0 3 0
% 3 1540 670 1540 770 2 0
% �����q
\put(15.4000,-7.7000){\makebox(0,0)[lb]{�����q}}%
% STR 2 0 3 0
% 3 700 1990 700 2090 2 0
% �z�q
\put(7.0000,-20.9000){\makebox(0,0)[lb]{�z�q}}%
\end{picture}%

    \end{center}
  \end{figure}

\item
  \begin{enumerate}
  \item
    重心系での陽子の運動エネルギー(=中性子の運動エネルギー)を$E^*$と書くことにします.
    実験室系で中性子は$\sqrt{2T/m}$の速度で進んでいたので,重心系ではその$1/2$の速度を持ちます.ゆえに
    \begin{equation}
      \sqrt{\dfrac{2E^*}{m}} = \frac 12 \sqrt{\dfrac{2T}{m}} , \quad \therefore E^* = \frac 14 T. \notag
    \end{equation}

    重心系での座標を右のようにとることにすると,重心系での陽子の運動量は
    \begin{equation}
      -\sqrt{2mE^*}
      \begin{pmatrix}
        \cos\theta^* \\ \sin\theta^*
      \end{pmatrix}
      =-\frac 12 \sqrt{2mT}
      \begin{pmatrix}
        \cos\theta^* \\ \sin\theta^* 
      \end{pmatrix} \notag
    \end{equation}
    とかけます.これを実験室系に戻すには$x$方向に運動量$\dfrac 12 \sqrt{2mT}$だけ進めばよいから,
    \begin{equation}
      \bm{p}_L = 
      \frac 12 \sqrt{2mT} 
      \begin{pmatrix}
        1-\cos\theta^* \\ -\sin\theta^*
      \end{pmatrix} \notag
    \end{equation}
    が,実験室系での運動量となります.よって運動エネルギーは
    \begin{equation}
      E = \frac{\bm{p}_L^2}{2m} = \frac T2 \left( 1 - \cos\theta^*\right). \ilabel{4energy}
    \end{equation}

  \item
    式\ieqref{4energy}より,$dE = \dfrac T2 \sin\theta^* \,d\theta^*$となります.したがって,エネルギーが$E\sim E+dE$に入るような,各$\theta^*$の幅は
    \begin{equation}
      2\times d\theta^* = 2\times \frac{2 dE}{T\sin\theta^*} \notag
    \end{equation}
    です.2倍したのは,各$E$に対して$\pm\theta^*$の二つが対応するので,それぞれの幅を加えたからです.
    重心系で散乱は等方的という仮定より,$\theta^*$軸には確率密度$1/2\pi$が入ります.こうして,$E\sim E+dE$に入る確率は
    \begin{equation}
      \frac{dW}{dE} dE = \frac {4dE}{T\sin\theta^*} \Big/ {2\pi} , \quad \therefore \odif{W}{E} = \frac{1}{\pi\sqrt{E(T-E)}}\notag .
    \end{equation}

  \end{enumerate}

\item
  入射した中性子は,シンチレータ中の陽子(水素原子核)に衝突します.
  陽子はCoulomb相互作用によってシンチレータ内の価電子たちを励起することで,
  衝突によってえたエネルギーをシンチレータに与えます.励起された価電子は脱励起し,エネルギーを光子として放出します.

  発生した光子の一部は光電子増倍管に向かい,光電陰極に衝突して電子を叩き出す.この電子が,電場による加速,極板との衝突を繰り返して増幅され,信号として取り出されます.

\item
  平均自由行程$\lambda$は,粒子が散乱されるまでに進める距離の目安を与えるものです.
  衝突断面積$\sigma$とするとき,体積$\sigma\lambda$辺りに散乱体が1つ程度あるべきなので$n\sigma \lambda = 1$,すなわち$\lambda = 1/n\sigma$.

\item
  中性子が$x$だけ進むまでに陽子に散乱される確率を$P(x)$とし,これを求めましょう.

  中性子が$dx$だけ進んだとき,散乱される確率は$dx/\lambda$であることに注意すれば
  \begin{align}
    P(x+dx)
    &= \text{(xまでに散乱される確率)} + \text{($x$までに散乱されない確率)}\times \text{(幅$dx$で散乱される確率)} \notag\\
    &= P(x) + [1-P(x)] \frac{dx}{\lambda} \notag
  \end{align}
  この微分方程式を解けば$P(x) = 1-e^{-x/\lambda}$となります.

\item
  プラスチックシンチレータの密度$10 \mathrm{g/cm^3}$から$n = \dfrac 1{13}\times 6\times 10^{23} \mathrm{/cm^3}$.

  また,運動エネルギー$20\mathrm{MeV}$は,運動量$\sqrt{2mc^2T}/c =\sqrt{2\times 1000\times 20} \mathrm{MeV}/c = 200 \mathrm{MeV}/c $なので,
  図より$\sigma = 0.5 \times 10^{-24} \mathrm{cm}^2$.

  したがって$\lambda = 1/n\sigma = \dfrac {130}{3} \mathrm{cm}$.ゆえに検出効率は
  \begin{equation}
    P(x = 10\mathrm{cm}) = 1-e^{-3/13} =\frac{3}{13} - \frac 12 \left(\frac{3}{13}\right)^2 + \cdots = 20 \% \notag
  \end{equation} 
\end{enumerate}
\end{answer}
