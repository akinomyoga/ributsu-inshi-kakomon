\documentclass[fleqn]{jbook}
\usepackage{physpub}

\begin{document}

\begin{question}{教育 数学}{}
\begin{subquestions}

\SubQuestion
3行3列の実行列
\[ A =\left( \begin{array}{ccc}
a+1 & 0 & 0 \\
0 & a+1 & -1 \\
-1 & 1 & a-1 
\end{array}\right)
\]
について以下の設問に答えよ。

\begin{subsubquestions}
\SubSubQuestion
$A\vec{e_1}=\lambda_1 \vec{e_1}, A\vec{e_2}=\lambda_2 \vec{e_2},
A\vec{e_3}=\vec{e_2}+\lambda_2 \vec{e_3}$を満たす実数値
$\lambda_1,\lambda_2$および$R^3$の単位ベクトル
$\vec{e_1},\vec{e_2},\vec{e_3}$を求めよ。ただし、
$\vec{e_1},\vec{e_2},\vec{e_3}$の第2成分は正とする。

\SubSubQuestion
$A^n\vec{e_1}, A^n\vec{e_2}, A^n\vec{e_3}$を
$\vec{e_1},\vec{e_2},\vec{e_3}$の一次結合で表せ。

\SubSubQuestion
$\vec{r},\vec{d}$を$R^3$のベクトルとする。このときベクトル列
$\vec{r_0},\vec{r_1},\vec{r_2},\cdots $を$\vec{r_0}=\vec{r},
\vec{r_{n+1}}=A\vec{r_n}+\vec{d} \quad (n=0,1,2,\cdots)$によって定める。ベク
トル列の極限$\ds{\lim_{n\rightarrow\infty} \vec{r_n}}$が任意の
$\vec{r},\vec{d}$について常に存在するような$a$の範囲を求めよ。

\end{subsubquestions}

\SubQuestion
関数$y(x)$に関する常微分方程式
\[ \frac{d^2 y}{dx^2}+\omega^2 y =f(x) \]
について以下の設問に答えよ。ただし$\omega$は正の定数であり、$x$の範囲
は$x\geq0$とする。

\begin{subsubquestions}
\SubQuestion
$y=a(x)\cos\omega x+b(x)\sin\omega x$とおいて上式に代入し, $\ds{\frac{da}{dx}}$と$\ds{\frac{db}{dx}}$に対する関係式を求めよ。ただし、\\
$\ds{\frac{da}{dx}\cos\omega x+\frac{db}{dx}\sin\omega x =0}$とする。

\SubQuestion
$a(x)$と$b(x)$を求めよ。

\SubQuestion
$f(x)=1/\lambda \quad (0 \leq x \leq \lambda),\quad 0 \quad (x >
\lambda)$のとき, $y(x)$を求めよ。ただし$\lambda$は正の定数であり、$x=0$で、$y=\frac{dy}{dx}=0$とする。

\SubQuestion

(iii)で得られた$y(x)$の$\lambda\rightarrow 0$の極限を求めよ。

\end{subsubquestions}

\SubQuestion
A,B2人があるゲームを繰り返し行なう。Aが2回続けて勝つまでゲームを続ける。各々のゲームでAが勝つ確率は2/3とする。

\begin{subsubquestions}
\SubSubQuestion
$N$回目のゲームでも終了しない確率を$x_N$とする。$x_N$を$x_{N-1}, x_{N-2}$で表せ。

\SubSubQuestion
$x_N$を$N$の関数として求めよ。

\SubSubQuestion
行なわれるゲームの回数の期待値を求めよ。

\end{subsubquestions}
\end{subquestions}
\end{question}
\begin{answer}{教育 数学}{}
\begin{subanswers}
\SubAnswer
\begin{subsubanswers}
\SubSubAnswer

行列$A$の固有値を$\lambda$とすると、
\begin{eqnarray*}
\det(\lambda E -A) & = & \left| \begin{array}{ccc}
\lambda-(a+1) & 0 & 0 \\
0 & \lambda-(a+1) & 1 \\
1 & -1 & \lambda-(a-1) 
\end{array}\right| \\
 & = & (\lambda - a)^2 \{ \lambda-(a+1) \}
\end{eqnarray*}
から、$\lambda_1=a+1,\lambda_2=a$となる。それぞれに対する固有ベクトルを求めると、
\[ \vec{e_1}=\left( \begin{array}{c}
\frac{1}{\sqrt{2}} \\
\frac{1}{\sqrt{2}} \\
0
\end{array}\right)
, \quad 
\vec{e_2}=\left( \begin{array}{c}
0 \\
\frac{1}{\sqrt{2}} \\
\frac{1}{\sqrt{2}} \\
\end{array}\right)
\]
さらに、$\vec{e_3}=(x,y,z)$とすると、$A\vec{e_3}=\vec{e_2}+\lambda_2 \vec{e_3}$に代入し、さらに、大きさが$1$なので、
\[ x=0 , \quad y-z=\frac{1}{\sqrt{2}}, \quad x^2+y^2+z^2=1 \]
第二成分が正であることを用いて解くと、$\ds{x=0, \quad  y=\frac{\sqrt{6}+\sqrt{2}}{4} , \quad z=\frac{\sqrt{6}-\sqrt{2}}{4}}$
となるから、
\[ \vec{e_3}=\left( \begin{array}{c}
0 \\
\frac{\sqrt{6}+\sqrt{2}}{4} \\
\frac{\sqrt{6}-\sqrt{2}}{4}
\end{array}\right)
\]

\SubSubAnswer
$\ds{B\equiv\left(\begin{array}{ccc}
\lambda_1 & 0 & 0 \\
0 & \lambda_2 & 0 \\
0 & 0  & \lambda_2 
\end{array}\right),\quad J\equiv\left(\begin{array}{ccc}
0 & 0 & 0 \\
0 & 0 & 0 \\
0 & 1 & 0 
\end{array}\right) }$とすると、\\
$\ds{BJ=JB=\left( \begin{array}{ccc}
0 & 0 & 0 \\
0 & 0 & 0 \\
0 & \lambda_2 & 0 
\end{array}\right) }$となるから、
$\ds{(J+B)^n=\left( \begin{array}{ccc}
\lambda_1^n & 0 & 0 \\
0 & \lambda_2^n & 0 \\
0 & n\lambda_2^{n-1} & \lambda_2^n 
\end{array}\right) }$ \\
よって、
\[ A^n\vec{e_1}=\lambda_1^n\vec{e_1},\quad A^n\vec{e_2}=\lambda_2^n\vec{e_2},\quad A^n\vec{e_3}=n\lambda_2^{n-1}\vec{e_2}+\lambda_2^n \vec{e_3} \]

\SubSubAnswer
\[ \vec{r_n}=A^n \vec{r_0}+(E+A+A^2+\cdots+A^{n-1})\vec{d} \]
と、任意の$\vec{r},\vec{d}$に対して収束するので、$\vec{e_1},\vec{e_2},\vec{e_3}$が、$\vec{r},\vec{d}$のとき、収束すれば良い。よって、$A^n$と$\ds{\sum_{k=0}^{\infty}A^k}$が収束すればよい。

$A^n$が収束するための必要十分条件は、$-1 <\lambda_1  \leq 1$かつ
$|\lambda_2| < 1$なので、$-1 < a \leq 0 $。さらに、
\[\sum_{k=0}^{\infty}k\alpha^{k-1}=\frac{\partial}{\partial \alpha}\sum_{k=0}^{\infty}\alpha^k\]
に注意して、$\ds{\sum_{k=0}^{\infty}A^k}$が収束するための必要十分条件
は、$|\lambda_1 |< 1$かつ$|\lambda_2| <1$なので、$-1<a<0$。よって求める
解は、$-1<a<0$である。

\end{subsubanswers}

\SubAnswer
ドットは、$\Deriver{}{x}$を表すものとする。($\Deriver{y}{x} \equiv \dot{y}$)
\begin{subsubanswers}
\SubSubAnswer
$\dot{a}\cos\omega x+\dot{b}\sin\omega x=0$に注意すると、
\[
\ddot{y} = \omega(-\dot{a}\sin\omega x+\dot{b}\cos\omega
x)-\omega^2y 
\qquad 
\Yueni \omega(-\dot a \sin\omega x + \dot b \cos\omega x)=f(x) 
\]

\SubSubAnswer
\[ 
\left\{ 
\begin{array}{l}
\dot a \cos\omega x + \dot b \sin\omega x=0\\
-\dot a \sin\omega x + \dot b \cos\omega x=\frac{f(x)}{\omega}
\end{array}
\right. 
\]
\[ 
\Yueni \dot a = -\frac{f(x)}{\omega}\sin\omega x,
\quad \dot b = \frac{f(x)}{\omega}\cos\omega x 
\]
\[ 
\Yueni a = - \int \frac{f(x)}{\omega}\sin\omega x \,dx,
\quad b = \int \frac{f(x)}{\omega}\cos\omega x \,dx 
\]

\SubSubAnswer
$x=0$で$y=\dot y=0$より、$a(0)=b(0)=0$であるから、
\[ a = - \int_0^x \frac{f(x)}{\omega}\sin\omega x \,dx
= \left\{ \begin{array}{ll}
\frac{1}{\omega^2
  \lambda}\cos\omega x-\frac{1}{\omega^2\lambda} & 0 \leq x \leq \lambda\\
\frac{1}{\omega^2
  \lambda}\cos\omega \lambda-\frac{1}{\omega^2\lambda} & x > \lambda
\end{array} \right. \]
\[ b = \int_0^x \frac{f(x)}{\omega}\cos\omega x\,dx=
\left\{ \begin{array}{ll}
\frac{1}{\omega^2\lambda}\sin\omega x & 0 \leq x \leq \lambda\\
\frac{1}{\omega^2\lambda}\sin\omega\lambda & x > \lambda
\end{array} \right. \]
以上まとめれば、
\[ y(x) = \left\{ \begin{array}{ll}
\frac{1}{\omega^2\lambda}(1-\cos\omega x) & 0 \leq x \leq \lambda\\
\frac{1}{\omega^2\lambda}(\cos\omega(x-\lambda)-\cos\omega x) & x > \lambda
\end{array} \right.
\]

\SubSubAnswer
$\lambda\rightarrow0$では、$x \neq 0$のとき$x > \lambda$なる
$\lambda$がつねに存在するので、
\[
y(x) = \lim_{\lambda\rightarrow0}
\frac{1}{\omega^2\lambda}(\cos\omega(x-\lambda)-\cos\omega x) 
= \frac{1}{\omega}\sin\omega x 
\]
これは、y(0)=0もみたしている。
\end{subsubanswers}

\SubAnswer
\begin{subsubanswers}
\SubSubAnswer
\[ x_N=\frac{1}{3}x_{N-1}+\frac{2}{9}x_{N-2} \]

\SubSubAnswer
$x_1=1,x_2=\frac{5}{9}$であるから、
\[ \left(x_N+\frac{1}{3}x_{N-1}\right)=\frac{2}{3}\left(x_{N-1}+\frac{1}{3}x_{N-2}\right)=2\left(\frac{2}{3}\right)^N \]
\[ \left(x_N-\frac{2}{3}x_{N-1}\right)=-\frac{1}{3}\left(x_{N-1}-\frac{2}{3}x_{N-2}\right)=-\left(-\frac{1}{3}\right)^N \]
従って、
\[ x_N=2\left(\frac{2}{3}\right)^{N+1}+\left(-\frac{1}{3}\right)^{N+1} \]

\SubSubAnswer
N回目におわる確率は、$x_{N-1}-x_N$であるから、求める期待値は、
\begin{eqnarray*}
\sum_{N=1}^{\infty}N(x_{N-1}-x_N)&=&\sum_{N=1}^{\infty}x_{N-1}\\
&=&\sum_{N=1}^{\infty}\left(2\left(\frac{2}{3}\right)^N+\left(-\frac{1}{3}\right)^N\right)\\
&=&\frac{15}{4}
\end{eqnarray*}
ただし、途中
$\sum_{N=1}^{\infty}Nx_N=\sum_{N=1}^{\infty}(N-1)x_{N-1}$を用いた。
\end{subsubanswers}

\end{subanswers}

\end{answer}


\end{document}

