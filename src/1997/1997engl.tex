\documentclass[fleqn]{jbook}
\usepackage{physpub}

\begin{document}

\begin{question}{教育 英語}{}
\begin{subquestions}
\SubQuestion
  次の文章は、中国の李鵬首相によって1996年4月に行なわれた中国
  における科学技術の役割についての講演の一部である。この文章を読ん
  で以下の設問{\bf(i)}および{\bf(ii)}に答えよ。

\baselineskip=12pt

   It has been eighteen years since China started to reform and open 
  up to the rest of the world. During this period, the national
  economy has been developing rapidly, with an average annual
  growth rate of 9\%. Recently, the Chinese People's Congress passed
  the next 5-year plan and a 15-year long-term development
  plan. The economy should maintain its good momentum with an
  estimated growth rate of 7 to 8\% during the next 5 years and
  about 7\% in the next century. In order to sustain this growth,
  we are undergoing two transitions --- from a planned economy to a
  market economy and from growth by increased development to
  growth by increased economic efficiency. In addition, we are
  using two strategies --- building the country through science and
  education and through sustainable development. It will take
  international cooperation to achieve these objectives.

   Why can't China do it alone? Because we have many
  problems. For example, we started to talk about 9-year
  compulsory education in 1985, but this goal has still not been
  achieved and will only be 85\% complete by the year 2000. In some 
  areas, we can only achieve 6-year compulsory education. In
  addition, development is not balanced across regions of
  China. Coastal areas are more advanced, but the middle and
  western regions are far behind. China still has 65 million
  people living in absolute poverty; the natural environments they 
  inhabit are extremely poor. This is not to say that there is no
  future for development in the western regions. They have
  abundant underground resources, but exploiting them will depend
  on science, technology, and education, as well as government
  subsidies. 

   The biggest problem in China, however, is agriculture. China
  must use 7\% of the world's arable land to feed 22\% of the
  world's population and has a shortage of water resources as
  well. Currently, only one-third of China's cultivated land
  produces high-yields; the other two thirds produce medium or low 
  yields. To develop agriculture, we need both practical
  techniques and high technology. For instances, the use of hybrid 
  rice can increase yields by another 20\% (after the most recent
  15\% increase); and with biological engineering, we have bred a
  new variety of cotton which is genetically resistant to
  bollworms, which have plagued our cotton production in recent
  years. Although China carries out a rather strict family
  planning policy, the population still increases by 13 million
  every year, and we must solve the problems of providing food,
  adequate living conditions, education, and employment for the
  increased population. I believe that science and technology can
  help us do so. 

   Meteorology is another field to which we pay special
  attention, because it is especially important to China to reduce 
  the effect of natural disasters. China is a country with
  frequent floods and droughts. Generally in a five-year period,
  we have two years with good harvests, two with average
  harvests, and one with disaster. We have dredged rivers and
  watercourses, built dams, and planted trees to prevent
  floods. But droughts are more serious than floods in
  China. Solving this problem will depend on more efficient use of 
  water resources. China also suffers typhoon attacks. Therefore,
  meteorology is very important to agriculture and people's
  lives. Accurate forecasts can reduce damage. We have established 
  a national meteorological network and jointed up with worldwide
  networks; we have used large computers to do
  forecasts. Currently we can make 5 to 7 days' advanced
  forecasts. 

   Chinese industry has high energy consumption, low efficiency,
  high materials consumption and low product quality. These
  problems also need science and technology solutions. In
  addition, our government needs to correctly handle the relation
  between basic research and applied science. Because applied
  science can increase productivity, it has been viewed as
  important by society. However, basic research is also important; 
  its development can bring about breakthroughs. China is a
  developing country and cannot afford to spend a lot of money on
  basic research. But the government does appropriate some money
  for it, and wider international cooperation would help expand
  China's basic research capabilities. 

   Since the founding of the People's Republic, we have gradually
  established a scientific research system that encompasses almost
  all fields of study. However, this system was based on the old
  Soviet model. Research was mainly carried out by institutes of
  the Chinese Academy of Sciences (CAS) and of the various
  ministries. In developed countries, research is mainly done at
  universities and by companies. We have encouraged research
  institutions to establish relations with business enterprises
  and encouraged large companies to do their own scientific
  research. CAS has also pioneered in letting research institutes
  set up enterprises to enter the market.

   Although China has made great strides in development, there
  are also many problems and difficulties, and it will take tens
  of years of arduous effort to solve them. We can't do it alone
  and would like to establish better cooperation with science and
  technology circles in the Asia-Pacific region and elsewhere in
  the world.
%
  \begin{flushright}
        (Science 巻頭言より)
  \end{flushright}
        
  arable      : 耕作に適する \quad
  bollworm    : 蛾の幼虫の一種 \quad
  dredge      : (しゅんせつ)する \quad
  meteorology : 気象学
\baselineskip=15pt

  \noindent{[設問]}
  \begin{subsubquestions}
  \SubSubQuestion
    李鵬首相は、中国における科学技術の果たす役割を4つあげている。
  中国のどのような現状に対して、 どのような役割を果たすのか、それ
  ぞれを箇条書きに整理して述べよ。

  \SubSubQuestion
    李鵬首相は、中国における科学研究を行うシステムと先進国のそれと
  はどのように違うと述べているか。また、この問題に対し中国はどのよ
  うに対処していると述べているか。それぞれを簡潔に述べよ。

  \end{subsubquestions}

\SubQuestion
    次の文章を読み、下記の下線(ア)および(イ)で示した部分を適切
  な日本語に訳せ。

\baselineskip=12pt

   Nearly a century ago, Basil Chamberlain opened his famous
  essay ``English as She is Japped'' with the sentence: ``English as
  she is spoken and written in Japan forms quite an enticing
  study''. We might well say the same thing today. Despite the
  tremendous effort and investment put into foreign-language
  teaching, Japan still abounds in the ``Japlish'' that Chamberlain
  found so entertaining.

   Three years of English is obligatory in most junior high
  schools, followed by three more years in senior high school, and
  usually continuing for another two years for those attending
  university. But with all that, few Japanese --- including those who
  end up as English teachers --- can converse freely in the language or 
  write it with any degree of proficiency. \underlineeng{(ア)English,
   or more correctly {\it Eigo}, is taught as an academic
  exercise with so much attention to memorizing fine points  
  of grammar and vocabulary that it ceases to be a means of
  communication. The examinations for which these methods prepare
  the student are in turn prepared by those who have gone through
  the same system, with the result that they do not test linguistic
  ability but merely the capacity to memorize relatively
  disconnected bits of information.}

   The situation becomes all the more ironic when one recalls
  that, despite the miseducation in English, thousands of English
  words have entered the Japanese language itself over the past few
  generations. Some of these loanwords represent new concepts for
  which there were no equivalents, or at least none that caught on,
  in Japanese itself. {\it Terebi} (``television'') is an
  example. Others have rendered the old Japanese synonyms archaic, 
  as {\it rabu-reta} (``love letter'') did to {\it koibumi}. Still 
  others coexist with Japanese words of the same meaning, such as
  {\it tsuma} and {\it waifu}(``wife''). Occasionally the foreign word
  carries a special meaning distinguishing it from the old Japanese
  word, as {\it raisu} (rice eaten on a plate with a fork) is
  differentiated from {\it gohan} (rice eaten in a bowl with
  chopsticks). Many English loanwords get new, restricted meanings
  in Japanese. Thus {\it mishin} (``machine'') always means sewing
  machine.
 
   Perhaps the most difficult class of all for the native speaker 
  to recognize is the words that are abbreviated in much the same
  way that the Japanese abbreviate their own Chinese compounds. {\it 
  Zenesuto} (``general strike'') and {\it sabunado} (``subterranean
  promenade,'' i.e., an underground walkway) are two of them. Finally, 
  some words are not standard English at all, but new inventions of
  the Japanese, such as {\it sarariiman} (``salary man''), which means 
  a white-collar worker.
 
   What helps to keep this {\it Eigo} from becoming English is
  that the words are written in the limited sound system of the
  Japanese syllabary, which produces pronunciations totally
  unrecognizable to the native ear. At the same time it retards the
  efforts of the Japanese to learn to speak English
  intelligibly. \underlineeng{(イ)On the positive side, these
  loanwords have helped the Japanese to cope with the demands of an
  international, technological world. Perhaps in the same way that
  English, as Anglo-Saxon, was enriched and beautified --- by Greek,
  French, and Latin --- Japanese can hope to be improved by its contact 
  with English. But for the moment at least, these hopes belong
  clearly to the future.}
%
  \begin{flushright}
        (Beverley D.Tuckerによる)
  \end{flushright}
\baselineskip=15pt
\SubQuestion
    下の文章の下線部を英訳せよ。

     あるとき自動車王フォードが、予告なしにエジソンの研究室を訪れた。
技師の一人が「先生は昼寝をしていらっしゃいます」とさえぎった。「あ
まり眠らないと聞いていますが」とフォード。「ええ、夜は少ししかお休
みになりません。ただ、昼寝をたくさんなさるだけです」。
\underlinejpn{ナポレオン(Napoleon)も4時間睡眠の人として名高い。 しかし、 彼は自分の睡眠時間を少なめに言う傾向があった。重要な会議や戦場に居合わせた者の証言によると、疲 れ果てて睡眠不足を訴えることも多かったらしい。睡眠不足のために、重大な局面で何度か判断を誤った、と指摘する歴史家もいるそうだ。}
「Sleep Thieves」を書いた心理学者スタンレー・コレンは、エジソンや
ナポレオンの例を引き、人間には十分な眠りが必要だと力説する。彼によ
れば、チェルノブイリの惨事もスリーマイル島事故もチャレンジャーの爆
発も、すべて関係者の睡眠不足が絡んでいた、という。
%
  \begin{flushright}
        (朝日新聞  天声人語による)
  \end{flushright}

\end{subquestions}
\end{question}
\begin{answer}{教育 英語}{}
\begin{subanswers}

\SubAnswer 
{\bf 全訳} 
        
   中国が改革開放政策を始めて18年になる。この間、中国経済は年平均9
%の成長率で急速に発展してきた。最近、中国人民大会で、次期5ヶ年計
画と15ヶ年の長期発展計画が可決された。中国経済は、好ましい勢いを
維持するべきであり、次の5年で7〜8%、来世紀には約7%で成長する
と予想している。このような成長を維持するために、我々は計画経済から
市場経済へ、また開発による成長から経済効率をあげることによる成長へ
─という2つの変革を経験しようとしている。それに加え、我々は2つの
計画─科学と教育、そして持続可能な発展によって国をつくること─をす
すめている。これらの目標を達成するためには、国際協力が必要となる。

   何故、中国一ヵ国でそれを達成することができないのであろうか。中国
は多くの問題を抱えているからである。例えば、我々は1985年に9ヵ
年の義務教育を提案したのだが、この目標はまだ達成されておらず200
0年までに義務教育を完全に受けることができるのは85%にすぎないだ
ろう。6ヵ年の義務教育を達成しているにすぎない地域もある。さらに開
発には地域格差がある。沿岸地域は発展しているが、中部と西部は遥かに
遅れている。中国にはまだきわめて貧しい生活をしている人々が6500
万人おり、彼らのおかれた環境は劣悪だ。西部地域に発展の見込みがない
と言っても過言ではない。豊富な地下資源はあるがそれを開発できるかど
うかは政府補助金だけでなく科学技術や教育にかかっている。

   しかし、中国における最大の問題は農業である。中国は世界の農地の7
%を使って人類の22%を養わなければならず、そのうえ水も不足してい
る。現在、中国の農地で高い収穫をあげているのは1/3にしかすぎない。
一方、残りの2/3は中程度か少ないのである。農業を発展させるためには
実行可能で高度な技術が必要である。例えば、米の交配種をつかうと、
(最近20%増産したにもかかわらず)さらに20%も収穫を増やすこと
ができる。また、バイオテクノロジーを使えば、近年綿花に損害を与えて
きたワタキバガの幼虫に遺伝的に強い新種の綿花を産み出した。中国では、
かなり厳しい家族計画政策が行われているが、人口はまだ年に1300万
人ずつ増加しているため、増加した人々のために食料、環境、教育、雇用
を供給しなければならない。私は、科学技術がこの問題を解決する助けに
なると信じている。

   気象学も、我々が特に注目している分野である。というのも、中国にとっ
ては災害の影響を減らすことが特に重要であるからだ。中国では、洪水や
干ばつが多い。平均して、5年のうち2年は豊作で、2年は平均、1年は
災害にあう。我々は、洪水を避けるために、水路を堀り、ダムをつくり、
植林してきた。しかし、中国では、干ばつは洪水より深刻である。この問
題を解決できるかどうかは、水資源をより効率的に利用できるかにかかっ
ているだろう。中国は、台風にも遭う。そのため、気象学は農業や人々の
生活にとっても重要である。正確に予報できれば、被害を減らすことがで
きる。我々は、全国気象ネットワークを作り、世界のネットワークに参加
してきた。そして、予報するために大型計算機を使用してきた。現在では、
5日〜7日先の予報をすることができる。

   中国の産業は、エネルギーの消費は多いが効率は低く、資源消費量は多
いが品質は低い。このような問題の解決にも、科学技術が必要である。そ
れに加えて、政府は基礎研究と応用科学とを正確に関係づけることが必要
である。応用科学は、生産性を向上させることができるため、社会にとっ
て重要であると見なされてきた。しかし、基礎研究も重要である。という
のも、基礎研究の発展によって、大躍進を遂げることができるからである。
中国は発展途上国であるため、基礎研究に多額の費用を出す余裕はない。
しかし、政府は基礎研究に十分な費用を出しており、より広い国際協力が
あれば、中国の基礎研究の研究能力をあげることができるだろう。

   人民共和国の設立以来、我々は、ほぼ全学問の研究領域を含む科学研究
システムを設立しつつある。しかし、このシステムは旧ソビエトをもとに
している。研究は主に中国科学アカデミー(CAS)や、様々な国家機関の
研究機関によって行われていた。先進国では、研究は主に大学や企業で行
われている。我々は、研究機関が企業と関係を持ったり、大企業が独自の
科学研究を行うことを勧めている。CASも、研究機関が市場に参入するた
めに事業を起こすように率先してきた。

   中国は、発展の中で研究が大きく躍進してきたが、多くの問題や困難も
残っており、それを解決するには、何十年も根気強く努力しなければなら
ないだろう。我々は、独力でそれを成し遂げることはできない。アジア─
太平洋地域、そしてその他の地域の科学技術団体とよりよい協力関係を築
きたいと考えている。

  \begin{subsubanswers}
  \SubSubAnswer
        {\bf 現状}:沿岸地域は発展しているが、中西部地域は開発が遅れてい
る。

        {\bf 役割}:西部地域の豊かな地下資源を開発する。


        {\bf 現状}:増え続ける人口を養うだけの食料と水が不足している。

        {\bf 役割}:生命科学を用いて、農作物の生産性を向上させる。


        {\bf 現状}:5年に1年は災害にあい、台風の被害にもよく遭う。

        {\bf 役割}:気象学を用いて、予報し、被害を軽減する。


        {\bf 現状}:中国の産業は効率が低く、品質も悪い。

        {\bf 役割}:応用科学と基礎研究の研究を進めて、高効率、高品質を目
指す。

  \SubSubAnswer

  中国では科学研究はCASや国家機関が行っているのに対し、先進国では
  主に大学や企業で行われている。

  この問題に対し、中国では、これらの研究機関が企業と関係を持ったり、
大企業が独自の科学研究を行うことを勧めている。

  \end{subsubanswers}

\SubAnswer 
{\bf 全訳} 
        
   1世紀近く前、Basil Chamberlainはその有名なエッセイ``English as
  She in Japped''を次の文章ではじめた。「日本において話されたり書
  かれたりする英語はきわめておもしろい研究となる。」今も同じことが
  言えよう。外国語教育に捧げられた莫大な努力と投資にも関わらず、日
  本には、いまだChamberlainが非常に面白がる``Japlish(和製英語)''
  が満ちている。

   たいていの中学校では英語は3年間必修であり、それに続いて高校でさ
らに3年間、大学に進学するものは普通さらに2年間学ぶ。しかしそれに
もかからわず、最終的に英語教師になるものも含めて、日本人にはこの言
語で自由に会話ができたりこれを書くことに多少なりとも熟練していたり
するものはほとんどいないのである。\underlinejpn{(ア)``English''、もしくはより正確には「英語」は、学問の訓練として文法や語彙の細かい点を覚えることに
あまりに多くの注意を払いすぎて教えられているため、伝達の手段でなくなってしまっているのである。この方法で生徒たちが(受験)準備する試験は、同じ制度を経験してきた人々にによって今度はつくられており、その結果言語能力ではなく、単に知識の比較的ばらばらな断片を記憶する能力を試験することになる。}

   次のことを思い出すとき、この状況はいっそう皮肉なものとなる。つま
り、英語の誤った教え方にもかかわらず多くの英単語が過去数世代にわたっ
て日本語そのものに入ってきていることである。これらの外来語のうちに
は、新しい概念を表すものがある。これら(の概念)には、それに相当す
る概念がなかったかまたは少なくとも日本語自身の中で受けいられなかっ
たのである。「テレビ」はその一例である。他には、「恋文」に対する
「ラブレター」のように、古い日本語での同意語を死語にしたものもある。
さらに他には、「妻」と「ワイフ」のように、同じ意味の日本語の単語と
共存しているものもある。時には、外来語は古い日本語の単語と異なる特
別な意味を持つこともある。たとえば「ライス」(フォークで皿から食べ
る米)は「ごはん」(箸で茶碗から食べる米)とは区別される。多くの英
語からの外来語は、日本語では新しい制限された意味を得る。それゆえに
「ミシン」は常に縫い物に使う機械を意味している。

   おそらく、(英語を)母国語とする人々にとって認識するのが全ての中
でもっとも難しいのは、日本人が独自の漢字の熟語を短縮するのと全く同
じやり方で短縮された単語である。ゼネストやサブナード(地下の散歩道)
はその2つの例である。最後に、全く標準英語ではなく日本語の新たな発
明であるもの、たとえばホワイトカラーを意味する「サラリーマン」など
の単語もある。

   「英語」が``English''になれない原因は、ネイティブな耳にとっては
全く認識不可能な発音をつくりだす日本語の五十音という限られた音の体
系で単語が書かれるということである。同時にこのことは、日本人が英語
を明瞭に話すことを学ぼうとする努力を妨げている。\underlinejpn{(イ)肯定的な立場では、これらの外来語は日本人が国際的な科学技術の世界の要求に対処するのに役立ってきている。おそらくアングロサクソン語としての英語がギリシャ語、フランス語、ラテン語によって豊かになり美しくなったのと同じようにして、日本語も英語との接触によって改良されていくことが望めるだろう。しかしすくなくともさしあたっては、これらの希望は明らかに先のものである。}

\SubAnswer
\baselineskip=12pt
Napoleon is also famous as a person who slept for only four
hours. However, he had the tendency of telling others shorter
hours of sleep than he actually had. According to those who were
present with him at important meetings and battlefields, there were 
many occasions that he complaint about physical exhaustion and
lack of sleep. There are also historians who pointed out that
Napoleon made a number of misjudgements during critical situations 
because of shortage of sleep.

[別解]

 It is well known that Napoleon slept only four hours, but he
 tended to say less than he had actually slept. According to the
 testimony of those who happened to be present in the important
 conferences and battlefields, he often complained of his
 exhaustion and lack of sleep. It is said that some historians
 point out that he sometimes made wrong judgements during critical 
 situations from lack of sleep.

\baselineskip=15pt
\end{subanswers}
\end{answer}

\end{document}


 
