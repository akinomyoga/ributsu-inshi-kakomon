\documentclass[fleqn]{jbook}
\usepackage{physpub}

\begin{document}

\begin{question}{英語}{1:yt 2:黒沼}
\begin{subquestions}
\SubQuestion
    次の文を読んで、自ら考えるところを\underline{英語で}、20行程度で記せ。
\baselineskip=12pt

Nuclear power plants currently provide about 18\% of the world's electricity.
Although global demand for electricity is increasing,
this figure is expected to fall as the construction of new nuclear power
stations winds down.
Some in the nuclear industry, however, are fighting back, arguing that
the only way to meet demand while cuting emissions of greenhouse gases
is to build more nuclear power plants.


But this fight back comes at a bad time for the industry.
Last year a nuclear worker in Japan died after mishandling enriched uranium.
Historically, governments invested heavily in the nuclear industry following
the oil crisis in the 1970's.
The price of oil then fell, new reserves of oil and gas were discovered.
However, much of the optimism within the nuclear industry comes from
its enviromental credentials.
The International Energy Agency estimates that world energy demand in 2020
will be two-thirds more than 1995 levels, and in 1997 at the Kyoto conference
industrialized nations pledged to cut greenhouse--gas emissions to
5\% below 1990 levels by 2012.
Nuclear energy produces almost no greenhouse gases.


Japan remains committed to nuclear power, despite recent problems,
and produces about one--third of its electricity from nuclear sources.
Japan expects to increase its nuclear capacity
by more than 20\% over the next two decades.
This is not typical of developed countries.
%\begin{flushright}
\hfil (adapted from \textit{Physics World}, April 2000)
%\end{flushright}


\SubQuestion
次の英文は、ノーベル平和賞を受賞した或る物理学者の文章である。
これを読み以下の設問に答えよ。

\begin{subsubquestions}
  \SubSubQuestion 下線部(a),(b)を和訳せよ。
  \SubSubQuestion 科学者になろうとする人に筆者が提唱していることを
	日本語で(直訳ではなく、自分の言葉で)説明せよ。
\end{subsubquestions}

%以下、英文全文丸写し
\baselineskip=12pt

The tremendous advances in pure science made during
the 20th century have completely changed the relation
between science and society. Through its technological
application, science has become a dominant element in
our lives. It has enormously improved the quality of life.
It has created great perils, threatening the very existence 
of the human species. Scientists can no longer claim that
their work has nothing to do with the welfare of the individual
or with state policies.

However,
%問題のやたら長い下線部(a)です。
\underlineeng[(a)]{many scientists still cling to an ivory tower mentality
founded on precepts such as "science should be done for its own sake",
"science is neutral", and "science cannot be blamed for its misapplication."}
%ここまで下線部
This amoral attitude is in my opinion actually immoral, because 
it eschews personal responsibility for the likely consequences of
one's actions.

The ever-growing interdependence of the world
community offers great benefits to individuals,
but by the same token it imposes responsibilities
on them. Every citizen must be accountable for his or
her deeds. This applies particularly to scientists,
for the reasons I have outlined. It is also in their
interest, because the public holds scientists responsible
for any misuse of science. 
%下線部(b)はここで〜ス
\underlineeng[(b)]{The public has the means to control science by withholding
the purse or imposing restrictive regulations. It is far better
that scientists themselves take appropriate steps to ensure
responsible application of their work.}
%以上下線部終了

Professional organization of scientists should work
out ethical codes of conduct for their members, including 
the monitoring of research projects for possible harm to
society. It is particularly important to ensure that new
entrants into the scientific profession are made aware
of their social and moral responsibilities. One way 
would be to initiate a pledge for scientists, a sort of
Hippocratic oath, to be taken at graduation. As in the
medical profession, the main value of such an oath might be 
symbolic, but I believe it would stimulate young scientists 
to reflect on the wider consequences of their intended field 
of work before embarking on a career in academia or industry.

I like the pledge initiated by the Student Pugwash Group
in the United States, which has already been signed by
thousands of students from many countries. It reads:
%このコメントの直前は
% ";" なのか ":" なのか印刷不明瞭でちょっと判別つきませんでした。
%英語のこの辺の区別が分かる方、なおしてちょ〜だい
"I promise to work for a better world, where science and
technology are used in socially responsible ways. I will
not use my education for any purpose intended to harm 
human beings or the environment. Through my career, I 
will consider the ethical implication of my work before 
I take action. While the demands placed upon me may be great,
I sign this declaration because I recognize that individual 
responsibility is the first step on the path to peace."

%原本にならって右寄せ
\begin{flushright}

%与えられた語彙
(precept: guide for behavior, eschew: keep oneself away from)

%引用
(adapted from \textit{Science}, \textbf{288}(1999))

\end{flushright}





\end{subquestions}
\end{question}
\begin{answer}{英語}{}

\begin{subanswers}
\SubAnswer 
\textbf{全訳}

原子力発電は現在世界の電力のおよそ18\% をまかなっている。
世界の電力需要は増加しているが、この数値からは減るものと見こまれている。
新規の原発の建設が減っているからである。
しかし、原発業界の一部は反撃する。% fight back にもっとまともな訳はないか。
温室効果ガスの排出を抑えつつ需要にこたえるには、もっと原発をつくるしかないと。

しかしこの反撃も時期が悪い。
昨年日本では、濃縮ウランの軽率なとりあつかいによってひとりが亡くなっている。
歴史的に見れば、政府は1970年代の石油危機以来、
原発業界にかなりの投資をしてきた。
その後石油の価格はさがり、新たな油田やガス田も発見されたが、
業界内部は上にあげたような環境配慮の面から楽観的であった。
国際エネルギー機関は2020年の世界のエネルギー需要は
1995年の5/3倍になるとしている。
一方、1997年の京都会議では工業先進国は2012年には温室効果ガスの排出を
1900年より5\% 少なくすると誓った。
原子力はほとんど温室効果ガスを出さないのである。

日本は昨今の不祥事にもかかわらず原子力に依存している。
実際、日本の電力の1/3は原発によるものである。
日本は原発による発電をこれからの20年で20\% 以上増やすことを計画している。
この行動は他の先進国とは異なっている。

\paragraph{解答例}

I forcus the discussion in the following on
the claim that nuclear plants produce no greenhouse gases,
which is seemingly mentioned to imply
that they are somewhat `safer' to the environment.

We often hear that global warming, caused possibly by greenhouse gases
emitted by us and accumulated in the air
already is threatening us,
for example small island countries in the South Pacific
are said to disappear under the water in the near future.
Although the debate whether there is global--warming, or if there is,
whether the cause is the human emission of greenhouse gases have 
not yet been settled even scientifically,
to start eliminating the emission in this stage is politically correct
I think, 
because it seems to me that we will need oil not only as the source of energy,
but also as the source of many chemicals, and that
the latter need will last even if all energy can be produced from
something other than oils.
In this respect, nuclear plants are certainly good,
they use no oils, emit no greenhouse gases, etc.

But of course nuclear fuel must be handled with great care,
during and after its use.
The accident last year in Japan is just one example.
You know many others, including one at Chernobyl.
The point is that even though otherwise very safe,
once an accident spreads the radioactive elements around,
the damage caused is enormous and lasts quite long.
And even if a plant ends its life with no accident,
the waste produced %along with the electricity
and the plant itself is dangerously radioactive,
so they are to be watched by us, not just us living today
but including generations many many after us.
Considering this, 
the cost imposed for human beings total is I think formidablly big.
Maybe to decrease the emission of carbon dioxide quickly
we must first rely on the nuclear plants.
But we should bear in mind that it is only an intermediate step to
a more, truly safer way of getting energy.

\SubAnswer
\textbf{全訳}

20世紀における純粋科学の大いなる進歩は、科学と社会の関係を
完全に変えた。
科学技術の応用を通して、科学は我々の生活の支配的要素となった。
科学は生活水準を大いに向上させた。
科学は大変な危険物をも生み出し、それは人類の存在をも脅かしている。
科学者は自分たちの研究が個人の幸福や国家の方針と
無関係であるなどとはもはや言っていられない。

しかしながら、
\underlinejpn[(a)]{
多くの科学者はまだ、
「科学はそれ自身のためになされるべきである」
「科学は中立である」
「科学は誤用を非難され得ない」
といった格言に基づく
象牙の塔的な思考に固執している。
}
この道徳観念を欠いた態度は、私の考えでは、
実際不道徳である。何故ならば、自分の行動によって
もたらされるかもしれない結果に対する
個人の責任を避けているからである。

国際社会の相互依存性は増大する一方であり、
そのことは個人に大きな利益をもたらすと同時に
個々人に責任を課している。
民間人各々が自分の行為に責任を持たなければならない。
私が概説したような理由によって、このことは特に科学者に当てはまる。
そのことはまた、科学者のためでもある。
何故ならば大衆は科学の誤用の責任が科学者にあると考えているからである。
\underlinejpn[(b)]{
大衆は、財源を抑えるとか制限的な規則を設けるといった、
科学を統制する手段を持っている。
科学者自らがその研究の応用に責任を持つことを確実にするため
適切な措置をとればなお一層良い。
}

社会に対してもたらされ得る危害を
調査するプロジェクトを監視することも含めて、
科学者の専門組織は構成員向けに研究上の倫理規約を設けるべきである。
科学に携わる職業に新規に就く者に
自身の社会的・道徳的責任を自覚させることは特に重要である。
1つの方法は、卒業時に同意すべき
科学者の誓約(一種のヒポクラテスの誓い)を設けることであろう。
医学の職においてのように、そのような誓いの主たる価値は象徴的であるだろうが、
その誓いが
若い科学者に
学界や産業でのキャリアに漕ぎ出す前に
研究の専攻分野の
帰結についてより広く
じっくり考えさせるきっかけを与えるであろう
と私は信じている。

私は合衆国の学生パグウォッシュグループが提唱した誓約が気に入っている。
それは既に多くの国々の何千もの学生によって署名されている。
それにはこう書かれている;
「
私は、
科学と技術が社会的に責任を持って用いられるようなより良い世界を実現するよう努める
ことを誓います。
私は、人類や自然環境に害を与えることを意図した如何なる目的のためにも
私の受けた教育を用いません。
私はこの職につく限りにおいて、行動する前に私の研究の倫理的影響を良く考えます。
私に課された要求は大きいかもしれませんが、
私は個々の責任が平和への道における第一歩であるということを
認識しているのでこの宣言に署名します。
」

\begin{flushright}
(\textit{Science},\textbf{288}(1999)より)
\end{flushright}

%うーん、翻訳はやっぱり難しいわ。
%日本語になんかしない方が明瞭な文だよねえ、絶対。
%へたくそな訳にしか出来ませんでした。申し訳ないです。

\paragraph{解答例 : 科学者になろうとする人に筆者が提唱していること}

科学は大いに発展し、その応用は
我々の生活水準の向上に貢献すると同時に
人類・自然環境に有害なものをも生み出している。
科学が個人の幸福から国家間の利害までも
左右するほどの強大な影響力を獲得した今、
もはや科学者が研究成果の応用に関して
無責任な態度をとることは許されない。
科学者は実際的な科学による害を調査するのみならず、
研究上の倫理規約を自らに課すべきである。
卒業して科学研究の世界に羽ばたく前に、
何らかの誓いを立て、
科学を扱う者としての責任、科学者としての倫理を
よく見つめなおしておくべきである。
即ち、
科学者を志す者は研究の応用に関して
研究者としての自分の責任をまず自覚すべきである、
と筆者は提唱しているのである。

%うーん、いまいちですねえ。時間があったらもうちょっと手直しします。
%実際の解答スペースがどんなサイズだったかわかんないから
%何となく長くなっちゃいました。

\end{subanswers}
\end{answer}

\end{document}

   





