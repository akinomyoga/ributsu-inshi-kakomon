\documentclass[fleqn]{jbook}
\usepackage{physpub}

\begin{document}

\begin{question}{$B1Q8l(B}{1:yt 2:$B9u>B(B}
\begin{subquestions}
\SubQuestion
    $B<!$NJ8$rFI$s$G!"<+$i9M$($k$H$3$m$r(B\underline{$B1Q8l$G(B}$B!"(B20$B9TDxEY$G5-$;!#(B
\baselineskip=12pt

Nuclear power plants currently provide about 18\% of the world's electricity.
Although global demand for electricity is increasing,
this figure is expected to fall as the construction of new nuclear power
stations winds down.
Some in the nuclear industry, however, are fighting back, arguing that
the only way to meet demand while cuting emissions of greenhouse gases
is to build more nuclear power plants.


But this fight back comes at a bad time for the industry.
Last year a nuclear worker in Japan died after mishandling enriched uranium.
Historically, governments invested heavily in the nuclear industry following
the oil crisis in the 1970's.
The price of oil then fell, new reserves of oil and gas were discovered.
However, much of the optimism within the nuclear industry comes from
its enviromental credentials.
The International Energy Agency estimates that world energy demand in 2020
will be two-thirds more than 1995 levels, and in 1997 at the Kyoto conference
industrialized nations pledged to cut greenhouse--gas emissions to
5\% below 1990 levels by 2012.
Nuclear energy produces almost no greenhouse gases.


Japan remains committed to nuclear power, despite recent problems,
and produces about one--third of its electricity from nuclear sources.
Japan expects to increase its nuclear capacity
by more than 20\% over the next two decades.
This is not typical of developed countries.
%\begin{flushright}
\hfil (adapted from \textit{Physics World}, April 2000)
%\end{flushright}


\SubQuestion
$B<!$N1QJ8$O!"%N!<%Y%kJ?OB>^$r<u>^$7$?0?$kJ*M}3X<T$NJ8>O$G$"$k!#(B
$B$3$l$rFI$_0J2<$N@_Ld$KEz$($h!#(B

\begin{subsubquestions}
  \SubSubQuestion $B2<@~It(B(a),(b)$B$rOBLu$;$h!#(B
  \SubSubQuestion $B2J3X<T$K$J$m$&$H$9$k?M$KI.<T$,Ds>'$7$F$$$k$3$H$r(B
	$BF|K\8l$G(B($BD>Lu$G$O$J$/!"<+J,$N8@MU$G(B)$B@bL@$;$h!#(B
\end{subsubquestions}

%$B0J2<!"1QJ8A4J84]<L$7(B
\baselineskip=12pt

The tremendous advances in pure science made during
the 20th century have completely changed the relation
between science and society. Through its technological
application, science has become a dominant element in
our lives. It has enormously improved the quality of life.
It has created great perils, threatening the very existence 
of the human species. Scientists can no longer claim that
their work has nothing to do with the welfare of the individual
or with state policies.

However,
%$BLdBj$N$d$?$iD9$$2<@~It(B(a)$B$G$9!#(B
\underlineeng[(a)]{many scientists still cling to an ivory tower mentality
founded on precepts such as "science should be done for its own sake",
"science is neutral", and "science cannot be blamed for its misapplication."}
%$B$3$3$^$G2<@~It(B
This amoral attitude is in my opinion actually immoral, because 
it eschews personal responsibility for the likely consequences of
one's actions.

The ever-growing interdependence of the world
community offers great benefits to individuals,
but by the same token it imposes responsibilities
on them. Every citizen must be accountable for his or
her deeds. This applies particularly to scientists,
for the reasons I have outlined. It is also in their
interest, because the public holds scientists responsible
for any misuse of science. 
%$B2<@~It(B(b)$B$O$3$3$G!A%9(B
\underlineeng[(b)]{The public has the means to control science by withholding
the purse or imposing restrictive regulations. It is far better
that scientists themselves take appropriate steps to ensure
responsible application of their work.}
%$B0J>e2<@~It=*N;(B

Professional organization of scientists should work
out ethical codes of conduct for their members, including 
the monitoring of research projects for possible harm to
society. It is particularly important to ensure that new
entrants into the scientific profession are made aware
of their social and moral responsibilities. One way 
would be to initiate a pledge for scientists, a sort of
Hippocratic oath, to be taken at graduation. As in the
medical profession, the main value of such an oath might be 
symbolic, but I believe it would stimulate young scientists 
to reflect on the wider consequences of their intended field 
of work before embarking on a career in academia or industry.

I like the pledge initiated by the Student Pugwash Group
in the United States, which has already been signed by
thousands of students from many countries. It reads:
%$B$3$N%3%a%s%H$ND>A0$O(B
% ";" $B$J$N$+!!(B":" $B$J$N$+0u:~ITL@NF$G$A$g$C$HH=JL$D$-$^$;$s$G$7$?!#(B
%$B1Q8l$N$3$NJU$N6hJL$,J,$+$kJ}!"$J$*$7$F$A$g!A$@$$(B
"I promise to work for a better world, where science and
technology are used in socially responsible ways. I will
not use my education for any purpose intended to harm 
human beings or the environment. Through my career, I 
will consider the ethical implication of my work before 
I take action. While the demands placed upon me may be great,
I sign this declaration because I recognize that individual 
responsibility is the first step on the path to peace."

%$B86K\$K$J$i$C$F1&4s$;(B
\begin{flushright}

%$BM?$($i$l$?8lWC(B
(precept: guide for behavior, eschew: keep oneself away from)

%$B0zMQ(B
(adapted from \textit{Science}, \textbf{288}(1999))

\end{flushright}





\end{subquestions}
\end{question}
\begin{answer}{$B1Q8l(B}{}

\begin{subanswers}
\SubAnswer 
\textbf{$BA4Lu(B}

$B86;RNOH/EE$O8=:_@$3&$NEENO$N$*$h$=(B18\% $B$r$^$+$J$C$F$$$k!#(B
$B@$3&$NEENO<{MW$OA}2C$7$F$$$k$,!"$3$N?tCM$+$i$O8:$k$b$N$H8+$3$^$l$F$$$k!#(B
$B?75,$N86H/$N7z@_$,8:$C$F$$$k$+$i$G$"$k!#(B
$B$7$+$7!"86H/6H3&$N0lIt$OH?7b$9$k!#(B% fight back $B$K$b$C$H$^$H$b$JLu$O$J$$$+!#(B
$B29<<8z2L%,%9$NGS=P$rM^$($D$D<{MW$K$3$?$($k$K$O!"$b$C$H86H/$r$D$/$k$7$+$J$$$H!#(B

$B$7$+$7$3$NH?7b$b;~4|$,0-$$!#(B
$B:rG/F|K\$G$O!"G;=L%&%i%s$N7ZN($J$H$j$"$D$+$$$K$h$C$F$R$H$j$,K4$/$J$C$F$$$k!#(B
$BNr;KE*$K8+$l$P!"@/I\$O(B1970$BG/Be$N@PL}4m5!0JMh!"(B
$B86H/6H3&$K$+$J$j$NEj;q$r$7$F$-$?!#(B
$B$=$N8e@PL}$N2A3J$O$5$,$j!"?7$?$JL}ED$d%,%9ED$bH/8+$5$l$?$,!"(B
$B6H3&FbIt$O>e$K$"$2$?$h$&$J4D6-G[N8$NLL$+$i3Z4QE*$G$"$C$?!#(B
$B9q:]%(%M%k%.!<5!4X$O(B2020$BG/$N@$3&$N%(%M%k%.!<<{MW$O(B
1995$BG/$N(B5/3$BG\$K$J$k$H$7$F$$$k!#(B
$B0lJ}!"(B1997$BG/$N5~ET2q5D$G$O9)6H@h?J9q$O(B2012$BG/$K$O29<<8z2L%,%9$NGS=P$r(B
1900$BG/$h$j(B5\% $B>/$J$/$9$k$H@@$C$?!#(B
$B86;RNO$O$[$H$s$I29<<8z2L%,%9$r=P$5$J$$$N$G$"$k!#(B

$BF|K\$O:r:#$NIT>M;v$K$b$+$+$o$i$:86;RNO$K0MB8$7$F$$$k!#(B
$B<B:]!"F|K\$NEENO$N(B1/3$B$O86H/$K$h$k$b$N$G$"$k!#(B
$BF|K\$O86H/$K$h$kH/EE$r$3$l$+$i$N(B20$BG/$G(B20\% $B0J>eA}$d$9$3$H$r7W2h$7$F$$$k!#(B
$B$3$N9TF0$OB>$N@h?J9q$H$O0[$J$C$F$$$k!#(B

\paragraph{$B2rEzNc(B}

I forcus the discussion in the following on
the claim that nuclear plants produce no greenhouse gases,
which is seemingly mentioned to imply
that they are somewhat `safer' to the environment.

We often hear that global warming, caused possibly by greenhouse gases
emitted by us and accumulated in the air
already is threatening us,
for example small island countries in the South Pacific
are said to disappear under the water in the near future.
Although the debate whether there is global--warming, or if there is,
whether the cause is the human emission of greenhouse gases have 
not yet been settled even scientifically,
to start eliminating the emission in this stage is politically correct
I think, 
because it seems to me that we will need oil not only as the source of energy,
but also as the source of many chemicals, and that
the latter need will last even if all energy can be produced from
something other than oils.
In this respect, nuclear plants are certainly good,
they use no oils, emit no greenhouse gases, etc.

But of course nuclear fuel must be handled with great care,
during and after its use.
The accident last year in Japan is just one example.
You know many others, including one at Chernobyl.
The point is that even though otherwise very safe,
once an accident spreads the radioactive elements around,
the damage caused is enormous and lasts quite long.
And even if a plant ends its life with no accident,
the waste produced %along with the electricity
and the plant itself is dangerously radioactive,
so they are to be watched by us, not just us living today
but including generations many many after us.
Considering this, 
the cost imposed for human beings total is I think formidablly big.
Maybe to decrease the emission of carbon dioxide quickly
we must first rely on the nuclear plants.
But we should bear in mind that it is only an intermediate step to
a more, truly safer way of getting energy.

\SubAnswer
\textbf{$BA4Lu(B}

20$B@$5*$K$*$1$k=c?h2J3X$NBg$$$J$k?JJb$O!"2J3X$H<R2q$N4X78$r(B
$B40A4$KJQ$($?!#(B
$B2J3X5;=Q$N1~MQ$rDL$7$F!"2J3X$O2f!9$N@83h$N;YG[E*MWAG$H$J$C$?!#(B
$B2J3X$O@83h?e=`$rBg$$$K8~>e$5$;$?!#(B
$B2J3X$OBgJQ$J4m81J*$r$b@8$_=P$7!"$=$l$O?MN`$NB8:_$r$b6<$+$7$F$$$k!#(B
$B2J3X<T$O<+J,$?$A$N8&5f$,8D?M$N9,J!$d9q2H$NJ}?K$H(B
$BL54X78$G$"$k$J$I$H$O$b$O$d8@$C$F$$$i$l$J$$!#(B

$B$7$+$7$J$,$i!"(B
\underlinejpn[(a)]{
$BB?$/$N2J3X<T$O$^$@!"(B
$B!V2J3X$O$=$l<+?H$N$?$a$K$J$5$l$k$Y$-$G$"$k!W(B
$B!V2J3X$OCfN)$G$"$k!W(B
$B!V2J3X$O8mMQ$rHsFq$5$lF@$J$$!W(B
$B$H$$$C$?3J8@$K4p$E$/(B
$B>]2g$NEcE*$J;W9M$K8G<9$7$F$$$k!#(B
}
$B$3$NF;FA4QG0$r7g$$$?BVEY$O!";d$N9M$($G$O!"(B
$B<B:]ITF;FA$G$"$k!#2?8N$J$i$P!"<+J,$N9TF0$K$h$C$F(B
$B$b$?$i$5$l$k$+$b$7$l$J$$7k2L$KBP$9$k(B
$B8D?M$N@UG$$rHr$1$F$$$k$+$i$G$"$k!#(B

$B9q:]<R2q$NAj8_0MB8@-$OA}Bg$9$k0lJ}$G$"$j!"(B
$B$=$N$3$H$O8D?M$KBg$-$JMx1W$r$b$?$i$9$HF1;~$K(B
$B8D!9?M$K@UG$$r2]$7$F$$$k!#(B
$BL14V?M3F!9$,<+J,$N9T0Y$K@UG$$r;}$?$J$1$l$P$J$i$J$$!#(B
$B;d$,35@b$7$?$h$&$JM}M3$K$h$C$F!"$3$N$3$H$OFC$K2J3X<T$KEv$F$O$^$k!#(B
$B$=$N$3$H$O$^$?!"2J3X<T$N$?$a$G$b$"$k!#(B
$B2?8N$J$i$PBg=0$O2J3X$N8mMQ$N@UG$$,2J3X<T$K$"$k$H9M$($F$$$k$+$i$G$"$k!#(B
\underlinejpn[(b)]{
$BBg=0$O!":b8;$rM^$($k$H$+@)8BE*$J5,B'$r@_$1$k$H$$$C$?!"(B
$B2J3X$rE}@)$9$k<jCJ$r;}$C$F$$$k!#(B
$B2J3X<T<+$i$,$=$N8&5f$N1~MQ$K@UG$$r;}$D$3$H$r3N<B$K$9$k$?$a(B
$BE,@Z$JA<CV$r$H$l$P$J$*0lAXNI$$!#(B
}

$B<R2q$KBP$7$F$b$?$i$5$lF@$k4m32$r(B
$BD4::$9$k%W%m%8%'%/%H$r4F;k$9$k$3$H$b4^$a$F!"(B
$B2J3X<T$N@lLgAH?%$O9=@.0w8~$1$K8&5f>e$NNQM}5,Ls$r@_$1$k$Y$-$G$"$k!#(B
$B2J3X$K7H$o$k?&6H$K?75,$K="$/<T$K(B
$B<+?H$N<R2qE*!&F;FAE*@UG$$r<+3P$5$;$k$3$H$OFC$K=EMW$G$"$k!#(B
1$B$D$NJ}K!$O!"B46H;~$KF10U$9$Y$-(B
$B2J3X<T$N@@Ls!J0l<o$N%R%]%/%i%F%9$N@@$$!K$r@_$1$k$3$H$G$"$m$&!#(B
$B0e3X$N?&$K$*$$$F$N$h$&$K!"$=$N$h$&$J@@$$$N<g$?$k2ACM$O>]D'E*$G$"$k$@$m$&$,!"(B
$B$=$N@@$$$,(B
$B<c$$2J3X<T$K(B
$B3X3&$d;:6H$G$N%-%c%j%"$KAf$.=P$9A0$K(B
$B8&5f$N@l96J,Ln$N(B
$B5"7k$K$D$$$F$h$j9-$/(B
$B$8$C$/$j9M$($5$;$k$-$C$+$1$rM?$($k$G$"$m$&(B
$B$H;d$O?.$8$F$$$k!#(B

$B;d$O9g=09q$N3X@8%Q%0%&%)%C%7%e%0%k!<%W$,Ds>'$7$?@@Ls$,5$$KF~$C$F$$$k!#(B
$B$=$l$O4{$KB?$/$N9q!9$N2?@i$b$N3X@8$K$h$C$F=pL>$5$l$F$$$k!#(B
$B$=$l$K$O$3$&=q$+$l$F$$$k(B;
$B!V(B
$B;d$O!"(B
$B2J3X$H5;=Q$,<R2qE*$K@UG$$r;}$C$FMQ$$$i$l$k$h$&$J$h$jNI$$@$3&$r<B8=$9$k$h$&EX$a$k(B
$B$3$H$r@@$$$^$9!#(B
$B;d$O!"?MN`$d<+A34D6-$K32$rM?$($k$3$H$r0U?^$7$?G!2?$J$kL\E*$N$?$a$K$b(B
$B;d$N<u$1$?650i$rMQ$$$^$;$s!#(B
$B;d$O$3$N?&$K$D$/8B$j$K$*$$$F!"9TF0$9$kA0$K;d$N8&5f$NNQM}E*1F6A$rNI$/9M$($^$9!#(B
$B;d$K2]$5$l$?MW5a$OBg$-$$$+$b$7$l$^$;$s$,!"(B
$B;d$O8D!9$N@UG$$,J?OB$X$NF;$K$*$1$kBh0lJb$G$"$k$H$$$&$3$H$r(B
$BG'<1$7$F$$$k$N$G$3$N@k8@$K=pL>$7$^$9!#(B
$B!W(B

\begin{flushright}
(\textit{Science},\textbf{288}(1999)$B$h$j(B)
\end{flushright}

%$B$&!<$s!"K]Lu$O$d$C$Q$jFq$7$$$o!#(B
%$BF|K\8l$K$J$s$+$7$J$$J}$,L@NF$JJ8$@$h$M$(!"@dBP!#(B
%$B$X$?$/$=$JLu$K$7$+=PMh$^$;$s$G$7$?!#?=$7Lu$J$$$G$9!#(B

\paragraph{$B2rEzNc(B : $B2J3X<T$K$J$m$&$H$9$k?M$KI.<T$,Ds>'$7$F$$$k$3$H(B}

$B2J3X$OBg$$$KH/E8$7!"$=$N1~MQ$O(B
$B2f!9$N@83h?e=`$N8~>e$K9W8%$9$k$HF1;~$K(B
$B?MN`!&<+A34D6-$KM-32$J$b$N$r$b@8$_=P$7$F$$$k!#(B
$B2J3X$,8D?M$N9,J!$+$i9q2H4V$NMx32$^$G$b(B
$B:81&$9$k$[$I$N6/Bg$J1F6ANO$r3MF@$7$?:#!"(B
$B$b$O$d2J3X<T$,8&5f@.2L$N1~MQ$K4X$7$F(B
$BL5@UG$$JBVEY$r$H$k$3$H$O5v$5$l$J$$!#(B
$B2J3X<T$O<B:]E*$J2J3X$K$h$k32$rD4::$9$k$N$_$J$i$:!"(B
$B8&5f>e$NNQM}5,Ls$r<+$i$K2]$9$Y$-$G$"$k!#(B
$BB46H$7$F2J3X8&5f$N@$3&$K1)$P$?$/A0$K!"(B
$B2?$i$+$N@@$$$rN)$F!"(B
$B2J3X$r07$&<T$H$7$F$N@UG$!"2J3X<T$H$7$F$NNQM}$r(B
$B$h$/8+$D$a$J$*$7$F$*$/$Y$-$G$"$k!#(B
$BB($A!"(B
$B2J3X<T$r;V$9<T$O8&5f$N1~MQ$K4X$7$F(B
$B8&5f<T$H$7$F$N<+J,$N@UG$$r$^$:<+3P$9$Y$-$G$"$k!"(B
$B$HI.<T$ODs>'$7$F$$$k$N$G$"$k!#(B

%$B$&!<$s!"$$$^$$$A$G$9$M$(!#;~4V$,$"$C$?$i$b$&$A$g$C$H<jD>$7$7$^$9!#(B
%$B<B:]$N2rEz%9%Z!<%9$,$I$s$J%5%$%:$@$C$?$+$o$+$s$J$$$+$i(B
%$B2?$H$J$/D9$/$J$C$A$c$$$^$7$?!#(B

\end{subanswers}
\end{answer}

\end{document}

   





