\documentclass[fleqn]{jbook}
\usepackage{physpub}

\begin{document}
\let\b\mathbf
\begin{question}{問題6}{yt}

\begin{enumerate}
\item ある半導体試料のホール係数の温度依存性を測定したところ、図1のようにA,B,Cの3つの温度域でそれぞれ特徴的な振る舞いが見られた。これに関して以下の問に答えよ。
但し、この半導体の伝導体の底は等方的で単一の極小をもつ単純な構造を持つものとする。また、この試料には単一種類のドナー(電子供与体)のみが含まれ、アクセプター(電子受容体)はないものとする。必要ならば次の数値を参考にせよ:\[
素電荷\ e=1.60\times 10^{-19}\hbox{C},\qquad
自然対数\ \ln 10=\log_e10=2.30
\]
\begin{center}
\input{2000phy6-1.tpc}
\end{center}
\begin{enumerate}
\item 室温(300K)におけるキャリアー密度(電子密度)を求めよ。
\item 前問で求めたキャリアー密度の値は、そのまま、この試料に含まれるドナー不純物の濃度を与えることになる。その理由を説明せよ。
\item ドナー準位の束縛エネルギー$E_D$(伝導体の底を基準とした値)を求めよ。
\item 前問で求めたエネルギーに相当する電磁波は次のどの波長領域に属するか、次のうちから選べ。\[
ラジオ波/マイクロ波/赤外線/可視光/紫外線/\hbox{X}線
\]
\item 半導体中のドナー束縛状態は、イオン化したドナー不純物$D^+$のクーロン場に電子が束縛されたものである。この束縛状態を水素原子モデルにならって考えてみよう。水素原子の基底状態の束縛エネルギー$E_0$は\[
E_0=\frac{1}{(4\pi\epsilon_0)^2}\frac{m_0e^4}{2\hbar^2}=13.6\hbox{eV}\qquad
(\epsilon_0:真空の誘電率,\ 
m_0:電子の質量)
\]である。今の場合、水素原子モデルと異なる点は、a) 電子が真空中ではなく誘電率$\epsilon$の媒質中にあること、および、b) 結晶の周期ポテンシャルの効果によって電子の有効質量$m^*$が自由電子のそれとは異なること、の2点である。これらのことを考慮して、ドナー束縛エネルギー$E_D$を$E_0$、$m^*/m_0$, $\epsilon/\epsilon_0$によって表わせ。
\item $\epsilon/\epsilon_0=10$であるとして、この半導体中のドナー束縛状態に付いて、水素原子のボーア半径に対応する長さを求めよ。その長さを、結晶の原子間距離及びドナー不純物原子間の平均距離と比較することにより、上記の水素原子モデルによる近似が正当化されることを示せ。
\end{enumerate}
\item ある物質を短冊型に切り出した試料に図2のように電極をつけ、端子 1--2間に定電流電源をつなぎ、端子1--2間および端子3--4間にそれぞれ高入力インピーダンスの電圧計をつないで測定を行った。図中に示した各部の寸法は$L=5$mm, $l=3$mm, $w=1$mm, $d=0.1$mm である。
\begin{enumerate}
\item 端子1--2間に一定電流$I=0.1 mA$を流したところ、端子1--2間には$V_{1-2}=90.0$mV, 端子3--4間には$V_{3-4}=24.0$mVの電圧が発生した。この物質の電気抵抗率$\rho$を計算せよ。
\item 電気抵抗率の測定では、このように試料の4本の電極をつけて、電流端子と電圧端子を別々にすることがしばしば行われる。何の為に、そのようにするのかを説明せよ。
\item 実際の測定では、電流を$I=+0.1$mA としたとき$V_{3-4}=+24.8$mVで、
$I=-0.1$mAとしたときは$V_{3-4}=-23.2$mV であった。(問(i)で用いた$V_{3-4}=24.0$mVはそれらの平均を取ったものである。)このように、電流を反転した時に電圧の大きさが同じにならない原因として、どのような物理的現象が考えられるか。
\end{enumerate}
\end{enumerate}
\begin{center}
\input{2000phy6-2.tpc}
\end{center}
\end{question}
\begin{answer}{問題6}{}

\begin{enumerate}
\item 
\begin{enumerate}
\item 図より300Kでの$R_H$は$10^{-3}\hbox{m}^3/\hbox{C}$である。
キャリア密度を$n$とすると$|R_H|=1/ne$なので、
$n=6.25\cdot10^{21}/\hbox{m}^3$。
\item 
ドナー不純物が$n$個、伝導帯準位が$N\gg n$個あるとすると、
伝導帯にいるキャリアーの数は$nN/(N+n\exp({E_D/kT}))$できまると考えられる。
よって$R_H$は $定数+\exp(E_D/kT)$と振る舞う。
Bではキャリアーの数が一定になっているので、
ドナーから伝導体に移りうるキャリアーはみな移ってしまっていることが判る。
一方で、価電子帯からの励起が効くのは図のAに移ってからであるから、
室温でのキャリアーの数はドナーの数に一致する。
\item Cのかたむきから$T \ln n\propto \ln 10\times 10^3\hbox{K} /8 $、よって
$E_D=k_B 288\hbox{K}\sim0.025\hbox{eV}$。
\item これは$\hbar c/E_D\sim8\mu$mに相当するので、マイクロ波。
\item 単純に比例計算から、$E_D=E_0(\epsilon/\epsilon_0)^{-2}(m^*/m_0)^{-1}$.
\item エネルギーとボーア半径は逆比例するので、
ドナー束縛状態のボーア半径は水素原子の$13.6\hbox{eV}/0.025\hbox{eV}=5.5\cdot 10^2$倍で$29$nm。Si の格子定数は2.3\AA だからこれは充分に広く、電子の質量を有効質量として、物質の誘電率を用いて計算することを妥当にしている。
また、ドナー濃度からドナー間隔は$54$nm, これはボーア半径の倍程度であり、
電子密度は指数関数的に落ちるので、ほかのドナー原子を考えない近似を妥当にしている。
\end{enumerate}
\item 
\begin{enumerate}
\item $V_{3-4}=I\rho l/(wd)$より$\rho=V_{3-4}wd/(lI)$= $8\Omega\cdot $mm.
\item 電流の流れている端子を用いて電圧を測ると、小さい試料の場合は
端子の線内での電圧降下が馬鹿にならないので、電流の流れない端子を別に用意して
測定する。四端子法と呼ばれる。
\item よわい整流作用?
\end{enumerate}
\end{enumerate}
\end{answer}


\end{document}