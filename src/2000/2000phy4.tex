\documentclass[fleqn]{jbook}
\usepackage{physpub}

\begin{document}

\begin{question}{問題4}{中山優}
ロケットの等加速度運動を相対論的に考察する。以下では一次元的な運動のみを考え、静止した観測者の系で時刻$t$におけるロケットの座標を$x(t)$と書く。
\begin{enumerate}
	\item ロケットに固定された時計が示す時間を固有時間$\tau$と呼ぶ。観測系における時刻$t_1$から$t_2$の間の固有時間の経過は光速を$c$として
	$$ \Delta\tau = \frac{1}{c}\int_{t1}^{t_2} dt \sqrt{c^2-\left(\frac{dx(t)}{dt}\right)^2} $$
	で与えられることをローレンツ変換を用いて示せ。ここでローレンツ変換とは、ある座標系$(t,x)$とそれに対して$x$軸方向に速度$v$で運動している座標系$(t',x')$の間では次のように定義される。
	$$ ct' = \frac{1}{\sqrt{1-(v/c)^2}}ct - \frac{v/c}{\sqrt{1-(v/c)^2}}x $$
	$$ x' =  - \frac{v/c}{\sqrt{1-(v/c)^2}}ct + \frac{1}{\sqrt{1-(v/c)^2}}x $$
	
	\item 相対論的な速度ベクトルおよび加速度ベクトルは
	
	$(u^{\mu}) = (c\frac{dt}{d\tau},\frac{dx}{d\tau}),a^{\mu} = \frac{du^{\mu}}{d\tau},(\mu = 0,1)$で記述される。このとき常に
	\begin{equation}
	u^{\mu}u_{\mu} = c^2, u^{\mu}a_{\mu} = 0
	\end{equation}
	が成立することを示せ。ただし$u^{\mu}v_{\mu} \equiv u^{0}v^{0}-u^{1}v^{1}$
	
	\item 以下ではロケットの等加速度運動、すなわち運動する物体の静止系において正方向に定数$g(>0)$で加速されている状況を考察する。このときロケットの加速度ベクトルが$a^{\mu}a_{\mu} = -g^2$をみたすことを示せ。またそれと(1)を用いて$u^{\mu}$に対する次の運動方程式を導け。
	\begin{equation}
	\frac{du^0(\tau)}{d\tau} = \frac{g}{c}u^1(\tau) \frac{du^1(\tau)}{d\tau} = \frac{g}{c}u^0(\tau)
	\end{equation}
	\item 初期条件$\tau = 0$で$t=x=\frac{dx}{d\tau} = 0$という初期条件のもとに運動方程式(2)を解き、$t,x$をロケットの固有時間$\tau$の関数として求めよ。
	\item これから例えば$g$を地表の重力加速度($\simeq$ 10m/秒$^2$)とした場合、地球から銀河の中心まで行くのに(距離約3万光年)ロケットの固有時間はどれくらい経過するか。有効数字一桁で答えよ。ただし$\ln(10) \simeq 2.3,1年 \simeq 3.1 \times 10^7 秒$を用いて良い。
	\SubQuestion 地球からこのロケットを観測するために、信号を光速で送りそれがロケットで反射して返ってくるまでの時間を計測する。時刻$t$に送った信号が返ってくるまでの時間間隔を求めよ。特にこの観測が不能になるのはどのような場合か理由とともに考察せよ。
	\end{enumerate}
\end{question}
\begin{answer}{問題4}{}
特殊相対論において、等加速度運動とはどのようなものだろうか。等速直線運動と異なり、あらゆる系から見てガリレイ運動学的な加速度運動を行うことは明らかに不可能である。なぜなら、十分な時間がたてばある慣性系から見てその運動は、光速を超えてしまう\footnote{光速になる瞬間までをガリレイ運動学的な等加速度運動を相対論的に扱い、相対論的な等加速度運動と定義することもやろうとすれば可能である。が、この定義は慣性系によってしまうのでローレンツ変換に共変な量で書き表すことができない。}。しかし、相対論においても等加速度運動と呼ぶにふさわしい運動が存在し、その最ももっともらしい定義は本問で与えられるものである。実際、一様な場のなかでの粒子の運動はこのような結果をもたらす。参考書:「相対性理論入門」ランダウ・ジューコフ著、東京図書。
\begin{enumerate}
	\item 固有時刻$\tau$とそこから微小固有時間$d\tau$経た時刻$\tau+d\tau$を考える。その時、観測系では時刻$t$と$t+dt$が対応し、観測系から見た粒子の位置を$x$、速度を$v = \frac{dx(t)}{dt} $とおく。ここで、ローレンツ変換より、
	$$ c\tau = \frac{1}{\sqrt{1-(v/c)^2}}ct - \frac{v/c}{\sqrt{1-(v/c)^2}}x $$
	$$ c(\tau+d\tau) = \frac{1}{\sqrt{1-(v/c)^2}}c(t+dt) - \frac{v/c}{\sqrt{1-(v/c)^2}}(x+dx) $$
	が成り立つので、辺々引き算して、
	$$ cd\tau = \frac{1}{\sqrt{1-(v/c)^2}}cdt - \frac{v/c}{\sqrt{1-(v/c)^2}}dx $$
	ここで、$dx = vdt$より、
	$$ d\tau = \frac{1}{c}\frac{1}{\sqrt{1-(v/c)^2}}\left(c-\frac{v^2}{c}\right)dt = \frac{1}{c}\sqrt{c^2-v^2}dt $$
となる。よって、これを$t_1$から$t_2$まで積分して、
$$  \Delta\tau = \frac{1}{c}\int_{t1}^{t_2} dt \sqrt{c^2-\left(\frac{dx(t)}{dt}\right)^2} $$
を得る。
\item 
$$ u^{\mu}u_{\mu} = c^2\left(\frac{dt}{d\tau}\right)^2-\left(\frac{dx}{d\tau}\right)^2 = \left(\frac{dt}{d\tau}\right)^2\left[c^2-\left(\frac{dx}{dt}\right)^2 \right] $$ 
であるが、一方1より、
$$ \frac{d\tau}{dt} = \sqrt{1-\frac{v^2}{c^2}} $$
なので、
$$ u^{\mu}u_{\mu}  = \frac{1}{1-(v/c)^2}(c^2-v^2) = c^2 $$
である。

次に、今得た$ u^{\mu}u_{\mu} = c^2 $の両辺を$\tau$で微分して、
$$ a^{\mu}u_{\mu} +u^{\mu}a_{\mu}= 0 $$
を得るが、与えられた計量では$u^{\mu}v_{\mu} = u_{\mu}v^{\mu}$なので、これは、直ちに$u^{\mu}a_{\mu}= 0$を意味する。

\item 物体の静止系で、$a^{\mu} = ( \frac{du^0}{d\tau},g ) $となるべきであるが、1より$dt = d\tau $なので、$u^0 = c $であるから、$a^{\mu} = (0,g) $。故に、
$$ a^{\mu}a_{\mu} = -g^2 $$
これはローレンツスカラーとして共変な式なのでどの系でも正しい。

運動方程式を導くために使う式は、
\begin{equation}
(u^0+u^1)(u^0-u^1) = c^2 
\end{equation}
\begin{equation}
\frac{u^0}{a^1} = \frac{u^1}{a^0}  
\end{equation}
\begin{equation}
(a^0+a^1)(a^0-a^1) = -g^2   
\end{equation}
である(指示されている式を変形しただけ)。ここで、(1)と(3)の比を取って、
$$ \frac{u^1+u^0}{a^0+a^1}\frac{u^1-u^0}{a^0-a^1} = \frac{c^2}{g^2} $$
となるが、(2)と加比の理より、
$$ \left(\frac{u^0}{a^1}\right)^2 = \left(\frac{u^1}{a^0}\right)^2 = \left(\frac{c}{g}\right)^2 $$
を得る。符号をうまく取れば、答えの運動方程式、
$$ 	\frac{du^0(\tau)}{d\tau} = \frac{g}{c}u^1(\tau) \frac{du^1(\tau)}{d\tau} = \frac{g}{c}u^0(\tau)$$
が導かれる\footnote{符号であるが、(2)の条件から	複号同順で$u$と$a$の符号が同じか違うかまでは求まる。$g$の定義を考えればこの答えがふさわしいことは自明である。なぜ符合の紛れが生じたかであるが、用いたのはローレンツスカラーの式のみである。ローレンツスカラーは符号反転に対して不変であるので、符号反転した運動方程式も導かれたわけである。}。
\item 微分方程式の対角化は容易であり、辺々足したり引いたりすれば良い。対角化した微分方程式は
$$ \frac{d}{d\tau}(u^0+u^1) = \frac{g}{c}(u^0+u^1)$$
$$ \frac{d}{d\tau}(u^0-u^1) = -\frac{g}{c}(u^0-u^1)$$
となり、一般解として、
$$ u^0 = A\exp(\frac{g}{c}\tau) + B\exp(-\frac{g}{c}\tau)、 u^1 = A\exp(\frac{g}{c}\tau) - B\exp(-\frac{g}{c}\tau) $$
を得る($A,B$は定数。)。初期条件より、
$$ A+B =c、 A-B = 0 $$ 
なので、定数を決定できて、解は、
$$ u^0 = c\cosh(\frac{g}{c}\tau)、 u^1 = c\sinh(\frac{g}{c}\tau) $$
となる。さらに、$u$の定義より、$t,x$はそれぞれを積分すれば求まるので、
$$ t = \frac{c}{g}\sinh(\frac{g}{c}\tau)$$
$$ x = \frac{c^2}{g}\left[\cosh(\frac{g}{c}\tau) -1\right] $$
である。積分定数は初期条件を加味してうまくえらんである。
\item 4で求めた$x$の式から、$\tau$を計算するが、十分時間がかかると思われるので、$\cosh(\frac{g}{c}\tau)-1$は$\frac{1}{2}\exp(\frac{g}{c}\tau)$で近似してよい。これより、
$$ \tau = \frac{c}{g}\ln\left(\frac{2xg}{c^2}\right) = \frac{3\cdot10^8}{10}\ln\left(\frac{2\cdot3\cdot10^4\cdot3\cdot10^8\cdot3.1\cdot10^7\cdot10}{(3\cdot10^8)^2}\right) \simeq 10 [年] $$
となる。
\item まず、ロケットの運動$x(\tau)$を$x(t)$に書き換えよう。双曲線関数の公式を用いて、
$$ x = \frac{c^2}{g}\left(\sqrt{1+\frac{g^2}{c^2}t^2}-1\right)$$
となる。求める時間間隔を$s$とおけば、解くべき方程式は、
$$ \frac{cs}{2} = \frac{c^2}{g}\left[\sqrt{1+\frac{g^2}{c^2}\left(t+\frac{s}{2}\right)^2}-1\right]$$
である。両辺を2乗して、$s$について解けば、
$$ s = \frac{gt^2}{c-gt} > 0$$
を得る。$s$が正でなくてはならないので、$t<c/g $が必要となる。この時刻はガリレイ運動学においてはロケットが光速に到達する時刻である。もちろん相対論的な運動を考えているのでこの解釈は正しくないが、結局等加速度運動は固定系から眺めると十分時間が経つと$x = c(t-t_0)$に漸近する。よって、ある程度時間が経ってしまうと光速で追いかけても追いつけなくなってしまうことがわかる。なお比較のためにガリレイ運動学ではこの半分の時間で到達できなくなることを付け加えておく。
\end{enumerate}

\end{answer}


\end{document}