\documentclass[fleqn]{jbook}
\usepackage{physpub}

\begin{document}
\def\b#1{\hbox{\boldmath$#1$}}
\def\eV{\mathrm{eV}}
\begin{question}{問題8}{立川}
% 冗長な日本語だ。吐き気がする。
図1のように、細胞の膜上にある円筒型蛋白質が並進拡散している。
蛋白質が衝突した時に安定な複合体を作ったかどうかを判定するために、蛋白質間の励起エネルギー移動を測定する方法について、以下の問に答えよ。
\begin{center}
\input{2000phy8-1.tpc}
\end{center}
\begin{enumerate}
\item 蛋白質に結合させた発光体は、基底状態$S_0$と第一励起状態$S_1$を持ち、可視光で励起されるとする。図2のように2つの異なる蛋白質$L$ 、$M$ 内に存在する発光体の電子状態を$(S_0^L,S_1^L)$, $(S_0^M,S_1^M)$とする。
可視光により蛋白質$L$の電子が$S_0^L$から$S_1^L$に励起された。
$S_1^L$は$S_1^M$よりごくわずかエネルギー順いが高いとする。発光体が励起状態
$S_1$から$S_0$に遷移する時に、発光を伴う速度定数を$k_f$, 無輻射的に遷移する時の速度定数を$k_{nf}$とする。ただし速度定数とは単位時間あたり吸収・発光する発光体の割合である。まず蛋白質 $L$ の発光体を$t=0$で励起したとき、時間$t$で励起状態にある発光体の数$N(t)$を$k_f$, $k_{nf}$を用いて表わせ。また時間$t$での発光強度$F(t)$を表わせ。
\begin{center}\input{2000phy8-2.tpc}\end{center}
\item 蛋白質$L$内の発光体の励起された電子のエネルギーが$S_1^L$から$S_1^M$に移動することを、励起エネルギー移動という。このとき、蛋白質間の励起エネルギー移動は発光遷移双極子モーメント$\b F$と吸収遷移双極子モーメント$\b A$の双極子相互作用で起こる。励起エネルギー移動の速度定数$k_T$は、$k_T=\alpha k_f J \kappa^2 /R^6$と書ける。ここで$\kappa^2$は配向因子と呼ばれ、$\b F$や$\b A$の方向に依存する量である。蛋白質$L$の発光遷移双極子モーメントは$\b F_L$, 蛋白質$M$の吸収遷移双極子モーメントは$\b A_M$のように表わす。$\b R$は$\b F_L$と$\b F_R$の間の距離ベクトルであり、$J$は蛋白質$L$の発光スペクトルと蛋白質$M$の吸収スペクトルの重なりの大きさである。$\alpha$は比例定数である。
\begin{enumerate}
\item 励起エネルギー移動がないときとある時の励起状態の蛋白質$L$の発光体の寿命を$\tau$と$\tau'$とする。$1/\tau$と$\tau/\tau'$を上に定義した速度定数を用いてあらわせ。
\item 励起エネルギー移動が起こり$\tau'$が$\tau$の$1/2$になる距離を$R_0$とするとき、$R_0$はどのように書けるか。但し$\tau$, $\tau'$を使わずに書くこと。さらに$\tau/\tau'$を$R$, $R_0$の関数として表わすとどうなるか。
\item 配向因子$\kappa^2$は$\b F_L$と$\b A_M$の相対角度は位置によって大きく変化するし、この相対配置を試料溶液中の蛋白質で決定するのは一般に大変困難である。従って$\tau/\tau'$の値の観測だけでは、蛋白質$L$と$M$の距離$R$の変化を決定することは出来ない。ところが、発光体が蛋白質内で3次元方向に全く自由に速く運動している時は、$\kappa^2$の値は$2/3$であり、このときは蛋白質$L$と$M$の距離を計算することが出来る。$\kappa^2=2/3$の場合、蛋白質$L$と$M$が複合体を作った場合と、複合体を作らずおたがい離れて2次元膜上を運動している時とを比べる。蛋白質$L$の$\tau'/\tau$は$R/R_0$の変化に対してどのように変化するか。定性的に図示し、縦軸と横軸に特徴的な値を記入せよ。但しこのとき、蛋白質の直径は同じ$R_0/2$であるとし、発光体の中心は円筒型蛋白質の中心と一致するとする。
\end{enumerate}\newpage
\item 蛋白質$L$と$M$が単独で存在する場合の吸収スペクトルと発光スペクトルを図3に示す。$L$と$M$の吸収強度の最大値の波長を各々$\lambda_L$と$\lambda_M$とし、
$L$と$M$の発光強度の最大値の波長を各々$F_L$と$F_M$とし、
$\lambda_L<F_L<\lambda_M<F_M$とする。
ここで注意すべきは、現実の蛋白質は複雑な構造を持っているので、$L$と$M$の吸収、発光は種々の要因により図3のように幅の広いスペクトルを持つ。発光スペクトルを測定することで、実験的に蛋白質$L$と$M$の間に、励起エネルギー移動が起こったか、そうでないかを判定できる。どの波長で励起し、どの発光スペクトル測定を行えば判定できるか。図3にならって図示して解答せよ。
\end{enumerate}
\begin{center}
\input{2000phy8-3.tpc}
\end{center}
\end{question}
\begin{answer}{問題8}{}
\begin{enumerate}
\item $dt$の間に$N(t)(k_f+k_{nf})dt$だけ減るので、
$dN/dt=-(k_f+k_{nf})N$である。よって$N(t)=N(0)\exp(-(k_f+k_{nf})t)$。
また$F(t)=k_fN(t)=k_fN(0)\exp(-(k_f+k_{nf})t)$。
\item 
\begin{enumerate}
\item 寿命とは数が$1/e$になる時間のことである。よって
励起エネルギー移動がなければ$\tau=(k_f+k_{nf})^{-1}$、
励起エネルギー移動があれば$\tau'=(k_f+k_{nf}+\alpha k_f J \kappa^2 R^{-6})^{-1}$。よって\[
\frac{\tau}{\tau'}=\frac{k_f+k_{nf}+\alpha k_f J \kappa^2 R^{-6}}{k_f+k_{nf}}。
\]
\item 直前の式を$=2$と置いて$\alpha k_f J \kappa^2 R_0^{-6}=k_f+k_{nf}$,
よって$R_0=(\alpha k_f J \kappa^2/(k_f+k_{nf})^{1/6}$。
また\[
\frac{\tau}{\tau'}=1+\left(\frac{R_0}{R}\right)^6。
\]
\item 蛋白質$L$と$M$が接した時の双極子間の距離は$R_0/2$である。
よって$\tau'/\tau=1/(1+(R/R_0)^6)$はおよそ以下のようになる:

\input{2000phy8-4.tpc}

\end{enumerate}
\item $L$を$\lambda_L$で励起し、$M$の$F_M$の波長の光を観測すれば良い。
すなわち、$\lambda_L$で励起された$L$が$F_L$で発光し、それが
スペクトルに重なりのある$M$の$\lambda_M$から吸収され、
結局$F_M$で放射されるからである(図)。
\begin{center}
\input{2000phy8-5.tpc}
\end{center}
\end{enumerate}
\end{answer}


\end{document}