%% -*- coding:sjis -*-
%%
%% ChangeLog
%% 2013-07-14, Koichi Murase, 作成・入力
%%
\begin{question}{第2問}{村瀬}
\begin{enumerate}
\item
  以下の連立偏微分方程式を考える。
  \begin{align}
    \left.\begin{aligned}
      \frac{\partial u_1}{\partial t}
      +4 \frac{\partial u_2}{\partial x} &= 0\\
      \frac{\partial u_2}{\partial t}
      + \frac{\partial u_1}{\partial x} &= 0
    \end{aligned}\right\} \ilabel{eq:2011mathQ2.1.1}
  \end{align}
  ここで $u_1(x, t), u_2(x, t)$ は $-\infty<t<+\infty$ および $-\infty<x<+\infty$
  で定義された2変数関数である。以下の設問に答えよ。

  \begin{enumerate}
  \item\ilabel{item:2011mathQ2.1.1}
    式(\iref{eq:2011mathQ2.1.1})を以下のようにベクトル表記する。
    \begin{align*}
      \frac{\partial \bm{u}}{\partial t} + A\frac{\partial \bm{u}}{\partial x}=0,\quad
      \bm{u}=\begin{pmatrix} u_1(x,t) \\ u_2(x,t) \end{pmatrix}.
    \end{align*}
    このとき, 係数行列 $A$ の固有値 $\lambda_1$, $\lambda_2$
    と固有ベクトル $\bm{q}_1, \bm{q}_2$ を求めよ。

  \item\ilabel{item:2011mathQ2.1.2}
    設問(\iref{item:2011mathQ2.1.1})で求めた $\bm{q}_1, \bm{q}_2$ を並べた2次元正方行列 $P=(\bm{q}_1\; \bm{q}_2)$ を用いて変換
    \begin{align*}
      \begin{pmatrix} s_1(x,t) \\ s_2(x,t) \end{pmatrix} &= P^{-1} \bm{u}.
    \end{align*}
    を行い, 式(\iref{eq:2011mathQ2.1.1}) を $s_1(x, t), s_2(x, t)$ に対する偏微分方程式に書き換えよ。

  \item
    設問(\iref{item:2011mathQ2.1.2})で得られた式は
    $s_1(x, t), s_2(x, t)$ に対してそれぞれ独立な線形方程式であり,
    初期条件を与えれば解くことができる。さらに, その解から $u_1(x, t), u_2(x, t)$
    を求めることができる。このことを用いて, 初期条件
    \begin{align*}
      u_1(x, t=0)=e^{-x^2},\quad
      u_2(x, t=0)=0
    \end{align*}
    のもとに, 連立偏微分方程式(\iref{eq:2011mathQ2.1.1})の解 $u_1(x, t), u_2(x, t)$ を求めよ。
    また $t=1$ での解の概形を図示せよ。
  \end{enumerate}

\item
  2つの関数 $f_1(x, t), f_2(x, t)$ に対する連立偏微分方程式
  \begin{align*}
    \left.\begin{aligned}
      \frac{\partial f_1}{\partial t}
      + \frac{\partial^2 f_1}{\partial x^2}
      -6 \frac{\partial^2 f_2}{\partial x^2} &= 0\\
      \frac{\partial f_2}{\partial t}
      + \frac{\partial^2 f_1}{\partial x^2}
      -4 \frac{\partial^2 f_2}{\partial x^2} &= 0
    \end{aligned}\right\}
  \end{align*}
  を初期条件
  \begin{align*}
    f_1(x, t=0)=e^{-x^2},\quad
    f_2(x, t=0)=0
  \end{align*}
  のもとに解け。関数の定義域は $0\le t<+\infty$ および $-\infty<x<+\infty$ とする。
\end{enumerate}
\end{question}
