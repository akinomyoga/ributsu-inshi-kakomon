\documentclass[fleqn]{jbook}
\usepackage{physpub}

\begin{document}

\begin{question}{$B650i(B $B1Q8l(B}{}

\begin{subquestions}
\SubQuestion
$B<!$N1QJ8$rFI$_0J2<$N@_Ld$KEz$($h!#(B
\baselineskip=12pt

$B!!(BI propose to analyze ``Talking about Science.'' How is it best done? Why is it that a subject presented by A is thrilling account which leaves a deep impression, whereas the very same material presented by B is dull and boring and produces no impression whatever? How should we present our branch of science to fellow scientists who work in quite another field? How can we present science to those who have little or no scientific background, as if often the case with men of high ability who are important in affairs of state? How can we make the non-scientist understand why its study means so much to us, a passion they sometimes find very difficult to understand?

$B!!(BWhat is a basic character of a \underline{($B%"(B)``talk''}? I think it can be expressed by saying that its primary object is to create a state of mind, or point of view, not to convey information. I can perhaps illustrate what I mean by dwelling on the vast difference between the spoken and written account. Under the heading ``talk,'' I am not including a course of lectures where students take notes and the lectures follow each other as a composite whole. Nor do I include the ``get together'' of two or three experts in the same line of research, for which no rules are necessary. I am considering the hour's talk to an audience whose interest one has to arouse. The written account can also aim at creating a viewpoint, but its main function is to be a storehouse of information. The argument can be meaty and condensed. It can be packed with tables, graphs, and mathematical equations. This is possible because the reader can always pause and digest it at his leisure, going back over parts which he finds to be difficult. I do not mean to imply that one can be irresponsible in a talk, but one need not cross all the ``t's'' and dot all the ``i's.'' In fact, the talk would be spoiled by attempt to do so.

$B!!(BA talk is therefore different altogether from a \underline{($B%$(B)``paper.''} To my mind the governing factor which determines its art form is this: The success of the way in which the subject has been presented is measured by the extent to which the average number of the audience remembers it next day.

$B!!(BThis may seem an obvious statement, but if we use this principle as a yardstick to assess a lecture we have listened to, or in planning to a lecture of our own, it creates a very significant viewpoint. The value of a lecture is not to be measured by how much one manages to cram into an hour, how much important information has been referred to, or how completely it covers the ground. It is to be measured by how much a listener can tell a friend about it next morning. If we honestly put this question to ourselves and think how little we can remember of talks we have heard, it gives us a sense of proportion and of values in planning a lecture and makes us realize that what we say will go over the heads of audience if we set our sights too high. I would like now to list what I believe to be some of the considerations which apply in planning a talk.

$B!!(BFor instance, suppose we ask how many points can we hope to ``get over'' in an hour? I think the answer should be ``one.'' If the average member of the audience can remember with interest and enthusiasm one main theme, the lecture has been a great success. I like to compare the composition of a lecture to that of a picture. Of course this is dangerous ground on which to venture, because art experts differ so much among themselves. But in simple terms, is it not held that a picture should have one main center of interest? It may have numerous subsidiary features, but the composition is so cunningly arranged that when the eye falls on these and follows their placing it is subtly led back to the main center of interest and does not fail out of picture frame. A lecture should be like that. There should be one main theme, and all the subsidiary interesting points, experiments, or demonstrations should be such that they remind hearer of the theme. As in a picture, so in a lecture, the force of the impression depends upon a ruthless sacrifice of unnecessary detail. It can be richly endowed with exciting details, but they must be of such a kind that recollection of them inevitably brings the main theme back to mind. In other words, the lecture must compose in the sense of having a pattern because it is this pattern which helps so much to impress it on one's memory.

$B!!(BA lecture is made or marred in the first 10 minutes. This is the time to establish the foundations, to remind the audience of things they half know already, and to define terms that will be used. Again this seems obvious, but I have listened to so much splendid material lost to the audience because the lecturer failed to realize that it did not know what he was talking about, whereas, if the precious first 10 minutes had been spent on preparation, he would have carried his listeners with him for the rest of the talk.\\

\fbox{\parbox[t]{16.15cm}{$B!!(BHere a most important principle comes in which I think of as the ``detective story'' principle. It is a matter of order. How dull a detective story would be if the writer told you who did it in the first chapter and then gave you the clues. Yet how many lectures do exactly this. One wishes to give the audience the esthetic pleasure of seeing how puzzling phenomena become crystal clear when one has the clue and thinks about them in the right way. So make sure the audience is first puzzled.

$B!!(BWe all know the tendency to go to sleep in lectures; how often have I felt ashamed at doing so myself. Though the best lecturer can never entirely escape from producing this effect, there is much that can be done to minimize it. A continuous even delivery is fatal. There is something hypnotic about it which induces sleep. Pauses and changes of tempo are essential. Above all, jokes have a marked and enduring effect.
}}
\begin{flushright}
\textit{The Art of Talking about Science,} Lawrence Bragg; Science {\bf 154}(1966)1614 $B$h$jH4?h(B
\end{flushright}
\baselineskip=15pt
\begin{subsubquestions}
\SubSubQuestion
$BI.<T$O!"2<@~It(B($B%"(B)$B$N(Btalk$B$H2<@~It(B($B%$(B)$B$N(Bpaper$B$H$NK\<AE*$JAj0c$O2?$@$H$$$C$F$$$k$N$+!"$^$?!"(Bsuccessful talk$B$H$O$I$N$h$&$J$b$N$@$H8@$C$F$$$k$N$+!"4J7i$KF|K\8l$G@bL@$;$h!#(B

\SubSubQuestion
$BI.<T$,!"(Bsuccessful talk$B$r$9$k>e$G!"(Bspeaker$B$,G[N8$9$Y$-$b$N$H$7$F$"$2$F$$$k;vJA$r(B4$B$D!"F|K\8l$G2U>r=q$-$K$7$F=R$Y$h!#(B

\SubSubQuestion
$BOH$G0O$C$?Fs$D$N%Q%i%0%i%U$K6&DL$7$?$R$H$D$NI{Bj$rIU$1$?$$!#0J2<$N(B5$B$D$NI{Bj$+$i:G$bE,$7$?$b$N$r0l$D$@$1A*$Y!#(B
\begin{itemize}
\item[($B%"(B)]\ A Detective Story
\item[($B%$(B)]\ The Arousing of Interest
\item[($B%&(B)]\ Tempo and Jokes
\item[($B%((B)]\ Giving Clues
\item[($B%*(B)]\ The Best Lecturer
\end{itemize}

\end{subsubquestions}

\SubQuestion
$B0J2<$NJ8>O$r1QLu$;$h!#$?$@$7!"I,MW$G$"$l$P0J2<$NC18l$r;29M$K$;$h!#(B

differential operator, simultaneity, transmitting medium, Galilean transformation

\begin{subsubquestions}
\SubSubQuestion
Schr\"{o}dinger$BJ}Dx<0$H8EE5E*$J%(%M%k%.!<1?F0NL$NJ}Dx<0$rHf$Y$F$_$k$H!">/$J$/$H$b<+M3N3;R$KBP$7$F$O!"%(%M%k%.!<$H1?F0NL$,GHF04X?t(B$\psi$$B$K:nMQ$9$k!"0J2<$NHyJ,:nMQAG$KBP1~$9$k$b$N$H9M$($i$l$k!#(B
\[
E \to i\hbar \frac{\del}{\del t} \quad,\quad p \to -i\hbar \nabla
\]

\SubSubQuestion
Michelson$B$H(BMorley$B$NM-L>$J<B83$O!"8wB.$,!"4QB,<T!"G^<A!"8w8;$N4V$NAjBP1?F0$KL54X78$K$9$Y$F$NJ}8~$KBP$7$FF1$8$G$"$k$3$H$r<($7$?!#(B

\SubSubQuestion
$B$7$?$,$C$F!"8EE5NO3X$rITJQ$KJ]$D%,%j%l%$JQ49$O@5$7$/$J$/!"8wB.$r0lDj$H$9$k$h$&$JJL$NJQ49$KCV$-49$($i$l$M$P$J$i$J$$!#(B

\SubSubQuestion
Einstein$B$O!"$3$N$h$&$JJQ49$,I,A3E*$K!"=>Mh$N;~4V$HF1;~@-$N35G0$rJQ99$7$F$7$^$&$3$H$rL@$i$+$K$7$?!#(B

\end{subsubquestions}

\end{subquestions}

\end{question}
\begin{answer}{$B650i(B $B1Q8l(B}{}
\begin{subanswers}
\SubAnswer
  {\bf $BA4Lu(B}

$B!!(B``Talking about Science''$B$K$D$$$FJ,@O$7$F$_$h$&!#$I$&$d$C$?$i0lHV$&$^$/$G$-$k$N$@$m$&$+!##A$K$h$C$FH/I=$5$l$?Bj:`$O?<$$0u>]$r;D$9%9%j%j%s%0$J$b$N$G$"$k$N$KBP$7!"F1$8$b$N$,#B$K$h$C$FH/I=$5$l$k$HB`6~$GA4$/0u>]$r;D$5$J$$$N$O$J$<$@$m$&$+!#A4$/JL$NJ,Ln$N2J3X<T$K<+J,$NJ,Ln$N$3$H$K$D$$$FH/I=$9$k$K$O$I$&$7$?$i$h$$$N$@$m$&$+!#@/<#2H$K$h$/$$$k$h$&$J!"2J3X$NCN<1$,$[$H$s$I!"$"$k$$$OA4$/$J$$?M!9$K2J3X$K$D$$$FOC$9$K$O$I$&$7$?$i$G$-$k$@$m$&$+!#2J3X<T0J30$N?MC#$K$=$N8&5f$,=EMW$J0UL#$r;}$C$F$$$kM}M3!"$9$J$o$AH`$i$K$H$C$F$O;~$KHs>o$KM}2r$7$,$?$$$b$N$G$"$k>pG.$r$I$N$h$&$K$7$?$i$o$+$C$F$b$i$($k$@$m$&$+!#(B

$B!!(B``Talk''$B$N4pK\E*$JFCD'$O2?$G$"$m$&$+!#$=$l$O(B``Talk''$B$NBh0l$NL\E*$O>pJs$rEA$($k$3$H$G$O$J$/!"?4$N>uBV$9$J$o$A9M$(J}$rC[$/$3$H$G$"$k!"$H8@$$I=$5$l$k$H;W$&!#;d$N8@$$$?$$;v$O$*$=$i$/8@MU$K$h$k@bL@$H3h;z$K$h$k@bL@$H$NBg$-$J0c$$$r$h$/9M$($F$_$k$3$H$G@bL@$G$-$k$@$m$&!#8+=P$7$N(B``Talk''$B$K$O!"3X@8$,%N!<%H$r<h$C$?$j0l$D0l$D$N9V5A$,$*8_$$$K%U%)%m!<$7$F$G$-$"$,$k$h$&$J9V:B$O4^$^$l$J$$!#$^$?!"F1$88&5fJ,Ln$N(B2$B!"(B3$B?M$N@lLg2H$N!"2?$i5,B'$bI,MW$H$7$J$$$h$&$J=8$^$j$b4^$^$J$$!#;d$,8@$C$F$$$k$N$O8B$i$l$?;~4VFb$GD0=0$N6=L#$r0z$-5/$3$5$J$1$l$P$J$i$J$$OC$N$3$H$G$"$k!#3h;z$K$h$k@bL@$bJ*$N8+J}$rC[$/$3$H$b$G$-$k$,!"$=$N<g$JL\E*$O>pJs$NC_@Q$G$"$k!#$=$N5DO@$OFbMF$,K-$+$GG;$$!#I=$d%0%i%U!"?t<0$r4^$_$&$k!#$3$l$OFI<T$,$$$D$bET9g$NNI$$;~$K>.5Y;_$7$F$^$H$a!"Fq$7$$$H;W$C$?ItJ,$KN)$ALa$k$3$H$,$G$-$k$+$i$3$=2DG=$J$N$G$"$k!#(BTalk$B$N:]$KL5@UG$$G$"$C$FNI$$$H$$$C$F$$$k$N$G$O$J$/!":Y$+$$$H$3$m$K$^$G5$$r;H$&I,MW$O$J$$$H$$$C$F$$$k$N$G$"$k!#<B:]!"$=$s$J$3$H$r$7$h$&$H$7$?$i(BTalk$B$OBfL5$7$K$J$C$F$7$^$&$@$m$&!#(B

$B!!$=$l$f$((BTalk$B$O!VO@J8!W$H$OA4$/0[$J$k$b$N$G$"$k!#;d$N9M$($G$O$=$N5;=Q$N7A<0$r7hDj$9$k<g$JMW0x$O<!$N$h$&$J$b$N$G$"$k!#!VBj:`$rH/I=$7$?J}K!$N@.2L$O<!$NF|$K$=$l$r3P$($F$$$kD0=0$NJ?6Q?t$NDxEY$GB,$i$l$k!#!W(B

$B!!$3$l$OEv$?$jA0$N0U8+$N$h$&$K;W$($k$+$b$7$l$J$$$,!"$b$7<+J,$,J9$$$?9V5A$rI>2A$9$k4p=`$H$7$F$3$N86M}$r;H$C$?$j!"$"$k$$$O<+J,<+?H$N9V5A$N7W2h$rN)$F$k:]$K;H$($P$H$F$b=EMW$J4QE@$,@8$_=P$5$l$k$N$G$"$k!#9V5A$N2ACM$O(B1$B;~4V$K$I$l$@$1$NNL$r5M$a9~$`$N$+$K$G$b!"$I$l$@$1B?$/$N=EMW$J>pJs$K?($l$?$+$K$G$b!"4pAC$r$I$l$@$1$7$C$+$j2!$5$($?$+$K$G$bB,$i$l$k$N$G$O$J$$!#MbD+$K$I$l$@$1$NJ9$-<j$,M'C#$K$=$l$K$D$$$FOC$;$k$+$GB,$i$l$k$N$G$"$k!#$b$72f!92J3X<T$,$3$NLdBj$r@5D>$K<u$1;_$a!"<+J,$,J9$$$?9V5A$r$I$l$[$I;W$$=P$9$3$H$,$G$-$J$$$N$+$H$$$&$3$H$r9M$($l$P!"2f!9$O9V5A$N7W2h$rN)$F$k:]$KH=CGNO$H2ACM4Q$,Hw$o$k$7!"L\I8$r9b$/$7$9$.$k$H2f!9$N8@$C$F$$$k$3$H$,D0=0$NM}2r$rD6$($F$7$^$&$3$H$K5$IU$/!#$3$3$G!"(BTalk$B$r=`Hw$9$k:]$KMxMQ$9$kG[N8$G$"$k$H;d$,;W$&$b$N$N$&$A$$$/$D$+$r5s$2$?$$$H;W$&!#(B

$B!!Nc$($P!"#1;~4V$K$I$l$@$1$N%]%$%s%H$rM}2r$5$;$k$3$H$,$G$-$k$+?R$M$F$_$h$&!#Ez$($O!V0l$D!W$@$H;d$O;W$&!#$b$7D0=0$NCf$NJ?6QE*$J?M$,6/$$6=L#$r;}$C$F0l$D$N%a%$%s%F!<%^$r3P$($F$$$k$J$i!"$=$N9V5A$OBg@.8y$G$"$k!#9V5A$H<L??$N9=@.$rHf$Y$F$_$h$&!#$b$A$m$s$3$l$O$d$k$K$O4m81$J;n$_$G$"$k!#$J$<$J$i7]=Q2H$K$OHs>o$KB?<oB?MM$J?M$?$A$,$$$k$+$i$G$"$k!#$7$+$7!"4JC1$K8@$C$F$7$^$&$H3(2h$K$O6=L#$N3K$H$J$k$b$N$,0l$D$"$k$H$O9M$($i$l$J$$$@$m$&$+!#3(2h$K$OB?$/$NI{<!E*$JFCD'$,$"$k$,!"$=$N9=@.$OHs>o$K9*L/$K;EAH$^$l$F$$$k$N$G!"$=$NFCD'$KL\$r$d$j!"G[CV$rDI$C$F$$$/$H$&$^$/6=L#$N3K$K0z$-La$5$l3($NOH$+$i$O$_=P$J$$$N$G$"$k!#9V5A$b$=$N$h$&$G$"$k$Y$-$G$"$k!#0l$D$N%a%$%s%F!<%^$,$"$j!"I{<!E*$J6=L#!"<B83!"<B>Z$H$$$&$N$O$9$Y$FJ9$-<j$K%F!<%^$r;W$$=P$5$;$k$h$&$J$b$N$G$"$k$Y$-$J$N$G$"$k!#3(2h$N$h$&$K9V5A$G$b!"0u>]$K;D$k$+$I$&$+$OITI,MW$J:Y$+$$E@$r%P%C%5%j$H@Z$jMn$H$;$k$+$I$&$+$K$+$+$C$F$$$k$N$G$"$k!#9V5A$K$O$b$H$b$H;I7cE*$J:YIt$,K-IY$K$"$k$+$b$7$l$J$$$,!"$=$N:YIt$O$=$l$i$r;W$$=P$7$F$$$/$HI,A3E*$K%a%$%s%F!<%^$r;W$$5/$3$7$F$7$^$&$h$&$J$b$N$G$J$/$F$O$J$i$J$$!#8@$$49$($l$P9V5A$O$"$k%Q%?!<%s$r;}$D$H$$$&0UL#$G9=@.$5$l$F$$$J$1$l$P$J$i$J$$!#$J$<$J$i$3$N%Q%?!<%s$G?M$N5-21$K0u>]$r;D$9$3$H$,$G$-$k$+$i$G$"$k!#(B

$B!!9V5A$O:G=i$N#1#0J,$G@.H]$,7h$^$C$F$7$^$&!#$3$l$OOC$NEZBf$r:n$j!"D0=0$KH`$i$,$9$G$KH>J,CN$C$F$$$k$3$H$r;W$$=P$5$;!";H$o$l$kMQ8l$rDj5A$9$k$?$a$N;~4V$G$"$k!#$d$O$j$3$l$bL@$i$+$J$3$H$K;W$($k$,!"AG@2$i$7$$Bj:`$G$"$C$F$b9V1i<T$,<+J,$NOC$rD0=0$O$o$+$C$F$$$J$$;v$rM}2r$G$-$F$$$J$$$?$a$K!"D0=0$,<h$j;D$5$l$F$7$^$C$F$$$k$b$N$r;d$OJ9$$$?$3$H$,$"$k!#$,!"$b$75.=E$J:G=i$N#1#0J,$r2<=`Hw$K=<$F$F$$$?$J$i;D$j$NOC$K$bJ9$-<j$O$D$$$F$$$/$3$H$,$G$-$?$@$m$&!#(B

$B!!$3$3$G;d$,!V?dM}>.@b!W$N86M}$H;W$C$F$$$k$H$F$b=EMW$J$b$N$r65$($h$&!#LdBj$J$N$O=gHV$G$"$k!#$b$7:n<T$,:G=i$N>O$KC/$,$d$C$?$+$r65$(!"<j$,$+$j$rM?$($F$7$^$C$?$i?dM}>.@b$O$D$^$i$J$$J*$K$J$C$F$7$^$&$@$m$&!#$7$+$7$3$l$HF1$8;v$r$d$C$F$$$k9V1i<T$,$I$s$J$KB?$$$3$H$+!*LdBj2r7h$N;e8}$r$D$+$_!"$=$N8=>]$K$D$$$F@5$7$/9M$($?;~!"IT2D2r$J8=>]$,$$$+$KL@Gr$J$b$N$K$J$C$F$$$/$N$+$r8+$F$$$/463PE*$J3Z$7$5$rD0=0$KL#$o$C$F$b$i$$$?$$$N$@!#$@$+$i:G=i$OD0=0$r8MOG$o$;$k$h$&$K$9$k$H$$$$!#(B

$B!!9V5A$N:]$K5oL2$j$r$7$,$A$@$H$$$&;v$O$_$s$J$o$+$C$F$$$k$7!";d<+?H5oL2$j$r$7$F2?EYCQ$:$+$7$$;W$$$r$7$?$3$H$+CN$l$J$$!#$I$s$J$K$$$$9V1i<T$G$5$(A4$/L25$$rM6$o$J$$$J$s$F;v$O$"$j$($J$$$,!"L25$$rM6$&$N$r:G>.8B$K?)$$;_$a$F$*$/$?$a$K$G$-$k$3$H$O$?$/$5$s$"$k!#$:$C$HF1$8$h$&$KOC$9$N$OCWL?E*$@!#L25$$rM6$&:EL2=Q$N$h$&$J$b$N$,$"$k$N$@!#4K5^$r$D$1$k$N$,Bg@Z$G$"$k!#$H$j$o$1%8%g!<%/$N8z2L$O:]N)$C$F$$$k$7!"D9;}$A$9$k!#(B

\begin{subsubanswers}
\SubSubAnswer
$B!&Aj0c(B

``talk''$B$OJ*$N8+J}$r9=C[$9$k$?$a$NJ*$G$"$k$N$KBP$7$F!"(B``paper''$B$O>pJs$NC_@Q$rL\E*$H$9$k!#(B

$B!&(Bsuccessful talk$B$H$O(B

$BOC$NFbMF$r<!$NF|$K3P$($F$$$k?M$,?tB?$/$$$k$h$&$J9V5A!#(B

\SubSubAnswer
\begin{itemize}
\item $B$"$k0l$D$N%a%$%s%F!<%^$rEA$($k$h$&$K9=@.$r9M$($k$3$H(B
\item $BOC$N;O$a$N#1#0J,4V$K$3$l$+$iOC$9FbMF$N2<=`Hw$r$9$k$3$H(B
\item $BD0=0$K$OOC$N;O$a$G8MOG$o$;$k$3$H(B
\item $BD0=0$,L2$i$J$$$h$&$KOC$K4K5^$r$D$1!"%8%g!<%/$r8@$&$3$H(B
\end{itemize}

\SubSubAnswer
\begin{itemize}
\item[($B%"(B)]\ $B#1$DL\$N%Q%i%0%i%U$N$_$K3:Ev!#ITE,@Z!#(B
\item[($B%$(B)]\ $BN>J}$N%Q%i%0%i%U$NFbMF$r;}$C$F$$$k$H$$$C$F$b$$$$$+$b$7$l$J$$!#@52r!#(B
\item[($B%&(B)]\ $B#2$DL\$N%Q%i%0%i%U$N$_$K3:Ev!#ITE,@Z!#(B
\item[($B%((B)]\ $B#1$DL\$N%Q%i%0%i%U$N$_$K3:Ev!#ITE,@Z!#(B
\item[($B%*(B)]\ $B$I$A$i$N%Q%i%0%i%U$N<gBj$G$b$J$$!#ITE,@Z!#(B
\end{itemize}

\end{subsubanswers}

\SubAnswer
\begin{subsubanswers}
\baselineskip=12pt
\SubSubAnswer
Comparing the Schr\"{o}dinger equation with the classical energy-momentum equation, we can see that, at least for a free particle, energy and momentum correspond to the differential operators which act on the wave function $\psi$, as follows:
\[
E \to i\hbar \frac{\del}{\del t} \quad,\quad p \to -i\hbar \nabla
\]

\SubSubAnswer
The famous experiment by Michelson and Morley showed that the speed of light is constant for all directions, irrespective of the relative motion among the observer, the transmitting medium,and the source.

\SubSubAnswer
Therefore, the Galilean transformation, under which classical mechanics is invariant, is found to be incorrect and should be replaced by another, under which the speed of light is constant.

\SubSubAnswer
Einstein showed that such a transformation would inevitably change the concept of time and simultaneity which had been believed differently before.
\baselineskip=15pt
\end{subsubanswers}
\end{subanswers}
\end{answer}
\end{document}