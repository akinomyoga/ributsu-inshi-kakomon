\documentclass[fleqn]{jbook}
\usepackage{physpub}

\begin{document}

\begin{question}{教育 英語}{}

\begin{subquestions}
\SubQuestion
次の英文を読み以下の設問に答えよ。
\baselineskip=12pt

 I propose to analyze ``Talking about Science.'' How is it best done? Why is it that a subject presented by A is thrilling account which leaves a deep impression, whereas the very same material presented by B is dull and boring and produces no impression whatever? How should we present our branch of science to fellow scientists who work in quite another field? How can we present science to those who have little or no scientific background, as if often the case with men of high ability who are important in affairs of state? How can we make the non-scientist understand why its study means so much to us, a passion they sometimes find very difficult to understand?

 What is a basic character of a \underline{(ア)``talk''}? I think it can be expressed by saying that its primary object is to create a state of mind, or point of view, not to convey information. I can perhaps illustrate what I mean by dwelling on the vast difference between the spoken and written account. Under the heading ``talk,'' I am not including a course of lectures where students take notes and the lectures follow each other as a composite whole. Nor do I include the ``get together'' of two or three experts in the same line of research, for which no rules are necessary. I am considering the hour's talk to an audience whose interest one has to arouse. The written account can also aim at creating a viewpoint, but its main function is to be a storehouse of information. The argument can be meaty and condensed. It can be packed with tables, graphs, and mathematical equations. This is possible because the reader can always pause and digest it at his leisure, going back over parts which he finds to be difficult. I do not mean to imply that one can be irresponsible in a talk, but one need not cross all the ``t's'' and dot all the ``i's.'' In fact, the talk would be spoiled by attempt to do so.

 A talk is therefore different altogether from a \underline{(イ)``paper.''} To my mind the governing factor which determines its art form is this: The success of the way in which the subject has been presented is measured by the extent to which the average number of the audience remembers it next day.

 This may seem an obvious statement, but if we use this principle as a yardstick to assess a lecture we have listened to, or in planning to a lecture of our own, it creates a very significant viewpoint. The value of a lecture is not to be measured by how much one manages to cram into an hour, how much important information has been referred to, or how completely it covers the ground. It is to be measured by how much a listener can tell a friend about it next morning. If we honestly put this question to ourselves and think how little we can remember of talks we have heard, it gives us a sense of proportion and of values in planning a lecture and makes us realize that what we say will go over the heads of audience if we set our sights too high. I would like now to list what I believe to be some of the considerations which apply in planning a talk.

 For instance, suppose we ask how many points can we hope to ``get over'' in an hour? I think the answer should be ``one.'' If the average member of the audience can remember with interest and enthusiasm one main theme, the lecture has been a great success. I like to compare the composition of a lecture to that of a picture. Of course this is dangerous ground on which to venture, because art experts differ so much among themselves. But in simple terms, is it not held that a picture should have one main center of interest? It may have numerous subsidiary features, but the composition is so cunningly arranged that when the eye falls on these and follows their placing it is subtly led back to the main center of interest and does not fail out of picture frame. A lecture should be like that. There should be one main theme, and all the subsidiary interesting points, experiments, or demonstrations should be such that they remind hearer of the theme. As in a picture, so in a lecture, the force of the impression depends upon a ruthless sacrifice of unnecessary detail. It can be richly endowed with exciting details, but they must be of such a kind that recollection of them inevitably brings the main theme back to mind. In other words, the lecture must compose in the sense of having a pattern because it is this pattern which helps so much to impress it on one's memory.

 A lecture is made or marred in the first 10 minutes. This is the time to establish the foundations, to remind the audience of things they half know already, and to define terms that will be used. Again this seems obvious, but I have listened to so much splendid material lost to the audience because the lecturer failed to realize that it did not know what he was talking about, whereas, if the precious first 10 minutes had been spent on preparation, he would have carried his listeners with him for the rest of the talk.\\

\fbox{\parbox[t]{16.15cm}{ Here a most important principle comes in which I think of as the ``detective story'' principle. It is a matter of order. How dull a detective story would be if the writer told you who did it in the first chapter and then gave you the clues. Yet how many lectures do exactly this. One wishes to give the audience the esthetic pleasure of seeing how puzzling phenomena become crystal clear when one has the clue and thinks about them in the right way. So make sure the audience is first puzzled.

 We all know the tendency to go to sleep in lectures; how often have I felt ashamed at doing so myself. Though the best lecturer can never entirely escape from producing this effect, there is much that can be done to minimize it. A continuous even delivery is fatal. There is something hypnotic about it which induces sleep. Pauses and changes of tempo are essential. Above all, jokes have a marked and enduring effect.
}}
\begin{flushright}
\textit{The Art of Talking about Science,} Lawrence Bragg; Science {\bf 154}(1966)1614 より抜粋
\end{flushright}
\baselineskip=15pt
\begin{subsubquestions}
\SubSubQuestion
筆者は、下線部(ア)のtalkと下線部(イ)のpaperとの本質的な相違は何だといっているのか、また、successful talkとはどのようなものだと言っているのか、簡潔に日本語で説明せよ。

\SubSubQuestion
筆者が、successful talkをする上で、speakerが配慮すべきものとしてあげている事柄を4つ、日本語で箇条書きにして述べよ。

\SubSubQuestion
枠で囲った二つのパラグラフに共通したひとつの副題を付けたい。以下の5つの副題から最も適したものを一つだけ選べ。
\begin{itemize}
\item[(ア)]\ A Detective Story
\item[(イ)]\ The Arousing of Interest
\item[(ウ)]\ Tempo and Jokes
\item[(エ)]\ Giving Clues
\item[(オ)]\ The Best Lecturer
\end{itemize}

\end{subsubquestions}

\SubQuestion
以下の文章を英訳せよ。ただし、必要であれば以下の単語を参考にせよ。

differential operator, simultaneity, transmitting medium, Galilean transformation

\begin{subsubquestions}
\SubSubQuestion
Schr\"{o}dinger方程式と古典的なエネルギー運動量の方程式を比べてみると、少なくとも自由粒子に対しては、エネルギーと運動量が波動関数$\psi$に作用する、以下の微分作用素に対応するものと考えられる。
\[
E \to i\hbar \frac{\del}{\del t} \quad,\quad p \to -i\hbar \nabla
\]

\SubSubQuestion
MichelsonとMorleyの有名な実験は、光速が、観測者、媒質、光源の間の相対運動に無関係にすべての方向に対して同じであることを示した。

\SubSubQuestion
したがって、古典力学を不変に保つガリレイ変換は正しくなく、光速を一定とするような別の変換に置き換えられねばならない。

\SubSubQuestion
Einsteinは、このような変換が必然的に、従来の時間と同時性の概念を変更してしまうことを明らかにした。

\end{subsubquestions}

\end{subquestions}

\end{question}
\begin{answer}{教育 英語}{}
\begin{subanswers}
\SubAnswer
  {\bf 全訳}

 ``Talking about Science''について分析してみよう。どうやったら一番うまくできるのだろうか。Aによって発表された題材は深い印象を残すスリリングなものであるのに対し、同じものがBによって発表されると退屈で全く印象を残さないのはなぜだろうか。全く別の分野の科学者に自分の分野のことについて発表するにはどうしたらよいのだろうか。政治家によくいるような、科学の知識がほとんど、あるいは全くない人々に科学について話すにはどうしたらできるだろうか。科学者以外の人達にその研究が重要な意味を持っている理由、すなわち彼らにとっては時に非常に理解しがたいものである情熱をどのようにしたらわかってもらえるだろうか。

 ``Talk''の基本的な特徴は何であろうか。それは``Talk''の第一の目的は情報を伝えることではなく、心の状態すなわち考え方を築くことである、と言い表されると思う。私の言いたい事はおそらく言葉による説明と活字による説明との大きな違いをよく考えてみることで説明できるだろう。見出しの``Talk''には、学生がノートを取ったり一つ一つの講義がお互いにフォローしてできあがるような講座は含まれない。また、同じ研究分野の2、3人の専門家の、何ら規則も必要としないような集まりも含まない。私が言っているのは限られた時間内で聴衆の興味を引き起こさなければならない話のことである。活字による説明も物の見方を築くこともできるが、その主な目的は情報の蓄積である。その議論は内容が豊かで濃い。表やグラフ、数式を含みうる。これは読者がいつも都合の良い時に小休止してまとめ、難しいと思った部分に立ち戻ることができるからこそ可能なのである。Talkの際に無責任であって良いといっているのではなく、細かいところにまで気を使う必要はないといっているのである。実際、そんなことをしようとしたらTalkは台無しになってしまうだろう。

 それゆえTalkは「論文」とは全く異なるものである。私の考えではその技術の形式を決定する主な要因は次のようなものである。「題材を発表した方法の成果は次の日にそれを覚えている聴衆の平均数の程度で測られる。」

 これは当たり前の意見のように思えるかもしれないが、もし自分が聞いた講義を評価する基準としてこの原理を使ったり、あるいは自分自身の講義の計画を立てる際に使えばとても重要な観点が生み出されるのである。講義の価値は1時間にどれだけの量を詰め込むのかにでも、どれだけ多くの重要な情報に触れたかにでも、基礎をどれだけしっかり押さえたかにでも測られるのではない。翌朝にどれだけの聞き手が友達にそれについて話せるかで測られるのである。もし我々科学者がこの問題を正直に受け止め、自分が聞いた講義をどれほど思い出すことができないのかということを考えれば、我々は講義の計画を立てる際に判断力と価値観が備わるし、目標を高くしすぎると我々の言っていることが聴衆の理解を超えてしまうことに気付く。ここで、Talkを準備する際に利用する配慮であると私が思うもののうちいくつかを挙げたいと思う。

 例えば、1時間にどれだけのポイントを理解させることができるか尋ねてみよう。答えは「一つ」だと私は思う。もし聴衆の中の平均的な人が強い興味を持って一つのメインテーマを覚えているなら、その講義は大成功である。講義と写真の構成を比べてみよう。もちろんこれはやるには危険な試みである。なぜなら芸術家には非常に多種多様な人たちがいるからである。しかし、簡単に言ってしまうと絵画には興味の核となるものが一つあるとは考えられないだろうか。絵画には多くの副次的な特徴があるが、その構成は非常に巧妙に仕組まれているので、その特徴に目をやり、配置を追っていくとうまく興味の核に引き戻され絵の枠からはみ出ないのである。講義もそのようであるべきである。一つのメインテーマがあり、副次的な興味、実験、実証というのはすべて聞き手にテーマを思い出させるようなものであるべきなのである。絵画のように講義でも、印象に残るかどうかは不必要な細かい点をバッサリと切り落とせるかどうかにかかっているのである。講義にはもともと刺激的な細部が豊富にあるかもしれないが、その細部はそれらを思い出していくと必然的にメインテーマを思い起こしてしまうようなものでなくてはならない。言い換えれば講義はあるパターンを持つという意味で構成されていなければならない。なぜならこのパターンで人の記憶に印象を残すことができるからである。

 講義は最初の10分で成否が決まってしまう。これは話の土台を作り、聴衆に彼らがすでに半分知っていることを思い出させ、使われる用語を定義するための時間である。やはりこれも明らかなことに思えるが、素晴らしい題材であっても講演者が自分の話を聴衆はわかっていない事を理解できていないために、聴衆が取り残されてしまっているものを私は聞いたことがある。が、もし貴重な最初の10分を下準備に充てていたなら残りの話にも聞き手はついていくことができただろう。

 ここで私が「推理小説」の原理と思っているとても重要なものを教えよう。問題なのは順番である。もし作者が最初の章に誰がやったかを教え、手がかりを与えてしまったら推理小説はつまらない物になってしまうだろう。しかしこれと同じ事をやっている講演者がどんなに多いことか!問題解決の糸口をつかみ、その現象について正しく考えた時、不可解な現象がいかに明白なものになっていくのかを見ていく感覚的な楽しさを聴衆に味わってもらいたいのだ。だから最初は聴衆を戸惑わせるようにするといい。

 講義の際に居眠りをしがちだという事はみんなわかっているし、私自身居眠りをして何度恥ずかしい思いをしたことか知れない。どんなにいい講演者でさえ全く眠気を誘わないなんて事はありえないが、眠気を誘うのを最小限に食い止めておくためにできることはたくさんある。ずっと同じように話すのは致命的だ。眠気を誘う催眠術のようなものがあるのだ。緩急をつけるのが大切である。とりわけジョークの効果は際立っているし、長持ちする。

\begin{subsubanswers}
\SubSubAnswer
・相違

``talk''は物の見方を構築するための物であるのに対して、``paper''は情報の蓄積を目的とする。

・successful talkとは

話の内容を次の日に覚えている人が数多くいるような講義。

\SubSubAnswer
\begin{itemize}
\item ある一つのメインテーマを伝えるように構成を考えること
\item 話の始めの10分間にこれから話す内容の下準備をすること
\item 聴衆には話の始めで戸惑わせること
\item 聴衆が眠らないように話に緩急をつけ、ジョークを言うこと
\end{itemize}

\SubSubAnswer
\begin{itemize}
\item[(ア)]\ 1つ目のパラグラフのみに該当。不適切。
\item[(イ)]\ 両方のパラグラフの内容を持っているといってもいいかもしれない。正解。
\item[(ウ)]\ 2つ目のパラグラフのみに該当。不適切。
\item[(エ)]\ 1つ目のパラグラフのみに該当。不適切。
\item[(オ)]\ どちらのパラグラフの主題でもない。不適切。
\end{itemize}

\end{subsubanswers}

\SubAnswer
\begin{subsubanswers}
\baselineskip=12pt
\SubSubAnswer
Comparing the Schr\"{o}dinger equation with the classical energy-momentum equation, we can see that, at least for a free particle, energy and momentum correspond to the differential operators which act on the wave function $\psi$, as follows:
\[
E \to i\hbar \frac{\del}{\del t} \quad,\quad p \to -i\hbar \nabla
\]

\SubSubAnswer
The famous experiment by Michelson and Morley showed that the speed of light is constant for all directions, irrespective of the relative motion among the observer, the transmitting medium,and the source.

\SubSubAnswer
Therefore, the Galilean transformation, under which classical mechanics is invariant, is found to be incorrect and should be replaced by another, under which the speed of light is constant.

\SubSubAnswer
Einstein showed that such a transformation would inevitably change the concept of time and simultaneity which had been believed differently before.
\baselineskip=15pt
\end{subsubanswers}
\end{subanswers}
\end{answer}
\end{document}