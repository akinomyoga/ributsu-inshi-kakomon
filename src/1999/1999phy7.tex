\documentclass[fleqn]{jbook}
\usepackage{physpub}
\begin{document}

\begin{question}{専攻 問題7}{石井}
タンパク質の高次構造決定法として、(A)電子顕微鏡法、(B)円偏光二色性(CD)、(C)X線結晶回折、(D)核磁気共鳴法(NMR)、(E)中性子散乱法などがある。

これらの方法のうちで、CD法以外では、タンパク質を構成する各原子の座標を決定できる。

\begin{subquestions}
\SubQuestion
CD法で決定できる高次構造は何か、また、それはタンパク質のどのような特徴を利用しているかを述べよ。また、この方法の長所は何か。

\SubQuestion
電子顕微鏡では電子を100kV程度の電場で加速して用いる。

\begin{subsubquestions}
\SubSubQuestion
この電子の波長($\lambda$)を非相対論的に求めよ(有効数字1桁で良い)。但し、電子質量($m_e=9.1\times 10^{-31}$ kg)、電子電荷($-e=-1.6 \times 10^{-19}$ C)、プランク定数($h=6.6\times 10^{-34}$ J$\cdot$s)を用いよ。

\SubSubQuestion
電子顕微鏡の分解能は0.1nm程度であり、上で求めた電子の波長と桁が違う。光学顕微鏡で得られる分解能は200nm程度であり、可視光の波長($400\sim 700$nm)と同程度である。この理由を考察せよ。
\end{subsubquestions}

\SubQuestion
X線結晶解析ではタンパク質分子の単結晶を作製し、その回折強度を計測する。

\begin{subsubquestions}
\SubSubQuestion
X線回折強度がタンパク質分子の原子構造(原子の種類及び配置)を反映するのはなぜか、簡潔に述べよ。

\SubSubQuestion
回折強度の計測だけでは原子構造を解くことが出来ない場合が大半である。それは何故か。

\SubSubQuestion
どのような方法でこの困難を解決するか、概略を述べよ。
\end{subsubquestions}

\SubQuestion
核磁気共鳴法では重水中のタンパク質のスペクトルを観察し、タンパク質を構成する水素原子核のそれぞれを区別して同定・帰属付けを行う。

\begin{subsubquestions}
\SubSubQuestion
核磁気共鳴法は、一様磁場$\vec{B}$中に試料を置き、磁気能率が$\vec{B}$のまわりで歳差運動する周波数を、マイクロ波共鳴によって測定する手法である。$B=1$Tの磁場中での電子スピンの共鳴周波数が約28GHzであることを既知として、同じ磁場中での水素原子核(陽子)の共鳴周波数を概算せよ。なお、電子と陽子の$g$因子は各々約2及び5.6である。

\SubSubQuestion
重水を溶媒として用いるのは何故か。

\SubSubQuestion
タンパク質の異なる位置にある水素原子のスペクトルが異なるのは何故か。

\SubSubQuestion
この方法で決定できる分子量の上限(約20,000)を決定している要素は何かを考察せよ。
\end{subsubquestions}

\SubQuestion
タンパク質と核酸の複合体を中性子線(波長0.1nm程度)を用いて構造研究すると、X線、電子線より有利な場合がある。どんな場合か、中性子の特徴と関連しつつ述べよ。

\end{subquestions}
\end{question}
\begin{answer}{専攻 問題7}{石井}
\begin{subanswers}
\SubAnswer
CD法で決定できる高次構造は2次構造、特に$\alpha$ヘリックスである。これは左及び右円偏光が光学活性な分子では異なった相互作用をするという性質による。CD法の利点は他の方法に比べて容易に$\alpha$ヘリックスの含有量を見積もることができるということである。また、低濃度のタンパク質溶液で測定可能であり、測定時間も短くてすみ容易に実験できることなども挙げられる。

\SubAnswer
\begin{subsubanswers}
\SubSubAnswer
電子線のエネルギーと運動量には
\begin{align*}
\frac{1}{2}m_{e}v^{2} &= eV  \\
m_{e}v &= \frac{h}{\lambda} 
\end{align*}
という関係があり、これより、波長$\lambda$を求めると
\[ \lambda = \frac{h}{m_e v}=\frac{h}{\sqrt{2m_e eV}}= 3.86 \cdots \times 10^{-12}~\text{[m]}  \]
となり、電子の波長は4[pm]程となる。

\SubSubAnswer
光学顕微鏡では凹凸レンズの組み合わせで色収差、球面収差を除くことができる。よって開口数で決まる回折収差のみが問題となるので波長と同程度の分解能が得られる。
電子顕微鏡では、電磁レンズとして凸レンズしか作れないために色収差、球面収差を除くことができない。それゆえ、球面収差を適当な範囲内におさめるために開口数を小さくする必要があり、解像度が回折収差で制限されてしまうからである。また、入射粒子と原子との散乱の多くがエネルギーのやりとりがある非弾性散乱であること、電子と物質との相互作用から化学反応を起こし、ラジカルを形成してしまうために照射する電子数が限られてしまい雑音が多くなることなども挙げられる。
\end{subsubanswers}

\SubAnswer
\begin{subsubanswers}
\SubSubAnswer
X線は、殻外電子と相互作用を起こし散乱するため、得られた回折像はタンパク質原子の電子密度を変換したものになっているために原子構造を反映している。

\SubSubAnswer
位相問題があるためである。反射$\vec{h}$の回折強度$I(\vec{h})$と結晶構造因子$F(\vec{h})$との間には
\[  I(\vec{h})=|F(\vec{h})|^{2}  \]
の関係がある。一方、結晶中の電子密度は
\[   \rho (\vec{r})= \sum \frac{1}{V} \sum_{\vec{h}} F(\vec{h}) \exp(- 2 \pi i \vec{h} \cdot \vec{r})  \]
で表される。一般に構造因子$F(\vec{h})$は複素量であるが、回折強度からわかるのはその振幅で、結晶の電子状態を知るには位相を決める必要がある。

\SubSubAnswer
多重同型置換法または、多波長異常散乱法を用いればよい。

多重同型置換法とはもとの結晶とは別に重原子を付加した重原子同型置換体を複数個つくり、生じる強度変化から位相を決定する方法である。

多波長異常散乱法とは、分子内にX線の波長に依存して原子散乱因子が変化する原子を含むときに行われ、波長を変えたときの強度変化から位相を決定する方法である。
\end{subsubanswers}

\SubAnswer
\begin{subsubanswers}
\SubSubAnswer
共鳴周波数$\omega_0$と因子$g$との関係は
\[   \omega_0 = g H  \]
であるから、陽子の共鳴周波数$\omega_{0p}$は、電子の共鳴周波数を$\omega_{0e}$とすると
\[  \omega_{0p} = \omega_{0e} \frac{g_{p}}{g_{e}}\fallingdotseq 78~\text{[GHz]}  \]
となり、約78[GHz]程である。($g_{p}$, $g_{e}$はそれぞれ陽子、電子の因子)

\SubSubAnswer
水分子中の水素原子Hの核スピンが、核磁気共鳴に用いられる電磁波に応答してしまうため。

\SubSubAnswer
プロトンがおかれた周囲の環境により、環電流効果などの影響で高磁場側にスペクトルがシフトする化学シフトが起こることや近傍プロトンの影響によるspin-spin相互作用でspin-spin分裂をおこすことによりスペクトルが変化する。
\SubSubAnswer 分子量が多くなるとスペクトル量も多くなり、さらに、鈍いピークの存在によりスペクトルが隠れてしまいスペクトルが分離できなくなってしまうことによる。
\end{subsubanswers}

\SubAnswer
中性子線は、X線のように殻外電子と相互作用するのではなく、原子核と強く相互作用するために水素分子がはっきり見え、かつ窒素分子を炭素・酸素分子と区別できる。タンパク質・核酸複合体は水分子を介して結合をつくっていることが多く、水分子がよく見える中性子線には利点がある。また、吸収断面積が小さいために物質透過力も強く、試料損傷が少ないこと、散乱が非弾性散乱であるためにエネルギー変化から散乱体の動力学的性質を知ることができるなどの利点もある。
\end{subanswers}
\end{answer}



\end{document}
