%% -*- coding:japanese-shift-jis-2004 -*-
\documentclass[fleqn]{jbook}
\usepackage{physpub}
\usepackage{bm}
\usepackage[dvips]{graphicx}
\usepackage{booktabs}
\usepackage{float}
\usepackage{amsmath,amssymb}

\begin{document}

\begin{question}{第1問}{山﨑雅人}
\begin{enumerate}
\item
  $n$個の$n$次元列ベクトル$u_j(j=1,\ldots, n)$を用いて、行列$U$を$U=(u_1,u_2,\ldots, u_n)$で定義するとき、$U$がユニタリー行列ならば、$\{u_j\}$は正規直交系をなすことをを示せ。

\item
  $A$を実対称行列、$\bm{x}=(x_1,\ldots,x_n)^{T}$を$n$次元ベクトルとする。
  ただし、$T$は転置を表す。このとき実2次形式$\phi(\bm{x})=\bm{x}^T A \bm{x}$に対し適当な直交変換$\bm{x}=P\bm{y}$を行うと、
  対角行列$B$を用いて$\phi=\bm{y}^T B \bm{y}=\sum_{i=1}^n \lambda_i y_i^2$の標準形に変換できる。\\
  $\phi(\bm{x})=4 x_1^x+2 x_2^2+2 x_3^2-2 x_1 x_2+2 x_2 x_3-2 x_3 x_1$について、
  $A,P,\lambda_i (i=1,2,3)$を求め標準形で表せ。

\item\ilabel{2004mathQ1.3}
  $A$を$n$次実対称行列として、次の微分方程式を考える。
  $$\frac{d\bm{x}}{dt}=A\bm{x}$$
  このとき、スカラー関数$\Phi(\bm{x})$を用いて$\frac{d\bm{x}}{dt}=-\Delta_x \Phi(\bm{x})$と書けることを$\Phi(\bm{x})$の具体形とともに示せ。
  ただし、$\frac{d\bm{x}}{dt}=(\frac{dx_1}{dt},\ldots,\frac{dx_n}{dt})^T, \Delta_x \Phi=(\frac{\partial \Phi}{\partial x_1},\ldots,\frac{\partial \Phi}{\partial x_n})^T$
  である。

\item
  設問\iref{2004mathQ1.3}で軌道$\bm{x}(t)$に沿った$\Phi$の微分$d\Phi(\bm{x}(t))/dt$は、
  $\frac{d\Phi(\bm{x}(t))}{dt}\le 0$を満たすことを示せ。
  また、任意の$\bm{x} (\bm{x}\ne 0$に対し$\Phi >0$が成り立つならば、
  解軌道は最終的に$\bm{x}=0$に漸近することを示せ。
\end{enumerate}
\end{question}

\begin{answer}{第1問}{宮川裕}
\begin{enumerate}
\item
  \begin{eqnarray*}
  U=\begin{pmatrix} \vec{u}_1 & \vec{u}_2 & \cdots & \vec{u}_n\end{pmatrix}と書くと、
  U^{*}= \begin{pmatrix} \vec{u}_1^* \\ \vec{u}_2^* \\ \vdots \\ \vec{u}_n^* \end{pmatrix}である。\\
  \end{eqnarray*}
  $U$ がユニタリー行列であるから、
  \begin{eqnarray*}
  U^{*}U=\begin{pmatrix} \vec{u}_1^* \cdot \vec{u}_1 &
                          \vec{u}_1^* \cdot \vec{u}_2 & \cdots \\
                          \vec{u}_2^* \cdot \vec{u}_1 &
                          \vec{u}_2^* \cdot \vec{u}_2 & \cdots \\
                          \vdots \end{pmatrix}=E
  \end{eqnarray*}
  各成分を見れば、題意は証明された。

\item\ilabel{2004mathA1.2}
  \begin{eqnarray*}
    \phi =\vec{x}^T A \vec{x} = \vec{y}^T P^T A P \vec{y}\\
  \end{eqnarray*}
  適当な直交行列$P$を用いて、$A$を対角化する。
  固有ベクトルが互いに直交することを利用する。
  まず$A$の固有値、固有ベクトルを求め、それを使って基底の変換をする。
  \begin{eqnarray*}
    A=\begin{pmatrix}
      4 & -1 & -1 \\
      -1 & 2  & 1  \\
      -1 & 1  & 2
    \end{pmatrix}から固有値を求めると、\\
    固有値\lambda =1に対して、固有ベクトル\begin{pmatrix} 0 \\ 1 \\ -1 \end{pmatrix}\\
    固有値\lambda =2に対して、固有ベクトル\begin{pmatrix} 1 \\ 1 \\  1 \end{pmatrix}\\
    固有値\lambda =5に対して、固有ベクトル\begin{pmatrix} -2 \\ 1 \\  1 \end{pmatrix}\\
  \end{eqnarray*}
  である。$P$を求めるには、$P$が直交変換であることに注意して、$P$の各列をノルム1の固有ベクトルにする。
  よって
  \begin{eqnarray*}
    P=\begin{pmatrix}
      0 & \frac{1}{\sqrt3} & \frac{2}{\sqrt6} \\
      \frac{1}{\sqrt2} & \frac{1}{\sqrt3}  & -\frac{1}{\sqrt6}  \\
      -\frac{1}{\sqrt2} & \frac{1}{\sqrt3}  & -\frac{1}{\sqrt6}
    \end{pmatrix}
  \end{eqnarray*}

\item
  $n=3$ くらいで実験するのが、わかりやすいかと思われる。\\
  $A$の成分を$a_{ij}$のように書くことにする。\\
  成分で書くと、
  \begin{align*}
    x_1&= a_{11} x_1 + a_{12} x_2 + a_{13} x_3 \\
    x_2&= a_{21} x_1 + a_{22} x_2 + a_{23} x_3 \\
    x_3&= a_{31} x_1 + a_{32} x_2 + a_{33} x_3 
  \end{align*}
  積分して、
  \begin{align*}
    x_1&= a_{11} \frac{x_1^2}{2} + a_{12} x_1 x_2 + a_{13} x_1 x_3 + f(x_2,x_3) \\
    x_2&= a_{21} x_1 x_2 + a_{22} \frac{x_2^2}{2} + a_{23} x_1 x_3 + f(x_1,x_3) \\
    x_3&= a_{31} x_1 x_3 + a_{32} x_2 x_3 + a_{33} \frac{x_3^2}{2} + f(x_1,x_2) \\
  \end{align*}
  ただし$f$は任意の関数でよい。$A$は対角行列であることに注意すると、次のようにすればよいことがわかる。
  \begin{align*}
    \Psi(x)
      &= -\left( \sum_{i=1}^{n} a_{ii}\frac{x_i^2}{2} + \sum_{i\neq j}^{n}a_{ij}\frac{x_i x_j}{2}\right) \\
      &= -\sum_{i,j}^{n} a_{ij}\frac{x_i x_j}{2}
  \end{align*}

\item
  \[
    \frac{dx}{dt}= A\vec{x}= -\nabla _x \Psi(\vec{x})
  \]
  から直接示そうとすると、成分計算がごちゃごちゃになって、ちょっと難しい。\\
  ここは設問\iref{2004mathA1.2}の結果を使って、基底の変換をする。
  \[
    \vec{x}\to P\vec{y}
  \]
  とすると、
  \[
    P^{-1} \nabla_x \to \nabla_y
  \]
  であり、$\Psi$ がスカラーなので変換を受けないことに注意すると、
  \[
    \frac{dy}{dt}= P^{-1}A P y = -\nabla_y \Psi(\vec{y})
  \]
  となり、$\Psi$ を対角化した、
  \begin{equation}
    \Psi = -\sum_{i}^{n} \lambda _i \frac{y_i^2}{2} \ilabel{eq:2004mathQ1.Psi1}
  \end{equation}
  だけで考えればいいことになる。こうしておけば、直接計算して、
  \begin{equation}
    \frac{d\Psi(y(t))}{dt}= -\sum_{i}^{n}  \lambda _i y_i \frac{dy_i}{dt}\\
    = + \sum_{i}^{n} \lambda _i y_i \frac{\partial \Psi(y)}{\partial y_i}\\
    =- \sum_{i}^{n} \lambda _i^2 y_i^2  \le 0  \ilabel{eq:2004mathQ1.Psi2}
  \end{equation}
  で問題の前半は終わった。\\

  また、$任意の\vec{x}(x \neq 0)に対して\Psi>0$が成り立つとき、\ieqref{eq:2004mathQ1.Psi1}式から$\lambda$は負。
  すると\ieqref{eq:2004mathQ1.Psi1}式の言っていることは、$y$が中心$y=0$から離れるほど、
  二次関数的に単調にポテンシャルが増加するということである。
  式\ieqref{eq:2004mathQ1.Psi2}の意味は、「解曲線に沿ったポテンシャルの時間変化は、マイナス」つまり、
  「ポテンシャルの低いほうへ$y$は移動」であるから、解軌道は最もポテンシャルの低い$y=0$へ漸近する。
\end{enumerate}
\end{answer} 

\begin{question}{第2問}{山﨑雅人}
$f(x,t)$についての
偏微分方程式
\begin{equation}
  \frac{\partial f}{\partial t} = \frac{\partial}{\partial x} (xf) + D \frac{\partial ^2 f}{\partial x^2} \ilabel{c}
\end{equation}
を考える。

$f$および$\partial f/\partial x$は、$x \rightarrow \pm \infty$に対して十分速やかに$0$に収束する. また、$D$は正の定数とする。

\begin{enumerate}
\item
  $x$についての$f$の定積分
  $$
  I=\int_{-\infty}^{\infty}f(x,t)dx
  $$
  が保存されること、すなわち$dI/dt=0$であることを示せ。
\item
  原点$x=0$を中心とするガウス分布関数
  $$
  g(x,\sigma)=\frac{1}{\sqrt{\mathstrut{2\pi}}\sigma}e^{-x^2/(2\sigma^2)} ~~ (\sigma\text{は正の定数})
  $$
  が、はじめの偏微分方程式の定常解(すなわち、$\partial f/\partial f=0$)であるとき、$\sigma$はどう表せるか。
\item
  中心$X(t)$、標準偏差$\sigma(t)$($X$および$\sigma$は$t$の関数)のガウス分布関数
  $$
  f(x,t)=g\left(x-X(t),\sigma(t)\right)
  $$
  が、はじめの偏微分方程式の解になるために、$X(t)$および$\sigma(t)$の満たすべき常微分方程式を求めよ。
\item
  はじめの偏微分方程式を、初期条件$f(x,0)=\delta(x-1)$のもとで解き、$t\to\infty$において設問(2)で求めた定常解に近づくことを示せ。
  (デルタ関数$\delta(x-1)$は、$g(x-1,\sigma)$の$\sigma\to 0$の極限であることを考慮せよ。)
\end{enumerate}
\end{question}

\begin{answer}{第2問}{宮川裕}
偏微分方程式
\begin{equation}
  \frac{\partial f}{\partial t} = \frac{\partial}{\partial x} (xf) + D \frac{\partial ^2 f}{\partial x^2} \ilabel{c2}
\end{equation}
を考える.

ただし、条件として
\begin{itemize}
\item{$f,\partial f/\partial x$が, $x \rightarrow \pm \infty$に対して十分速やかに$0$に収束する. }
\item{Dは正の定数}
\end{itemize}
が与えられている.

\begin{enumerate}
\item
  \begin{equation}
  I=\int^\infty_{-\infty}f(x,t)dx
  \end{equation}
  として, $dI/dt=0$を示す. 

  \begin{align}
    \frac{dI}{dt} &= \int^\infty_{-\infty} \frac{\partial f}{\partial t}dx \notag \\
      &= \int^\infty_{-\infty} \left( \frac{\partial}{\partial x} (xf) + D \frac{\partial ^2 f}{\partial x^2} \right) dx \notag \\
      &= \left[ xf + D \frac{\partial f}{\partial x} \right]^\infty_{-\infty} \ilabel{a}
  \end{align}

  条件より, 
  \begin{gather}
    xf \rightarrow 0 ( x \rightarrow \pm \infty ) \\
    x \frac{\partial f}{\partial x} \rightarrow 0 ( x \rightarrow \pm \infty ) 
  \end{gather}
  であるから, \ieqref{a}に代入して, $dI/dt=0$が云える. $_\Box$

\item\ilabel{2004mathA2.2}
  \begin{equation}
    g(x,\sigma)=\frac{1}{\sqrt{2\pi} \sigma} e^{-x^2/(2\sigma^2)} \ilabel{b}
  \end{equation}
  に対して, 
  \begin{equation}
    \frac{\partial}{\partial x} (xg) + D \frac{\partial ^2 g}{\partial x^2} =0
  \end{equation}
  であるための$\sigma$の条件を求める. 

  計算すると, 
  \begin{gather}
    \frac{\partial g}{\partial x} = -\frac{x}{\sigma^2}g \\
    \frac{\partial^2 g}{\partial x^2} = -\frac{g}{\sigma^2}+\frac{x^2}{\sigma^4}g
  \end{gather}
  であることから, \ieqref{b}にこれらを代入すると, 
  \begin{align}
    g-\frac{x^2}{\sigma^2}g+D\left[\frac{x^2}{\sigma^4}g-\frac{g}{\sigma^2}\right] &=0 \notag \\
    \Longleftrightarrow  \left(1-\frac{D}{\sigma^2} \right) \left(1-\frac{x^2}{\sigma^2} \right) g &= 0 
  \end{align}
  これが$x$に関わらず成立するので, $1-D/\sigma^2=0$が云える. 
  \begin{align}
    \therefore\quad & \sigma = \sqrt{D} \ _\Box
  \end{align}

\item
  ガウス分布関数
  \begin{equation}
    f(x,t)=g(x-X(t),\sigma(t)) \ilabel{d}
  \end{equation}
  が\ieqref{c2}の解になるための条件を考える. 

  \ieqref{d}の$f$の微分は
  \begin{gather}
    \frac{\partial f}{\partial t} = \left[\frac{1}{\sigma(t)^3}\left\{(X(t)-x)^2\frac{d \sigma}{dt}-\sigma(t)(X(t)-x)\frac{dX}{dt} \right\}
    -\frac{1}{\sigma(t)}\frac{d\sigma}{dt} \right]g(x-X(t),\sigma(t)) \\
    \frac{\partial f}{\partial x} = - \frac{x-X(t)}{\sigma(t)^2}g(x-X(t),\sigma(t)) \\
    \frac{\partial^2 f}{\partial x^2} = \left[\frac{(x-X(t))^2}{\sigma(t)^4}-\frac{1}{\sigma(t)^2}\right]g(x-X(t),\sigma(t))
  \end{gather} 
  となるので, これを\ieqref{c2}に代入して, 
  \begin{gather}
    \left(\frac{1}{\sigma(t)^3}\frac{d\sigma}{dt}\right)x^2+\left(\frac{1}{\sigma(t)^2}\frac{dX}{dt}-\frac{2X(t)}{\sigma(t)^3}\frac{d\sigma}{dt}\right)x
    +\left\{\left(\frac{X(t)^2}{\sigma(t)^3}-\frac{1}{\sigma(t)}\right)\frac{d\sigma}{dt}-\frac{x(t)}{\sigma(t)^2}\frac{dX}{dt}\right\} \notag \\
    =\left(\frac{D}{\sigma(t)^4}-\frac{1}{\sigma(t)^2}\right)x^2+\left(\frac{X(t)}{\sigma(t)^2}-\frac{2DX(t)}{\sigma(t)^4}\right)x
    +\left\{1+D\left(\frac{X(t)^2}{\sigma(t)^4}-\frac{1}{\sigma(t)^2}\right)\right\}
  \end{gather}
  を得るが, これが$x$によらずに成り立つための条件は各係数が等しいことであるので, 結果として, 
  \begin{gather}
    \frac{d\sigma}{dt} = \frac{D}{\sigma}-\sigma \ilabel{e} \\
    \frac{dX}{dt}=-X \ilabel{f}
  \end{gather}
  を得る. $_\Box$

\item
  常微分方程式\ieqref{e}, \ieqref{f}を初期条件$\sigma(t)=0,X(t)=1$の下で解けばよい. 

  常微分方程式\ieqref{e}の一般解は, 
  \begin{equation}
    \sigma(t)=\sqrt{D \pm \exp \{-2(t-C)\}} \ \text{($C$は任意定数)}
  \end{equation}
  であるから, 初期条件より, 
  \begin{equation}
    \sigma(t)=\sqrt{D(1-\exp(-2t))}
  \end{equation}
  が解である. 

  同様にして, 
  \begin{equation}
    X(t)=\exp(-t)
  \end{equation}
  がわかる. 

  したがって, $x \rightarrow \infty$で, $\sigma \rightarrow \sqrt{D},X \rightarrow 0$であるから, 
  確かに解は, $g(x,\sqrt{D})$, すなわち設問\iref{2004mathA2.2} でもとめた定常解に近づくことになる. $_\Box$
\end{enumerate}
\end{answer}

\end{document}
