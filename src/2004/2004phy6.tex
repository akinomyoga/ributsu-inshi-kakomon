\documentclass[fleqn]{jbook}
\usepackage{amsmath}
\usepackage{txfonts}
\usepackage[T1]{fontenc}
\begin{document}

\begin{question}{第6問}{}
実験においては、熱力学的な孤立系を取り扱うことは極めてまれである。例えば、温度を一定に保つために恒温槽を用いたり、圧力を一定に保つために大気圧下で実験を行う。このような系では、外界と系との間で熱の出入りや仕事のやりとりがあり、実験に都合のよい熱力学量を用いることが必要である。その一つがギブズ自由エネルギー($G$)であり、
$$
G=H-TS
$$
で与えられる。ここで、$H,T,S$は、それぞれ、エンタルピー、絶対温度、エントロピーである。以下の設問に答えよ。ただし、$H=E+pV$である。($E,p,V$は、それぞれ、系の内部エネルギー、圧力、体積である)。

\begin{enumerate}
\item 可逆過程において、外界より系に流入する微少な熱量$\delta Q$とエントロピーの微小変化$dS$との間に成り立つ関係式を示せ。

\item 可逆過程において、$G$の全微分$dG$が、
$$
dG=Vdp-SdT
$$
で与えられることを、熱力学の第1法則から出発して示せ。

\item
一般に、以下の関係式が成立することを証明せよ。

$$
S-\left(\frac{\partial G}{\partial T}\right)_p
$$
$$
H=-T^2\left(\frac{\partial (G/T)}{\partial T}\right)_p
$$

\item
次に、大気圧下で温度上昇によりもたらされる。鎖状分子の立体構造変化(熱転移)について考えてみよう。この鎖状分子(例えば、蛋白質などの生体高分子を考える)は$N$このユニットが直鎖状につながったものであり、低温では唯一の立体構造を取るが、温度を上昇させると、ある温度で協同的に熱転移してランダム構造に変わる。ランダム構造では、1ユニットあたり$n$通りの状態を自由に取ることができるとかていすると、一つの鎖状分子あたりの可能な状態数は$n^N$通りあることになる。$N=100$かつ$n=10$のとき、低温構造から低温ランダム構造への変化に伴って起こる、分子1モルあたりのエントロピーの変化$\Delta S_C$を求めよ。必要であれば、$\ln 10=2.3$、気体定数$R=8.3 \text{J} \text{K}^{-1}\text{mol}~{-1}$を使ってよい。

\item
設問4における熱転移は一次相転移として扱うことができる。転移に伴うエントロピー変化が設問4の$\Delta S_C$のみであると仮定すると、熱転移温度が90℃の時、熱転移に伴うエンタルピー変化はいくらか。
\end{enumerate}


\end{question}


\begin{answer}{第6問}{}



1.
 \begin{equation}
    \delta Q = T ds
 \end{equation}


2.
  $G = H - TS$、$H = U + pV$、第一法則$dU = \delta Q - pdV$より、
 \begin{align}
    dG &= dU + Vdp + pdV - TdS - SdT \\ 
         &= \delta Q + Vdp - Tds - SdT \\
         &= Vdp - SdT
 \end{align}


3.(2)式より偏微分して、
 \begin{equation}
    -( \frac{\partial G}{\partial T} )_p = S
 \end{equation}
次に、$G/T$を$T$で偏微分すると、
 \begin{align}
    (\frac{\partial (G /T)}{\partial T})_p &= -\frac{G}{T^2} + \frac{1}{T}( \frac{\partial G}{\partial T} )_p \\
                                           &= -\frac{G}{T^2} - \frac{S}{T} 
 \end{align}
となるので、
 \begin{equation}
    -T^2 (\frac{\partial (G /T)}{\partial T})_p = G + ST = H  
 \end{equation}
となる。



4.エントロピーは状態数$W$を用いて、$S = k \ln W$と表せるので、エントロピー変化$\triangle S_C$は、
 \begin{align}
    \triangle S_C &= N_A(k \ln 10^{100} - k \ln 1 = 100 k \ln 10 ) \\
                    &= R \cdot 100\ln 10 \\
                    &\simeq 8.3 \cdot 100\ln 10 \\
                    &= 1.909 \times 10^3 \\
                    &\simeq 1.9 \times 10^3 [J/K]\\ 
 \end{align}
となる。


5.
 \begin{equation}
    dH = dG + TdS + SdT = Vdp + TdS
 \end{equation}
圧力は変化しないそうなので、
 \begin{equation}
    \triangle H=T \triangle S_C = 90 \times 1909 = 171810 \simeq 1.7 \times 10^5 [J] 
 \end{equation}
となる。

\end{answer}
\end{document}