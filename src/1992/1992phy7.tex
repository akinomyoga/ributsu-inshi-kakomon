\documentclass[fleqn]{jbook}
\usepackage{physpub}

\begin{document}

\begin{question}{専攻 問題7}{}

\begin{subquestions}
\SubQuestion
タンパク質の階層構造を示すのに、一次構造、二次構造、超二次構造、三次構造、
四次構造という述語が使われている。それぞれの述語がタンパク質のどのようなレベル
の構造をしめすかを簡単に述べよ。(一述語あたり二行以内)。
\item タンパク質の二次構造は、ポリペプチド主鎖の二つの二面角$\phi$
(C'--N--C--C')
と$\psi$(N--C--C'--N)のみで表すことができる。その理由を述べよ。

\SubQuestion
化学結合したホモポリペプチドであっても、$\alpha$-らせんや$\beta$-構造
のような立体構造を示すことがある。しかし、ホモポリペプチドは、天然タンパク質が
持つ、より高次の立体構造をとることはない。これらの事実より、タンパク質の二次
構造を安定化する相互作用と、より高次の立体構造を安定化する相互作用との間に
どのような違いがあると考えられるか。十行以内で述べよ。
\item タンパク質の天然構造は、そのアミノ酸配列(遺伝情報)により一意的に
決定されると考えられている。このような考えの根拠になっている既知の実験事実
について解説せよ。また、アミノ酸配列より天然構造へと至る構造形成の仕組みを
明らかにするには、どの様な研究が必要と考えられるか。それぞれ10行以内で述べよ。

\end{subquestions}
\end{question}
\begin{answer}{専攻 問題7}{}

\begin{subanswers}
\SubAnswer

\begin{description}
\item[{\bf 一次構造}]\quad 20種のアミノ酸による配列のこと
\item[{\bf 二次構造}]\quad ペプチド結合部の水素結合によってつくられる
構造で$\alpha$ヘリックスと$\beta$シートのこと
\item[{\bf 超二次構造}]\quad $\alpha$ヘリックスと$\beta$シートとランダム
コイルの組み合わせによって作られるモチーフ
\item[{\bf 三次構造}]\quad 上記の構造をふまえたうえで一本のアミノ酸鎖が
最終的にとる立体構造
\item[{\bf 四次構造}]\quad 多くの酵素は一本のアミノ酸からつくられるいくつかの
サブユニットから成り、そのサブユニットの組合せを四次構造という。
\end{description}

\SubAnswer
タンパク質のbackboneをたどると一つのペプチドあたりC'--N,
N--C,C--C,の3つの
回転自由度がある。しかし、実際はペプチド結合部すなわちN--Cは、C=O二重結合
と共鳴しており回転できない。そのため自由度が2個になる。
\item まず、二次構造が対称構造であることがわかる。すなわちある性質を持った
残基が周期的に現れることが(homopolypeptideではいうまでもなく残基の性質は
周期的である)、二次構造を安定化する要因となる。これに対しより高次の構造は
基本的に対象とは無関係である。ではより高次に構造は何によって決まるかと言うと、
むしろ残基の性質が違うこと、つまりhydrophilicityやchargeの違いが、その要因と
なっている。また、より高次の構造は二次構造とくらべて、ランダムな状態との
chemical 
energyの差が小さく反応速度が小さいために触媒を必要とする。in situでは、
folding enzymeがその役目を果たす。

\SubAnswer
ribonucleaseをureaで変性し失活させた後、徐々にureaを除去すると
再び活性を取り戻した(C.B.Anfinsen)。この実験を解釈すると、ureaによって
一次構造は保持しているが高次構造を失っているペプチド鎖がureaの除去により
再び高次構造を取り戻したということである。これはin vitroで行なわれた実験で
あるのでfolding enzymeなどの助けがなくても一次構造のみから高次構造を
再生したことになる。

タンパク質の一残基あるいは少数のみ改変し高次構造の変化を見ることによって
その部位の構造をつくるために必要な相互作用を知ることができる。すなわち、改変
蛋白が構造を失わないならば残基のいかなる違いによるものかを調べることである。
また同じことを、異なる生物の同じ機能を持つ蛋白を見比べることによっても
できる。

\end{subanswers}
\end{answer}


\end{document}
