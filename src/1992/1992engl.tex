\documentclass[fleqn]{jbook}
\usepackage{physpub}

\begin{document}

\begin{question}{$B650i(B $B1Q8l(B}{}
\begin{subquestions}
\SubQuestion
    $B0J2<$N1QJ8$rFI$_!"@_Ld$KEz$($h!#2rEz$O2rEzMQ;f$N=jDjMs$K5-$;!#(B
\baselineskip=12pt

    $B!!(BFor a news conference at the Japan Press Center in Tokyo, it was
   a humble affair. Only about a dozen reporters were present Thursday
   when three people from India made a quiet appeal to the Japanese
   people about a giant dam project on the Narmada River, which
   empties into the Gulf of Cambay 400 kilometers north of Bombay.

   $B!!(BAs a result of the project 25,000 hectares of fertile farmland and
   forests will be submerged under water, the Indians said. About
   100,000 people are being driven out of their homes in 248 villages
   along the 200 kilometer upper reaches of the river. Sixty percent
   of these people are of ethnic minority groups.

    $B!!(BThe relocation plans are inadequate. The dam project will not
    only cause environmental destruction through the loss of forests
    but also destroy the traditional lifestyles of local people.
    \underlineeng{(a)Would urban residents, the three Indians
    asked, wish for an increase in the power supply in exchange for
    the life of people who pick fruit in the primeval forests and make
    their living by selling wood products?} A total of 12,000 people
    staged a 24-hour sit-down demonstration against the project last
    month.  \underlineeng{(b)People around the world are casting
    critical eyes on Japan now. Together with the World Bank, Japan is
    helping finance the Indian dam project through its Official
    Development Assistance program.}  The order for power generators
    to be used in the project has been placed with a Japanese company. 
    ``Do the people of Japan know that part of the taxes they pay is
    being used in this way?'' the three Indians asked at their news
    conference. Honestly, we didn't know. We haven't been informed.

    $B!!(B\underlineeng{(c)The Overseas Economic Cooperation Fund, the
    agency which represents Japan in the project, have not even
    bothered to study the impact the project will have on the
    environment and the area's inhabitants.}  The three Indians urged
    the Japanese to pay attention to the matter of who will lose from
    the project and who will gain. I think they were greately reserved
    in making their point.
%
 \begin{flushright}
     --quoted, with modification, from \em Tensei Jingo-'90 Summer Edition,
\em Hara Shoboh.
  \end{flushright}


Official Development Assistance : $B@/I\3+H/1g=u(B\\
Overseas Economic Cooperation Fund : $B3$307P:Q6(NO4p6b(B\\
\baselineskip=15pt

 \noindent {[$B@_Ld(B]}
 \begin{subsubquestions}
 \SubSubQuestion
    $B2<@~It(B(a)$B$r(B 100 $B;z0JFb$GOBLu$;$h!#(B
 \SubSubQuestion
    $B2<@~It(B(b)$B$r(B 100 $B;z0JFb$GOBLu$;$h!#(B
 \SubSubQuestion
    $B2<@~It(B(c)$B$r(B 100 $B;z0JFb$GOBLu$;$h!#(B
 \end{subsubquestions}





\SubQuestion
   $B0J2<$N1QJ8$rFI$_!"@_Ld$KEz$($h!#2rEz$O2rEzMQ;f$N=jDjMs$K5-$;!#(B
\baselineskip=12pt

    $B!!(BA new survey of extraordinary prolific researchers raises once
   again the question who should be an author and who should not.
   Twenty researchers worldwide have published an article at least
   once every 11.3 days over the past decade, according to a report
   released last week. The top five researchers have published more
   than once a week.

   $B!!(BSome of these authors are crystallographers who, by nature of
   their sophisticated equipment and experiences, collaborate with
   dozens of researchers who need their services. Co-authorship is
   usually given in return. Most of the others are scientists who run
   medium-sized to large laboratories or research groups that have
   tapped into a particularly fruitful line of research in their
   field. \underlineeng{(a) In the top twenty, the single
   discipline most heavily representer is basic molecular chemistry,
   followed by transplant surgery.} Biomedicalresearch, in general,
   accounts for more than half of the total.

   $B!!(BAs science finds itself under scrutiny as never before, many
   researchers are considering the old problem of authorship. Is the
   loan of a key reagent worth a co-authorship? How about a good idea? 
   Regular guidance?

   $B!!(B\underlineeng{(b)As a result of cases in which prominent
   researchers were damaged by the revelation that papers on which
   they were listed as authors contained fablicated data (even though
   they had not done the research themselves),the debate over
   authorship is no longer academic.} A laboratory director who
   initiates or supervises a project is now considered responsible for
   the accuracy of the data itself if he or she shares authorship.

   $B!!(BAn informal survey of many of the researchers on the list reveals
   that there are as many authorship policies as there are
   authors. Differences in authorship policies tend to be vary by
   discipline. Because chemistry experiments are relatively
   straight-forward and self-checking, the heads of prolific chemistry
   laboratories say they feel confident in simply providing ideas and
   oversight, along with regular review, to work
   \underlineeng{(c)they will eventually co-sign.} Frank Cotton, a
   Tezas A\&M University chemist, says he selects --- and co
   authors --- most of the research projects in his laboratory because
   the grants are in his name, and are based on his
   proposals. Although he does little of the bench research, he says
   he guides his 15 to 25 researchers, examines their data, and has a
   hand in writing almost all the papers. On the other hand, ``I only
   put my name on if I had the idea, started the study, or played an
   active part in it,'' says Julia Polak, a Universilty of London
   pathologist. ``My own criterion is that if I don't understand it
   and haven't been part of the writing of the paper, I don't want my
   name on it.''

  $B!!(B\underlineeng{(d)Although there may never be firm,
   interdisciplinary rules about authorship, prolific researchers do
   seem to be aware now of the perils of appending their names to work
   they have not closely supervised.}
%
 \begin{flushright}
    --quoted, with modifications, from \em Nature. \em
 \end{flushright}

fabricate : $B!J$3$N>l9g!KYTB$$9$k(B \\
\baselineskip=15pt

 \noindent{[$B@_Ld(B]}
 \begin{subsubquestions}
 \SubSubQuestion
   $B2<@~It(B(a)$B$r(B80$B;z0JFb$GOBLu$;$h!#(B
 \SubSubQuestion
   $B2<@~It(B(b)$B$r(B140$B;z0JFb$GOBLu$;$h!#(B
 \SubSubQuestion
    $B2<@~It(B(c)$B$O6qBNE*$K$O$I$&$$$&$3$H$+5-$;!#(B
 \SubSubQuestion
    $B2<@~It(B(d)$B$r(B100$B;z0JFb$G5-$;!#(B
 \end{subsubquestions}





\SubQuestion
  
$B0J2<$N1QJ8$rFI$_!$<!$NLd$KEz$($h!#2rEz$O2rEzMQ;f$N=jDjMs$K$7$k$;!#(B
\baselineskip=12pt

   $B!!(BI must reiterate my feeling that experimental physicists always
   welcome the \em suggestions \em of the theorists. But the present
   situation is ridiculous. Theorists now sit on the scheduling
   committees at the large particle accelerators and can exercise veto
   power over proposals by the best experimentists. In the days when I
   was active in nuclear and particle physics, the theorists exercised
   their veto power in an acceptable manner--they held up to ridicule
   anyone who did ``stupid experiments.'' So, if anyone could be
   intimidated by such social pressures, and nearly everyone can be,
   they were effectively kept from doing stupid result, to find
   himself ridiculed by the much smarter theorists. A classic example
   was the experiment in which R.T.Cox found, in 1928, that the
   electrons emitted in the beta decay of radium E must be
   polarized--because they ``double-scattered'' with different
   probabilities to the left and to the right.  Theorists couldn't
   accept this result because it violated the principle of
   consetvation of parity, which they held to be sacred. So the
   important Cox discovery disappeared from sight until it was
   rediscovered ``properly.''  Nineteen years later, T.D.Lee and Frank
   Yang, to solve a serious problem in particle physics, proposed that
   parity might not be conserved in weak interactions. Several teams
   set up experiments to look for parity violations, and Lee and
   Yang's suggestion was quickly found to conform to the structure of
   the real world. But the Nobel Prize for this change in paradigm
   went neither to Cox nor to Madam Wu and her collaborators, who
   proved Lee and Yang to be correct. Lee and Yang certainly earned
   their prize, but I believe the experimentalists should have shared
   in the glory.

   $B!!(BNext year is the 100th anniversary of the Michelson-Morley
   experiment. That terribly important experiment couldn't be done at
   the present time, as I think the following imagenary scene will
   show. Michelson and Morley tell the ``scheduling committee'' that
   they plan to measure the velocity of the earth through the ether by
   means of Michelson's new inferometer. The theoretical
   astrophysicist on the committee asks, ``How accurately will your
   method measure the velocity?'' Michelson says, ``To about one
   significant digit.'' The theorist responds, ``But we already know the
   velocity of the earth through the ether from astronomical
   observations to two or three significant figures.'' All the menbers
   of the committee agree that the proposal is one of the nuttiest
   they've ever heard, and Michelson and Morley find themselves turned
   down flat. So the experiment that led to the abandonment of the
   concept of the ether isn't done.

   $B!!(B\underlineeng{I wonder how many physical concepts that
   ``everyone knows'' to be true, as everyone knew parity conservation
   and the existence of the ether were true, would turn out to be
   false if experiments were again allowed to do nutty experiments.}
%
 \begin{flushright}
   --quoted, with modifications, from Luis W.Alvarez, \em Adventures of a 
physicist, \em Basic Books, N.Y.,1987.
  \end{flushright}
$B!!(B
  nutty$B!JJFB/8l!K(B : foolish 
\baselineskip=15pt

  \noindent{[$B@_Ld(B]}
  \begin{subsubquestions}
  \SubSubQuestion
    $B2<@~ItJ,$N1QJ8$r(B140$B;z0JFb$GOBLu$;$h!#(B
  \SubSubQuestion
     $B$3$N1QJ8$NMW;]$r(B\underline{$B1Q8l$G(B}$B;XDj$5$l$?9T?tFb!J(B100--200
$B8lDxEY0JFb!K$G=q$1!#(B
   \end{subsubquestions}






\SubQuestion
$B<!$NJ8$r(B80$B8lDxEY0JFb$K1QLu$;$h!#2rEz$O2rEzMQ;f$N=jDjMs$K5-$;!#(B

  $B!!(B 25$BG/$[$IA0$^$G$O!$M[;R$O$=$l0J>eJ,3d$G$-$J$$N3;R$G$"$k$H9M$($i$l(B
   $B$F$$$?!#$7$+$7!$M[;R$HM[;R!$$"$k$$$OM[;R$HEE;R$r9bB.$G>WFM$5$;$k<B(B
   $B83$r9T$C$?$H$3$m!$<B:]$K$OM[;R$O$5$i$K>.$5$JN3;R$+$i9=@.$5$l$F$$$k(B
   $B$3$H$,$o$+$C$?!#$=$NN3;R$O!$!V%/%)!<%/!W$HL>IU$1$i$l$?!#(B
%
 \begin{flushright}
    --quoted with modifications from \em A Brief History of Time, \em
by S.W.Hawking, Bantam Books, N.Y., 1988.
 \end{flushright}

 $B!!(B $BM[;R(B : proton \quad 
    $BN3;R(B : particle \quad 
    $BEE;R(B : electron \quad 
    $B%/%)!<%/(B : quark 



\SubQuestion
 $B!!2<5-$NJ8!J$$$o$f$k%<%N%s$N5UM}!K$NO@M}$K$OITHw$,$"$k!#$3$NJ8$rFI$_!$(B
  $B@_Ld$KEz$($h!#2rEz$O2rEzMQ;f$N=jDjMs$K5-$;!#(B

  $B!!@h9T$9$k55$r8eJ}$+$i%"%-%l%9$,DI$$$+$1$?$H$9$k!#(B
   \underlinejpn{$BDI$$$+$1;O$a$?;~E@$G55$N$$$?>l=j$K%"%-%l%9$,E~C#$7(B $B$?$H$-!$55$OA0J}$K$*$j!$$=$N>l=j$K%"%-%l%9$,E~C#$7$?$H$-$K$O55$O$5$i$KA0J}$K$$$k!#(B}

$B%<%N%s(B : Zeno \quad %$B"+$J$s$+2x$7$/$J$$!)(B
$B%"%-%l%9(B : Achilles

\begin{subsubquestions}
\SubSubQuestion
  $B2<@~ItJ,$NOBJ8$r(B60$B8lDxEY0JFb$G1QLu$;$h!#(B
\SubSubQuestion
   $B=PMh$k$@$1C;$/!$(B\underlinejpn{\bf $B1Q8l$G(B}$B$3$NJ8$NO@M}$NITHw$r@bL@$;$h!#(B
\end{subsubquestions}   

\end{subquestions}
\end{question}
\begin{answer}{$B650i(B $B1Q8l(B}{}

\begin{subanswers}
\SubAnswer 
{\bf $BA4Lu(B}

  $BEl5~!&Fb9,D.$N%W%l%9%;%s%?!<$H$$$&>l=jJA$+$i$9$k$H!"$=$l$O$5$5$d$+$J5-(B
 $B<T2q8+$@$C$?!#%$%s%I$+$i$d$C$F$-$?CK=w;0?M$,:rF|!"==?t?M$N5-<T$rA0$K(B
 $B!VF|K\$N;TL1$KNI$/CN$C$F$b$i$$$?$$$3$H!W$r@E$+$J8}D4$G8l$C$?!#(B

  $B%\%s%Y%$$+$iKL$X;MI4%-%m!#0!BgN&$N@>It$rN.$l$kBg2O!"%J%k%^%@@n$GCe9)$5(B
 $B$l$?5pBg%@%`$NOC$@!#$3$N7z@_$G!"FsK|%X%/%?!<%k$NK-$+$JG@CO$H?9NS$,?e(B
 $BKW$9$k!#>eN.0hFsI4%-%m$K$o$?$C$FFsI4;M==H,$NB<$,$"$j!"=;L1Ls==K|?M$,(B
 $BDI$$N)$F$i$l$F$$$k!#O;3d$,>/?tL1B2$@$H$$$&!#(B

 $B0\=;7W2h$bK~B-$K$G$-$F$$$J$$!#Bg5,LO$J?9NSAS<:$K$h$k4D6-GK2u$HEAE}E*(B
 $B$J@83hJ82=$NJx2u$H!#(B\underlinejpn{$B!J(Ba$B!K!VET2q$N=;(B $B?M$O!"86;ONS$G2L<B(B
 $B$r:N<h$7!"LZ@=IJ$rGd$k$3$H$G!"@87W$rN)$F$F$$$k?M!9$NJk$i$7$H0z$-49$((B
 $B$K$7$F$^$G!"EENO(B $B$N6!5k$,A}$($k$3$H$rK>$`$G$7$g$&$+!W$H!";0?M$N%$%s%I(B
 $B?M$OLd$$$+$1$?!#(B } (91$B;z(B) $B@h7n$K$O0lK|Fs@i?M$,CkLk!"H?BP$N:B$j9~(B
 $B$_$r$7$?!#(B
 
 \underlinejpn{$B!J(Bb$B!K8=:_!"@$3&Cf$N?M$,!"F|K\$KBP$7HcH=E*$JL\$r8~$1$F(B
$B$$$^$9!#F|K\$O!"@/I\3+H/1g=u7W2h$rDL$8!"@$3&6d9T(B $B$H6&F1$G!"$3$N%$%s%I(B
$B$N%@%`7W2h$N;q6b$rM;;q$7$h$&$H$7$F$$$k!#(B }(81$B;z(B) $BMH?eH/EE5!$r<uCm$7$?(B
$B$N$b!"F|K\$N4k6H$@!#2q8+$G!";0?M$O!VF|K\$N9qL1$O!"@G6b$,$3$s$J7A$G;H$o(B
$B$l$F$$$k$N$rCN$C$F$$$^$9$+!W$HLd$$$+$1$F$$$?!#(B

$B!VCN$j$^$;$s$G$7$?!#CN$i$5$l$F$$$^$;$s$G$7$?!W$H@5D>$KEz$($k$7$+$J$$!#(B
\underlinejpn{$B!J(Bc$B!KF|K\$rBeI=$9$k5!4X$G$"$k3$30(B $B7P:Q6(NO4p6b$O!"7W2h(B
$B$K$h$C$F4D6-$,<u$1$k1F6A$d!"CO0h$N=;(B $B?M$,<u$1$k1F6A$rD4::$7$h$&$H$9$i(B
$B$7$J$+$C$?!#(B }(75$B;z(B) $B!V$@$l$N5>@7$N>e$K!"$@$l$,Mx1W$rF@$F$$$k$N$+!W$K(B
$B$D$$$F4X?4$r;}$C$F$[$7$$!"$H$$$&;0?M$N8@MU$O$:$$$V$s95$(L\$@$C$?$H;W$&!#(B

 \begin{flushright}
     --1990.4.20 $BE7@<?M8l$+$iH4?h(B
  \end{flushright}

\SubAnswer
 {\bf $BA4Lu(B}

$B!!O@J8$rHs>o$KB?$/=q$/8&5f<T$K$D$$$F$N?7$7$$D4::7k2L$r8+$k$H!"$^$?$b!"C/(B
$B$rCx<T$H8F$V$Y$-$+H]$+$H$$$&5?Ld$r46$8$k!#@h=5H/I=$5$l$?D4::7k2L$K$h$k(B
$B$H!"@$3&Cf$GFs==?M$N8&5f<T$,!"2a5n==G/4V!">/$J$/$H$b(B11.3$BF|$K0l2s$O5-;v(B
$B$r=PHG$7$F$$$k!#%H%C%W$N8^?M$O!"0l=54V$K0l2s0J>e$N3d9g$G!"5-;v$r=P$7$F(B
$B$$$k!#(B

$B!!$3$l$i$NCx<T$NCf$K$O!"7k>=3X<T$,$$$k!#H`$i$O!"9bEY$KJ#;($J<B83@_Hw$H7P(B
$B83$r;}$C$F$*$j!"I,A3E*$K!"H`$i$N=u$1$rI,MW$H$9$k?tB?$/$N8&5f<T$H6&F1<B(B
$B83$r9T$J$C$F$$$k!#IaDL!"$=$N8+JV$j$H$7$F!"6&Cx<T$N8*=q$r$rF@$k$N$G$"$k!#(B
$BB>$N$[$H$s$I$N3X<T$O!"Cf0L$+!"Bg$-$J8&5f<<$b$7$/$O8&5f%0%k!<%W$r<h$j$^(B
$B$H$a$F$$$k2J3X<T$G!"$=$N%0%k!<%W$,!"@l96J,Ln$N$J$+$G$b!"FC$K@.2L$NB?$$(B
$B0lO"$N8&5f$K9T$-Ev$?$C$?$h$&$J?MC#$G$"$k!#(B
\underlinejpn{$B!J(Ba$B!K:G(B $B$bCx:n$NB?$$Fs==?M$N@lLgJ,Ln$O!"C1FH$NJ,Ln$G(B
$B$O!"4pACJ,;R2=3X$,:G$bB?$/!"<!$$$G0\?"302J0e3X$,L\N)$C$?!#(B }(52$B;z(B)
$B$h$j9-$/J,Ln$rB*$($k$H!"@8L?0e3X$,A4BN$NH>J,0J>e$r@j$a$?!#(B

$B!!:#$^$G$K$J$/2J3X$K87L)$5$,MW5a$5$l$k:#F|!"B?$/$N3X<T$,!"!VO@J8$NCx<T!W(B
$B$H$$$&@N$J$,$i$NLdBj$K$D$$$F9M$(D>$7$F$$$k!#H?1~$N80$H$J$k;nLt$rB_$9$3(B
$B$H$O!"6&Cx<T$H$J$k$KB-$k$3$H$G$"$m$&$+!#$G$O!"$h$$%"%$%G%#%"$rM?$($?$H(B
$B$$$&$N$O(B? $BF|>oE*$J;XF3$O(B? 
\underlinejpn{$B!J(Bb$B!KCxL>$J8&5f<T$,!"YTB$%G!<%?$r4^$s$G$$$k$3$H$,L@(B
$B$i$+$H$J$C$?O@(B $BJ8$K!"Cx<T$H$7$FL>$rO"$M$?$P$+$j$K!"(B($BK\?M$O$=$N8&5f$r$7(B
$B$J$+$C$?$K$K$b$+$+$o$i$:(B)$BHo32$rHo$C$?$H$$$&;v(B $B7o$,5/$3$k$K;j$C$F$O!"(B
$B!VCx<T!W$K$D$$$F$NO@Ah$O!"$b$O$d%"%+%G%_%C%/$H$$$&0h$r$O$:$l$F$$$k!#(B }
(125$B;z(B)
$B:#F|$G$O!"8&5f$N$F$[$I$-$d4FFD$r$9$k!"8&5f<<$N;XF3<T$O!"O@J8$K6&Cx<T$H(B
$B$7$FL>A0$rO"$M$k>l9g$O!"$=$NO@J8$N%G!<%?$N@53N$5$KBP$7@UG$$rLd$o$l$k$N(B
$B$G$"$k!#(B

$B!!B?:nCx<T$N%j%9%H$K:\$C$F$$$k8&5f<T$NB?$/$K$D$$$F$NHs8x<0$JD4::$+$i!"(B
$B!VCx<T!W$K$D$$$F$N9M$(J}$O!"==?M==?'$G$"$k$3$H$,$o$+$k!#!VCx<T!W$K$D$$(B
$B$F$N9M$(J}$O!"@l96$9$kJ,Ln$K$h$C$F0c$&798~$K$"$k!#2=3X<B83$O!"Hf3SE*0l(B
$BK\F;$G$"$j!"3NG'$b<+J,$G9T$J$&$3$H$,B?$$$N$G!"O@J8$r$?$/$5$s=P$7$F$$$k(B
$B2=3X$N8&5f<<$N%j!<%@!<$K$h$k$H!"H`$i$O!"IaCJ8&5f<<Fb$N8&5f$r8+$F$^$o$C(B
$B$F$$$k$N$K2C$(!"%"%$%G%#%"$rM?$($?$j!"4FFD$7$?$j$9$k$@$1$N$3$H$b=<J,Bg(B
$B@Z$J$3$H$G$"$k$H3N?.$7$F$*$j!"(B
\underlinejpn{$B!J(Bc$B!K7k6I!"O@J8$N6&Cx<T$H$J$k$N$G$"$k!#(B }
$B%F%-%5%9(BA\&M$BBg3X$N2=3X<T$G$"$k%U%i%s%/(B=$B%3%C%H%s$K$h$k$H!"H`$N8&5f<<$G(B
$B9T$J$o$l$k<B83$N$[$H$s$I$O!"H`$,A*$s$@$b$N$G$"$j!"$=$7$F!"H`$,6&Cx<T$H(B
$B$J$C$F$$$k!#$H$$$&$N$b!"H`$,<B83$N5v2D$rM?$($k$N$G$"$j!"H`$NDs0F$K4p$E(B
$B$$$F<B83$O9T$J$o$l$F$$$k$+$i$G$"$k!#H`<+?H$,<B:]$N<B83$r$d$k$3$H$O$[$H(B
$B$s$I$J$$$,!"H`$O!"8&5f<<$N(B15$B$+$i(B30$B?M$N8&5f<T$r;XF3$7!"H`$i$N%G!<%?$KL\(B
$B$rDL$7!"$[$H$s$I$9$Y$F$NO@J8$K<j$rF~$l$F$$$k$H$N$3$H$G$"$k!#0lJ}$G!"%m(B
$B%s%I%sBg3X$NIBM}3X<T$G$"$k%8%e%j%"(B=$B%]%i%C%/$O!"<!$N$h$&$K8@$C$F$$$k!#(B
$B!V;d$O!"$=$N%"%$%G%#%"$,<+J,$G;W$$$D$$$?$b$N$G$"$k$+!"8&5f$r;O$a$?$N$,(B
$B<+J,$G$"$k$+!"$b$7$/$O8&5f$N=EMW$JItJ,$r<+J,$,9T$J$C$?>l9g$G$J$1$l$P!"(B
$BO@J8$KL>A0$r:\$;$^$;$s!#M}2r$b$7$F$*$i$:!"O@J8=q$-$r0lIt$G$b9T$J$C$F$$(B
$B$J$$>l9g!"$=$NO@J8$KL>A0$,:\$k$N$O$$$d$G$"$k!"$H$$$&$N$,!";d$N4p=`$G$9!#!W(B 

$B!!(B\underlinejpn{$B!J(Bd$B!KO@J8$NCx<T$K4X$7$F!"A4J,Ln$K6&DL$N3N8G$H$7$?7h(B
$B$^$j$,$G$-$k$3$H$O$J$$$+$b$7$l$J$$$,!"O@J8$rB?:n$9(B $B$k8&5f<T$b!":#$G$O!"(B
$B=<J,4FFD$7$?$o$1$G$J$$8&5f$NO@J8$K<+J,$NL>A0$r:\$;$k$3$H$N4m81@-$rCN$C(B
$B$F$$$k$h$&$@(B $B!#(B}(99$B;z(B)
%$BLdBj(B(c)$B$N2rEz$,ITB-$7$F$$$k!#(B



\SubAnswer
 {\bf $BA4Lu(B}
%$BFC$KA0H>$,$R$I$$Lu$G$7$?$N$G!"FHCG$GJQ99$7$?2U=j$,$$$/$D$+$"$j$^$9!#(B

$B!!;d$O!"<+J,$N5$;}$A$r8}$r;@$C$Q$/$9$k$[$I8@$o$:$K$O$$$i$l$J$$!#<B83J*(B
$BM}3X<T$O$$$D$bM}O@2H$N!VDs0F!W$r<u$1F~$l$F$7$^$C$F$$$k!#$7$+$7!"8=:_$N(B
$B>u67$OGO</$2$F$$$k!#:#$d!"M}O@2H$O!"5pBg$JN3;R2CB.4o$NF|Dx2q5D$N@J>e$G!"(B
$BHs>o$K$9$0$l$?<B832H$N=P$9Ds0F$KBP$7!"5qH]8"$r9T;H$9$k$3$H$,$G$-$k$N$G(B
$B$"$k!#;d$,!"86;R3K$*$h$SAGN3;RJ*M}$G3hF0$7$F$$$?$3$m$O!"M}O@2H$N5qH]8"(B
$B$N9T;H$N;EJ}$O!"$^$@!"<u$1F~$l$i$l$k$b$N$G$"$C$?!#H`$i$O!"!VGO</$J<B83!W(B
$B$r$7$??M$rC/$G$b$$$$$+$i0z$-9g$$$K=P$7!"$"$6>P$C$?$N$G$"$k!#$H$$$&$o$1(B
$B$G!"$b$7!"$=$N$h$&$J<R2qE*05NO$K$h$j0R3E$5$l$k?M$,$$$k$H$9$l$O!"(B($B<B:](B
$B$K$O$[$H$s$I$N?M$,0R3E$5$l$F$7$^$&$N$G$"$k$,(B\ldots )$B8z2LE*$K!"GO</$J<B(B
$B83$r?M$,$7$J$/$9$k$3$H$,$G$-$k$N$G$"$k!#;~!9!"GO</$J<B83$@$H5$IU$+$:$K!"(B
$BGO</$J<B83$r$7$F$7$^$&<B832H$,$$$F!"$=$NGO</$2$?7k2L$rO@J8$K$7$F$7$^$&(B
$B$N$@$,!"$=$&$9$k$H!"$O$k$+$KF,$N$$$$M}O@2HC#$K>P$o$l$F$7$^$&$N$G$"$C$?!#(B
$BE57?E*$JNc$O!"(BR.T.Cox $B$,(B1928$BG/$KH/8+$7$?<B837k2L$G$"$k!#$=$l$O!"%i%8%&(B
$B%`(BE$B$N(B$\beta$$BJx2u$K$h$C$F!"J|=P$5$l$kEE;R$O!"(B($B1&$H:8$G0c$&3NN($G!VFs=E(B
$B;6Mp!W$9$k$N$G(B)$BJP6K$9$k!"$H$$$&$b$N$G$"$k!#M}O@2H$O!"$3$N7k2L$r<u$1F~(B
$B$l$k$3$H$,=PMh$J$+$C$?!#$H$$$&$N$O!"$=$l$O!"H`$i$,?@@;;k$7$F$$$?%Q%j%F%#!<(B
$BJ]B8B'$r?/$7$F$$$?$+$i$G$"$k!#$=$N$?$a$K!"(BCox $B$N=EMW$JH/8+$O!"!V@5$7$/!W(B
$B:FH/8+$5$l$k$^$G;k3&$+$i>C$(5n$C$?$N$G$"$k!#(B19$BG/8e!"(BT.F.Lee $B$H(BFrank
Yang $B$,!"AGN3;RJ*M}$N=EBg$JLdBj$r2r7h$9$k$?$a$K!"$b$7$+$9$k$H<e$$Aj8_(B
$B:nMQ$G$O%Q%j%F%#!<$,J]B8$5$l$J$$$N$G$O$J$$$+$b$7$l$J$$!"$H$$$&$3$H$rDs(B
$B0F$7$?!#$$$/$D$+$N%0%k!<%W$,%Q%j%F%#!<$NHsJ]B8$r3NG'$9$k<B83$r@_Dj$7!"(B
$B$9$0$K!"(BLee $B$H(BYang$B$NDs0F$,!"<B:]$N@$3&$N9=B$$KE,9g$7$?$b$N$G$"$k$3$H$,(B
$B8+$$=P$5$l$?!#$7$+$7$J$,$i!"$3$N!V>o<1!W$NJQ3W$KBP$9$k%N!<%Y%k>^$O!"(B
Cox $B$,Lc$C$?$o$1$G$b!"(BWu $BIW?M$H$=$N6&F1<B83<TC#(B(Lee $B$H(B Yang $B$,@5$7$$$3(B
$B$H$r>ZL@$7$?(B)$B$,Lc$C$?$o$1$G$b$J$+$C$?!#(BLee $B$H(B Yang $B$,3N$+$K1I4'$r>!$A(B
$BF@$?$o$1$@$,!";d$O!"<B832H$b$=$l$rJ,$+$D$Y$-$G$"$C$?$H;W$&!#(B

$B!!MhG/$O!"(BMichelson-Morley $B$N<B83$+$i$A$g$&$II4G/$G$"$k!#$"$NHs>o$K=EMW(B
$B$J<B83$b!":#F|$N$h$&$J>u67$G$O9T$J$o$l$J$+$C$?$G$"$m$&!#$H$$$&$N$b!"<!(B
$B$N$h$&$JA[A|>e$N>lLL$,E83+$5$l$?$G$"$m$&$+$i$@!#(BMichelson $B$H(BMorley $B$,(B
$B!VF|Dx2q5D!W$G!"(BMichelson $B$N?7$7$$43>D7W$r;H$C$F!"%(!<%F%kCf$NCO5e$NB.(B
$BEY$rB,Dj$9$k7W2h$G$"$k$3$H$rJs9p$9$k!#2q5D$K;22C$7$F$$$kM}O@E7BNJ*M}3X(B
$B<T$,<ALd$9$k!#!V$"$J$?J}$NJ}K!$@$H!"$I$NDxEY$N@:EY$G!"CO5e$NB.EY$,B,$l(B
$B$k$N$G$9$+(B?$B!W(BMichelson $B$,8@$&!#!VM-8z?t;z0l7e$G$9!#!WM}O@2H$,1~$($k!#(B
$B!V$7$+$7!"%(!<%F%kCf$G$NCO5e$NB.EY$O!"4{$K!"E7J84QB,$+$i!"M-8z?t;z(B2,~3
$B7e$G5a$a$i$l$F$$$k$s$G$9$h!#!W2q5D$NA40w$,!"(BMichelson $B$H(BMorley $B$NDs0F(B
$B$,A0BeL$J9$N$P$+$2$?$b$N$G$"$k$H$$$&$3$H$G0lCW$7!"(B2$B?M$N0F$O!"$K$Y$b$J(B
$B$/5Q2<$5$l$F$7$^$&!#$H$$$&6q9g$K!"%(!<%F%k$N35G0$rJ|4~$9$k$-$C$+$1$H$J$C(B
$B$?<B83$b9T$J$o$l$J$$$3$H$K$J$C$F$7$^$&!#(B

$B!!(B\underlinejpn{$B$+$D$F$NMM$K<B832H$,$P$+$J<B83$r$d$i$;$F$b$i$((B
$B$k$N$J$i!"$+$D$F%Q%j%F%#$NJ]B8$d%(!<%F%k$NB8:_$,@5$7$$(B $B$3$H$r$_$s$J$,CN$C(B
$B$F$$$?$h$&$K!"!V$_$s$J$,@5$7$$$3$H$rCN$C$F$$$k!WJ*M}35G0$N$$$C$?$$$$$/(B
$B$D$,!"<B$O4V0c$$(B $B$G$"$k$HH=L@$9$k$3$H$+$H;W$&!#(B }(118$B;z(B)


  \begin{subsubanswers} 
    \SubSubAnswer
            $BA4Lu;2>H$N$3$H!#(B
  
    \SubSubAnswer
\baselineskip=12pt

    $B!!(BTheoretical physicists always try to keep experimental
    physicists from doing experiments that seem to be stupid from what
    is ``known'' to be true. In the past, theorists held up to
    ridicule anyone who did ``stupid experiments'', and by the social
    pressure, experimentalists were intimidated and kept from such
    experiments.  But still, it was possible to do such experiments
    accidentally by not knowing that it was stupid. And also, there
    were some ``stupid experiments'' which eventually revealed the
    falseness of a fact that everyone believed to be true. The present
    situation is ridiculous because theorists sit on the scheduling
    committees and can exercise veto power over the proposals by the
    experimentalists directly.  But if the experimentalists were again
    allowed to do ``stupid experiments'', they may reveal some of our
    physical misconception.  $B!J(B137$B8l!K(B
  \end{subsubanswers}




\SubAnswer
    
    $B!!(BUp to about twenty-five years ago, it was thought that
    protons were ``elementary'' particles, but the experiment in which
    protons collided with other protons or electrons at high speeds
    indicated that they were in fact made up of smaller
    particles. These particles were named quarks.


\SubAnswer

\begin{subsubanswers}

 \SubSubAnswer
  By the time Achilles reaches the place where the turtle stayed 
at first, the turtle is proceeded ahead. When Achilles again reaches
the point where the turtle were, he has gone further.
 \SubSubAnswer
  The above statement neglected the convergence of the accumulated
time that Achilles spent in each steps.

\baselineskip=15pt
\end{subsubanswers}

\end{subanswers}
\end{answer}

\end{document}

   





