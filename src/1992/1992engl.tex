\documentclass[fleqn]{jbook}
\usepackage{physpub}

\begin{document}

\begin{question}{教育 英語}{}
\begin{subquestions}
\SubQuestion
    以下の英文を読み、設問に答えよ。解答は解答用紙の所定欄に記せ。
\baselineskip=12pt

     For a news conference at the Japan Press Center in Tokyo, it was
   a humble affair. Only about a dozen reporters were present Thursday
   when three people from India made a quiet appeal to the Japanese
   people about a giant dam project on the Narmada River, which
   empties into the Gulf of Cambay 400 kilometers north of Bombay.

    As a result of the project 25,000 hectares of fertile farmland and
   forests will be submerged under water, the Indians said. About
   100,000 people are being driven out of their homes in 248 villages
   along the 200 kilometer upper reaches of the river. Sixty percent
   of these people are of ethnic minority groups.

     The relocation plans are inadequate. The dam project will not
    only cause environmental destruction through the loss of forests
    but also destroy the traditional lifestyles of local people.
    \underlineeng{(a)Would urban residents, the three Indians
    asked, wish for an increase in the power supply in exchange for
    the life of people who pick fruit in the primeval forests and make
    their living by selling wood products?} A total of 12,000 people
    staged a 24-hour sit-down demonstration against the project last
    month.  \underlineeng{(b)People around the world are casting
    critical eyes on Japan now. Together with the World Bank, Japan is
    helping finance the Indian dam project through its Official
    Development Assistance program.}  The order for power generators
    to be used in the project has been placed with a Japanese company. 
    ``Do the people of Japan know that part of the taxes they pay is
    being used in this way?'' the three Indians asked at their news
    conference. Honestly, we didn't know. We haven't been informed.

     \underlineeng{(c)The Overseas Economic Cooperation Fund, the
    agency which represents Japan in the project, have not even
    bothered to study the impact the project will have on the
    environment and the area's inhabitants.}  The three Indians urged
    the Japanese to pay attention to the matter of who will lose from
    the project and who will gain. I think they were greately reserved
    in making their point.
%
 \begin{flushright}
     --quoted, with modification, from \em Tensei Jingo-'90 Summer Edition,
\em Hara Shoboh.
  \end{flushright}


Official Development Assistance : 政府開発援助\\
Overseas Economic Cooperation Fund : 海外経済協力基金\\
\baselineskip=15pt

 \noindent {[設問]}
 \begin{subsubquestions}
 \SubSubQuestion
    下線部(a)を 100 字以内で和訳せよ。
 \SubSubQuestion
    下線部(b)を 100 字以内で和訳せよ。
 \SubSubQuestion
    下線部(c)を 100 字以内で和訳せよ。
 \end{subsubquestions}





\SubQuestion
   以下の英文を読み、設問に答えよ。解答は解答用紙の所定欄に記せ。
\baselineskip=12pt

     A new survey of extraordinary prolific researchers raises once
   again the question who should be an author and who should not.
   Twenty researchers worldwide have published an article at least
   once every 11.3 days over the past decade, according to a report
   released last week. The top five researchers have published more
   than once a week.

    Some of these authors are crystallographers who, by nature of
   their sophisticated equipment and experiences, collaborate with
   dozens of researchers who need their services. Co-authorship is
   usually given in return. Most of the others are scientists who run
   medium-sized to large laboratories or research groups that have
   tapped into a particularly fruitful line of research in their
   field. \underlineeng{(a) In the top twenty, the single
   discipline most heavily representer is basic molecular chemistry,
   followed by transplant surgery.} Biomedicalresearch, in general,
   accounts for more than half of the total.

    As science finds itself under scrutiny as never before, many
   researchers are considering the old problem of authorship. Is the
   loan of a key reagent worth a co-authorship? How about a good idea? 
   Regular guidance?

    \underlineeng{(b)As a result of cases in which prominent
   researchers were damaged by the revelation that papers on which
   they were listed as authors contained fablicated data (even though
   they had not done the research themselves),the debate over
   authorship is no longer academic.} A laboratory director who
   initiates or supervises a project is now considered responsible for
   the accuracy of the data itself if he or she shares authorship.

    An informal survey of many of the researchers on the list reveals
   that there are as many authorship policies as there are
   authors. Differences in authorship policies tend to be vary by
   discipline. Because chemistry experiments are relatively
   straight-forward and self-checking, the heads of prolific chemistry
   laboratories say they feel confident in simply providing ideas and
   oversight, along with regular review, to work
   \underlineeng{(c)they will eventually co-sign.} Frank Cotton, a
   Tezas A\&M University chemist, says he selects --- and co
   authors --- most of the research projects in his laboratory because
   the grants are in his name, and are based on his
   proposals. Although he does little of the bench research, he says
   he guides his 15 to 25 researchers, examines their data, and has a
   hand in writing almost all the papers. On the other hand, ``I only
   put my name on if I had the idea, started the study, or played an
   active part in it,'' says Julia Polak, a Universilty of London
   pathologist. ``My own criterion is that if I don't understand it
   and haven't been part of the writing of the paper, I don't want my
   name on it.''

   \underlineeng{(d)Although there may never be firm,
   interdisciplinary rules about authorship, prolific researchers do
   seem to be aware now of the perils of appending their names to work
   they have not closely supervised.}
%
 \begin{flushright}
    --quoted, with modifications, from \em Nature. \em
 \end{flushright}

fabricate : (この場合)捏造する \\
\baselineskip=15pt

 \noindent{[設問]}
 \begin{subsubquestions}
 \SubSubQuestion
   下線部(a)を80字以内で和訳せよ。
 \SubSubQuestion
   下線部(b)を140字以内で和訳せよ。
 \SubSubQuestion
    下線部(c)は具体的にはどういうことか記せ。
 \SubSubQuestion
    下線部(d)を100字以内で記せ。
 \end{subsubquestions}





\SubQuestion
  
以下の英文を読み,次の問に答えよ。解答は解答用紙の所定欄にしるせ。
\baselineskip=12pt

    I must reiterate my feeling that experimental physicists always
   welcome the \em suggestions \em of the theorists. But the present
   situation is ridiculous. Theorists now sit on the scheduling
   committees at the large particle accelerators and can exercise veto
   power over proposals by the best experimentists. In the days when I
   was active in nuclear and particle physics, the theorists exercised
   their veto power in an acceptable manner--they held up to ridicule
   anyone who did ``stupid experiments.'' So, if anyone could be
   intimidated by such social pressures, and nearly everyone can be,
   they were effectively kept from doing stupid result, to find
   himself ridiculed by the much smarter theorists. A classic example
   was the experiment in which R.T.Cox found, in 1928, that the
   electrons emitted in the beta decay of radium E must be
   polarized--because they ``double-scattered'' with different
   probabilities to the left and to the right.  Theorists couldn't
   accept this result because it violated the principle of
   consetvation of parity, which they held to be sacred. So the
   important Cox discovery disappeared from sight until it was
   rediscovered ``properly.''  Nineteen years later, T.D.Lee and Frank
   Yang, to solve a serious problem in particle physics, proposed that
   parity might not be conserved in weak interactions. Several teams
   set up experiments to look for parity violations, and Lee and
   Yang's suggestion was quickly found to conform to the structure of
   the real world. But the Nobel Prize for this change in paradigm
   went neither to Cox nor to Madam Wu and her collaborators, who
   proved Lee and Yang to be correct. Lee and Yang certainly earned
   their prize, but I believe the experimentalists should have shared
   in the glory.

    Next year is the 100th anniversary of the Michelson-Morley
   experiment. That terribly important experiment couldn't be done at
   the present time, as I think the following imagenary scene will
   show. Michelson and Morley tell the ``scheduling committee'' that
   they plan to measure the velocity of the earth through the ether by
   means of Michelson's new inferometer. The theoretical
   astrophysicist on the committee asks, ``How accurately will your
   method measure the velocity?'' Michelson says, ``To about one
   significant digit.'' The theorist responds, ``But we already know the
   velocity of the earth through the ether from astronomical
   observations to two or three significant figures.'' All the menbers
   of the committee agree that the proposal is one of the nuttiest
   they've ever heard, and Michelson and Morley find themselves turned
   down flat. So the experiment that led to the abandonment of the
   concept of the ether isn't done.

    \underlineeng{I wonder how many physical concepts that
   ``everyone knows'' to be true, as everyone knew parity conservation
   and the existence of the ether were true, would turn out to be
   false if experiments were again allowed to do nutty experiments.}
%
 \begin{flushright}
   --quoted, with modifications, from Luis W.Alvarez, \em Adventures of a 
physicist, \em Basic Books, N.Y.,1987.
  \end{flushright}
 
  nutty(米俗語) : foolish 
\baselineskip=15pt

  \noindent{[設問]}
  \begin{subsubquestions}
  \SubSubQuestion
    下線部分の英文を140字以内で和訳せよ。
  \SubSubQuestion
     この英文の要旨を\underline{英語で}指定された行数内(100--200
語程度以内)で書け。
   \end{subsubquestions}






\SubQuestion
次の文を80語程度以内に英訳せよ。解答は解答用紙の所定欄に記せ。

    25年ほど前までは,陽子はそれ以上分割できない粒子であると考えられ
   ていた。しかし,陽子と陽子,あるいは陽子と電子を高速で衝突させる実
   験を行ったところ,実際には陽子はさらに小さな粒子から構成されている
   ことがわかった。その粒子は,「クォーク」と名付けられた。
%
 \begin{flushright}
    --quoted with modifications from \em A Brief History of Time, \em
by S.W.Hawking, Bantam Books, N.Y., 1988.
 \end{flushright}

   陽子 : proton \quad 
    粒子 : particle \quad 
    電子 : electron \quad 
    クォーク : quark 



\SubQuestion
  下記の文(いわゆるゼノンの逆理)の論理には不備がある。この文を読み,
  設問に答えよ。解答は解答用紙の所定欄に記せ。

   先行する亀を後方からアキレスが追いかけたとする。
   \underlinejpn{追いかけ始めた時点で亀のいた場所にアキレスが到達し たとき,亀は前方におり,その場所にアキレスが到達したときには亀はさらに前方にいる。}

ゼノン : Zeno \quad %←なんか怪しくない?
アキレス : Achilles

\begin{subsubquestions}
\SubSubQuestion
  下線部分の和文を60語程度以内で英訳せよ。
\SubSubQuestion
   出来るだけ短く,\underlinejpn{\bf 英語で}この文の論理の不備を説明せよ。
\end{subsubquestions}   

\end{subquestions}
\end{question}
\begin{answer}{教育 英語}{}

\begin{subanswers}
\SubAnswer 
{\bf 全訳}

  東京・内幸町のプレスセンターという場所柄からすると、それはささやかな記
 者会見だった。インドからやってきた男女三人が昨日、十数人の記者を前に
 「日本の市民に良く知ってもらいたいこと」を静かな口調で語った。

  ボンベイから北へ四百キロ。亜大陸の西部を流れる大河、ナルマダ川で着工さ
 れた巨大ダムの話だ。この建設で、二万ヘクタールの豊かな農地と森林が水
 没する。上流域二百キロにわたって二百四十八の村があり、住民約十万人が
 追い立てられている。六割が少数民族だという。

 移住計画も満足にできていない。大規模な森林喪失による環境破壊と伝統的
 な生活文化の崩壊と。\underlinejpn{(a)「都会の住 人は、原始林で果実
 を採取し、木製品を売ることで、生計を立てている人々の暮らしと引き換え
 にしてまで、電力 の供給が増えることを望むでしょうか」と、三人のインド
 人は問いかけた。 } (91字) 先月には一万二千人が昼夜、反対の座り込
 みをした。
 
 \underlinejpn{(b)現在、世界中の人が、日本に対し批判的な目を向けて
います。日本は、政府開発援助計画を通じ、世界銀行 と共同で、このインド
のダム計画の資金を融資しようとしている。 }(81字) 揚水発電機を受注した
のも、日本の企業だ。会見で、三人は「日本の国民は、税金がこんな形で使わ
れているのを知っていますか」と問いかけていた。

「知りませんでした。知らされていませんでした」と正直に答えるしかない。
\underlinejpn{(c)日本を代表する機関である海外 経済協力基金は、計画
によって環境が受ける影響や、地域の住 人が受ける影響を調査しようとすら
しなかった。 }(75字) 「だれの犠牲の上に、だれが利益を得ているのか」に
ついて関心を持ってほしい、という三人の言葉はずいぶん控え目だったと思う。

 \begin{flushright}
     --1990.4.20 天声人語から抜粋
  \end{flushright}

\SubAnswer
 {\bf 全訳}

 論文を非常に多く書く研究者についての新しい調査結果を見ると、またも、誰
を著者と呼ぶべきか否かという疑問を感じる。先週発表された調査結果による
と、世界中で二十人の研究者が、過去十年間、少なくとも11.3日に一回は記事
を出版している。トップの五人は、一週間に一回以上の割合で、記事を出して
いる。

 これらの著者の中には、結晶学者がいる。彼らは、高度に複雑な実験設備と経
験を持っており、必然的に、彼らの助けを必要とする数多くの研究者と共同実
験を行なっている。普通、その見返りとして、共著者の肩書をを得るのである。
他のほとんどの学者は、中位か、大きな研究室もしくは研究グループを取りま
とめている科学者で、そのグループが、専攻分野のなかでも、特に成果の多い
一連の研究に行き当たったような人達である。
\underlinejpn{(a)最 も著作の多い二十人の専門分野は、単独の分野で
は、基礎分子化学が最も多く、次いで移植外科医学が目立った。 }(52字)
より広く分野を捉えると、生命医学が全体の半分以上を占めた。

 今までになく科学に厳密さが要求される今日、多くの学者が、「論文の著者」
という昔ながらの問題について考え直している。反応の鍵となる試薬を貸すこ
とは、共著者となるに足ることであろうか。では、よいアイディアを与えたと
いうのは? 日常的な指導は? 
\underlinejpn{(b)著名な研究者が、捏造データを含んでいることが明
らかとなった論 文に、著者として名を連ねたばかりに、(本人はその研究をし
なかったににもかかわらず)被害を被ったという事 件が起こるに至っては、
「著者」についての論争は、もはやアカデミックという域をはずれている。 }
(125字)
今日では、研究のてほどきや監督をする、研究室の指導者は、論文に共著者と
して名前を連ねる場合は、その論文のデータの正確さに対し責任を問われるの
である。

 多作著者のリストに載っている研究者の多くについての非公式な調査から、
「著者」についての考え方は、十人十色であることがわかる。「著者」につい
ての考え方は、専攻する分野によって違う傾向にある。化学実験は、比較的一
本道であり、確認も自分で行なうことが多いので、論文をたくさん出している
化学の研究室のリーダーによると、彼らは、普段研究室内の研究を見てまわっ
ているのに加え、アイディアを与えたり、監督したりするだけのことも充分大
切なことであると確信しており、
\underlinejpn{(c)結局、論文の共著者となるのである。 }
テキサスA\&M大学の化学者であるフランク=コットンによると、彼の研究室で
行なわれる実験のほとんどは、彼が選んだものであり、そして、彼が共著者と
なっている。というのも、彼が実験の許可を与えるのであり、彼の提案に基づ
いて実験は行なわれているからである。彼自身が実際の実験をやることはほと
んどないが、彼は、研究室の15から30人の研究者を指導し、彼らのデータに目
を通し、ほとんどすべての論文に手を入れているとのことである。一方で、ロ
ンドン大学の病理学者であるジュリア=ポラックは、次のように言っている。
「私は、そのアイディアが自分で思いついたものであるか、研究を始めたのが
自分であるか、もしくは研究の重要な部分を自分が行なった場合でなければ、
論文に名前を載せません。理解もしておらず、論文書きを一部でも行なってい
ない場合、その論文に名前が載るのはいやである、というのが、私の基準です。」 

 \underlinejpn{(d)論文の著者に関して、全分野に共通の確固とした決
まりができることはないかもしれないが、論文を多作す る研究者も、今では、
充分監督したわけでない研究の論文に自分の名前を載せることの危険性を知っ
ているようだ 。}(99字)
%問題(c)の解答が不足している。



\SubAnswer
 {\bf 全訳}
%特に前半がひどい訳でしたので、独断で変更した箇所がいくつかあります。

 私は、自分の気持ちを口を酸っぱくするほど言わずにはいられない。実験物
理学者はいつも理論家の「提案」を受け入れてしまっている。しかし、現在の
状況は馬鹿げている。今や、理論家は、巨大な粒子加速器の日程会議の席上で、
非常にすぐれた実験家の出す提案に対し、拒否権を行使することができるので
ある。私が、原子核および素粒子物理で活動していたころは、理論家の拒否権
の行使の仕方は、まだ、受け入れられるものであった。彼らは、「馬鹿な実験」
をした人を誰でもいいから引き合いに出し、あざ笑ったのである。というわけ
で、もし、そのような社会的圧力により威嚇される人がいるとすれは、(実際
にはほとんどの人が威嚇されてしまうのであるが\ldots )効果的に、馬鹿な実
験を人がしなくすることができるのである。時々、馬鹿な実験だと気付かずに、
馬鹿な実験をしてしまう実験家がいて、その馬鹿げた結果を論文にしてしまう
のだが、そうすると、はるかに頭のいい理論家達に笑われてしまうのであった。
典型的な例は、R.T.Cox が1928年に発見した実験結果である。それは、ラジウ
ムEの$\beta$崩壊によって、放出される電子は、(右と左で違う確率で「二重
散乱」するので)偏極する、というものである。理論家は、この結果を受け入
れることが出来なかった。というのは、それは、彼らが神聖視していたパリティー
保存則を侵していたからである。そのために、Cox の重要な発見は、「正しく」
再発見されるまで視界から消え去ったのである。19年後、T.F.Lee とFrank
Yang が、素粒子物理の重大な問題を解決するために、もしかすると弱い相互
作用ではパリティーが保存されないのではないかもしれない、ということを提
案した。いくつかのグループがパリティーの非保存を確認する実験を設定し、
すぐに、Lee とYangの提案が、実際の世界の構造に適合したものであることが
見い出された。しかしながら、この「常識」の変革に対するノーベル賞は、
Cox が貰ったわけでも、Wu 夫人とその共同実験者達(Lee と Yang が正しいこ
とを証明した)が貰ったわけでもなかった。Lee と Yang が確かに栄冠を勝ち
得たわけだが、私は、実験家もそれを分かつべきであったと思う。

 来年は、Michelson-Morley の実験からちょうど百年である。あの非常に重要
な実験も、今日のような状況では行なわれなかったであろう。というのも、次
のような想像上の場面が展開されたであろうからだ。Michelson とMorley が
「日程会議」で、Michelson の新しい干渉計を使って、エーテル中の地球の速
度を測定する計画であることを報告する。会議に参加している理論天体物理学
者が質問する。「あなた方の方法だと、どの程度の精度で、地球の速度が測れ
るのですか?」Michelson が言う。「有効数字一桁です。」理論家が応える。
「しかし、エーテル中での地球の速度は、既に、天文観測から、有効数字2,~3
桁で求められているんですよ。」会議の全員が、Michelson とMorley の提案
が前代未聞のばかげたものであるということで一致し、2人の案は、にべもな
く却下されてしまう。という具合に、エーテルの概念を放棄するきっかけとなっ
た実験も行なわれないことになってしまう。

 \underlinejpn{かつての様に実験家がばかな実験をやらせてもらえ
るのなら、かつてパリティの保存やエーテルの存在が正しい ことをみんなが知っ
ていたように、「みんなが正しいことを知っている」物理概念のいったいいく
つが、実は間違い であると判明することかと思う。 }(118字)


  \begin{subsubanswers} 
    \SubSubAnswer
            全訳参照のこと。
  
    \SubSubAnswer
\baselineskip=12pt

     Theoretical physicists always try to keep experimental
    physicists from doing experiments that seem to be stupid from what
    is ``known'' to be true. In the past, theorists held up to
    ridicule anyone who did ``stupid experiments'', and by the social
    pressure, experimentalists were intimidated and kept from such
    experiments.  But still, it was possible to do such experiments
    accidentally by not knowing that it was stupid. And also, there
    were some ``stupid experiments'' which eventually revealed the
    falseness of a fact that everyone believed to be true. The present
    situation is ridiculous because theorists sit on the scheduling
    committees and can exercise veto power over the proposals by the
    experimentalists directly.  But if the experimentalists were again
    allowed to do ``stupid experiments'', they may reveal some of our
    physical misconception.  (137語)
  \end{subsubanswers}




\SubAnswer
    
     Up to about twenty-five years ago, it was thought that
    protons were ``elementary'' particles, but the experiment in which
    protons collided with other protons or electrons at high speeds
    indicated that they were in fact made up of smaller
    particles. These particles were named quarks.


\SubAnswer

\begin{subsubanswers}

 \SubSubAnswer
  By the time Achilles reaches the place where the turtle stayed 
at first, the turtle is proceeded ahead. When Achilles again reaches
the point where the turtle were, he has gone further.
 \SubSubAnswer
  The above statement neglected the convergence of the accumulated
time that Achilles spent in each steps.

\baselineskip=15pt
\end{subsubanswers}

\end{subanswers}
\end{answer}

\end{document}

   





