\documentclass[fleqn]{jbook}
\usepackage{physpub}

\begin{document}

\begin{question}{専攻 問題3}{}
面積$S=L_x\times L_y$の平面上は自由に動けるが,面に垂直な方向には動けない電子
の系(2次元電子系)を考える。$L_x,L_y$は,それぞれ平面の$x$方向,$y$方向の長
さであり,巨視的な大きさであるとする。また,電子間の相互作用は無視できるもの
とする。電子の質量を$m$,電子密度を$n$,ボルツマン定数を$k_B$,プランク定数を
$2\pi\hbar$,絶対温度を$T$とする。このとき,以下の設問に答えよ。

\begin{subquestions}
\SubQuestion
周期的境界条件を用いて,1個の電子に対する固有状態,固有エネルギーを求めよ。

\SubQuestion
状態密度$\rho(E)$は$dN(E)/dE$で定義される。$\rho(E)$および化学ポテンシ
ャル$\mu$を用いて,有限温度での全電子数を与える式をしるせ。この式のここでの状
態密度の表式を代入して,$\mu$を$n$および$T$の関数として求めよ。また,$E_F$と
$T$の関数としても表してみよ。

\SubQuestion
$k_BT\ll E_F$のときの比熱の最重要項を求めよ。このとき,必要ならば次の公
式を用いよ。
\[\int_{-\infty}^{\infty}x^2\frac{df}{dx}dx=-\frac{\pi^2}{3},
\hspace{1cm} ここで、f(x)=\frac{1}{e^x+1}\] である。

\SubQuestion
電子の状態密度が$E_F-\Delta E\leq E\leq E_F+\Delta E$で
$\rho(E)=|E-E_F|^{\alpha}$である系を考える。ここで,$\alpha>-1$である。
このとき十分低温($k_BT\ll\Delta E$)での比熱が温度の関数としてどのような振る
舞いをするか調べよ。

\end{subquestions}
\end{question}
\begin{answer}{専攻 問題3}{}

\begin{subanswers}
\SubAnswer
2次元での自由粒子の Schr\"{o}dinger 方程式は、 
\begin{equation}
 -\frac{\hbar^2}{2m}\left(\frac{\partial^2}{\partial x^2}+\frac{\partial^2}{\partial y^2}\right)\Psi 
=E\Psi \eqname{A1} \end{equation}
である。この式を$\Psi=X(x)Y(y)$と変数分離して解く\eqhref{A1}へ代入すれば、
\[ -\frac{\hbar^2}{2m}\left(\frac{X''}{X}+\frac{Y''}{Y}\right)=E \]
を得る。ここで、$X''/X=-k_x^2,Y''/Y=-k_y^2$とおけば、
\[X=A_xe^{ik_xx},Y=A_ye^{ik_yy}\]
と解が求まる。次に周期的境界条件により
$k_x,k_y$を求めると、
$X(x)=X(x+L_x)$より$e^{ik_xL_x}=1$、つまり、
$k_x=2\pi n_x/L_x\;(n_x=0,\pm1,\pm2,\cdots)$
が得られ、同様にして、$k_y=2\pi n_y/L_y\;(n_y=0,\pm1,\pm2,\cdots)$
が得られる。

従って、\eqhref{A1}の解は、
\[\Psi=X(x)Y(y)=Ae^{2i\pi\left(\frac{n_x}{L_x}x+\frac{n_y}{L_y}y\right)}
\;\;\;\;\;(A=A_xA_y) \]
と書け、さらに、規格化条件から、
$A=1/\sqrt{S}$となる。結局固有ベクトル、固有値はそれぞれ、
\begin{eqnarray*}
\Psi&=&\frac{1}{\sqrt{S}}e^{2i\pi\left(\frac{n_x}{L_x}x+
\frac{n_y}{L_y}y\right)} \\
E&=&\frac{\hbar^2}{2m}(k_x^2+k_y^2) =\frac{(2\pi\hbar)^2}{2m}\left\{\left(\frac{n_x}{L_x}\right)^2+\left(\frac{n_y}{L_y}\right)^2\right\}, \;\;\;\;( n_x,n_y=0,\pm1,\pm2,\cdots)
\end{eqnarray*}
となる。

\SubAnswer
あるエネルギー$E$までの状態数は、$n_x$-$n_y$面での$E=$一定
という線内の格子点数に対応する。

いま、$L_x,L_y$は十分に大きいとしているので、エネルギー準位の間隔は小さい
とできる。そこで、状態数を、軸の長さがそれぞれ$\sqrt{2mE}L_x/\pi\hbar,
\sqrt{2mE}L_y/\pi\hbar$の楕円の面積と考えることができる。

従って、
\begin{equation}
N(E)= 2\cdot\pi\cdot\sqrt{2mE}\frac{L_x}{2\pi \hbar}\cdot\sqrt{2mE}\frac{
L_y}{2\pi \hbar} = \frac{mS}{\pi \hbar^2}E \eqname{A2}
\end{equation}
ここで、因数2はスピンの自由度によるものである。

また、$N(E_F)=nS$より、
\[\frac{mS}{\pi\hbar^2}E_F=nS
\Longrightarrow E_F=\frac{\pi\hbar^2}{m}n
\]
が得られる。

有限温度$T$での全電子数$N$は、fermi 分布関数$f(E)$を用いて、
\[
N=\int_0^{\infty}\rho(E)f(E)dE 
=\int_{0}^{\infty}\rho(E)\frac{1}{e^{\frac{E-\mu}
{k_{\scriptscriptstyle{B}}T}}+1}dE
\]
と書き表せる。

さて、\eqhref{A2}式より、
$\rho(E)$は、
\[\rho(E)=\frac{dN(E)}{dE}=\frac{mS}{\pi\hbar^2}\]
を代入して、
\[
N=\frac{mS}{\pi\hbar^2}\int_0^{\infty}\frac{dE}{e^{
\frac{E-\mu}{k_{\scriptscriptstyle{B}}T}}+1} 
=\frac{mS}{\pi\hbar^2}
k_{\scriptscriptstyle{B}}T\ln\left(
e^{\frac{\mu}{k_{\scriptscriptstyle{B}}T}}+1\right)=nS
\]
となる。よって、$\mu$は、$T$と$n$、$T$と$E_F$を使って、それぞれ
\begin{equation}
\mu=k_{\scriptscriptstyle{B}}
T\ln\left(e^{\frac{\pi\hbar^2}{m}n
\frac{1}{k_{\scriptscriptstyle{B}}T}}-1\right)= k_{\scriptscriptstyle{B}}T\ln\left(
e^{\frac{E_F}{k_{\scriptscriptstyle{B}}T}}-1\right) \eqname{A3}
\end{equation}
と表せる。

\SubAnswer
内部エネルギー$U(T)$は、
\[
U(T)=\int_0^{\infty}E\rho(E)f(E)dE 
=\frac{mS}{\pi\hbar^2}\int_0^{\infty}\frac{EdE}
{e^{\frac{E-\mu}{k_{\scriptscriptstyle{B}}T}}+1} 
\]
である。ここで、低温近似
$k_{\scriptscriptstyle{B}}T\ll E_F$を使うと、\eqhref{A3}
式より$\mu\simeq E_F$として良いから、
$\beta\equiv 1/k_{\scriptscriptstyle{B}}T$として、
\[U(T)\simeq\frac{mS}{\pi\hbar^2}\int_0^{\infty}\frac{EdE}{e^{\beta(E-E_F)}
+1} \]
と表せる。すると、比熱$C(T)$は、
\[
C(T)=\frac{\partial U(T)}{\partial T} 
\simeq-\frac{1}{k_{\scriptscriptstyle{B}
}T^2}\frac{\partial }{\partial \beta}\left(\frac{mS}{\pi\hbar^2}
\int_0^{\infty}\frac{EdE}{e^{\beta(E-E_F)}+1}\right)
=\frac{mS}{\pi\hbar^2}\frac{1}{k_{\scriptscriptstyle{B}}T^2}\int_0^{\infty}
\frac{E(E-E_F)e^{\beta(E-E_F)}}{\left(e^{\beta(E-E_F)}+1\right)^2}dE
\]
ここで、$x\equiv \beta(E-E_F)$とすると、$dx=\beta dE$などより、
\[
C(T)=\frac{mS}{\pi\hbar^2}\frac{1}
{k_{\scriptscriptstyle{B}}T^2}\int_{-\beta E_F}^{\infty}
\frac{\left(\frac{x}{\beta}+E_F\right)\frac{x}{\beta}e^x}
{\left(e^x+1\right)^2}\frac{dx}{\beta}
= \frac{mS}{\pi\hbar^2}\frac{1}{k_{\scriptscriptstyle{B}
}T^2}\int_{-\beta E_F}^{\infty}
\frac{1}{\beta^3}\frac{(x+\beta E_F)xe^x}{\left(e^x+1\right)^2}dx 
\]

$1/\beta \ll E_F$、つまり、$\beta E_F\gg 1$より、
\begin{eqnarray*}
&\simeq&\frac{mS}{\pi\hbar^2}\frac{1}{k_{\scriptscriptstyle{B}}T^2}
\int^{\infty}_{-\infty}\frac{1}{\beta^3}\frac{(x+\beta E_F)x
e^x}{\left(e^x+1\right)^2}dx
\end{eqnarray*}
が得られる。

積分を計算する際に、$(x+\beta E_F)$という部分に注目すると、
残った部分が奇関数になっているから、第二項から来る積分は消える。また、
\[\frac{d}{dx}f(x)=\frac{d}{dx}\left(\frac{1}{e^x+1}\right)=
\frac{-e^x}{\left(e^x+1\right)^2}\]
であるから、結局、
\[
C(T)\simeq-\frac{mS}{\pi\hbar^2}
\frac{1}{k_{\scriptscriptstyle{B}}T^2}\int_{-\infty}^{\infty}
\frac{1}{\beta^3}x^2\frac{d}{dx}f(x)dx 
=-\frac{mS}{\pi\hbar^2}\frac{1}
{k_{\scriptscriptstyle{B}}T^2}(k_{\scriptscriptstyle{B}}T)^3
\left(-\frac{\pi^2}{3}\right) 
=\frac{mS\pi k_{\scriptscriptstyle{B}}^2}{3\hbar^2}T
\]
となる。

\SubAnswer
状態密度の仮定式$\rho(E)=|E-E_F|^{\alpha},\alpha>-1(
{\rm for} \; E_F-\Delta E \leq E \leq
 E_F+\Delta E)$を代入すると、内部エネルギーとして、
\[U(T)=\int_{E_F-\Delta E}^{E_F+\Delta E}E|E-E_F|^{\alpha}
\frac{dE}{e^{\beta(E-\mu)}+1} \]
を考えればよく、
$k_{\scriptscriptstyle{B}}T\ll E_F,\Delta E$で低温近似をすれば、
\begin{eqnarray*}
C(T)&\simeq& -\frac{1}{k_{\scriptscriptstyle{B}}T^2}
\int_{E_F-\Delta E}^{E_F+\Delta E}
\frac{\partial }{\partial \beta}\left(E|E-E_F|^{\alpha}\frac{1}{e^{\beta(E-E_F)}+1}\right)dE\\
&=&\frac{1}{k_{\scriptscriptstyle{B}}T^2}\int_{E_F-\Delta E}^{E_F+\Delta E}
E|E-E_F|^{\alpha}\frac{(E-E_F)e^{\beta(E-E_F)}}{\left(e^{\beta(E-E_F)}
+1\right)^2}dE 
\end{eqnarray*}
$x\equiv\beta(E-E_F)$とすると、
\begin{eqnarray*}
&=& \frac{1}{k_{\scriptscriptstyle{B}}T^2}\int_{-\beta\Delta E}^{\beta\Delta E}
\left(\frac{x}{\beta}+E_F\right)\left|\frac{x}{\beta}\right|^{\alpha}
\frac{\frac{x}{\beta}e^x}{\left(e^x+1\right)^2}\frac{dx}{\beta} \\
&=& \frac{1}{k_{\scriptscriptstyle{B}}T^2}
\cdot\beta^{-\alpha-3}\int_{-\infty}^{\infty}
(x+\beta E_F)|x|^{\alpha}\frac{xe^x}{\left(e^x+1\right)^2}dx \\
&=& \frac{1}{k_{\scriptscriptstyle{B}}T^2}
\beta^{-\alpha-3}\int_{-\infty}^{\infty}
\frac{|x|^{\alpha}x^2e^x}{\left(e^x+1\right)^2}dx 
\;\;\;(偶奇性より。{\bf{4}}番を参照)\\
&\propto& T^{\alpha+1}
\end{eqnarray*}
という結果を得る。

\end{subanswers}
\end{answer}


\end{document}
