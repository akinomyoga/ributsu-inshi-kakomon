\documentclass[fleqn]{jbook}
\usepackage{physpub}

\begin{document}

\begin{question}{専攻 問題1}{}
\begin{subquestions}
\SubQuestion

次のような一次元ポテンシャル$V(x)$中にある質量$m$,電荷$e$を持つ粒子を
考える。
\[V(x)=\left\{\begin{array}{ll}-V_0\cos x&-\pi\leq x\leq\pi \\
\infty & x<-\pi,x>\pi \end{array}\right.\]
ただし,$V_0$は正とする。

\begin{subsubquestions}
\SubSubQuestion

$x=0$の近傍でポテンシャルが調和近似できるとし
$(V(x)\sim-V_0(1-x^2/2))$,エネルギー固有値と固有関数を求めよ。また,$n$
番目の量子状態に対してこの近似が成り立つ条件を示せ。

ここで,$n$を整数とした微分方程式
\[\frac{d^2f}{d\xi^2}-2\xi\frac{df}{d\xi}+2nf=0\]
の解は,次の二つの式を満たすエルミート多項式$H_n(\xi)$を用いて与えられること
に注意せよ。
\begin{eqnarray*}
&\displaystyle{
H_n(\xi)=(-1)^n\exp(\xi^2)\frac{d^n}{d\xi^n}(\exp(-\xi^2))} &\\
&\displaystyle{\int_{-\infty}^{\infty}H_n(\xi)H_m(\xi)\exp(-\xi^2)d\xi=
\sqrt{\pi}
2^nn!\delta_{n,m}} \end{eqnarray*}

\SubSubQuestion
$x=0$の近傍で,ポテンシャルの非調和項
$\Delta V(x)\equiv V(x)+V_0(1-x^2/2)$を$x$について最低次の項で近似する。
エネルギー固有値のずれをこの項について一次の摂動計算で求めよ。

ここで,必要ならば,以下に与えられるエルミート多項式の漸化式を用いよ。
\[\xi H_n(\xi)=\frac{1}{2}H_{n+1}(\xi)+nH_{n-1}(\xi) \]

\SubSubQuestion
$x$方向に弱い電場$E$をかけたときのエネルギー固有値を,$V(x)$が調和近似
できるとして求めよ。このとき,$V(x)$の調和近似が成り立つための電場$E$の条件を
示せ。
\end{subsubquestions}


\SubQuestion
 次のような一次元ポテンシャル$V(x)$中にある質量$m$を持つ粒子を考える。
\[ V(x)=\left\{\begin{array}{ll}V_0(\cos x +1)&-2\pi\leq x\leq 2\pi \\
\infty & x<-2\pi,x>2\pi \end{array}\right.\]
ただし,$V_0$は正とする。

この粒子の$x$の期待値が$t=0$で正であるとき,$x$の期待値の時間発展を述べよ。
ここで,粒子のエネルギー期待値は$V_0$より十分低く,状態は基底状態と第一励起状
態だけを用いて記述できるとする。
\end{subquestions}

\end{question}
\begin{answer}{専攻 問題1}{}
\begin{subanswers}
\SubAnswer

\begin{subsubanswers}
\SubSubAnswer
$x\simeq 0$ で
\begin{eqnarray}
V(x) &\simeq & -V_{0}(1-\frac{x^{2}}{2}+\frac{x^{4}}{24}+\cdots )\nonumber\\
&\simeq &-V_{0}(1-\frac{x^{2}}{2}) \eqname{eq1}
\end{eqnarray}
と調和近似できたとする。この時Schr\"{o}dinger eq.は、
\begin{eqnarray}
\frac{\hbar^{2}}{2m}\frac{d^{2}}{dx^{2}}\psi +\left(
E+V_{0}-\frac{m\omega^{2}}{2}
x^{2}\right)\psi&=&0\nonumber \\
\frac{d^{2}}{dx^{2}}\psi +\left\{
 \frac{2m(E+V_{0})}{\hbar^{2}}-\left( \frac{m\omega}
{\hbar}\right)^{2}x^{2}\right\} \psi &=&0 \eqname{eq2}
(V_{0}=m\omega^{2}とした)
\end{eqnarray}
$x=\alpha X$の変数変換を考えると
\begin{displaymath}
\frac{d^{2}}{dx^{2}}=\frac{1}{\alpha^{2}}\frac{d^{2}}{dX^{2}},\ x^{2}=(\alpha X)^{2} 
\end{displaymath}
より、
\begin{equation}
\alpha^{4}=\left( \frac{\hbar}{m\omega}\right)^{2},\ \lambda =\frac{2m(E+V_{0})}
{\hbar^{2}}\alpha^{2}=\frac{2(E+V_{0})}{\hbar\omega} \eqname{eq3}
\end{equation}
とすると式\eqhref{eq2}は
\begin{equation}
\frac{d^{2}}{dX^{2}}\psi +(\lambda -X^{2})\psi=0 \eqname{eq4}
\end{equation}
となる。

ここで$u=e^{\pm \frac{X^{2}}{2}}$は$u''+(\mp 1-X^{2})u=0$をみたし、これは$X
\rightarrow \pm \infty$で式\eqhref{eq4}と一致するので式\eqhref{eq4}の解は近似の効く範囲で
$e^{\pm \frac{X^{2}}{2}}$に漸近する解を持つ。波動関数の収束からこのうち
$e^{-\frac{X^{2}}{2}}$が許される。式\eqhref{eq4}の一般解を求めるために、
\begin{displaymath}
\psi(X)=e^{-\frac{X^{2}}{2}}\varphi(X)
\end{displaymath}
と置いて式\eqhref{eq4}に代入すると、
\begin{equation}
\frac{d^{2}}{dX^{2}}\varphi
-2X\frac{d\varphi}{dX}+(\lambda-1)\varphi=0 \eqname{eq5}
\end{equation}
となる。この方程式の解は$(\lambda-1)=2n(n=0,1,2\cdots)$のときのみ$X\rightarrow
\infty$で$e^{-\frac{X^{2}}{2}}$をかけて収束する(そうでないときは、
$e^{-\frac{X^{2}}{2}}$をかけても発散する)。よって、
\begin{equation}
\lambda=2n+1(n=0,1,2\cdots) \eqname{eq6}
\end{equation}
式\eqhref{eq3}からこれは
\begin{equation}
E=E_{n}=(n+\frac{1}{2})\hbar\omega-V_{0} \eqname{eq7}
\end{equation}
一方、式\eqhref{eq7}に対する固有関数は式\eqhref{eq6}をみたすとき式\eqhref{eq5}の解が$n$次Hermite
多項式になることから、規格化定数$N$を用いて
\begin{equation}
\psi _{n}(X)=Ne^{-\frac{X^{2}}{2}}H_{n}(X)
\end{equation}
規格化条件$\displaystyle{\int_{-\infty}^{\infty}|\psi(x)|^{2}dx=1}$より
\begin{eqnarray*}
1&=&\int_{-\infty}^{\infty}N^{2}e^{-X^{2}}{H_{n}(X)}^{2}\alpha dX=\alpha N^{2}
\sqrt{\pi}2^{n}n!\\
よってN&=&(\alpha \sqrt{\pi}2^{n}n!)^{-\frac{1}{2}}=
\left\{\sqrt{\frac{\pi \hbar}
{m\omega}}2^{n}n!\right\}^{-\frac{1}{2}}\\
よって\psi_{n}(x)&=&\left\{\sqrt{\frac{\pi 
\hbar}{m\omega}}2^{n}n!\right\}^{-\frac{1}{2}}
\exp\left[-\frac{m\omega}{2\hbar}x^{2}\right]
H_{n}\left(\sqrt{\frac{m\omega}{\hbar}}x\right)
\end{eqnarray*}
ただしこの調和近似が良いのは
\begin{displaymath}
x\sim 0 \Longleftrightarrow E_{n}\sim -V_{0}\Longleftrightarrow 
\left(n+\frac{1}{2}\right)\hbar
\omega \ll V_{0}
\end{displaymath}
%

\SubSubAnswer
摂動項
\begin{eqnarray*}
\Delta V(x)&=&-V_{0}\left(\cos x -1+\frac{x^{2}}{2}\right)
\simeq -\frac{V_{0}}{24}x^{4}\\
\Delta E_{n}&=&\langle n|\Delta V(x)|n\rangle
=\int_{-\infty}^{\infty}N^{2}e^{-
\left(\frac{x}{\alpha}\right)^{2}}H_{n}^{2}\left(\frac{x}{\alpha}\right)
\left(-\frac{V_{0}}{24}x^{4}\right)dx\\
&=&\int_{-\infty}^{\infty}N^{2}e^{-X^{2}}H_{n}^{2}(X)\left(-\frac{V_{0}}{24}
\alpha^{4}X^{4}\right)\alpha dX\\
&=&-\frac{V_{0}}{24}N^{2}\alpha^{5}\int_{-\infty}^{\infty}e^{-X^{2}}H_{n}^{2}(X)X^{4}dX
\end{eqnarray*}
\begin{eqnarray*}
\lefteqn{\int_{-\infty}^{\infty}e^{-X^{2}}H_{n}^{2}(X)X^{4}dX}&&\\
&=&\int_{-\infty}^{\infty}e^{-X^{2}}(XH_{n}(X))^{2}X^{2}dX\\
&=&\int_{-\infty}^{\infty}e^{-X^{2}}\left[
\frac{1}{2}H_{n+1}+nH_{n-1}\right]^{2}X^{2}dX\\
&=&\int_{-\infty}^{\infty}e^{-X^{2}}\left[\frac{1}{4}(XH_{n+1})^{2}
+n(XH_{n+1})(XH_{n-1})+n^{2}(XH_{n-1})^{2}\right]dX\\
&=&\int_{-\infty}^{\infty}e^{-X^{2}}\left[\frac{1}{4}\left\{\frac{1}{2}H_{n+2}
+(n+1)H_{n}\right\}^{2}+n\left\{
\frac{1}{2}H_{n+2}+(n+1)H_{n}\right\}\right.\cdot\\
&&\cdot\left.\quad\left\{\frac{1}{2}H_{n}+(n-1)
H_{n-2}\right\}+n^{2}\left\{\frac{1}{2}H_{n}+(n-1)H_{n-2}\right\}^{2}\right]dX
\end{eqnarray*}
ここで$H_{n}$の直交性から積分すると$[\cdots]$の中のクロスタームは消えるので
\begin{eqnarray*}
&=&\int_{-\infty}^{\infty}e^{-X^{2}}\left[\frac{1}{4}\frac{1}{4}H_{n+2}^{2}
+\left\{\frac{1}{4}(n+1)^{2}+n(n+1)\frac{1}{2}+n^{2}\frac{1}{4}\right\}
H_{n}^{2}
+n^{2}(n-1)^{2}H_{n-2}^{2}\right]dX\\
&=&\int_{-\infty}^{\infty}e^{-X^{2}}\left[\frac{1}{16}H_{n+2}^{2}
+\frac{(2n+1)^{2}}{4}H_{n}^{2}+n^{2}(n-1)^{2}H_{n-2}^{2}\right]dX\\
&=&\frac{1}{16}\sqrt{\pi}2^{n+2}(n+2)!+\frac{(2n+1)^{2}}{4}\sqrt{\pi}2^{n}n!+
n^{2}(n-1)^{2}\sqrt{\pi}2^{n-2}(n-2)!\\
&=&\sqrt{\pi}2^{n}n!\left\{\frac{(n+1)(n+2)}{4}+\frac{(2n+1)^{2}}{4}+
\frac{n(n-1)}{4}\right\}\\
&=&\sqrt{\pi}2^{n}n!\frac{6n^{2}+6n+3}{4}
\end{eqnarray*}
よって
\begin{eqnarray*}
\Delta E_n&=&\langle n|\Delta V(x)|n\rangle
=-\frac{V_{0}}{24}N^{2}\alpha^{5}\sqrt{\pi}2^{n}n!
\frac{6n^{2}+6n+3}{4}\\
&=&-\frac{V_{0}}{24}\frac{\alpha^{5}\sqrt{\pi}2^{n}n!}{\alpha\sqrt{\pi}2^{n}n!}
\frac{6n^{2}+6n+3}{4}\\
&=&-\frac{V_{0}}{32}\left(\frac{\hbar}{m\omega}\right)^{2}(2n^{2}+2n+1)\\
&=&-\frac{\hbar^{2}}{32m}(2n^{2}+2n+1)\quad (V_{0}=m\omega^{2})
\end{eqnarray*}
%

\SubSubAnswer
\begin{eqnarray*}
&&\\*[-15mm]
V(x)&=&-V_{0}\cos x-eEx\\
&=&-V_{0}\left(1+\frac{eE}{V_{0}}x-\frac{1}{2}x^{2}+
\frac{1}{24}x^{4}-\cdots\right)\\
&=&-V_{0}\left(1-\frac{1}{2}\left(x-\frac{eE}{V_{0}}\right)^{2}
+\frac{e^{2}E^{2}}{2V_{0}^{2}}+
\frac{1}{24}x^{4}-\cdots\right)
\end{eqnarray*}
これが調和近似できるためには$\left(x-\frac{eE}{V_{0}}\right)^{2}$
の項の$x^{2}$からのずれが
\begin{displaymath}
\frac{eE}{V_{0}} \ll 1
\end{displaymath}
となれば良い。この時エネルギー固有値は$\displaystyle{
\frac{e^{2}E^{2}}{2V_{0}^{2}}}$の項の
ずれの効果で、
\begin{displaymath}
E_{n}=\left(n+\frac{1}{2}\right)\hbar\omega-V_{0}-\frac{e^{2}E^{2}}{2V_{0}}
\end{displaymath}

\end{subsubanswers}

%

\SubAnswer
系はground stateと1st excited stateだけで記述される。それぞれの状態の
エネルギー固有関数とエネルギー固有値は求まっているとする。\\
\begin{tabular}{ccc}
{ } & {エネルギー} & {波動関数}\\
ground & $E_{0}$ & $\psi_{0}(x)$\\
1st excited & $E_{1}$ & $\psi_{1}(x)$
\end{tabular}\\
一般にground stateと1st excited stateのパリティはそれぞれ正と負で
ある。

(注)正確には例えば$\psi_{0}$は中央がへこんでいるかも知れないがここで問題
なのはパリティだけである。

ここで$\psi_{0}(x),\psi_{1}(x)$の時間発展はSchr\"{o}dinger eq.
\begin{displaymath}
i\hbar\frac{\partial}{\partial t}\psi_{n}(x,t)=E_{n}\psi_{n}(x,t)\quad (n=0,1)
\end{displaymath}
で決まるので、
\[
\psi_{0}(x,t)=\exp\left[-\frac{iE_{0}t}{\hbar}\right]\psi_{0}(x), 
\qquad
\psi_{1}(x,t)=\exp\left[-\frac{iE_{1}t}{\hbar}\right]\psi_{1}(x)
\]
となる。したがって
\[
\psi(x,t)=C_0\psi_{0}(x,t)+C_1\psi_{1}(x,t)=\exp\left[-\frac{iE_{0}t}{\hbar}\right]\left\{C_0\psi_{0}(x)+C_1\exp\left[\frac{-i(E_{1}-E_{0})t}{\hbar}\right]\psi_{1}(x)\right\}
\]
よって、パリティを考慮すれば、
\[
\psi(x,t+\frac{\pi\hbar}{E_1-E_0}) = \psi(-x,t), \qquad
\psi(x,t+\frac{2\pi\hbar}{E_1-E_0})= \psi(x,t)
\]
よって、粒子は角振動数$\displaystyle{\omega=
\frac{E_{1}-E_{0}}{\hbar}}$で$x>0$と$x<0$との間を
行き来する。

\end{subanswers}

\end{answer}



\end{document}
