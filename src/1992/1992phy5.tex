\documentclass[fleqn]{jbook}
\usepackage{physpub}

\begin{document}

\begin{question}{専攻 問題5}{}
面積$S$の2次元空間を運動する2個の電子が短距離間の引力ポテンシャル
$-V_0\delta(\vec{r}_1-\vec{r}_2)$($V_0$は正の定数,$\delta(\vec{r})$は2次元
空間でのディラックのデルタ関数,$\vec{r}_1,\vec{r}_2$は電子の座標)によって相
互作用している。この時,2個の電子のスピンが逆向きであれば,束縛状態が出現す
る。重心は静止しているとして,次の順序で束縛エネルギーを求めよ。この束縛エネ
ルギーはシュレディンガー方程式
\begin{equation}
\left[\frac{\vec{p}_1^2}{2m}+\frac{\vec{p}_2^2}{2m}-V_0\delta(\vec{r}_1-
\vec{r}_2)\right]\psi(\vec{r}_1,\vec{r}_2)=\varepsilon\psi(\vec{r}_1,
\vec{r}_2) \eqname{Q1} \end{equation}
の最低固有値により決定される。($m$は電子の質量,$\vec{p}_1,\vec{p}_2$
は運動量)

\begin{subquestions}
\SubQuestion
2個の電子のスピンが逆向きの場合,空間部分の波動関数
$\psi(\vec{r}_1,\vec{r}_2)$は$\vec{r}_1$と$\vec{r}_2$の入れ換えに関してどのよ
うな性質を持つか説明せよ。2個の電子のスピンが同じ向きの場合の\\
$\psi(\vec{r}_1,\vec{r}_2)$はどうなるか。また,このとき引力ポテンシャルは
$\psi(\vec{r}_1,\vec{r}_2)$にどのような影響を及ぼすか。

\SubQuestion
 $\psi(\vec{r}_1,\vec{r}_2)$をフーリエ展開すると波数$\vec{k}$に関して偶対
称である関数$g(\vec{k})$を用いて
\begin{equation}
\psi(\vec{r}_1,\vec{r}_2)=\sum_{\vec{k}}e^{i\vec{k}\cdot(\vec{r}_1-\vec{r}_2)}
g(\vec{k}) \eqname{Q2} \end{equation}
と書けることを示せ。

\SubQuestion
シュレディンガー方程式を$g(\vec{k})$を用いて表現すると,次式が得られる
ことを示せ。
\begin{equation}
\left[2\varepsilon(\vec{k})g(\vec{k})-V_0\frac{1}{S}\sum_{\vec{q}}
g(\vec{k}-\vec{q})\right]=\varepsilon g(\vec{k}) \eqname{Q3} 
\end{equation}
ここで,$\varepsilon(\vec{k})=\hbar^2k^2/2m$であり,$\delta(\vec{r})
=\frac{1}{S}\sum_{\vec{q}}e^{i\vec{q}\cdot\vec{r}}$に注意する。

\SubQuestion
\eqhref{Q3}式を解くために,$\sum_{\vec{q}}g(\vec{k}-\vec{q})$は$k$に依存しない
ことに注意し,$\frac{1}{S}\sum_{\vec{q}}g(\vec{k}-\vec{q})=C$と書く。この定数
$C$を用いると,\eqhref{Q3}は$g(\vec{k})$について解けて
\[g(\vec{k})=\frac{V_0C}{2\varepsilon(\vec{k})-\varepsilon} \]
となり,$\varepsilon$を決定する方程式が次式で与えられることを示せ。
\begin{equation}
\frac{V_0}{S}\sum_{\vec{k}}\frac{1}{2\varepsilon(\vec{k})-\varepsilon}=1
\eqname{Q4}
\end{equation}

\SubQuestion
 \eqhref{Q4}の左辺を,$\vec{k}$に関する積分に書き直す。このとき,$\vec{k}$についての積分の領域が$\varepsilon(\vec{k})\leq \varepsilon_0$($\varepsilon_0$は適
当な正のエネルギー)であるとして$\varepsilon$を求めよ。\eqhref{Q4}を満たす
$\varepsilon$は負となることを確認せよ。特に,$mV_0/2\pi\hbar^2\ll1$の場合,束
縛エネルギー$E=-\varepsilon$は
\[
E=2\varepsilon_0\exp\left[-\frac{4\pi\hbar^2}{mV_0}\right]
\]
で与えられることを示せ。

\end{subquestions}
\end{question}
\begin{answer}{専攻 問題5}{}

\begin{subanswers}
\SubAnswer
2つの電子のスピンが逆向きのとき、
つまり状態ケットのスピン成分が反対称のときは、
\[
\psi(\vec{r}_1,\vec{r}_2)=\psi(\vec{r}_2,\vec{r}_1)
\]
逆に、状態ケットのスピン成分が対称(スピンが同じ向き)ならば、
\[
\psi(\vec{r}_1,\vec{r}_2)=-\psi(\vec{r}_2,\vec{r}_1)
\]
となる。

$\psi(\vec{r}_1,\vec{r}_2)$ が反対称状態のとき $r_1=r_2$ なら $\psi=0$ より
\[
\delta(\vec{r}_1-\vec{r}_2)\psi(\vec{r}_1,\vec{r}_2)=0
\]
になり、このとき Schr\"{o}dinger 方程式は自由粒子の場合と同じで、
束縛状態は出現しない。

$\psi(\vec{r}_1,\vec{r}_2)$ が対称状態のとき、結合エネルギーの期待値は
\[
\int \psi^{*}(\vec{r}_1,\vec{r}_2)
\{-V_0 \delta(\vec{r}_1-\vec{r}_2) \} \psi(\vec{r}_1,\vec{r}_2)
d^{2}r_1 d^{2}r_2 = -V_0 \int | \psi(\vec{r},\vec{r})|^2 d^{2}r < 0
\]
となりエネルギーが下がる。これにより束縛状態が出現する。

\SubAnswer
重心は静止しているから
\begin{equation}
(\vec{p_{1}}+\vec{p_{2}}) \psi(\vec{r}_1,\vec{r}_2) = 0
\eqname{cm}
\end{equation}
である。ここで重心座標$\vec{R}$、相対座標$\vec{r'}$を
\[
\vec{R} \equiv \frac {1}{2} (\vec{r_{1}} + \vec{r}_2),\hspace{0.5cm}
\vec{r'} \equiv \vec{r_{1}} - \vec{r}_2
\]
とおき
\[
\psi(\vec{r}_1,\vec{r}_2) \equiv \psi(\vec{R},\vec{r'})
\]
とすると\eqhref{Q1}式は、\eqhref{cm}式を使って、
\[
(\vec{p_{1}}+\vec{p_{2}}) \psi(\vec{r}_1,\vec{r}_2) =
 \frac {\hbar}{i} \left(\frac{\partial}{\partial \vec{r}_1 } + 
\frac{\partial}{\partial \vec{r}_2 }\right)\psi(\vec{r}_1,\vec{r}_2) =
\frac {\hbar}{i} \frac{\partial}{\partial \vec{R}} \psi(\vec{R},\vec{r'}) = 0
\]

となり、$\psi(\vec{R},\vec{r'})$ は $\vec{r'}$ のみの関数であることがわかる。
したがって
\[ 
\psi(\vec{r}_1,\vec{r}_2) = \sum_{\vec{k}} e^{i\vec{k}(\vec{r}_1 - \vec{r}_2)}
g(\vec{k})
\]
のように展開することができる。

束縛状態では $\psi(\vec{r}_1,\vec{r}_2)=\psi(\vec{r}_2,\vec{r}_1)$
だから

\[
\sum_{\vec{k}} e^{i\vec{k}(\vec{r}_1 - \vec{r}_2)}g(\vec{k}) =
\sum_{\vec{k}} e^{i\vec{k}(\vec{r}_1 - \vec{r}_2)}g(-\vec{k})
\]
となり、$g(\vec{k}) = g(-\vec{k})$である。すなわち$g(\vec{k})$は偶対称である。

\SubAnswer
\begin{eqnarray*}
\\*[-15mm]
\vec{p_{1}}^2+\vec{p_{2}}^2 &=& \frac {1}{2} \left\{ 
(\vec{p_{1}}+\vec{p_{2}})^2 - 
 (\vec{p_{1}}-\vec{p_{2}})^2 \right\}\\
\vec{p_{1}}+\vec{p_{2}} &=& \frac {\hbar}{i} 
\left (\frac{\partial}{\partial \vec{r}_1 } +
\frac{\partial}{\partial \vec{r}_2 } \right) =
\frac {\hbar}{i} \frac{\partial}{\partial \vec{R}}\\
\vec{p_{1}}-\vec{p_{2}} &=& \frac {\hbar}{i} 
\left(\frac{\partial}{\partial \vec{r}_1 } -
\frac{\partial}{\partial \vec{r}_2 }\right) =
\frac {2\hbar}{i} \frac{\partial}{\partial \vec{r'}}
\end{eqnarray*}
を用いて Schr\"{o}dinger 方程式を書き換える。$\psi(\vec{R},\vec{r'})$
は$\vec{R}$に依存しないので
\[
\left[\frac {\vec{p_{1}}^2}{2m} + \frac {\vec{p_{2}}^2}{2m} -
V_0 \delta(\vec{r'}) \right]   
\psi(\vec{r'}) =
\left[ - \frac {\hbar^2} {m} \frac {d^2}{d \vec{r'}^2} - V_0 \delta(\vec{r'}) 
\right] 
\psi(\vec{r'}) =
\varepsilon \psi(\vec{r'})
\]
となる。この式に
\[
\psi(\vec{r'}) = \sum_{\vec{k}} e^{i\vec{k}\vec{r'}} g(\vec{k}), 
\hspace{0.5cm}
\delta(\vec{r'}) = \frac {1}{S} \sum_{\vec{q}} e^{i\vec{q}\vec{r'}}
\]
を代入すると
\[
\sum_{\vec{k}} \frac {\hbar^2 k^2} {m} e^{i\vec{k}\vec{r'}} g(\vec{k}) -
\frac {V_0}{S} \sum_{\vec{k}} \sum_{\vec{q}} e^{i(\vec{k}+\vec{q})\vec{r'}}
g(\vec{k}) = \varepsilon \sum_{\vec{k}} e^{i\vec{k}\vec{r'}} g(\vec{k})
\]
となり、$e^{i\vec{k}\vec{r'}}$の成分に着目すれば
\begin{equation}
\left[\frac {\hbar^2 k^2} {m} g(\vec{k}) - \frac {V_0}{S} \sum_{\vec{q}}
g(\vec{k}-\vec{q})\right] = \varepsilon g(\vec{k})
\eqname{8}
\end{equation}
を得る。

\SubAnswer
\begin{eqnarray*}
\\*[-12mm]
\frac {1}{S} \sum_{\vec{q}} g(\vec{k}-\vec{q}) = C
\end{eqnarray*}
を\eqhref{8}に代入すれば直ちに
\begin{equation}
g(\vec{k})=\frac{V_0C}{2\varepsilon(\vec{k})-\varepsilon}
\eqname{g9}
\end{equation}
を得る。そして
\[
\frac{1}{S}\sum_{\vec{q}}g(\vec{k}-\vec{q})= C
\]
に$\vec{k} = 0$を代入し$g(\vec{q}) = g(-\vec{q})$を用い、添字を$\vec{q}$
から$\vec{k}$に
変えると
\begin{equation}
\frac {1}{S} \sum_{\vec{k}} g(\vec{k}) = C
\eqname{sum}
\end{equation}

となり\eqhref{g9}を\eqhref{sum}に代入すると
\[
\frac{V_0}{S}\sum_{\vec{k}}\frac{1}{2\varepsilon(\vec{k})-\varepsilon}=1
\]
となる。
\SubAnswer
\begin{eqnarray*}
\\*[-15mm]
\frac{V_0}{S}\sum_{\vec{k}}\frac{1}{\frac {\hbar^2 k^2} {m}-\varepsilon}=1
\end{eqnarray*}
これを積分に直すと
\[
\frac {S} {4\pi^2} \frac {V_0}{S}
\int^{k_0}_{0} \frac {2\pi k}{\frac {\hbar^2 k^2} {m}-\varepsilon} dk =
\frac {mV_0}{4\pi \hbar^2} \int^{\varepsilon_0}_{0}
\frac {dE} {E- \frac {\varepsilon} {2}} = 1
\]
となる。
$\varepsilon > 0$ のときは積分が発散するので $\varepsilon < 0$ でなければ
ならない。このとき
\[
\frac {mV_0}{4\pi \hbar^2} \log (1-\frac {2\varepsilon_0}{\varepsilon}) = 1
\]
\[
\varepsilon = \frac {2\varepsilon_0}{1- \exp \frac{4\pi \hbar^2}{mV_0}} < 0
\]
特に$\displaystyle{
\frac {mV_0}{2\pi \hbar^2} \ll 1}$の場合、分母の 1 が無視できて
\[
E = 2\varepsilon_0 \exp \left(- \frac {4\pi \hbar^2}{mV_0}\right)
\]
となる。

\end{subanswers}

\end{answer}


\end{document}



