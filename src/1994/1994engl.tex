\documentclass[fleqn]{jbook}
\usepackage{physpub}

\begin{document}

\begin{question}{$B650i(B $B1Q8l(B}{}

\begin{subquestions}
\SubQuestion
  $B0J2<$NJ8>O$O2J3XO@J8$NCx<T$H$7$F<i$k$Y$-$3$H$r=R$Y$?$b$N$G$"$k!#(B
  $B3F9`L\$rOBLu$;$h!#(B

\baselineskip=12pt
  \begin{subsubquestions}
  \SubSubQuestion
    An author's central obligation is to present a concise,
    accurate account of the research performed as well as an objective
    discussion of its significance.

  \SubSubQuestion
    A paper should contain sufficient detail and reference
    to public sources of information to permit the author's peers to
    repeat the work.

  \SubSubQuestion
    An author should cite those publications that have been influential
    in determining the nature of the reported work and that will guide 
    the reader quickly to the earlier work that is essential for
    understanding the present investigation.

  \SubSubQuestion
    Fragmentation of research papers should be avoided. A scientist
    who has done extensive work on a system should organize 
    publication so that each paper gives a complete account of a
    particular aspect of the general study.
 
  \SubSubQuestion
    It is inappropriate for an author to submit manuscripts describing
    essentially the same research to more than one journal of primary
    publication.

  \SubSubQuestion
    A criticism of a published paper may sometimes be justified;
    however, in no case is personal criticism considered to be
    appropriate.

  \SubSubQuestion
    Only persons who have significantly contributed to the research 
    and paper preparation should be listed as authors. The author
    who submits a manuscript for publication attests to the fact that
    any others named as authors have seen the final version of the
    paper and have agreed to its submission for publication.

  \end{subsubquestions}
\baselineskip=15pt



\SubQuestion
  $B0J2<$NJ8>O$O!"$"$k9q:]2q5D$K$*$1$k%]%9%?!<%;%C%7%g%s$G$N2qOC$G$"$k!#(B
  $BF|K\8lItJ,$r1Q8l$KJQ$($h!#(B

  \begin{list}{}{\itemindent=0mm \topsep=0mm \itemsep=-0.5mm \labelsep=1mm}
  \item[A:]  Hello. Is this poster yours?
  \item[B:]  Yes.
  \item[A:]  Would you mind if I ask you some questions?
  \item[B:]  (1)$B$($(!"$I$&$>!#(B
  \item[A:]  We tried this.......
  \item[B:]  (2)$B$"$J$?$NOC$7$F$$$k$3$H$,!"$o$+$j$^$;$s!#(B\\
             (3)$B$9$_$^$;$s$,!"$b$&0lEY$$$C$F$$$?$@$1$^$;$s$+!#(B
  \item[A:]  We tried this before --- but....
  \item[B:]  (4)$B$^$@$h$/$o$+$j$^$;$s!#(B\\
             (5)$B$9$_$^$;$s$,!"$b$C$H$f$C$/$jOC$7$F$$$?$@$1$^$;$s$+!#(B\\
             (6)$B$(!<$H!"K\$rFI$`$_$?$$$K!#(B
  \item[A:]  All right. People --- in my group...
  \item[B:]  (7)$B$O$$!"$o$+$j$^$9!#(B\\
             (8)$B$=$l$G!"$4<ALd$O2?$G$9$+!)(B
  \item[A:]  I'd like to know --- how.....
  \item[B:]  (9)A $B<R$NH?1~%-%C%H(B(reaction kit)$B$r;dC#$N8&5f<<$G2~NI$7$?(B
             $BJ}K!$G;H$$$^$7$?!#(B\\
             (10)$BJ}K!$K$D$$$F$O!"(B3$BKgL\$N?^$r8fMw2<$5$$!#(B\\
             (11)$B@:EY!"46EY!":F8=@-$H$b$KM%$l$?$3$NJ}K!$N>\:Y$O!"J88%(B4
             $B$K$"$j$^$9!#(B\\
             (12)$BB,Dj$O(B B $B<R$NAuCV$GDL>o$NJ}K!$G9T$J$$$^$7$?!#(B\\
             (13)$BH?1~;~4V$O(B30$BJ,$G$9!#(B\\
             (14)$B?^(B7$B$N%0%i%U$O!";dC#$N7k2L$r%W%m%C%H$7$?$b$N$G$9!#(B\\
             (15)$BFs<oN`$N:YK&$K$OH?1~@-$KBg$-$J0c$$$,$"$j$^$9!#(B
  \item[A:]  My question is .....
  \item[B:]  (16)$BNI$/$o$+$j$^$;$s!#(B
  \item[A:]  (In a lower voice) Maybe it's all my fault.---- 
             I am not clear enough.
  \item[B:]  (17)$B$(!"$J$s$G$9$C$F!#(B\\
             (18)$B$^$@2?$+<ALd$,$"$k$N$G$9$+!#(B\\
             (19)$B$I$&$>1sN8$J$/$*?R$M2<$5$$!#(B
  \item[A:]  (20)$B$3$N;E;v$OO@J8$K$J$C$F$$$^$9$+!#(B
  \item[B:]  (21)$BEj9F$7$?$H$3$m$G$9!#(B\\
             (22)$B<uM}$5$l$?$i!"$*Aw$j$7$^$7$g$&$+!#(B
  \item[A:]  (23)$B$O$$!"@'Hs$*4j$$$7$^$9!#(B\\
             (24)$B$3$l$,;d$N%"%I%l%9$G$9!#(B\\
             Thank you very much for your helpful discussion.
  \item[B:]  (25)$B$$$($I$&$b!#(B
  \end{list}




\SubQuestion
  $B0J2<$NJ8>O$rFI$_J8Cf$K=R$Y$i$l$F$$$kFbMF$K1h$$@_Ld$KF|K\8l$GEz$($h!#(B
\baselineskip=12pt

  $B!!(BHydrogen was prepared many years before it was recognized as a
  distinct substance by Cavendish in 1766. It was named by Lavoisier.
  Hydrogen is the most abundant of all elements in the universe, and
  it is thought that the heavier elements were, and still are, being
  built from hydrogen and helium. It has been estimated that hydrogen
  makes up more than 90\% of all the atoms or three quarters of the
  mass of the universe. It is found in the sun and most stars, and
  plays an important part in the proton-proton reaction and
  carbon-nitrogen-oxygen cycle, which accounts for the energy of the
  sun and stars. It is thought that hydrogen is a major component of
  the planet Jupiter and that at some depth in the planet's interior 
  the pressure is so great that solid molecular hydrogen is converted
  into solid metallic hydrogen. On earth, hydrogen occurs chiefly in
  combination with oxygen in water, but it is also present in organic
  matter such as living plants, petroleum, coal, etc. It is present
  as the free element in the atmosphere, but only to the extent of
  less than 1 ppm, by volume. It is the lightest of all gases, and
  combines with other elements, sometimes explosively, to form
  compounds. Great quantities of hydrogen are required commercially
  for the fixation of nitrogen from the air in the Haber ammonia
  process and for the hydrogenation of fats and oils. It is also used
  as a rocket fuel, for welding, for production of hydrochloric acid, 
  for the reduction of metallic ores, and for filling ballons. It is
  prepared by the action of steam on heated carbon, by decomposition
  of certain hydrocarbons with heat, by the electrolysis of water, or
  by displacement from acids by certain metals. Liquid hydrogen is
  important in cryogenics and in the study of superconductivity as its
  melting point is only about ten degrees above absolute zero. In 1932,
  Urey announced the preparation of a table isotope, deuterium with an
  atomic weight of 2. Two years later an unstable isotope, tritium, 
  with an atomic weight of 3 was discovered. Tritium has a half-life
  of about 12.5 years. The atom of deuterium is found mixed in with
  about 6000 ordinary hydrogen atoms. Tritium atoms are also present
  but in much smaller proportion. Tritium is readily produced in
  nuclear reactors and is used in the production of the hydrogen bomb.
%
  \begin{flushright}
  -- quoted, with modiffications,   from CRC Handbook of Chemistry and Physics,1984.
  \end{flushright}
\baselineskip=15pt
%
  \begin{subsubquestions}
  \SubSubQuestion
    $B?eAG$NCO5e>e$G$N:G$b<gMW$JB8:_7ABV$O$J$K$+!#(B
  \SubSubQuestion
    $B?eAGJ,;R$OCO5eBg5$>eCf$K$I$NDxEY$"$k$+!#(B
  \SubSubQuestion
    $B?eAG$N@=B$K!$rFs$D$"$2$h!#(B
  \SubSubQuestion
    $B?eAG$O<BMQ>e2?$NLr$KN)$C$F$$$k$+!#Fs$D$"$2$h!#(B
  \SubSubQuestion
    $B1'Ch$K$*$$$F?eAG$O!"<ANLHf$K$7$F$I$l$@$1B8:_$9$k$+!#(B
  \SubSubQuestion
    $B?eAG$NF10LBN$K$O$I$N$h$&$J$b$N$,$"$j!"$=$l$i$NB8:_Hf$O(B
    $B$I$N$h$&$K$J$C$F$$$k$+!#(B
  \SubSubQuestion
    $B?eAG$NM;E@$O$I$NDxEY$+!#(B
  \end{subsubquestions}

\end{subquestions}
\end{question}
\begin{answer}{$B650i(B $B1Q8l(B}{}

\begin{subanswers}
\SubAnswer

  \begin{subsubanswers}
  \SubSubAnswer
    $BCx<T$N<gMW$J@UL3$O!"8&5f$N0U5A$K$D$$$F$N5R4QE*$J5DO@$K2C$(!"9T$C$?(B
    $B8&5f$K$D$$$F$N4J7i$G@53N$J5-=R$r9T$J$&$3$H$K$"$k!#(B

  \SubSubAnswer
   $BO@J8$K$O!"Cx<T$NF16H<T$,$=$N8&5f$r:F8=$9$k$3$H$,$G$-$k$h$&$K!"==J,(B
   $B$J>\:Y$H8x$N>pJs8;$X$N>H2q$,<}$a$i$l$F$$$k$Y$-$G$"$k!#(B

  \SubSubAnswer
    $BCx<T$O!"8xI=$7$?8&5f$NK\<A$r7hDj$E$1$k$&$($G1F6ANO$N$"$C$?=PHGJ*(B
    $B$d!"$=$N8&5f$rM}2r$9$k>e$GIT2D7g$J0JA0$N8&5f$rFI<T$,$9$_$d$+$K8+(B
    $B$D$1$k$?$a$N<j=u$1$H$J$k=PHGJ*$K$D$$$F8@5Z$9$Y$-$G$"$k!#(B

  \SubSubAnswer
    $BCGJRE*$JO@J8$N4s$;=8$a!"$H$$$&>uBV$OHr$1$i$l$k$Y$-$G$"$k!#(B
    $B$R$H$D$N$3$H$,$i$K$D$$$F9-HO0O$K$o$?$k8&5f$r9T$C$?2J3X<T$O!"$R$H$D(B
    $B$R$H$D$NO@J8$,!"8&5fA4BN$N$&$A$N$"$k0l$D$NLL$r40A4$K5-=R$9$k$h$&$K(B
    $B!"=PHG$r7OE}N)$F$F9T$&$Y$-$G$"$k!#(B

  \SubSubAnswer
    $BCx<T$,!"K\<AE*$K$OF1$88&5f$r=R$Y$?869F$r!":G=i$K8xI=$r9T$C$?;(;o0J(B
    $B30$KEj9F$9$k$N$O!"ITE,@Z$G$"$k!#(B

  \SubSubAnswer
    $B8xI=$5$l$?O@J8$KBP$9$kHcH=$O!"$H$-$K$O@5Ev2=$5$l$k$@$m$&!#$7$+$7!"(B
    $B$I$s$J$3$H$,$"$C$F$b8D?M8~$1$NHcH=$,BEEv$@$H$_$J$5$l$k$3$H$O$J$$!#(B

  \SubSubAnswer
    $B8&5f$dO@J8$N=`Hw$K=EMW$J9W8%$r$7$??MJ*$N$_$,!"Cx<T$H$7$FL>A0$r7G:\(B
    $B$5$l$k$Y$-$G$"$k!#=PHGMQ$N869F$rEj9F$9$kCx<T$O!"B>$NCx<T$H$7$FL>A0(B
    $B$r$"$2$i$l$F$$$k$9$Y$F$N?M$,O@J8$N:G=*HG$r3NG'$7$?$3$H$H!"$=$l$r=P(B
    $BHGMQ$KEj9F$9$k$3$H$KF10U:Q$_$G$"$k$3$H$N>Z?M$J$N$G$"$k!#(B
  \end{subsubanswers}

\baselineskip=12pt
\SubAnswer
\begin{list}{}{\itemindent=0mm \topsep=0mm \labelsep=1mm}
  \item[(1)]  Sure, go ahead please. / Of course, please go ahead.
% added 24 jul. 2001 by yuji tachikawa
$B$H$$$&$N$OE57?E*$J4V0c$$$G!"(B I \textbf{don't} mind. $B$r<($9$?$a$K(B Of course not. $B$J$I$H8@$o$M$P$J$i$J$$!#(B
  \item[(2)]  I don't understand what you meanare talking about.
  \item[(3)]  Sorry, but could you repeat it? / Excuse me, but could you explain it again?
  \item[(4)]  I still cannot understand you. / Still it's not clear for me.
  \item[(5)]  Excuse me, but would you please speak slower? /  ........., but please speak more slowly.
  \item[(6)]  Uh, as if you are reading me a book. / Uh, like reading a book.
  \item[(7)]  Yes, I understand.
  \item[(8)]  So, what is your answer?
  \item[(9)]  We used the reaction kit of ``A'' corporation in a method we improved in our group. / We used the ``A'' corporation reaction kit and we used it with an improved method which we developed in our group.
  \item[(10)]  The method is shown in the figure on 3rd page. / You can see the method .....
  \item[(11)]  For further details of this highly accurate, sensitive, and reproducible method you should refer to literature 4.
  \item[(12)]  We measured them in a usual method with the device of ``B'' corporation. / The measurements were taken in a usual way......
  \item[(13)]  The reaction time was 30 minutes.
  \item[(14)]  Figure 7 is our plotted results.
  \item[(15)]  There is a large difference in the reactivity between these two kinds of cell.
  \item[(16)]  I don't know well. / I'm not sure.
  \item[(17)]  What?
  \item[(18)]  Do you have any other answer?
  \item[(19)]  Please tell me without reserve.
  \item[(20)]  Is this work published in a paper? / Have you written this work down in any paper?
  \item[(21)]  I've just submitted it.
  \item[(22)]  Shall I send it to you when it is accepted?
  \item[(23)]  Yes, please.
  \item[(24)]  This is my address.
  \item[(25)]  You are welcome.
  \end{list}
\baselineskip=15pt

\SubAnswer 


{\bf $BA4Lu(B} 

 $B!!(B1776$BG/!"(BCavendish$B$K$h$C$F#1$D$N85AG$G$"$k$HG'<1$5$l$k$:$C$HA0$+$i!"(B
 $B?eAG$O@=B$$5$l$F$$$?!#(BLavoisier$B$K$h$C$F!"?eAG$HL>IU$1$i$l$?!#?eAG$O1'(B
 $BCh$K$*$$$F$9$Y$F$N85AG$NCf$G$b$C$H$bK-IY$KB8:_$7$F$*$j!"=E85AG$O!"?e(B
 $BAG$d%X%j%&%`$+$i!":n$i$l!"8=:_$G$b:n$i$l$F$$$k$H9M$($i$l$F$$$k!#?eAG(B
 $B$OA486;R$NB8:_Hf$G(B99$B!s!"1'Ch$NA4<ANL$N#4J,$N#3$r@j$a$F$$$k$HM=A[$5$l(B
 $B$F$$$k!#?eAG$OB@M[$d$[$H$s$I$N@1$G8+$D$+$j!"B@M[$d@1!9$GH/@8$9$k%(%M(B
 $B%k%.!<8;$H$J$k(Bp-p$BH?1~$d(BCNO$B%5%$%/%k$K=EMW$JLr3d$r2L$?$7$F$$$k!#?eAG$O(B
 $BLZ@1$N<g@.J,$G$"$j!"LZ@1$NFbIt$G$O!"9b05$N$?$a!"?eAG$,J,;R@-8GBN$+$i(B
 $B6bB0@-8GBN$X$HJQ2=$7$F$$$k?<$5$,$"$k$H!"9M$($i$l$F$$$k!#CO5e>e$G$O<g(B
 $B$K?eCf$G;@AG$H7k9g$7$?7A$GB8:_$9$k$,!"?"J*!&@PL}!&@PC:Ey$NM-5!J*Cf$K(B
 $B$bB8:_$7$F$$$k!#?eAG$O!"Bg5$Cf$KJ,;R$H$7$F$bB8:_$7$F$$$k$,!"(B1ppm$B$K$b(B
 $BK~$?$J$$!#?eAG$O!"A45$BN$NCf$G:G$b7Z$$5$BN$G!"GzH/$7$J$,$i7k9g$72=9g(B
 $BJ*$r:n$k$3$H$,$"$k!#?eAG$O9)6HE*$K%O!<%P!<%"%s%b%K%"K!$K$h$k6u5$Cf$N(B
 $BCbAG$N8GDj$d!"L};i$d@PL}$N?eAGE:2C$KBgNL$K;HMQ$5$l$k!#$^$?!"%m%1%C%H(B
 $BG3NA$dMO@\!"1v;@$N@=B$!"9[@P$N4T85!"IwA%$X$N=<E6$K$b;HMQ$5$l$F$$$k!#(B
 $B$^$?!"@=B$K!$H$7$F$O!"2CG.C:AGI=LL$G$N?e>x5$$NH?1~!&C:2=?eAG$NG.J,2r!&(B
 $B?e$NEE5$J,2r!&6bB0$H;@$NCV49H?1~$,$"$k!#1UBN?eAG$O!"Dc293X!"$^$?!"D6(B
 $BEAF3$N8&5f$K$*$$$F=EMW$G$"$k!#$H$$$&$N$b!"?eAG$NM;E@$OLs(B10K$B$7$+$J$$$?(B
 $B$a$G$"$k!#(B1932$BG/!"(BUrey$B$O86;RNL#2$N=E?eAG$H$$$&J|<M@-85AG$NB8:_$rH/8+(B
 $B$7$?!##2G/8e!"86;RNL#3$N%H%j%A%e%&%`$,H/8+$5$l$?!#%H%j%A%e%&%`$NH>8:(B
 $B4|$OLs(B12.5$BG/$G$"$k!#=E?eAG$OLs(B6000$B8D$NDL>o$N?eAG$K#18D4^$^$l$F$$$k!#(B
 $B%H%j%A%e%&%`$bB8:_$7$F$$$k$,!"$=$N3NN($OMZ$+$K>.$5$$!#%H%j%A%e%&%`$O(B
 $B86;R3KH?1~$K$h$j:n@.$5$l!"?eGz$N@=B$$K;HMQ$5$l$F$$$k!#(B

%
  \begin{flushright}
  -- CRC $BJ*M}2=3X%O%s%I%V%C%/(B(1984)$B$+$i=$@5$r;\$7$F0zMQ(B
  \end{flushright}
%

 
\begin{subsubanswers} 

 \SubSubAnswer
  $B;@AG$H2=9g$7$F?e$H$7$FB8:_$7$F$$$k!#(B

 \SubSubAnswer
 $BMF@Q$K$7$F(B 1ppm $BL$K~$G$"$k!#(B

 \SubSubAnswer
 $BLdBjJ8Cf$K$O(B4$BDL$j$N@=K!$,$+$+$l$F$$$k!#(B

 \begin{itemize} 
   \item $B2CG.$5$l$?9u1t(B($BC:AG(B)$B$H?e>x5$$H$NH?1~(B
   \item $B$"$k<o$NC:2=?eAG$NJ,2r(B
   \item $B?e$NEE5$J,2r(B
    \item $B;@$+$i6bB0$HCV49!#(B($BMW$9$k$K;@$K6bB0$rMO$+$7$FH/@8$5$;$k(B) 
 \end{itemize}

 \SubSubAnswer
 $B$3$l$KBP$9$kEz$($bLdBjJ8Cf$K$O$?$/$5$s$"$k!#(B
 \begin{quote}
 $B%O!<%P!<%"%s%b%K%"K!$K$h$kCbAG8GDj!"L};i$N?eAGE:2C(B 
 \par
 $B%m%1%C%HG3NA!"MO@\!"1v;@$N86NA!"6bB09[@P$N4T85(B 
 \par
 $BIwA%$KF~$l$kL\E*(B 
 \par
 $BDc29J*M}!"D6EAF3$N8&5fMQ(B
 \end{quote} 

 \SubSubAnswer 
  4$BJ,$N(B3
 \SubSubAnswer 
  $B<ANL?t(B2$B$N=E?eAG$,Ls(B6000$BJ,$N(B1$B!#<ANL?t(B3$B$N;0=E?eAG$,$5$i$K(B
 $BHyNL$"$k!#(B
 \SubSubAnswer
  $BLs@dBP(B10$BEY(B

\end{subsubanswers}
\end{subanswers}
\end{answer}


\end{document}
