\documentclass[fleqn]{jbook}
\usepackage{physpub}

\begin{document}

\begin{question}{教育 英語}{}

\begin{subquestions}
\SubQuestion
  以下の文章は科学論文の著者として守るべきことを述べたものである。
  各項目を和訳せよ。

\baselineskip=12pt
  \begin{subsubquestions}
  \SubSubQuestion
    An author's central obligation is to present a concise,
    accurate account of the research performed as well as an objective
    discussion of its significance.

  \SubSubQuestion
    A paper should contain sufficient detail and reference
    to public sources of information to permit the author's peers to
    repeat the work.

  \SubSubQuestion
    An author should cite those publications that have been influential
    in determining the nature of the reported work and that will guide 
    the reader quickly to the earlier work that is essential for
    understanding the present investigation.

  \SubSubQuestion
    Fragmentation of research papers should be avoided. A scientist
    who has done extensive work on a system should organize 
    publication so that each paper gives a complete account of a
    particular aspect of the general study.
 
  \SubSubQuestion
    It is inappropriate for an author to submit manuscripts describing
    essentially the same research to more than one journal of primary
    publication.

  \SubSubQuestion
    A criticism of a published paper may sometimes be justified;
    however, in no case is personal criticism considered to be
    appropriate.

  \SubSubQuestion
    Only persons who have significantly contributed to the research 
    and paper preparation should be listed as authors. The author
    who submits a manuscript for publication attests to the fact that
    any others named as authors have seen the final version of the
    paper and have agreed to its submission for publication.

  \end{subsubquestions}
\baselineskip=15pt



\SubQuestion
  以下の文章は、ある国際会議におけるポスターセッションでの会話である。
  日本語部分を英語に変えよ。

  \begin{list}{}{\itemindent=0mm \topsep=0mm \itemsep=-0.5mm \labelsep=1mm}
  \item[A:]  Hello. Is this poster yours?
  \item[B:]  Yes.
  \item[A:]  Would you mind if I ask you some questions?
  \item[B:]  (1)ええ、どうぞ。
  \item[A:]  We tried this.......
  \item[B:]  (2)あなたの話していることが、わかりません。\\
             (3)すみませんが、もう一度いっていただけませんか。
  \item[A:]  We tried this before --- but....
  \item[B:]  (4)まだよくわかりません。\\
             (5)すみませんが、もっとゆっくり話していただけませんか。\\
             (6)えーと、本を読むみたいに。
  \item[A:]  All right. People --- in my group...
  \item[B:]  (7)はい、わかります。\\
             (8)それで、ご質問は何ですか?
  \item[A:]  I'd like to know --- how.....
  \item[B:]  (9)A 社の反応キット(reaction kit)を私達の研究室で改良した
             方法で使いました。\\
             (10)方法については、3枚目の図を御覧下さい。\\
             (11)精度、感度、再現性ともに優れたこの方法の詳細は、文献4
             にあります。\\
             (12)測定は B 社の装置で通常の方法で行ないました。\\
             (13)反応時間は30分です。\\
             (14)図7のグラフは、私達の結果をプロットしたものです。\\
             (15)二種類の細胞には反応性に大きな違いがあります。
  \item[A:]  My question is .....
  \item[B:]  (16)良くわかりません。
  \item[A:]  (In a lower voice) Maybe it's all my fault.---- 
             I am not clear enough.
  \item[B:]  (17)え、なんですって。\\
             (18)まだ何か質問があるのですか。\\
             (19)どうぞ遠慮なくお尋ね下さい。
  \item[A:]  (20)この仕事は論文になっていますか。
  \item[B:]  (21)投稿したところです。\\
             (22)受理されたら、お送りしましょうか。
  \item[A:]  (23)はい、是非お願いします。\\
             (24)これが私のアドレスです。\\
             Thank you very much for your helpful discussion.
  \item[B:]  (25)いえどうも。
  \end{list}




\SubQuestion
  以下の文章を読み文中に述べられている内容に沿い設問に日本語で答えよ。
\baselineskip=12pt

   Hydrogen was prepared many years before it was recognized as a
  distinct substance by Cavendish in 1766. It was named by Lavoisier.
  Hydrogen is the most abundant of all elements in the universe, and
  it is thought that the heavier elements were, and still are, being
  built from hydrogen and helium. It has been estimated that hydrogen
  makes up more than 90\% of all the atoms or three quarters of the
  mass of the universe. It is found in the sun and most stars, and
  plays an important part in the proton-proton reaction and
  carbon-nitrogen-oxygen cycle, which accounts for the energy of the
  sun and stars. It is thought that hydrogen is a major component of
  the planet Jupiter and that at some depth in the planet's interior 
  the pressure is so great that solid molecular hydrogen is converted
  into solid metallic hydrogen. On earth, hydrogen occurs chiefly in
  combination with oxygen in water, but it is also present in organic
  matter such as living plants, petroleum, coal, etc. It is present
  as the free element in the atmosphere, but only to the extent of
  less than 1 ppm, by volume. It is the lightest of all gases, and
  combines with other elements, sometimes explosively, to form
  compounds. Great quantities of hydrogen are required commercially
  for the fixation of nitrogen from the air in the Haber ammonia
  process and for the hydrogenation of fats and oils. It is also used
  as a rocket fuel, for welding, for production of hydrochloric acid, 
  for the reduction of metallic ores, and for filling ballons. It is
  prepared by the action of steam on heated carbon, by decomposition
  of certain hydrocarbons with heat, by the electrolysis of water, or
  by displacement from acids by certain metals. Liquid hydrogen is
  important in cryogenics and in the study of superconductivity as its
  melting point is only about ten degrees above absolute zero. In 1932,
  Urey announced the preparation of a table isotope, deuterium with an
  atomic weight of 2. Two years later an unstable isotope, tritium, 
  with an atomic weight of 3 was discovered. Tritium has a half-life
  of about 12.5 years. The atom of deuterium is found mixed in with
  about 6000 ordinary hydrogen atoms. Tritium atoms are also present
  but in much smaller proportion. Tritium is readily produced in
  nuclear reactors and is used in the production of the hydrogen bomb.
%
  \begin{flushright}
  -- quoted, with modiffications,   from CRC Handbook of Chemistry and Physics,1984.
  \end{flushright}
\baselineskip=15pt
%
  \begin{subsubquestions}
  \SubSubQuestion
    水素の地球上での最も主要な存在形態はなにか。
  \SubSubQuestion
    水素分子は地球大気上中にどの程度あるか。
  \SubSubQuestion
    水素の製造法を二つあげよ。
  \SubSubQuestion
    水素は実用上何の役に立っているか。二つあげよ。
  \SubSubQuestion
    宇宙において水素は、質量比にしてどれだけ存在するか。
  \SubSubQuestion
    水素の同位体にはどのようなものがあり、それらの存在比は
    どのようになっているか。
  \SubSubQuestion
    水素の融点はどの程度か。
  \end{subsubquestions}

\end{subquestions}
\end{question}
\begin{answer}{教育 英語}{}

\begin{subanswers}
\SubAnswer

  \begin{subsubanswers}
  \SubSubAnswer
    著者の主要な責務は、研究の意義についての客観的な議論に加え、行った
    研究についての簡潔で正確な記述を行なうことにある。

  \SubSubAnswer
   論文には、著者の同業者がその研究を再現することができるように、十分
   な詳細と公の情報源への照会が収められているべきである。

  \SubSubAnswer
    著者は、公表した研究の本質を決定づけるうえで影響力のあった出版物
    や、その研究を理解する上で不可欠な以前の研究を読者がすみやかに見
    つけるための手助けとなる出版物について言及すべきである。

  \SubSubAnswer
    断片的な論文の寄せ集め、という状態は避けられるべきである。
    ひとつのことがらについて広範囲にわたる研究を行った科学者は、ひとつ
    ひとつの論文が、研究全体のうちのある一つの面を完全に記述するように
    、出版を系統立てて行うべきである。

  \SubSubAnswer
    著者が、本質的には同じ研究を述べた原稿を、最初に公表を行った雑誌以
    外に投稿するのは、不適切である。

  \SubSubAnswer
    公表された論文に対する批判は、ときには正当化されるだろう。しかし、
    どんなことがあっても個人向けの批判が妥当だとみなされることはない。

  \SubSubAnswer
    研究や論文の準備に重要な貢献をした人物のみが、著者として名前を掲載
    されるべきである。出版用の原稿を投稿する著者は、他の著者として名前
    をあげられているすべての人が論文の最終版を確認したことと、それを出
    版用に投稿することに同意済みであることの証人なのである。
  \end{subsubanswers}

\baselineskip=12pt
\SubAnswer
\begin{list}{}{\itemindent=0mm \topsep=0mm \labelsep=1mm}
  \item[(1)]  Sure, go ahead please. / Of course, please go ahead.
% added 24 jul. 2001 by yuji tachikawa
というのは典型的な間違いで、 I \textbf{don't} mind. を示すために Of course not. などと言わねばならない。
  \item[(2)]  I don't understand what you meanare talking about.
  \item[(3)]  Sorry, but could you repeat it? / Excuse me, but could you explain it again?
  \item[(4)]  I still cannot understand you. / Still it's not clear for me.
  \item[(5)]  Excuse me, but would you please speak slower? /  ........., but please speak more slowly.
  \item[(6)]  Uh, as if you are reading me a book. / Uh, like reading a book.
  \item[(7)]  Yes, I understand.
  \item[(8)]  So, what is your answer?
  \item[(9)]  We used the reaction kit of ``A'' corporation in a method we improved in our group. / We used the ``A'' corporation reaction kit and we used it with an improved method which we developed in our group.
  \item[(10)]  The method is shown in the figure on 3rd page. / You can see the method .....
  \item[(11)]  For further details of this highly accurate, sensitive, and reproducible method you should refer to literature 4.
  \item[(12)]  We measured them in a usual method with the device of ``B'' corporation. / The measurements were taken in a usual way......
  \item[(13)]  The reaction time was 30 minutes.
  \item[(14)]  Figure 7 is our plotted results.
  \item[(15)]  There is a large difference in the reactivity between these two kinds of cell.
  \item[(16)]  I don't know well. / I'm not sure.
  \item[(17)]  What?
  \item[(18)]  Do you have any other answer?
  \item[(19)]  Please tell me without reserve.
  \item[(20)]  Is this work published in a paper? / Have you written this work down in any paper?
  \item[(21)]  I've just submitted it.
  \item[(22)]  Shall I send it to you when it is accepted?
  \item[(23)]  Yes, please.
  \item[(24)]  This is my address.
  \item[(25)]  You are welcome.
  \end{list}
\baselineskip=15pt

\SubAnswer 


{\bf 全訳} 

  1776年、Cavendishによって1つの元素であると認識されるずっと前から、
 水素は製造されていた。Lavoisierによって、水素と名付けられた。水素は宇
 宙においてすべての元素の中でもっとも豊富に存在しており、重元素は、水
 素やヘリウムから、作られ、現在でも作られていると考えられている。水素
 は全原子の存在比で99%、宇宙の全質量の4分の3を占めていると予想され
 ている。水素は太陽やほとんどの星で見つかり、太陽や星々で発生するエネ
 ルギー源となるp-p反応やCNOサイクルに重要な役割を果たしている。水素は
 木星の主成分であり、木星の内部では、高圧のため、水素が分子性固体から
 金属性固体へと変化している深さがあると、考えられている。地球上では主
 に水中で酸素と結合した形で存在するが、植物・石油・石炭等の有機物中に
 も存在している。水素は、大気中に分子としても存在しているが、1ppmにも
 満たない。水素は、全気体の中で最も軽い気体で、爆発しながら結合し化合
 物を作ることがある。水素は工業的にハーバーアンモニア法による空気中の
 窒素の固定や、油脂や石油の水素添加に大量に使用される。また、ロケット
 燃料や溶接、塩酸の製造、鉱石の還元、風船への充填にも使用されている。
 また、製造法としては、加熱炭素表面での水蒸気の反応・炭化水素の熱分解・
 水の電気分解・金属と酸の置換反応がある。液体水素は、低温学、また、超
 伝導の研究において重要である。というのも、水素の融点は約10Kしかないた
 めである。1932年、Ureyは原子量2の重水素という放射性元素の存在を発見
 した。2年後、原子量3のトリチュウムが発見された。トリチュウムの半減
 期は約12.5年である。重水素は約6000個の通常の水素に1個含まれている。
 トリチュウムも存在しているが、その確率は遥かに小さい。トリチュウムは
 原子核反応により作成され、水爆の製造に使用されている。

%
  \begin{flushright}
  -- CRC 物理化学ハンドブック(1984)から修正を施して引用
  \end{flushright}
%

 
\begin{subsubanswers} 

 \SubSubAnswer
  酸素と化合して水として存在している。

 \SubSubAnswer
 容積にして 1ppm 未満である。

 \SubSubAnswer
 問題文中には4通りの製法がかかれている。

 \begin{itemize} 
   \item 加熱された黒鉛(炭素)と水蒸気との反応
   \item ある種の炭化水素の分解
   \item 水の電気分解
    \item 酸から金属と置換。(要するに酸に金属を溶かして発生させる) 
 \end{itemize}

 \SubSubAnswer
 これに対する答えも問題文中にはたくさんある。
 \begin{quote}
 ハーバーアンモニア法による窒素固定、油脂の水素添加 
 \par
 ロケット燃料、溶接、塩酸の原料、金属鉱石の還元 
 \par
 風船に入れる目的 
 \par
 低温物理、超伝導の研究用
 \end{quote} 

 \SubSubAnswer 
  4分の3
 \SubSubAnswer 
  質量数2の重水素が約6000分の1。質量数3の三重水素がさらに
 微量ある。
 \SubSubAnswer
  約絶対10度

\end{subsubanswers}
\end{subanswers}
\end{answer}


\end{document}
