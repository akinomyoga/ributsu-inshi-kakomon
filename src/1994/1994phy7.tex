\documentclass[fleqn]{jbook}
\usepackage{physpub}

\begin{document}

\begin{question}{専攻 問題7}{}

生物の細胞分化に関する次の文章を読んで、それにつづく問に簡潔に
答えよ。\\

ヒトの脳は約100億の神経細胞からできているがヒトの遺伝子の総数は約10万
に過ぎない。このように多細胞生物に細胞の数に比べて遺伝子の数は極めて
少ない。多数の細胞の細胞分化とその個性を少数の遺伝子によって決定せね
ばならないというこの困難を、生物は進化の過程で「(転写制御)遺伝子の
組合せによるコーディング(Combinatorial Coding)」という機構を獲得する
ことによって克服した。

\begin{subquestions}
\SubQuestion
  生物が少数の遺伝子の組合せによる細胞の分化・多様性を実現していると
  いう事実について考えてみよう。

  \begin{subsubquestions}
  \SubSubQuestion
    各遺伝子の ON/OFF の組合せで100億の脳細胞の一つ一つに”番号づけ”
    をするとしたら、遺伝子は何個以上必要か? $\log_{10}2 = 0.3 $ と
    せよ。

  \SubSubQuestion
    上で計算した最小に近い数の遺伝子ですべての脳細胞の分化を実現した
    とすると、その生物は進化の過程で不利であると予想される。その理由
    を述べよ。

  \SubSubQuestion
    遺伝子 A と B が ON で、かつ C が OFF の時に遺伝子 X の転写活性が
    高まるという場合、遺伝子 X の調節領域でどの様な分子機構が働いて
    いるかを予想して述べよ。特に転写調節のレベルで考えられる可能性
    をあげよ。

  \end{subsubquestions}

\SubQuestion
  生物における「組合せによるコーディング」のもう一つの例は遺伝子の本体
  である DNA にみられる。現在地球上の生物の DNA は、A(アデニン)・
  T(チミン)・ G(グアニン)・ C(シトシン)の4塩基の''文字列''ですべての
  情報を暗号化している。

  \begin{subsubquestions}
  \SubSubQuestion
    DNA が2塩基( P と Q と名づける)のみから構成される生物が宇宙の
    どこかに存在するとしたら、P と Q との間にはどの様な関係があると
    期待されるか?

  \SubSubQuestion
    この宇宙生物が2塩基組成の RNA を介して20種類のアミノ酸からなる
    タンパク質をコードするとしたら、遺伝暗号表はどの様なものになると
    予想されるか?

  \SubSubQuestion
    2つの塩基のみからなる DNA を持つこの宇宙生物が、4塩基 DNA を持つ
    地球上の生物に比べて生存に不利な点をあげよ。

  \SubSubQuestion
    タンパク質についても同様に考えてみよう。2種のアミノ酸のみからなる
    タンパク質で構成される生物が存在し得るか? タンパク質の構造と
    機能の多様性を実現するのに何種類のアミノ酸が必要かを推論し、
    その理由を述べよ。

  \end{subsubquestions}
\end{subquestions}
\end{question}
\begin{answer}{専攻 問題7}{}

\begin{subanswers}
\SubAnswer
  \begin{subsubanswers}
  \SubSubAnswer
    $100$億の脳細胞の一つ一つに2進数の”番号づけ”をする。遺伝子が
    $n$個以上必要であるとすると、$2^{n} = 10^{10} $ だから、 
    $n=33.3$ となり、遺伝子は $ 34 $ 以上必要である。

  \SubSubAnswer
    すべての脳細胞の1つ1つを $ 34 $ 個の遺伝子でコードしたとすると、
    進化の過程で起こる突然変異や転写のミスにより、必要な脳細胞が発現
    できなくなったりすると生存できなくなる恐れがあるので、進化の過程
    で不利となる。

  \SubSubAnswer
    遺伝子 X を構造遺伝子と考え、その転写活性が高まるという場合に
    ついて考える。このとき遺伝子 X の調節領域では、プロモーターに
    RNA ポリメラーゼが結合し、かつオペレーターにはリプレッサーが
    結合していない時転写が開始される。さらに転写量を増加させる
    エンハンサーなどの配列が活性化されると転写活性が高まる。よって、
    遺伝子 A と B としてプロモーターとエンハンサーを対応させ、C を
    オペレーターに対応させると遺伝子 A と B が ON で、かつ C が OFF
    の時に遺伝子X の転写活性が高まる。 

  \end{subsubanswers}


\SubAnswer
  \begin{subsubanswers}
  \SubSubAnswer
    DNA が2塩基( P と Q と名づける)のみから構成される生物でも DNA の
    2重らせんで遺伝情報が保存されているとすると、P と Q との間には
    (P-P 、Q-Q )、P-Q  のいずれかのペアができるがこのうち、後者の場合
    は、2つの塩基とも同じ数の水素結合のための手を持たねばならないので
    塩基対の形成時にミスが生じやすいので不可。一方前者の場合、2つの
    塩基の水素結合の手の数を変えておけば上のようなミスは起きずに済む。
    よって、前者のような関係があると思われる。

  \SubSubAnswer
    2塩基組成の RNA を介して20種類のアミノ酸からなるタンパク質を
    コードする時のコドンの大きさは $ 2^5 = 32 $ より、5文字であれば
    よい。遺伝暗号表は5文字のコドンを表現するため普通のコドン表で、
    縦に始めの2文字、横に次の2文字で分類し、最後の文字でコラムを分
    ければよい。

  \SubSubAnswer
    4塩基の DNA では $ 4^3 = 64 $ 個の組合せで20個のアミノ酸をコード
    しているので、コドンの縮訳が起こりコード情報の保存性が高められた
    り、転写時のミスが少なくなるが、2塩基の場合32個しかコードして
    いないので上に述べた利点は少なくなり、さらにコドンが長くなるので
    突然変異の確率が高くなるので不利である。

  \SubSubAnswer
    2種のアミノ酸のみからなるタンパク質で構成される生物が存在し得ない。
    理由としてはタンパク質は、生体内では親水基を外側に、疎水基を内側
    にして存在しているので、最低親水基と疎水基を持ったアミノ酸が必要
    である。さらにタンパク質の機能の多様性を得るためにいくつかのアミノ
    酸が必要なので、2種アミノ酸のみからなる生物は存在しないと考えら
    れる。では何種類のアミノ酸が必要になるかを考える。生体内のアミノ
    酸は側鎖の化学的性質により、中性で非極性、中性で極性、酸性、塩基
    性の4つに分けられることから、まずその4種類は最低必要である。この
    他に、タンパク質の3次構造形成のためグリシンのような側鎖の小さい
    ものや、S-S 結合を作るために S を側鎖に持ったものが必要である。
    この他タンパク質の酵素反応の活性部位でのタンパク質の識別のために
    いくつかのアミノ酸が必要だと思われる。ゆえに最低6種類以上のアミノ
    酸が必要だと予想される。

  \end{subsubanswers}
\end{subanswers}
\end{answer}


\end{document}
