\documentclass[fleqn]{jbook}
\usepackage{physpub}
\usepackage{txfonts}
\ifx\inshiNotDefined\varg
  \def\varg{g}
\fi
\begin{document}
\begin{question}{問題9}{山地 洋平}
慣性も面と$I$を持った異核$2$原子分子の回転運動について、次の問に答えよ。ただし、ボルツマン定数を$k_B$とし、比熱は1分子あたりの比熱とする。
\begin{enumerate}
\item 古典統計力学を使って、この系の温度$T$での回転運動の比熱を求めよ。
\item この系の回転運動を表すハミルトニアンは
\begin{eqnarray}
H=\frac{L^2}{2I}=\frac{\hbar^2}{2I}\cdot\frac{-1}{\sin^2\theta}\Bigl( \sin\theta\frac{\partial}{\partial\theta} \sin\theta\frac{\partial}{\partial\theta}+\frac{\partial^2}{\partial^2\phi} \Bigl)
\end{eqnarray}
である。回転量子数$l$($l=0,1,2,\cdots$)を使って、このハミルトニアンの固有値$E_l$とその縮重度$\varg_l$を表せ(導出過程は不要)。
\item この系の分配関数$Z$とエネルギー期待値$\langle E\rangle$を、$E_l$、$\varg_l$、温度$T$を使って表せ。
\item 低温極限で有効な近似を用いて$Z$の表式を求め、その近似の範囲で、系の比熱$C(T)$を$E_l$、$\varg_l$、$T$を使って表せ。またその近似が成り立つ条件について述べよ。
\end{enumerate}
次に慣性モーメント$I$の等角2原子分子の回転運動について考える。2つの原子の原子核はともにスピン1/2のフェルミ粒子であるとして、次の問に答えよ。ただし電子系の合成スピンはゼロであるとする。
\begin{enumerate}
\item 核の合成スピンはどのような値を取りうるかを考え、それぞれの場合について回転運動のエネルギー固有値と縮重度を求めよ。
\item 設問5で求めたエネルギー固有値の具体的な表式を使って、低温極限での回転運動の比熱を求めよ。
\item 十分高温で熱平衡に達した分子気体を急冷した直後に測定した低温比熱$C_1$と、熱平衡を保ちながらきわめてゆっくり冷却した後測定した低温比熱$C_2$では、違いがあったという。その理由として考えられることを簡単に述べよ。ただしいずれの場合も、分子気体は冷却後も気体のままであったとする。
\end{enumerate}
\end{question}
\begin{answer}{問題9}{高見 一}
参考)オルソ水素、パラ水素
\begin{enumerate}

\item Hamiltonianの回転部分は、$\displaystyle H_{rot} = 
\frac{1}{2 I}\left( {p_{\theta}}^2 + \frac{1}{\sin ^2 \theta} {p_{\phi}}^2
 \right)$
と書けるのでpartition functionは、
\begin{eqnarray*}
Z_{rot} &=& \frac{1}{(2 \pi \hbar)^2} \int_0 ^\pi d\theta 
\int_0 ^{2 \pi} d\phi \int _{-\infty} ^ {\infty} dp_{\theta} 
\int _{-\infty} ^ {\infty} dp_{\phi} 
\exp \left\{ -\frac{1}{k_B T}\frac{1}{2I} 
\left( {p_{\theta}}^2 + \frac{1}{\sin ^2 \theta} {p_{\phi}}^2 \right) 
\right\}\\
&=& \frac{2 I k_B T}{\hbar^2}
\end{eqnarray*}
である。従ってHelmholzのfree energyは、
\begin{eqnarray*}
F = -k_B T \ln Z_{rot} = -k_B T \ln \frac{2 I k_B T}{\hbar^2}
\end{eqnarray*}
となる。$\displaystyle S = -\left( \frac{\partial F}{\partial T} 
\right)_V ~~~~ C = T \left( \frac{\partial S}{\partial T} \right)$だから、
\begin{eqnarray*}
C = k
\end{eqnarray*}

(別)$\displaystyle \langle E\rangle = - \frac{\partial}{\partial \beta} \ln Z_{rot} = k_BT
, C = \frac{\partial E}{\partial T} = k$

\item $\displaystyle \Lambda \equiv \frac{1}{\sin \theta} 
\frac{\partial}{\partial \theta}\left( \sin \theta 
\frac{\partial}{\partial \theta} \right) + \frac{1}{\sin^2 \theta}
\frac{\partial^2}{\partial \phi^2}$の固有値は$l(l + 1)$であるから、
\begin{eqnarray*}
E_l = \frac{\hbar^2 l(l + 1)}{2 I}
\end{eqnarray*}
縮重度は$g_l = 2l + 1$。

\item partition functionは
\begin{eqnarray*}
Z_{rot} = \sum_{l=0}^{\infty} g_l \exp \left(- \frac{E_l}{k_BT} \right)
\end{eqnarray*}
またエネルギーは$\displaystyle \beta \equiv \frac{1}{k_BT}$として、
$\displaystyle \langle E\rangle = - \frac{\partial}{\partial \beta} \ln Z_{rot}$であるから、
$\beta$の微分を$T$の微分に書き換えて、
\begin{eqnarray*}
\langle E\rangle = k_B T^2 \frac{\partial}{\partial T} \ln \left\{ \sum_{l=0}^{\infty}
g_l \exp \left(- \frac{E_l}{k_BT} \right)\right\}
\end{eqnarray*}

\item 低温ではほとんど回転励起はしないと考えられるので、級数をl=1までとって、
\begin{eqnarray*}
Z_{rot} = 1 + 3 \exp \left( -  \frac{E_1}{k_B T} \right)
\end{eqnarray*}
とできる。($E_0$は簡単なので、数字に直してしまった。)前問の結果を
用いれば比熱は、
\begin{eqnarray*}
C &=& \frac{\partial \langle E\rangle}{\partial T} \\
&=& \frac{\partial}{\partial T}\frac{3 E_1}{\exp \left( \frac{E_1}{k_BT} + 3 
\right)} \\
&=& \frac{3 {E_1}^2}{k_B T^2}\frac{\exp \left(\frac{E_1}{k_BT} \right)}
{\left[ \exp \left( \frac{E_1}{k_BT} \right) + 3 \right]^2}
\end{eqnarray*}

\item 核の合成スピンは1, 0, -1をとりうる。
このうちs=1,-1,0の片方はspin tripretを作り(対称スピン波動関数)、
s=0のもう片方はspin singletを作る(反対称スピン波動関数)。
このスピン波動関数を用いて全体の波動関数は、(回転波動関数)$\times$
(スピン波動関数)で表される。さて、核がfermionであることから、
この波動関数は全体として反対称でなくてはならず、

  1.(対称)$\times$(反対称)
  
  2.(反対称)$\times$(対称)

のみ許される。ここで、回転の波動関数がl:evenの時対称、l:oddの時反対称で
あることを考えると、エネルギー固有値は、
\begin{eqnarray*}
E_l = \frac{\hbar^2 l(l+1)}{2 I}
\end{eqnarray*}で変わらず、ただ1.の時l:even、2.の時l:oddになる。
同様に縮重度は$g_l = 2l + 1$で1.の時l:even、2.の時l:oddになる。

\item 5.の事情を加味するとpartition functionは、
\begin{eqnarray*}
Z_{rot} = 3 Z_{rott} + 1 Z_{rots}
\end{eqnarray*}
となる。ただし、
\begin{eqnarray*}
Z_{rott} \equiv \sum_{l = 0,even} (2l + 1) \exp \left( - \frac{\hbar^2 l (l+1)}
{2Ik_BT}\right)
\end{eqnarray*}
\begin{eqnarray*}
Z_{rots} \equiv \sum_{l = odd} (2l + 1) \exp \left( - \frac{\hbar^2 l (l+1)}
{2Ik_BT}\right)
\end{eqnarray*}
低温極限では4.と同様にして求められる。やはりl=1までをとると、
$\displaystyle Z_{rott} = 1 , Z_{rots} = 3 \exp \left(-\frac{\hbar^2}{Ik_BT}
\right)$
となるので、比熱は
\begin{eqnarray*}
C = \frac{3}{4}C_{rott} + \frac{1}{4}C_{rots} = 0
\end{eqnarray*}


\item 核がspin tripretを作っている場合の分子とsingletを作っている場合の
分子の数を考えてみる。partition functionの比が個数比になることに注意する。
即ち、
\begin{eqnarray*}
n \equiv = \frac{N_{rott}}{N_{rots}} = \frac{3 Z_{rott}}{1 Z_{rots}}
\end{eqnarray*}
十分高温ではZの中のexponentをTaylor展開して一次項を取るという近似が
許されるので、実際計算してみるとn = 3と見なせる。同様の計算を
低温極限で行うと、n = 0となる。実際の実験では高温状態から冷却すると
準静的過程にくらべ、急速に冷却することになる。するともはや平衡状態の
統計力学は正しい結果を与えず、実際にはn=3で「凍り付いたまま」低温状態に
移行することになる。すると比熱は$\displaystyle C = \frac{3}{4}C_{rott} + 
\frac{1}{4}C_{rots}$で計算できることになる。一方で低温極限ではn=0となるので、
比熱は$C = C_{rots}$で与えられることになる。$C_{rott}$と$C_{rots}$は
高次項を計算してみれば分かるように
微妙に違うので、上の二通りの場合で比熱が異なることになる。 
\end{enumerate}
\end{answer}
\end{document}
