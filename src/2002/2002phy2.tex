\documentclass[fleqn]{jbook}
\usepackage{physpub}
\usepackage{txfonts}

\begin{document}
\begin{question}{問題2}{高島宏和}
イオン半径のほぼ等しい2種の元素A、Bを高温で溶かして均一に混ぜ合わせて、その後
急冷して、温度Tで合金(固溶体)を作る。元素A、Bは濃度比によらず格子定数が一定の単純立方格子を作ると仮定する。また、原子AとA、BとBが最近接で隣合う時の相互作用エネルギーは等しくその大きさをa,AとBが最近接で隣合うときの相互作用エネルギーの大きさをbであるとし、これ以外には相互作用は働かないとする。Aどうし、Bどうしがとなりあう方が、A、Bがとなりあうよりもエネルギーが低く、a<bが満たされているとする。ただし、合金中のA、Bの原子数濃度は等しいとする。
\begin{enumerate}
\item 格子点iがA原子で占められている時を$\sigma_i=1$、B原子で占められている時を$\sigma_i=-1$とする変数$\sigma_i$を使うと、この合金のモデルのハミルトニアンは、
\begin{equation}
H=\sum _{<i,j>}{ (P \sigma_i \sigma_j +Q)}
\end{equation}
という形に書けることを示し、PとQをa,bを用いてあらわせ。ただし、<a,b>は最近接格子の組みである。ここで今の場合、A、Bが50%ずつなので、対称性から化学ポテンシャルの項$-\mu \sum _i \sigma_i$は必要ないことを注意しておく。
\item この系の場合、$\sigma_i$の平均値$p=< \sigma_i >$の値はいくらか。
\item ハミルトニアンの中の相互作用を表す項$ \sigma_i \sigma_j$を定数mを用いて、$( \sigma_i +\sigma_j)m-m^2$と置き換える近似を行ない、
\begin{equation}
H=\sum _{<i,j>} { P((\sigma_i+\sigma_j)m-m^2)+Q}
\end{equation}
と表す。mを設問2で決まっている$ \sigma_i$の一様な平均値$m=p=<\sigma_i>=<\sigma_j>$で与えるならば、この近似は相互作用の相手を平均値で置き換えることになり、「平均場近似」と呼ばれる。しかし、ここではより一般にmをパラメタと考える。すなわち、上記の近似での平均値mが設問2で与えられる系全体の平均値pと関係なく与えられるとした時に、上記のハミルトニアンで与えられる系の分配関数Zと自由エネルギーFをm、N,a,bおよび温度Tの関数として表せ。ただし系の全格子点数は$N$であり、周期的境界条件下にあるとしてよい。以下、ボルツマン定数を$k_B$とする。
\item 設問3で求めた自由エネルギーFをmが小さいと考えてmの4次まで展開した表式は$F=u+vm^2+wm^4$と書ける。u,v,wをN,T,a,bを用いて表せ。
\item もしもmが自由に選べて、設問4で展開して4次まで求めた自由エネルギーが最小になるmが実現されるとすると、mは温度の関数としてどのように振舞うか。$k_b T>2(b-a)$の範囲でおおざっぱにmとTの関係を表すグラフもかけ。
\item 設問4で求めた自由エネルギーの表式を用いて、この系の温度を下げてくる時に、この合金系の熱平衡状態でどのような相転移が生ずるか、説明せよ。ただしA、Bの組成比が決まっている時に、物体全体では$ \sigma_i$の平均値mの値は設問2の値で決まっていて固定されていることに注意せよ。
\end{enumerate}
\end{question}
\begin{answer}{問題2}{高島宏和}
\begin{enumerate}
\item $P=(a-b)/2,Q=(a+b)/2$
\item 0
\item 格子の形を考慮すると一粒子ハミルトニアンは
\begin{equation}
H=3(a-b)m\sigma_i+3/2((a+b)-(a-b)m^2)
\end{equation}
であるから、
\begin{equation}
Z={\left(2\cosh{\frac{3(a-b)m}{kT}}\right)}^N e^{-\frac{3N}{2KT}{((a+b)-(a-b)m^2)}}
\end{equation}
\begin{equation}
F=-kTN\log \left(2\cosh \left(\frac{3(a-b)m}{kT}\right)  \right)+\frac{3N}{2}((a+b)-(a-b)m^2)
\end{equation}
となる。
\item $\log (2\cosh(x))$を展開すると$\log2 +\frac{x^2}{2}-\frac{x^4}{12}$に
なることを利用して

\begin{equation}
 u=\frac{3N}{2}(a+b)-kTN\log 2
\end{equation}
\begin{equation}
 v=-\frac{9(b-a)^2N}{2kT}+\frac{3N(b-a)}{2}
\end{equation}
\begin{equation}
 w=\frac{27N(a-b)^4}{4(kT)^3}
\end{equation}
\item Fをmで微分してそれを0とおく。無次元パラメータ$\xi=kT/(b-a)$を
導入すると、$\xi<3$でmが0でない値をとり、m=$\frac{\xi}{3}\sqrt{3-\xi}$
となるのでこれをグラフに書けば良い。
\item 相分離
\end{enumerate}
\end{answer}

\end{document}
