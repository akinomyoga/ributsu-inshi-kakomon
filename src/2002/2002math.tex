\documentclass[fleqn]{jbook}
\usepackage{physpub}
\usepackage{txfonts}

%%%% TEXT START %%%%
\begin{document}

\begin{question}{問題1}{桂  法称}
定数ベクトル
$\overrightarrow{v} = \left(
\begin{array}{c}
   2\\
   1\\
  -1
\end{array}\right)$
を用いた微分方程式

$$\frac{d}{dt}\overrightarrow{x} =
  \overrightarrow {v} \times \overrightarrow{x}$$

を考える。\\

\begin{enumerate}
\item
上の方程式に$A$を用いて
$$\frac {d}{dt} \overrightarrow{x} =
  A \bullet \overrightarrow{x}$$
という形に書き換える。行列$A$を求めよ。

\item
次の性質を持つ正規直行基底
    $\overrightarrow{e_i}(i=1,2,3)
    (\overrightarrow{e_i} \bullet
     \overrightarrow{e_j} = \delta_ij)$
と係数$\lambda (\lambda > 0)$を求めよ。

$$\overrightarrow{v} \times \overrightarrow{e_1} =
      \lambda \overrightarrow{e_2} , \qquad
      \overrightarrow{v} \times \overrightarrow{e_2} =
     -\lambda \overrightarrow{e_1} , \qquad
      \overrightarrow{v} \times \overrightarrow{e_3} =
      0$$

\item
上で得られた結果を用いると、行列$A$は実直行行列$U$を用いて

$$ A = U  \left(
      \begin{array}{ccc}
        0 & -\lambda & 0 \\
        \lambda  & 0 & 0 \\
        0 & 0 & 0        \\
      \end{array} \right)
      U^t , \qquad
       UU^t = \left(
      \begin{array}{ccc}
        1 & 0 & 0 \\
        0 & 1 & 0 \\
        0 & 0 & 1 \\
      \end{array} \right)$$

とかけることを示せ(ただし$U^t$は$U$の転地行列)。
また$U$求めよ。

\item
$\overrightarrow{x}$を基底$\overrightarrow{e_i}$を用いて

$$\overrightarrow{x} =
      \sum_{i=1} ^3 q_i(t) \overrightarrow{e_i}$$

と展開する。係数$q_i(t)$に対する微分方程式に書き下し、その一般解を求めよ。
また解の振る舞いを定性的に論じよ。
\end{enumerate}
\end{question}
\begin{answer}{問題1}{桂  法称}
\begin{enumerate}
\item 
$\overrightarrow{x}=\left(
     \begin{array}{c}
       x_1\\
       x_2\\
       x_3
     \end{array}\right)$とすると、
$\overrightarrow{v}$と$\overrightarrow{x}$の外積は、

$\overrightarrow{v} \times \overrightarrow{x} =
     \left(
     \begin{array}{c}
       x_2 +x_3\\
       -2x_3 - x_1\\
       -x_1 + 2x_2
     \end{array}\right)$\\
よって、

$$\frac{d}{dt}\overrightarrow{x} =
      \overrightarrow{v} \times \overrightarrow{x} = \left(
      \begin{array}{ccc}
        0 & 1 & 1 \\
       -1 & 0 & -2 \\
       -1 & 2 & 0 \\
      \end{array} \right)\left(
      \begin{array}{c}
       x_1\\
       x_2\\
       x_3
      \end{array}\right) =
      A \overrightarrow{x}$$

ゆえに求める行列は、
$$\left(
      \begin{array}{ccc}
        0 & 1 & 1 \\
       -1 & 0 & -2 \\
       -1 & 2 & 0 \\
      \end{array} \right) $$
である。

\item
$\overrightarrow{e_1} = \left(
     \begin{array}{c}
       0\\
       a\\
       b
     \end{array}\right), (a^2+b^2=1)$
とおくと、

$\overrightarrow{v} \times \overrightarrow{e_1} =  \left(
     \begin{array}{c}
       a+b\\
       -2b\\
       2a
     \end{array}\right) = \lambda \overrightarrow{e_2}$

この関係式を、
$\overrightarrow{v} \times \overrightarrow{e_2} =
     -\lambda \overrightarrow{e_1}$
に代入すると、\\
$\overrightarrow{v} \times \lambda \overrightarrow{e_2} = \left(
     \begin{array}{c}
       2a-2b\\
       -5a-b\\
       -a-5b
     \end{array}\right), \qquad
-{\lambda}^2 \overrightarrow{e_1}=\left(
     \begin{array}{c}
       0\\
       -{\lambda}^2 a\\
       -{\lambda}^2 b \\
    \end{array}\right)$\\
これより、
$$a = \frac{1}{\sqrt 2} , \qquad  b = \frac{1}{\sqrt 2} , \qquad \lambda = 
\sqrt 6$$
ゆえに、\\
$\overrightarrow{e_1} = \left(
     \begin{array}{c}
       0\\
       \frac{1}{\sqrt 2}\\
       \frac{1}{\sqrt 2}
     \end{array}\right), \qquad
\overrightarrow{e_2} = \left(
     \begin{array}{c}
       \frac{1}{\sqrt 3}\\
      -\frac{1}{\sqrt 3}\\
       \frac{1}{\sqrt 3}
     \end{array}\right), \qquad
\overrightarrow{e_3} = \left(
     \begin{array}{c}
       \frac{2}{\sqrt 6}\\
       \frac{1}{\sqrt 6}\\
      -\frac{1}{\sqrt 6}
     \end{array}\right)$
\item

$\overrightarrow{x}=q_1 \overrightarrow{e_1}
                   +q_2 \overrightarrow{e_2}
                   +q_3 \overrightarrow{e_3}$とおくと、
(2)より、

$\left(
     \begin{array}{c}
       x_1\\
       x_2\\
       x_3
     \end{array}\right) =
\left(
      \begin{array}{ccc}
        0 & \frac{1}{\sqrt 3}& \frac{2}{\sqrt 6}\\
        \frac{1}{\sqrt 2}& -\frac{1}{\sqrt 3}& \frac{1}{\sqrt 6}\\
        \frac{1}{\sqrt 2} & \frac{1}{\sqrt 3}& -\frac{1}{\sqrt 6}\\
      \end{array} \right)
\left(
     \begin{array}{c}
       q_1\\
       q_2\\
       q_3
     \end{array}\right)$\\
上を形式的に$\overrightarrow{x}=U\overrightarrow{q}$
とかくことにする。\\

ここで、(2)で導入した$\overrightarrow{e_i}$の性質から、\\
$\overrightarrow{v} \times \overrightarrow{x} = 
q_1(\overrightarrow{v} \times \overrightarrow{e_1})
+q_2(\overrightarrow{v} \times \overrightarrow{e_2})
+q_3(\overrightarrow{v} \times \overrightarrow{e_3}) =
\lambda q_1 \overrightarrow{e_2} 
-\lambda q_2 \overrightarrow{e_1} = U  \left(
     \begin{array}{c}
       \lambda q_2\\
      -\lambda q_1\\
       0
     \end{array}\right) = U \left(
      \begin{array}{ccc}
        0 & -\lambda & 0 \\
        \lambda  & 0 & 0 \\
        0 & 0 & 0        \\
      \end{array} \right)  \left(
     \begin{array}{c}
       q_1\\
       q_2\\
       q_3
     \end{array}\right)$\\

$\overrightarrow{x}=U\overrightarrow{q}$より、
$\overrightarrow{q}=U^{-1} \overrightarrow{x}$
(なぜならば$U$は実直行行列)\\
これを前の式に代入すれば、
$\overrightarrow{v} \times \overrightarrow{x} = U \left(
      \begin{array}{ccc}
        0 & -\lambda & 0 \\
        \lambda  & 0 & 0 \\
        0 & 0 & 0        \\
      \end{array} \right)  U^t \overrightarrow{x}$\\
(1)で書き換えた関係から、$A$は上の
$U \left(
      \begin{array}{ccc}
        0 & -\lambda & 0 \\
        \lambda  & 0 & 0 \\
        0 & 0 & 0        \\
      \end{array} \right)  U^t$
であることが分かる。\\
また、$U = \left(
      \begin{array}{ccc}
        0 & \frac{1}{\sqrt 3}& \frac{2}{\sqrt 6}\\
        \frac{1}{\sqrt 2}& -\frac{1}{\sqrt 3}& \frac{1}{\sqrt 6}\\
        \frac{1}{\sqrt 2} & \frac{1}{\sqrt 3}& -\frac{1}{\sqrt 6}\\
      \end{array} \right)$
\item
$\overrightarrow{x} =
      \sum_{i=1} ^3 q_i(t) \overrightarrow{e_i}$を
$\frac{d}{dt}\overrightarrow{x}=
      \overrightarrow{v} \times \overrightarrow{x}$に代入して、
両辺の$\overrightarrow{e_i}$の係数を比較すれば、

$$\dot q_1(t) = -\lambda q_2(t), \qquad
  \dot q_2(t) = \lambda q_1(t), \qquad
  \dot q_3(t) = 0$$

これより、\\
$$q_1(t) = Ae^{i \lambda t} + B e^{-i \lambda t}, \qquad
  q_2(t) = -iAe^{i \lambda t} + iB e^{-i \lambda t}, \qquad
  q_3(t) = Const.$$
$\overrightarrow{x}$が実のベクトルであるとすると、
$\overrightarrow{q}$も実のベクトルなので、$A=-B$
であることが必要である。
よって、
$$q_1(t) = 2A \cos(\lambda t), \qquad
  q_2(t) = 2A \sin(\lambda t), \qquad
  q_3(t) = Const.$$

解の振る舞いは、$\overrightarrow{v}$を中心軸とする円運動を
表していると考えられる。\\
これは回転座標系で見たベクトル$\overrightarrow{x}$の変化が,
角速度ベクトルを$\overrightarrow{\omega}$とすると、\\
$$\frac{d \overrightarrow{x}}{dt} = 
\overrightarrow{\omega} \times \overrightarrow{x}$$
となることを示している。
\end{enumerate}
\end{answer}

\begin{question}{問題2}{桂 法称}
球対称の関数$u(t, r)$に対する波動方程式は、以下のように書くことができる:
\begin{eqnarray*}
\frac{\partial^2 u}{\partial t^2} = c^2 \left( 
\frac{\partial^2 u}{\partial r^2} + \frac{2}{r} 
\frac{\partial u}{\partial r} \right)
\end{eqnarray*}
ここで、$r$は動径座標、$t$は時間座標、$c$は速度の次元の定数である。
以下の問に答えよ。
\begin{enumerate}
\item 上の偏微分方程式の一般解を求めよ。なおここで一般解とは
2個の任意関数で表された解を指す。
\ilabel{1-2002-math2}

\item 初期条件、$t = 0$で$u = e^{-r^2/(2r_0^2)}$、$\partial u / 
\partial t = 0$の場合の$t > 0$に対する解を求めよ。
ただし、$r_0$は波束の広がりをあらわす実定数である。

\item 初期条件、$t = 0$で$u = 0$、$\partial u / 
\partial t = e^{-r^2/(2r_0^2)}$の場合の$t > 0$に対する解を求めよ。
\ilabel{3-2002-math2}

\item \iref{3-2002-math2}で求めた関数を、時間に対してフーリエ変換せよ。
ただしここでフーリエ変換は、
\begin{eqnarray*}
U(\omega, r) = \int _{- \infty} ^{\infty} u(t ,r)e^{i\omega t}dt
\end{eqnarray*}
と定義する。

\item $g(\omega) \equiv U(\omega, r=0)$は、$\omega$の関数である。
$g(\omega)$が最大値となる$\omega$の値を求めよ。
\end{enumerate}
\end{question}

\begin{answer}{問題2}{桂 法称}
\begin{enumerate}
\item 右辺の$\displaystyle \frac{\partial^2 u}{\partial r^2}$は
$\displaystyle \frac{1}{r}\frac{\partial^2}{\partial r^2}(ru)$と書き換える
ことができるので、新たに$\displaystyle u(t, r) \equiv 
\frac{\chi(t, r)}{r}$と定義すれば、関数$\xi(t, r)$について、
\begin{eqnarray*}
\frac{\partial^2 \chi}{\partial t^2} = c^2 \frac{\partial^2 \chi}{\partial r^2}
\end{eqnarray*}
が成立する。これは1次元の波動方程式で一般解は
\begin{eqnarray*}
\chi(t, r) = f(r - ct) + g(r + ct)
\end{eqnarray*}
(ダランベールの解。$f(r - ct)$が$r$の正方向に、$g(r + ct)$が$r$の
負方向に進む波を表す。)
よって、与えられた偏微分方程式の一般解は、
\begin{eqnarray*}
u(t, r) = \frac{1}{r} \left\{ f(r - ct) + g(r + ct) \right\}
\end{eqnarray*}

\item 初期条件より、
\[
\left\{
\begin{array}{@{\,}ll}
f(r) + g(r) = r e^{-r^2/2r_0^2} \\
f'(r) - g'(r) = 0
\end{array}
\right.
\]
よって、$\displaystyle f(r) = g(r) = \frac{1}{2}r e^{-r^2/(2r_0^2)}$
\iref{1-2002-math2}で得られた結果に代入して、
\begin{eqnarray*}
u(t, r) = \frac{1}{2r} \left\{ (r - ct)e^{-(r-ct)^2/(2r_0^2)} 
+ (r + ct)e^{-(r+ct)^2/(2r_0^2)} \right\}
\end{eqnarray*}

\item 初期条件より、
\[
\left\{
\begin{array}{@{\,}ll}
f(r) + g(r) = 0\\
f'(r) - g'(r) = -\frac{1}{c}r e^{-r^2/(2r_0^2)}
\end{array}
\right.
\]
よって、$\displaystyle f(r) = \frac{r_0^2}{2c}e^{-r^2/(2r_0^2)}, 
g(r) = -\frac{r_0^2}{2c}e^{-r^2/(2r_0^2)}$
\iref{1-2002-math2}で得られた結果に代入して、
\begin{eqnarray*}
u(t, r) = \frac{r_0^2}{2c}\frac{1}{r} \left\{ e^{-(r-ct)^2/(2r_0^2)} 
- e^{-(r+ct)^2/(2r_0^2)} \right\}
\end{eqnarray*}

\item \iref{3-2002-math2}で求めた$u(t, r)$のフーリエ変換は、
\begin{eqnarray*}
U(\omega, r) &=& \int_{-\infty}^{\infty} u(t, r)e^{i\omega t}dt \\
&=& \frac{r_0^2}{2c}\frac{1}{r}e^{-r^2/(2r_0^2)}
\int_{-\infty}{\infty}\left\{ e^{-c^2t^2/(2r_0^2) + 
(i \omega + \frac{cr}{r_0^2})t}-e^{-c^2t^2/(2r_0^2) + 
(i \omega - \frac{cr}{r_0^2})t}\right\} dt
\end{eqnarray*}
ここでGauss積分の公式
\begin{eqnarray*}
\int_{-\infty}^{\infty}e^{-at^2 + bt}dt = \sqrt{\frac{\pi}{a}}
e^{\frac{b^2}{4a}}
\end{eqnarray*}
を用いれば、
\begin{eqnarray*}
U(\omega, r) &=& \frac{r_0^2}{2c}\frac{1}{r}e^{-r^2/2r_0^2}
\frac{\sqrt{2 \pi r_0^2}}{c}\left\{ e^{\frac{r_0^2}{2c^2}}
(i \omega + \frac{cr}{r_0^2})^2 
- e^{\frac{r_0^2}{2c^2}}(i \omega - \frac{cr}{r_0^2})^2 \right\} \\
&=& \sqrt{2 \pi}i \frac{r_0^3}{r c^2}e^{-\frac{r_0^2}{2c^2}\omega^2}
\sin (\frac{r}{c}\omega)
\end{eqnarray*}

\item \begin{eqnarray*}
g(\omega) \equiv U(\omega, r=0) &=& i \sqrt{2 \pi}\frac{r_0^3}{c^3}
\omega e^{-\frac{r_0^2}{2c^2}\omega^2 }\lim_{r \rightarrow 0}
\frac{\sin(\frac{\omega}{c}r)}{\frac{\omega}{c}r} \\
&=& i \sqrt{2 \pi}\frac{r_0^3}{c^3}
\omega e^{-\frac{r_0^2}{2c^2}\omega^2} 
\end{eqnarray*}
$|g(\omega)| \propto \omega e^{-r_0^2\omega^2/(2c^2)}$より
\begin{eqnarray*}
\frac{d}{d\omega}\left( \omega e^{\frac{r_0^2}{2c^2}\omega^2} \right)
= \left( 1 - \frac{r_0^2}{c^2}\omega^2 \right)
e^{\frac{r_0^2}{2c^2}\omega^2} = 0
\end{eqnarray*}
とすると、$\omega = c/r_0$の時、$g(\omega)$最大。

\end{enumerate}
\end{answer}
\end{document}
