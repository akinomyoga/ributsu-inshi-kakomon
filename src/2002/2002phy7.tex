\documentclass[fleqn]{jbook}
\usepackage{physpub}
\usepackage{txfonts}

%%%% TEXT START %%%%
\begin{document}

\begin{question}{問題7}{高見 一}

相対論的な高エネルギー陽子(p)ビームを静止している陽子(p)ターゲットに
照射することにより、陽子(p)と反陽子($\bar{p}$)の対を生成する反応:
\begin{eqnarray*}
p + p \longrightarrow p + p + p + \bar{p}
\end{eqnarray*}
を考える。ここで陽子の質量を$m_p$ ( = 1.7 $\times 10^{-27}$kg )、
陽子の電荷を$e$ ( = 1.6 $\times 10^{-19}$C )、
真空中の光速度を$c$ ( = 3.0 $\times 10^8$m/s )とする。

\begin{enumerate}
\item 左手座標系で$+z$方向に磁束密度の大きさが$B$ ( Wb/m$^2$ )の
一様磁場をかけたとする。実験室系で運動量の大きさが$P_p$ ( kg m/s )の
高エネルギー陽子が$x$軸上を$+x$方向に磁場と垂直に入射された場合、
陽子は回転運動を始める。その回転軸と回転方向を磁場と陽子の運動方向も
含めて図示せよ。また、その回転運動の曲率半径を$P_p$、$B$、$e$
を用いて表せ。

\item 陽子と反陽子の持つ固有の物理量のうち、等しくなる例と
異なる例をそれぞれ2つずつ挙げよ。

\item 設問2.で挙げた陽子と反陽子の性質の差異を用いて、
陽子と反陽子を区別する手段を2通り簡潔に述べよ。

\item 上記反応の始状態で$x$軸上を$+x$方向に入射された
ビーム陽子が実験室系で持つ運動量の大きさを$P_p$とする。
ビーム陽子とターゲット陽子の実験室系での全エネルギーと
運動量を$P_p$、$m_p$、$c$を用いて各々書き表せ。
また、ビーム陽子が重心系で持つ運動量の大きさを$P_p^*$とする。ビーム陽子と
ターゲット陽子の重心系での全エネルギーと運動量を
$P_p^*$、$m_p$、$c$を用いて各々書き表せ。ただし、
本設問以下においては磁場はかかっていないものとする。

\item 上記反応を生じさせるのに必要な高エネルギー陽子ビームの
実験室系での最低運動エネルギー$E_{min}$を$m_p$と$c$を
用いて表せ。

\item 陽子ビームの運動エネルギーを上記反応の起こる最低運動エネルギー
$E_{min}$とする。この反応の始状態を重心系で見たときに、
実験室系で静止していたターゲット陽子の重心系での速さ$v^*$は
光速度$c$の何倍になるかを計算せよ。また、始状態の反陽子の実験室系での
運動量の大きさを$m_p$と$c$を用いて表せ。
\end{enumerate}
\end{question}


\begin{answer}{問題7}{高見 一}

\begin{enumerate}
\item 
運動方程式は、
\begin{eqnarray}
\dot{\textbf{p}} = e \textbf{v} \times \textbf{B} \qquad
\left( \textbf{p} = \frac{m_0 \textbf{v}}{\sqrt{1 - \beta^2}} \right)
\ilabel{motion}
\end{eqnarray}
と書ける。式(\iref{motion})の両辺に$\textbf{p}$を内積して、
\begin{eqnarray}
\textbf{p} \cdot \dot{\textbf{p}} &=& e \textbf{p} \cdot 
( \textbf{v} \times \textbf{B} ) = 0 \nonumber \\
\frac{d}{dt}\textbf{p}^2 &=& 0 \nonumber \\
p^2 &=& Const. = {P_p}^2
\ilabel{conservation}
\end{eqnarray}
さて、式(\iref{motion})を陽に書くと、
\[
\left(
\begin{array}{@{\,}ccc}
\dot{p_x} \\
\dot{p_y} \\
\dot{p_z}
\end{array}
\right)
= e
\left(
\begin{array}{@{\,}ccc}
v_y B \\
-v_x B \\
0
\end{array}
\right)
\]
$v_x = \dot{x}$等より、
\[
\left(
\begin{array}{@{\,}ccc}
p_x \\
p_y \\
p_z
\end{array}
\right)
=
\left(
\begin{array}{@{\,}ccc}
e y B + C_1 \\
-e x B + C_2 \\
C_3
\end{array}
\right)
\]
今、$t=0$で$x=y=z=0$、$p_x = P_p , p_y = p_z = 0$とすると、
\[
\left(
\begin{array}{@{\,}ccc}
p_x \\
p_y \\
p_z
\end{array}
\right)
=
\left(
\begin{array}{@{\,}ccc}
e y B + P_p \\
-e x B \\
0
\end{array}
\right)
\]
これを式(\iref{conservation})に代入して、
\begin{eqnarray}
(e y B + P_p )^2 + (- e x B )^2 = {P_p}^2 \nonumber \\
x^2 + \left(y + \frac{P_p}{e B} \right)^2 = \left(\frac{P_p}{e B} \right)^2
\end{eqnarray}
よって陽子は中心$(0, -\frac{P_p}{eB}, 0 )$、半径$\frac{P_p}{eB}$の
円運動をする。(図略)


\item 
等しい物理量: (静止)質量、寿命、スピン 等  
異なる物理量: 電荷、磁気モーメント

\item 
陽子と反陽子は電荷が反対であることを用いる。
\begin{enumerate}
\item{電場をかけ、陽子と反陽子をsplittingさせる。}
\item{磁場をかけ、陽子と反陽子の回転方向が逆であることを利用する。}
\end{enumerate}

\item
実験室系ではターゲットは静止しているので、全運動量$\textbf{P}$は、
\[
\textbf{P} = \left(
\begin{array}{ccc}
P_p \\
0 \\
0 
\end{array}
\right)
+ \left(
\begin{array}{ccc}
0 \\
0 \\
0 
\end{array}
\right) 
= \left(
\begin{array}{ccc}
P_p \\
0 \\
0 
\end{array}
\right)
\]
ビーム陽子とターゲット陽子にはそれぞれ相対論的な関係式
\begin{eqnarray*}
E^2 = (pc)^2 + (m_p c^2)^2
\end{eqnarray*}
が成立するから、全エネルギー$E$は、
\begin{eqnarray*}
E = \sqrt{(P_p c)^2 + (m_p c^2)} + m_p c^2
\end{eqnarray*}
となる。重心系($\textbf{p}_{beam} + \textbf{p}_{target} =0$)でも同様に
\begin{eqnarray*}
\textbf{P} = \textbf{0} \\
E = 2 \sqrt{(P_p^* c)^2 + (m_p c^2)}
\end{eqnarray*}
となる。

\item
重心系での全エネルギー$M$が生成粒子の(静止)質量の和より
大きいときに反応が起こる。4元ベクトルの内積はLorentz invariantだから、
実験室系と重心系でこの値は等しく、
\begin{eqnarray*}
\left(\frac{E_{beam} + E_{target}}{c} \right)^2 - (\textbf{p}_{beam} + 
\textbf{p}_{target})^2 = \left(\frac{M}{c} \right)^2 \\
M = \sqrt{2(m_p c^2)^2 + 2m_p c^2 E_{beam}}
\end{eqnarray*}
$M \geq 4m_p c^2$であれば良いので、$E_{beam} = E_{min} + m_p c^2$に
注意して、
\begin{eqnarray*}
E_{min} = 6 m_p c^2
\end{eqnarray*}

\item
ターゲット陽子に固定した実験室系から$x$軸方向に$v^*$だけboostした系が
重心系になるとする。(この時ターゲット陽子の速度は$-v^*$となる。)
このとき、$E$,$\textbf{P}$はLorentz変換に従い、$E'$,$\textbf{p}'$に
変換される。具体的には$\beta^* = \frac{v^*}{c}$として、
\[ \displaystyle 
\left\{
\begin{array}{cc}
{p'}_{beam}^x = \frac{p_{beam}^x - \beta^* \frac{E_{beam}}{c}}
{\sqrt{1 - {\beta^*}^2}} \\
{p'}_{target}^x = \frac{p_{target}^x - \beta^* \frac{E_{target}}{c}}
{\sqrt{1 - {\beta^*}^2}} 
\end{array}
\right.
\]
いま、$E_{kin} = E_{min} = 6 m_p c^2$より、$E_{beam} = 7 m_p c^2$、
$E_{target} = m_p c^2$、$p_{target}^x = 0$であり、
ビーム粒子についての相対論的な関係式より$p_{beam}^x = 4 \sqrt{3}m_p c$が
導ける。これらを上式に代入して、$p'_{beam} + p'_{target} = 0$を
用いれば、
\begin{eqnarray*}
\beta^* = \frac{\sqrt{3}}{2}
\end{eqnarray*}
$E_{kin} = E{min}$の時、終状態の運動量は$z$成分のみで、
4粒子に等分配される(重心系でみると生成した粒子はすべて静止しているので、
生成した粒子の運動量の大きさは等しい)
ので、$p_{\bar{p}} = \sqrt{3} m_p c$
\end{enumerate}
\end{answer}
\end{document}
