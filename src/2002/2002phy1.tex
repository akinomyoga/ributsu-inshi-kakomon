\documentclass[fleqn]{jbook}
\usepackage{physpub}
\usepackage{txfonts}

\begin{document}
\begin{question}{問題1}{高島宏和}
\begin{figure}[htbp]
\begin{center}
%\includegraphics[width=.7\linewidth]{2002phy1q.eps}
\includegraphics[width=.55\linewidth]{2002physQ1_1.eps}
\end{center}
\end{figure}
高さ無限大の壁に囲まれたはば2Lの一次元井戸型ポテンシャルに、高さ無限大で幅がL二比べて十分小さいポテンシャル障壁を設け、質量mの粒子を左半分(幅L)に閉じ込める(図1)。井戸型ポテンシャルの底をエネルギーの原点とする。
\begin{enumerate}
\item 粒子が量子力学的な基底状態にあるとして、そのエネルギー$E_0$を求めよ。
\item ポテンシャル障壁を突然取り去って、粒子が幅2Lのポテンシャル全体を動けるようにする。この時、粒子が基底状態にある確率を求めよ。また、エネルギー期待値はいくらか。
\item ポテンシャル障壁を取り去った時刻からt秒後の粒子の波動関数を$\psi (x,t)$とする。今仮に$\psi (x,t)$がわかったとしよう。時刻tにおいて粒子の位置を観測したとき、粒子が井戸の左半分で見つかる確率は、この$\psi (x,t)$を用いてどのように書けるか。ただし、$\psi (x,t)$は規格化されているとする。
\item 上記の$\psi (x,t)$の表式を求めよ。ただし級数和はそのまま残して良い。
\item 今度は設問1の状態からポテンシャル障壁を極めてゆっくりと右端まで動かす場合について考えよう。このようなゆっくりとしたポテンシャルの変化に対しては断熱定理が成り立ち、粒子は連続的に移り代わることのできる量子数の同じ固有状態にとどまる。このことを使って、粒子が幅2Lのポテンシャル井戸全体を動けるようにした後の粒子のエネルギー期待値を、設問1で求めた$E_0$を用いて表せ。
\item 上記の断熱定理が成り立つことを、以下の手順で示せ。左側のポテンシャル井戸幅ががaの時に粒子が取りうる固有状態を$ \phi_n(x,a)$その固有エネルギーを$e_n (a)$(nは負でない整数)と書く。$ \phi_n(x,a)$は規格直交化されているものとする。t=0で基底状態にいた粒子の、時間tでの波動関数を次のように展開する。
\begin{equation}
\phi (x,t)=\sum _n c_n(t) \phi_n(x,a) e^{ -\frac{i}{\hbar} \int _0 ^t {e_n(a)d\tau}}
\end{equation}
aが時間に依存することに注意して、時間に依存する波動方程式から、係数ベクトル${C_n}$の満たすべき関係式が次の形に書けることを示し、aやtによらない係数行列$J_{mn}$の表式を求めよ。
\begin{equation}
\frac{dc_n(t)}{dt}=-\sum _m c_m(t) \frac{1}{a} \frac{da}{dt} J_{mn} exp \left(\frac{i}{\hbar} \int _0 ^t e_n(a)-e_m(a) d\tau \right)
\end{equation}
ただし、$J_{mn}=0$である。次にこの関係式を用いて、ある条件の下で断熱定理が成り立つことを説明せよ。
\item ポテンシャル井戸の左半分に閉じ込められた粒子を、運動エネルギー$e_0$の子・ん粒子だとする。設問5と同様に壁をゆっくりと(断熱的)右端まで動かしたあとのエネルギーが、設問5で求めた量子系の結果と同じ出あることを示せ。ただし、粒子は壁と弾性衝突(跳ね返り係数1)であるとし、壁を動かしている間に粒子は壁と十分多くの回数衝突するとせよ。
\end{enumerate}
\end{question}
\begin{answer}{問題1}{高島宏和}
\begin{enumerate}
\item 固有値は
  $E_n=\frac{-n^2 h^2}{8mL}$、固有関数は      
  $ \psi_n=\sqrt{\frac{2}{L}}\sin{\frac{n\pi x}{L}}$  \\
  と求まるので、$n=1$を代入する。\\
\item $u_n$を障壁を取り去ったあとの固有関数とする。\\
 これらでの基底状態の展開係数$a_n=<u_n | \psi_{init}> $
 は前問を使うと、
\begin{eqnarray}
a_n=\int _0 ^L \left(\sqrt{\frac{1}{L}}\sin{\frac{n \pi x}{2L}}\right)\left(\sqrt{\frac{2}{L}}\sin{\frac{\pi x}{L}}\right)dx
\end{eqnarray}
 積分を実行して$a_n$は
\begin{itemize}
\item nが偶数の時 0
\item $n=4m+1$の時$\frac{-4\sqrt{2}}{(n^2-4)\pi}$
\item $n=4m+3$の時$\frac{4\sqrt{2}}{(n^2-4)\pi}$
\end{itemize}
となるから、求める確率$P=|{a_1}^2|=\frac{32}{9\pi^2}$となる。\\
期待値は展開する固有関数系によらないので$E_0$\\
\item $ \int _0 ^L |\psi \left(x \right)^2 | dx$
\item $\psi (x,t)=\sum _n a_n u_n e^{-\frac{i}{\hbar}e_n t}$に、
 $a_n , u_n,e_n$の値を代入する。
\item $E_0/4$
\item 波動方程式に代入して整理して両辺に$\psi^*$をかけて積分すると\\
\begin{equation}
 J_{mn}=a \int _0 ^a {\psi_n}^*\frac{\psi_m}{da} dx
\end{equation}
${\psi_n}^*(a,t)$の値は最初の問題で求まっているからこれを代入して
\begin{equation}
 J_{mn}=\frac{2nm(-1)^{n-m}}{n^2-m^2}
\end{equation}
となりaやtによらない。
断熱定理は$\frac{da}{dt} $がaに比べ小さいという意味で壁がゆっくり
動く時に成り立つ。\\
\item 古典的粒子が壁にぶつかって壁を動かしたときにした仕事の分だけ
エネルギーを失ったと考えてみると.壁の位置変化$\Delta a$,粒子のエネルギー
変化$\Delta E$とすると、
\begin{equation}
\Delta E=- (2mv)(2a/v)^{-1} \Delta a
\end{equation}
これを整理すると$\frac{\Delta E}{E}=-2 \frac{\Delta a}{a}$より、E
がaの2乗に反比例することがわかる。
\end{enumerate}
\end{answer}
\end{document}
