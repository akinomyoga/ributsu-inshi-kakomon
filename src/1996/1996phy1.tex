\documentclass[fleqn]{jbook}
\usepackage{physpub}

\begin{document}

\begin{question}{専攻 問題1}{}

% Definition of local macros
\def\WF#1{\psi\raisebox{-0.5ex}{\scriptsize$#1$}}
\def\WFHC#1{\psi\raisebox{1.0ex}{\hbox to 0pt{\scriptsize$\ast$\hss}}\raisebox{-0.5ex}{\scriptsize$#1$}}
\def\vr{\vec{r}\,}


水素原子の中の陽子と電子の相対運動のシュレディンガー方程式は、
相対座標を$\vr$、換算質量を$m$とすると、cgsガウス系を単位系に
用いた場合、通常次の形に書かれる。
%
\begin{equation} 
  \Bigl[ -\frac{\hbar^2}{2m} \nabla^2 +V(\vr) \Bigr]\psi(\vr)%
  = E \psi(\vr) \eqname{Q1}
\end{equation}
\begin{equation}
  V = -e^2/r \eqname{Q2}
\end{equation}
%
この固有エネルギー$E_n$と固有関数$\psi_{\rm n\ell m}(\vr)$は
次のように求まる:
%
\begin{equation}
  E_n =-\frac{R_{\rm y}}{n^2} \qquad (n=1,2,...) \eqname{Q3}
\end{equation}
%
\begin{equation}
  \WF{\rm n\ell m}(\vr)%
  = R_{\rm n\ell}(r)Y_{\rm \ell m}(\theta,\phi) \qquad%
  (\ell=0,1,...,n\!-\!\!1; \quad m=0,\pm 1,...,\pm \ell)  \eqname{Q4}
\end{equation}
%
ただし $R_{\rm y}\equiv me^4/2\hbar^2$は Rydberg定数である。
ここで、$(r,\theta,\phi)$は極座標、$Y_{\rm \ell m}(\theta,\phi)$は
規格化された球面調和関数であり、たとえば、$Y_{00}=1/\sqrt{4\pi}$で
ある。波動関数の動径部分 $R_{\rm n\ell}(r)$ は、
\Orbit{1s}状態($n\!=\!1,\,\ell\!=\!0$)、
\Orbit{2s}状態($n\!=\!2,\,\ell\!=\!0$)および、
\Orbit{2p}状態 ($n\!=\!2,\,\ell\!=\!1$)についてはそれぞれ、
%
\begin{eqnarray}
  R_{10}(r) &=& \frac{2}{\sqrt{a^3}}%
                \exp{\Bigl( -\frac{r}{a}\Bigr)}   \eqname{Q5} \\
  R_{20}(r) &=& \frac{1}{\sqrt{8a^3}}%
                \Bigl( 2-\frac{r}{a} \Bigr)%
                \exp{\bigl( -\frac{r}{2a} \Bigr)} \eqname{Q6} \\
  R_{21}(r) &=& \frac{1}{\sqrt{24a^5}}%
                r \exp{\Bigl(-\frac{r}{2a}\Bigr)} \eqname{Q7}
\end{eqnarray}
%
で与えられる。ただし、$a \equiv \hbar^2/me^2$ は Bohr 半径である。\\
以上の知識を手がかりにして、以下の問いに答えよ。



\begin{subquestions}
\SubQuestion
  絶縁体に弱い光を当てると、電子と正孔のペアができる。絶縁体の中では、
  電子は電荷$-e$で質量が$m_e$の粒子のように、正孔は電荷$+e$で質量が
  $m_h$の粒子のようにふるまい、両者の間にはたらくポテンシャルは、
  絶縁体の(比)誘電率を$\varepsilon$(定数とする)として、
%
  \begin{equation}
    V = - \frac{e^2}{\varepsilon r} \eqname{Q8}
  \end{equation}
%
  で与えられると仮定する。電子と正孔の束縛状態について、以下の問いに
  答えよ。

  \begin{subsubquestions}
  \SubSubQuestion
    最も低いエネルギーを持つ束縛状態の束縛エネルギーを求めよ。

  \SubSubQuestion
    この束縛状態の空間的広がりの目安として、$r$の期待値$\Mean{r}$を
    求めよ。

  \end{subsubquestions}

\SubQuestion
  電子が実際に感じるポテンシャルは、式\eqhref{Q2}や式\eqhref{Q8}
  から少しずれていることが分かっている。たとえば水素原子の場合、
  陽子のごく近くでは、電子・陽子対が仮想的に生成・消滅している
  「雲」の内側にはいるためにポテンシャルが修正される。それが何を
  もたらすかを大ざっぱにみるために、式\eqhref{Q2}のポテンシャルを
  次のように修正してみよう。
%
  \begin{equation}
    V(\vr) = -\frac{e^2}{r} - \lambda \delta^3(\vr) \eqname{Q9}
  \end{equation}
%
  ただし、$\lambda$はある正の定数であり、$\delta^3(\vr)$ は3次元の
  デルタ関数である。

\newpage
  \begin{subsubquestions}
  \SubSubQuestion
    $V$の右辺第2項を摂動とみなし、摂動を受けた\Orbit{1s}状態、
    \Orbit{2s}状態、\Orbit{2p}状態のエネルギーを、それぞれ
    $\lambda$の一乗の精度で求めよ。

  \SubSubQuestion
    摂動を受けた\Orbit{1s}状態の電子について、電子を見いだす確率密度
    の$r=0$における値が、$\lambda=0$の場合の何倍になっているかを
    求めよ。ただし、摂動計算に用いる非摂動状態としては、
    \Orbit{1s},\Orbit{2s},\Orbit{2p}状態だけを考慮すればよく、また、
    $\lambda$の一乗の精度まででよい。

  \SubSubQuestion
    この確率密度が$\lambda=0$の場合に比べて増加しているか減少して
    いるかを述べ、式\eqhref{Q9}の形のポテンシャルがそのような確率
    密度の増加(あるいは減少)をもたらした理由を直感的に説明せよ。

  \end{subsubquestions}

\SubQuestion
  通常のクーロンポテンシャルで記述される系、式\eqhref{Q1}-\eqhref{Q7}
  に戻って、量子力学における確率の意味を考えよう。\\
  時刻$t=t_0$に、電子を、\Orbit{1s}状態と \Orbit{2s}状態の次のような
  重ね合わせ状態に置く:
%
  \begin{equation}
    \psi(\vr,t\!=\!t_0)%
    = \frac{1}{\sqrt{2}}\bigl[%
        \psi_{100}(\vr) + \psi_{200}(\vr)%
      \bigr]%
    \eqname{Q10}
  \end{equation}
%

  \begin{subsubquestions}
  \SubSubQuestion
    この状態では、確率密度$\Norm{\psi(\vr,t)}^2$が$t$の周期関数
    になることを示し、その周期を求めよ。

  \SubSubQuestion
    電子のエネルギーの期待値$\Mean{E}$はいくらか? また、実際に電子の
    エネルギー$E$を一度だけ測定した場合に測定値として得られる可能性
    がある$E$の値をすべて挙げ、その値が得られる確率を書け。

  \SubSubQuestion
    一度目の測定を終了してから、もういちど式\eqhref{Q10}の状態の電子
    を用意し、その$E$を測定する。さらにもう一度式\eqhref{Q10}の状態
    の電子を用意し、その$E$を測定する、...、ということを繰り返し、
    全部で$N$回の測定を行うとする。$i$番目$(i=1,2,...,N)$の測定で得た
    $E$の測定値を$\eps_i$と記すと、通常、その平均値
%
    \begin{equation}
      \bar{\eps}_N = \frac{1}{N} \sum_{i=1}^N \eps_i \eqname{Q11}
    \end{equation}
%
    を「$E$の実験値」として、理論値と比較することになる。しかしながら、
    測定の回数$N$が有限である以上、$\bar{\eps}_N$は一般には
    上で計算した$\Mean{E}$とは異なる。この、実験値
    $\bar{\eps}_N$と理論値$\Mean{E}$ のずれを$\delta$と記すこと
    にする。その自乗期待値
%
    \begin{equation}
      \Mean{\delta^2}  \equiv \Mean{(\bar{\eps}_N - \Mean{E} )^2 }
      \eqname{Q12}
    \end{equation}
%
    を求めよ。そして、無限回の測定を繰り返せば$\bar{\eps}_N$が
    $\Mean{E}$に一致すると期待できること、即ち、$N \to \infty$ で 
    $\Mean{\delta^2} \to 0$となることを確かめよ。

  \end{subsubquestions}
\end{subquestions}
\end{question}
\begin{answer}{専攻 問題1}{}
% Definition of local macros
\def\WF#1{\psi\raisebox{-0.5ex}{\scriptsize$#1$}}
\def\WFHC#1{\psi\raisebox{1.0ex}{\hbox to 0pt{\scriptsize$\ast$\hss}}\raisebox{-0.5ex}{\scriptsize$#1$}}
\def\vr{\vec{r}\,}
\def\Comb#1#2{\mbox{\tiny$#1$}{\rm C}\mbox{\tiny$#2$}}
\def\HI{{\cal H}^\prime\,}
\def\zpar{\kern-0.7ex\raisebox{1.1ex}{\tiny$(0)$}}
\def\fpar{\kern-0.7ex\raisebox{1.1ex}{\tiny$(1)$}}


\begin{subanswers}
\SubAnswer
  \begin{subsubanswers}
  \SubSubAnswer 
    この系では換算質量が $m=\ds\frac{m_em_h}{m_e+m_h}$
    、電荷が $e^2\rightarrow \ds\frac{e^2}{\varepsilon}$と
    置き換えられるので、Rydberg定数は、\vspace*{-2mm}
%
    \[ R_{\rm y} = \frac{m_em_he^4}{2\hbar^2(m_e\!+\!m_h)} \frac{1}{\varepsilon^2} \]
%
    と変更される。よって最低エネルギーは、
%
    \[ E_1 = -R_{\rm y}%
           = -\frac{m_em_he^4}{(m_e\!+\!m_h)2\varepsilon^2\hbar^2} \]
%
    である。ついでにBohr半径 $a$ は
%
    \[ a = \frac{\hbar^2\varepsilon(m_e\!+\!m_h)}{m_em_he^2} \]
%
    と変更される。以後この表式での$R_{\rm y}$と$a$を用いる。

  \SubSubAnswer
    \Orbit{1s}状態での$r$の期待値$\Mean{r}$は
%
    \[ \Mean{r}%
       = \Bracket{\Orbit{1s}}{\hat{r}}{\Orbit{1s}}%
       =  \Uint{\d{\vr}}%
            \WFHC{\Orbit{1s}}(\vr)\,r\,\WF{\Orbit{1s}}(\vr)%
       =  \Dint{0}{\infty}{\d{r}}\bquad
          \Dint{0}{\pi}{\d{\theta}}\bquad%
          \Dint{0}{2\pi}{\d{\varphi}} r^2\sin\theta\cdot%
          r\bigl| R_{10}(r)Y_{00}(\theta,\varphi) \bigr|^2 \]

    球面調和関数$Y_{00}$は球面積分に対して規格化されているので
%
    \[ \Dint{0}{\pi}{\d{\theta}}\bquad\Dint{0}{2\pi}{\d{\varphi}}%
       \sin\theta\bigl|Y_{00}(\theta,\varphi)\bigr|^2 = 1 \]
%
    である。先の積分を続けると、
%
    \[ \Mean{r}%
       = \Dint{0}{\infty}{\d{r}} r^3 \Bigl[\frac{2}{\sqrt{a^3}}%
         \exp\left(-\frac{r}{a}\right)\Bigr]^2%
       = \frac{a}{4} \Dint{0}{\infty}{\d{\left(\frac{2r}{a}}\right)}%
         \left(\frac{2r}{a}\right)^3 \exp\left(-\frac{2r}{a}\right)%
       = \frac{a}{4}3! = \frac{3a}{2} \]
%
    となる。


  \end{subsubanswers}


\SubAnswer
  \begin{subsubanswers}
  \SubSubAnswer
    ハミルトニアンの摂動項を$\HI\equiv -\lambda\delta^3(\vr)$
    で表すことにする。$4\pi r^2\delta^3(\vr)=\delta(r)$であることに
    注意する。\\
%
    \Orbit{1s}状態のエネルギーの1次摂動 $E_{\Orbit{1s}}\fpar$ は
%
    \begin{eqnarray*}
      E_{\Orbit{1s}}\fpar%
        &=&  \Bracket{\Orbit{1s}}\HI{\Orbit{1s}}%
         =  -\lambda\Uint{\d{\vr}}%
             \WFHC{\Orbit{1s}}\zpar(\vr)%
             \delta^3(\vr)%
             \WF{\Orbit{1s}}\zpar(\vr)\\
        &=& -\lambda\Dint{0}{\infty}{\d{r}}%
             r^2\frac{\delta(r)}{4\pi r^2}|R_{10}(r)|^2%
             \Dint{0}{\pi}{\d{\theta}}\bquad%
             \Dint{0}{2\pi}{\d{\varphi}}\sin\theta
             \left|Y_{00}(\theta,\varphi)\right|^2%
         =  -\frac{\lambda}{4\pi} |R_{10}(0)|^2%
         =  -\frac{\lambda}{a^3\pi}
    \end{eqnarray*}
%
    同様に\Orbit{2s}状態のエネルギーの1次摂動
    $E_{\Orbit{2s}}\fpar$ は
%
    \[ E_{\Orbit{2s}}\fpar%
       =  \Bracket{\Orbit{2s}}\HI{\Orbit{2s}}%
       =  -\frac{\lambda}{4\pi} |R_{20}(0)|^2%
       =  -\frac{\lambda}{8a^3\pi} \]
%
    となる。\Orbit{2p}の各状態の1次摂動は $R_{21}(0)=0$なので
%
    \[ E_{\Orbit{2p}}\fpar = 0 \]
%

  \SubSubAnswer
    \Orbit{1s}状態が\Orbit{2s}または\Orbit{2p}状態に遷移する
    状況を考えるので、1次摂動まで考慮した$\WF{\Orbit{1s}}$ は
%
    \[ \WF{\Orbit{1s}}%
       = \WF{\Orbit{1s}}\zpar + \sum_i c_i\WF{i}\zpar%
       \hspace{20mm}%
       i=\{\Orbit{2s},\Orbit{2p_x},\Orbit{2p_y},\Orbit{2p_z}\} \]
%
    と表される。この係数 $c_i$ は次の通り。
%
    \[ c_i = \frac{\Bracket{{\;}i{\;}}{\HI}{\Orbit{1s}}}%
                  {E_1\zpar - E_i\zpar}%
           =  \frac{\Bracket{{\;}i{\;}}{\HI}{\Orbit{1s}}}%
                  {E_1\zpar - E_2\zpar}%
           =  \frac{\Bracket{{\;}i{\;}}{\HI}{\Orbit{1s}}}%
                  {-R_{\rm y}+R_{\rm y}/4}%
           =  -\frac{4}{3R_{\rm y}}%
               \Bracket{{\;}i{\;}}{\HI}{\Orbit{1s}} \]
%
    \Orbit{2p}の各状態の$c_i$は、$\HI=-\lambda\delta^3(\vr)$
    、$R_{21}(0)=0$を考慮すると $0$であることがわかる。
    \Orbit{2s}の状態の$c_i$のみ$0$でない値をもつ。
%
    \begin{eqnarray*}
      c_{\Orbit{2s}}%
        &=&  \frac{4\lambda}{3R_{\rm y}}%
             \Dint{0}{\infty}{\d{r}}%
             r^2\frac{\delta(r)}{4\pi r^2}R_{20}(r)R_{10}(r)%
             \Dint{0}{\pi}{\d{\theta}}\bquad%
             \Dint{0}{2\pi}{\d{\varphi}}\sin\theta%
             \left|Y_{00}(\theta,\varphi)\right|^2 \\
        &=&  \frac{4\lambda}{3R_{\rm y}}\cdot \frac{1}{4\pi}\cdot%
             R_{20}(0)R_{10}(0)%
         =   \frac{\lambda}{3\pi R_{\rm y}}\cdot%
             \frac{2}{\sqrt{8a^3}}\cdot%
             \frac{2}{\sqrt{a^3}}%
         =   \frac{\sqrt{2}\lambda}{3\pi a^3 R_{\rm y}}
    \end{eqnarray*}
%
    よって
%
    \[ \WF{\Orbit{1s}}(\vr)%
       = \WF{\Orbit{1s}}\zpar(\vr)
         +\frac{\sqrt{2}\lambda}{3\pi a^3 R_{\rm y}}
         \WF{\Orbit{2s}}\zpar(\vr) \]
%
    である。中心$(\vr=0)$における電子密度は
%
    \[ |\WF{\Orbit{1s}}(0)|^2%
       \simeq |\WF{\Orbit{1s}}\zpar(0)|^2%
             +\frac{2\sqrt{2}\lambda}{3\pi a^3 R_{\rm y}}%
              \WFHC{\Orbit{2s}}\zpar(0)\WF{\Orbit{1s}}\zpar(0)%
       =      \frac{2}{a^3}%
              \Bigl( 1+\frac{2\lambda}{3a^3\pi R_{\rm y}} \Bigr) \]
%
    となり、$\lambda=0$にくらべて$(\ )$で囲んだ係数だけ密度が高く
    なっている。

  \SubSubAnswer
    ポテンシャルが深くなって電子がより集まったから。

  \end{subsubanswers}


\SubAnswer
  \begin{subsubanswers}
  \SubSubAnswer
    エネルギー$E$の定常状態の波動関数の時間依存性は$e^{-iEt/\hbar}$
    であるので、式\eqhref{Q10}は次のように時間発展する。ただし$t_0=0$
    としている。
%
    \[ \psi(\vr,t)%
       = \frac{1}{\sqrt{2}}\Bigl(%
           \WF{100}(\vr)\cdot e^{-iE_1t/\hbar}%
          +\WF{200}(\vr)\cdot e^{-iE_2t/\hbar}%
         \Bigr) \]
%
    $\WF{100},\WF{200}$は実数なので$\psi(\vr,t)$の複素共役は
%
    \[ \psi^\ast(\vr,t)%
       = \frac{1}{\sqrt{2}}\Bigl(%
           \WF{100}(\vr)\cdot e^{+iE_1t/\hbar}%
          +\WF{200}(\vr)\cdot e^{+iE_2t/\hbar}%
         \Bigr) \]
%
    となるので確率密度は
%
    \begin{eqnarray*}
      |\psi(\vr,t)|^2%
      &=& \frac{1}{2}\Bigl(%
            |\WF{100}|^2 + |\WF{200}|^2%
            + \WF{100}\WF{200}\cdot e^{+i(E_1\!-\!E_2)t/\hbar}%
            + \WF{200}\WF{100}\cdot e^{-i(E_1\!-\!E_2)t/\hbar}%
          \Bigr) \\
      &=& \frac{1}{2}\Bigl(%
            |\WF{100}|^2 + |\WF{200}|^2
            +2\WF{100}\WF{200}\cdot\cos{\frac{E_1\!-\!E_2}{\hbar}t}
          \Bigr) \\
    \end{eqnarray*}
%
    よって時間の周期関数である。その周期$T$は、
%
    \[ T = \frac{2\pi\hbar}{E_2-E_1} = \frac{8\pi\hbar}{3R_y} \]

  \SubSubAnswer
    式\eqhref{Q10}で表される状態を測定して$\WF{100}$を見出す
    確率$P_1$は
%
    \[ P_1 = \left| \Uint{\d{\vr}}\WFHC{100}%
             \frac{1}{\sqrt{2}}\bigl[%
               \psi_{100}(\vr) + \psi_{200}(\vr)%
             \bigr]%
             \right|^2 = \frac{1}{2} \]
%
    この時、電子のエネルギーは$E_1$である。$\WF{200}$を見出す
    確率$P_2$も同じである。\\
%
    よって電子のエネルギーの平均値$\Mean{E}$は
%
    \[ \Mean{E} = \frac{1}{2}(E_1+E_2) \]


  \SubSubAnswer
    $N$回の測定でエネルギー$E_1$を$n$回測定し、
    $E_2$を$N-n$回測定する確率$P(n)$は、
%
    \[ P(n)%
       = \Bigl(\frac{1}{2}\Bigr)^{n}%
         \Bigl(\frac{1}{2}\Bigr)^{N-n}%
         \Comb{N}{n}%
       = \frac{\Comb{N}{n}}{2^N} \]
%
    である。次に $\IDelta E \equiv (E_2-E_1)/2 = 3R_{\rm y}/8$
    と定義して $E_1 = \Mean{E}-\IDelta E$、
    $E_2 = \Mean{E}+\IDelta E$と表すことにする。
    $N$回の測定でのエネルギーの平均値 $\bar{\eps}\mbox{\tiny$N$}$ は
%
    \[ \bar{\eps}\mbox{\tiny$N$}%
       = \frac{E_1n+E_2(N-n)}{N}%
       = \Mean{E} + \frac{\IDelta E}{N}(2n-N) \]
%
    となる。よって$\Mean{\delta^2}$は
%
    \begin{eqnarray*}
     \Mean{\delta^2}%
      &=& \Mean{(\bar{\eps}\mbox{\tiny$N$} - \Mean{E} )^2 }
       =  \sum_{n=0}^{N}(\bar{\eps}_N-\Mean{E})^2 P(n)
       =  \sum_{n=0}^{N}%
          \frac{\IDelta E^2}{N^2}(2n-N)^2%
          \frac{\Comb{N}{n}}{2^N} \\
      &=& \frac{\IDelta E^2}{2^NN^2} \Bigl[%
              4 \sum_{n=0}^{N} n^2 \Comb{N}{n}%
            -4N \sum_{n=0}^{N} n   \Comb{N}{n}%
            +N^2\sum_{n=0}^{N}     \Comb{N}{n} \Bigr]
    \end{eqnarray*}
%
    ここで $\Comb{N}{n}$ の種々の公式
%
    \[ n   \Comb{N}{n} = N \Comb{N\!-\!1}{n\!-\!1}%
       \hspace{10mm}%
       n^2 \Comb{N}{n} = N(N\!-\!1) \Comb{N\!-\!2}{n\!-\!2}%
                       + N \Comb{N\!-\!1}{n\!-\!1}%
       \hspace{10mm}%
       \sum_{n=0}^{N} \Comb{N}{n} = 2^N \]
%
    を思い出して計算を進めると、
%
    \begin{eqnarray*}
     \Mean{\delta^2}%
      &=& \frac{\IDelta E^2}{2^NN^2} \Bigl[%
            4N(N\!-\!1)\sum_{n=0}^{N} \Comb{N\!-\!2}{n\!-\!2}%
          + 4N     \sum_{n=0}^{N} \Comb{N\!-\!1}{n\!-\!1}%
          - 4N^2   \sum_{n=0}^{N} \Comb{N\!-\!1}{n\!-\!1}%
          +  N^2   \sum_{n=0}^{N} \Comb{N}{n}  \Bigr] \\
      &=& \frac{\IDelta E^2}{2^NN^2} \Bigl[%
            4N(N\!-\!1)2^{N-2}%
          + 4N     2^{N-1}%
          - 4N^2   2^{N-1}%
          +  N^2   2^{N}  \Bigr] \\
     &=& \frac{\IDelta E^2}{N^2} \Bigl[%
            N^2-N+2N-2N^2+N^2 
         \Bigr] \\
     &=& \frac{\IDelta E^2}{N}%
      =  \frac{1}{N}\Bigl(\frac{3R_{\rm y}}{8}\Bigr)^2
    \end{eqnarray*}
%
    となる。確かに$N\to\infty$で$\Mean{\delta^2}\to 0$となる。
       
  \end{subsubanswers}
\end{subanswers}
\end{answer}

\end{document}
