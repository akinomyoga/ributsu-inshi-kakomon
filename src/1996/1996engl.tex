\documentclass[fleqn]{jbook}
\usepackage{physpub}

\begin{document}

\begin{question}{教育 英語}{}
\begin{subquestions}
\SubQuestion
  次の文章を読み、下の設問に答えなさい。
\baselineskip=12pt

   Modern computers have greatly extended the scope of glacier
  modeling. \underlineeng{(a)A model is a  means of reducing a
  complex real situation to a simple closed system that represents
  the essential features and to which the laws of physics can be 
  applied. Modeling can serve three purposes: experimentation, 
  explanation, and prediction.} Experimentation, discovering the 
  effect of changing the values of the controlling variables, is
  often the most useful; it can never be done in the real worlds.
  An explanation sometimes be an illusion; the fact that the model 
  with adjustable parameters produces plausible numerical values 
  does not prove that the underlying assumptions are correct. Most 
  models can be used for prediction, but first they must be tested 
  against data. Unambiguous testing is difficult and the temptation
  to use all the data to ``tune'' the model by adjusting parameters
  must be resisted. Although these \underline{(b)pitfalls}
  have not always been avoided, the scope of some recent ice-sheet
  models is impressive.

   The approach here emphasizes the physics, combined where necessary
  with mathematics. No apology is made for introducing mathematics.
  \underline{(c)In the author's opinion}, a mere handful of
  mathematical physicists, who may seldom set foot on a glacier, have
  contributes far more to the understanding of the subject than have a
  hundred measures of ablation stakes or recorders of advances and
  retreats of glacier termini. This is not to say that the latter are
  unimportant; in glaciology, as in other branches of science, there
  is a place for both the theoretical and the experimental approach.
  But the two should be coordinated, and the experiments should be
  designed to solve specific problems. \underlineeng{(d)Too often
  in the past, glaciological measurements have been made on the
  premise that the mere acquisition of data is a useful contribution
  itself. This is seldom the case.}

  glacier : 氷河 \quad
  stake   : 杭   \quad
  termini : 終端
\baselineskip=15pt
  \begin{subsubquestions}
  \SubSubQuestion
    下線部(a)および(d)を内容がわかるように和訳せよ。
  \SubSubQuestion
    下線部(b) pitfallsは何を指しているか。日本語で説明せよ。
  \SubSubQuestion 
    下線部(c) the author's opinion に反対の立場で、英語で50--100words
    で自分の意見を述べよ。
  \end{subsubquestions}


\SubQuestion
  以下の文章を読み、文中の内容に沿って、下の問いに日本語で答えなさい。
\baselineskip=12pt

   The important thing about Newton's theory of colours is not just
  that he was right, but the way in which he arrives at his
  conclusions. Before Newton, the way philosophers developed their
  ideas about the natural world was largely through pure thought.
  Descartes, for example, thought about the way in which light might
  be transmitted from a bright object to the eye, but he did not
  carry out experiments to test his ideas. Of course, Newton was
  not the first experimenter --- Galileo, in particular, pointed the
  way with studies of the way in which balls rolled down inclined
  planes, and with his work on pendulums. But Newton was the first
  person to express clearly the basis of what became the scientific
  method --- the combination of ideas (hypothesis), observation and
  experiment on which modern science rests.

   Newton's theory of colours emerged from experiments he carried
  out during his enforced sabbatical from Cambridge. By 1665, the
  fact that a ray of sunlight could be turned into a rainbow-like
  spectrum of colours by passing it through a triangular glass
  prism was well known. The standard explanation of the effect
  was based on the Aristotelian idea that white light represented
  a pure, unadulterated form, and that it became corrupted by
  passing through the glass. When the light enters the prism, it
  is bent, and then follows a straight line to the other side of
  the triangle, where it bends again as it emerges into the air.
  At the same time, the light is spread out, from a single spot
  of white light into a bar of colours. Working downwards from
  the point of the triangle, the light at the top is bent least,
  and travels the shortest distance through the glass, emerging
  as red. Lower down, where the triangular wedge of glass is wider,
  light which has been bent slightly more as it enters the prism
  travels further through the glass, and emerges into the air on
  the other side as violet. In between, there are all the colours
  of the rainbow --- red, orange, yellow, green, blue, indigo, and
  violet. Using a prism held up to the ray of light entering a
  darkened room through a small hole in the curtain (rather like
  the camera obscura set-up), the spectrum of colours can be
  displayed on the wall opposite the window.

   On the Aristotelian view, white light that had travelled the
  shortest distance through the glass was modified least, and
  become red light. While light that had travelled a little further
  through the glass was modified more, and become yellow --- and
  so on all the way down to violet.

   Newton actually tested these ideas, using both prisms and lenses
  which he ground himself, trying to minimize the colour change
  by making lenses in different shapes. He was the first person
  to distinguish the rays of different colours, and he named the
  seven colours of the spectrum.

   But the most important experiment Newton carried out at this
  time simply consisted of placing a second triangular wedge of
  glass behind the first prism but the other way up. The first
  prism, point uppermost, spread a spot of white light into a
  rainbow spectrum. The second prism, point downwards, combined
  the spread-out colours of the spectrum back into a spot of white
  light. Even though the light had passed through a further
  thickness of glass, it had not become more corrupted, but had
  returned to its former purity.

   As Newton realized, this shows that white light is not `pure`
  at all, but is a mixture of all the colours of the rainbow.
  Different colours of light bend by different amounts when they
  are refracted, but all the colours are present in the original
  white spot of light. It was a revolutionary idea, both because
  it overturned a basic tenet of Aristotelian philosophy and
  because it rested upon the secure foundation of experiment.

   \underlineeng{``The best and safest method of philosophizing
  seems to be, first to enquire diligently into the properties of
  things, and to establish those properties by experiment and then
  to proceed more slowly to hypotheses for the explanation of them.
  For hypotheses should be employed only in explaining the
  properties of things, but not assumed in determining them;
  unless so far as they may furnish experiments.''}

  \rightline{John Gribbin(1995) : Schr\"{o}dinger's kittens and 
  the search for reality}

  unadulterated  : まぜ物のない \quad
  camera obscura : (カメラの)暗箱 \quad
  tenet          : 教義 \quad
  enquire        : inquire
\baselineskip=15pt
  \begin{subsubquestions}
  \SubSubQuestion
    白色光がガラスの三角プリズムを通過すると虹色に分かれること(分光)
    についてアリストテレス学派とニュートンの考え方の違いを述べよ
    (200字以内)。

  \SubSubQuestion
    下線部のニュートンの言葉を和訳せよ。
  \end{subsubquestions}

\SubQuestion
  下の文章を英訳せよ。

   私たちの太陽以外の星のまわりにも惑星があるかもしれないという
  考えは、長い間人々の想像をかきたててきました。けれども、これまで
  科学者はこの好奇心を満足させるほどの精度で惑星を検出することが
  できませんでした。たとえ、近くの星に私たちの太陽系にあるような惑星
  があったとしても、暗くて小さすぎるので見わけるのが難しかったから
  です。しかし、もし木星のような巨大惑星を持つ星がたくさんあるならば、
  今日あるさまざまな探索方法によって惑星が見つかってもよい時期に
  なりました。いろいろな技術を使っての検出のための感度と時間は惑星
  の軌道半径によって異なります。すなわち、問題は星と惑星の間の距離
  がいくらかということです。

  太陽系   : Solar System \quad
  木星     : Jupiter \quad
  軌道半径 : orbital radius

  \end{subquestions}
\end{question}
\begin{answer}{教育 英語}{}
\begin{subanswers}
\SubAnswer
  {\bf 全訳}

   現代のコンピューターは大きく氷河モデルの領域を発展させた。
  \underlinejpn{(a)複雑な現実の環境を、必要な特徴を持ち物理法則が適用可能な単純閉鎖系に還元する手段がモデルである。モデリングには3つの目的がある。すなわち、実験、説明、予言である。}
  実世界では決してできない対象変数の値を変えることによる効果を発見
  するにおいて、実験はしばしばもっとも有用である。説明は時々思い
  違いとなっている。つまり、調節したパラメーターによるモデルが
  もっともらしい値をたくさんつくっても、その基礎となっている仮定が
  正しいことを示したことにならない。多くのモデルは予言に用いることが
  できるが、最初はデータに対するテストをする必要がある。曖昧でない
  テストは難しく、すべてのデータを用いて調節したパラメーターによる
  モデルを調節するという誘惑は退けなければならない。これらの落とし穴
  は時々回避されていないが、最近の氷河層モデルの範囲は印象的なもので
  ある。

   ここでのアプローチは物理を強調し、必要なところに数学を組み
  入れる。数学を導入することに対する釈明はしない。著者の意見によると、
  ほんの一握りの物理数学者は、彼らはほとんど氷河に足を踏み込むことは
  ないが、数百回の削摩杭の測定や氷河の終端の進退の記録よりも事象の
  理解への貢献度が大きい。これは後者の方が重要でないことをいうわけ
  ではない。ほかの科学と同様に、氷河学において理論的な面と実験的な
  面の両面がある。しかしその二つは協力をすべきであり、実験は特別な
  問題を解くために設計されなければならない。
  \underlinejpn{(d)以前は、氷河学における測定は、単なるデータ取得が氷河学に対する有効な貢献になる、という前提の下に行なわれることがほとんどだった。しかしこのようなことはめったにないものである。}

  \begin{subsubanswers}
  \SubSubAnswer
    全訳参照のこと。

  \SubSubAnswer
     次の3つの落とし穴のこと。\\
     実験について…%
       実際に行なうことができない。\\
     説明について…%
       可変パラメータを含む結果はある仮定の証明にはならない。\\
     予言について…%
       実験データの方を予言に合わせてしまうことがある。

  \SubSubAnswer
  \baselineskip=12pt
     Most of the mathematical physicists have not visited a glacier,
    so that sometimes they do not really understand about the glacier
    and the theory they constructed might become a only brain exercise.
    In the other hand, the experimentists can use the data to assume
    what have happened in the past, so the results are quite close to
    the real world phenomenon.
  \baselineskip=15pt
  \end{subsubanswers}

\SubAnswer
  {\bf 全訳}

   ニュートンの色の理論の重要性はその正しさにあるだけでなく、彼が如何
  にしてその結論に達したかにもある。ニュートン以前は、哲学者達が自然
  界についての考えを発展させた方法は純粋な思考を通じてであった。
  例えば、デカルトは光は明るい物体から目に伝わってくる仕方について
  考えていたが、彼はその考えを試す実験を行わなかった。もちろん、
  ニュートンは最初の実験家ではなかった。特にガリレオはボールが斜面を
  転がり落ちる仕方についての研究においてと、振り子での彼の仕事において
  実験するというその方法を指摘した。しかしニュートンが初めて科学的な
  方法――現代科学が頼っている仮説、観察と実験の組み合わせ――になった
  ことの基礎についてはっきりと示した人である。

   ニュートンの色の理論はケンブリッジから与えられた休暇年度の間に
  行った実験から生まれた。1665年までに、太陽光が三角ガラスプリズムを
  通すことで虹みたいな色のスペクトルにする事ができるという事実はよく
  知られていた。その効果についての基本的な説明は白色光は純粋で混ぜ
  ものがない物体であり、ガラスを通ることで乱されるようになるという
  アリストテレスの考えに基づいていた。光がプリズムに入ると、曲がり、
  三角形のほかの一辺に向かって、まっすぐ進み、再びその辺で曲がり、空気
  中にでる。と同時に、光は、白色光の一点から色棒に分散する。三角形の
  一点から下向きに進むときは、一番上にある光が一番小さく曲げられ、
  ガラスを通して、もっとも短い距離を移動して、赤として出現する。
  下の方ほど、ガラスの三角のくさびは広くなり、光はさらに少し曲げられ、
  プリズムに入ると、ガラスをさらに長く移動して反対側から空気中に紫
  としてでる。その間には虹のすべての色、赤、橙、黄、緑、青、藍、紫が
  ある。プリズムをカーテンの小さな穴を通して暗室に入ってきた光線に
  対して、直立させておく(カメラの中を暗くするやり方のようにする)と、
  色のスペクトルを、窓の反対側の壁に表示することができる。

   アリストテレス的な見方において、ガラスを通して最短距離を移動して
  きた白色光は、もっとも変化を受けず、赤い光となる。ガラスを通して、
  もう少し長く移動してきた白色光は、より変化を受け、黄色になる、等々
  して紫までいく。

   ニュートンは実際にこの考えについて自分で作ったプリズムとレンズの
  両方を用いて実験し、レンズを異なる形にすることで、色の変化を最小
  にしようとした。彼は、異なる色の光線を初めて区別した人であり、
  スペクトルの七色について名前を付けた。

   しかし、このときニュートンが行った実験で最も重要なのは、二つ目の
  三角のくさびを最初のプリズムのうしろに、しかし反対向きにつけただけ
  のものである。最初の、一番上向きのプリズムは白色光の点を、虹の
  スペクトルに分散させた。二つ目の下向きのプリズムは、スペクトルの
  分散した色をもとの白色光の点に集めた。光をもっとあついガラスを通した
  のに、それはそれ以上乱れることはなく、しかし以前のまとまった状態に
  戻ったのである。

   ニュートンが気づいたように、これは白色光が、そもそも「純粋」では
  なく、虹のすべての色の混合であることを示している。異なる光の色は、
  屈折されるとき、異なる量曲がるが、すべての色は元の白色光の点の中に
  存在する。それは画期的なアイディアであり、なぜなら、アリストテレス
  派哲学の基礎を覆し、しかも堅固な実験根拠に基づいているからである。

   最初に物の性質を念入りに調べ、実験によりその性質を確認し、そして
  あせらずに性質を説明する仮説へと進むのが、もっとも安全でもっとも良い
  自然哲学の方法であろう。実験が仮説の上に成り立っているのでない限り、
  仮説が物の性質を決めるべきではなく、物の性質により仮説を採用すべき
  なのである。

  \begin{subsubanswers}
  \SubSubAnswer
    {\bf アリストテレス学派}\\
     白色光は``純粋な''光であり、ガラスの中を通ることにより、
    ``汚され''てしまう。(プリズム中で光路の短い光はより少なく変化し
    赤色になり、光路の長い光は大きく変化し青色になる。)

    {\bf ニュートン}\\
     白色光は虹の七色の混合した光である。(二つのプリズムによりまた
    白色に戻すことができる。これはアリストテレス学派の説では説明
    できない。)

  \SubSubAnswer
    全訳参照のこと。

  \end{subsubanswers}

\SubAnswer
\baselineskip=12pt
   The thought that there would be another solar system have invoked
  imagination of people for a long time. But scientists still cannot
  detect any planets at the precision to satisfy their curiosity.
  Even if the close fixed stars had planets like ones of our solar
  system, it would be difficult to distinguish these planets because
  they would be too small and dark. If there are many stars that have
  a lot of big planets like Jupiter, however, it's time for us to be
  able to find out these planets with various research methods.
  Sensitivity and time on detecting them with many technique depend
  on their orbital radius, that is, the problem is how long the
  distance between fixed star and planet is.
\baselineskip=15pt

\end{subanswers}
\end{answer}



\end{document}
