\documentclass[fleqn]{jbook}
\usepackage{physpub}

\begin{document}

\begin{question}{$B650i(B $B1Q8l(B}{}
\begin{subquestions}
\SubQuestion
  $B<!$NJ8>O$rFI$_!"2<$N@_Ld$KEz$($J$5$$!#(B
\baselineskip=12pt

  $B!!(BModern computers have greatly extended the scope of glacier
  modeling. \underlineeng{(a)A model is a  means of reducing a
  complex real situation to a simple closed system that represents
  the essential features and to which the laws of physics can be 
  applied. Modeling can serve three purposes: experimentation, 
  explanation, and prediction.} Experimentation, discovering the 
  effect of changing the values of the controlling variables, is
  often the most useful; it can never be done in the real worlds.
  An explanation sometimes be an illusion; the fact that the model 
  with adjustable parameters produces plausible numerical values 
  does not prove that the underlying assumptions are correct. Most 
  models can be used for prediction, but first they must be tested 
  against data. Unambiguous testing is difficult and the temptation
  to use all the data to ``tune'' the model by adjusting parameters
  must be resisted. Although these \underline{(b)pitfalls}
  have not always been avoided, the scope of some recent ice-sheet
  models is impressive.

  $B!!(BThe approach here emphasizes the physics, combined where necessary
  with mathematics. No apology is made for introducing mathematics.
  \underline{(c)In the author's opinion}, a mere handful of
  mathematical physicists, who may seldom set foot on a glacier, have
  contributes far more to the understanding of the subject than have a
  hundred measures of ablation stakes or recorders of advances and
  retreats of glacier termini. This is not to say that the latter are
  unimportant; in glaciology, as in other branches of science, there
  is a place for both the theoretical and the experimental approach.
  But the two should be coordinated, and the experiments should be
  designed to solve specific problems. \underlineeng{(d)Too often
  in the past, glaciological measurements have been made on the
  premise that the mere acquisition of data is a useful contribution
  itself. This is seldom the case.}

  glacier : $BI92O(B \quad
  stake   : $B9:(B   \quad
  termini : $B=*C<(B
\baselineskip=15pt
  \begin{subsubquestions}
  \SubSubQuestion
    $B2<@~It(B(a)$B$*$h$S(B(d)$B$rFbMF$,$o$+$k$h$&$KOBLu$;$h!#(B
  \SubSubQuestion
    $B2<@~It(B(b) pitfalls$B$O2?$r;X$7$F$$$k$+!#F|K\8l$G@bL@$;$h!#(B
  \SubSubQuestion 
    $B2<@~It(B(c) the author's opinion $B$KH?BP$NN)>l$G!"1Q8l$G(B50--100words
    $B$G<+J,$N0U8+$r=R$Y$h!#(B
  \end{subsubquestions}


\SubQuestion
  $B0J2<$NJ8>O$rFI$_!"J8Cf$NFbMF$K1h$C$F!"2<$NLd$$$KF|K\8l$GEz$($J$5$$!#(B
\baselineskip=12pt

  $B!!(BThe important thing about Newton's theory of colours is not just
  that he was right, but the way in which he arrives at his
  conclusions. Before Newton, the way philosophers developed their
  ideas about the natural world was largely through pure thought.
  Descartes, for example, thought about the way in which light might
  be transmitted from a bright object to the eye, but he did not
  carry out experiments to test his ideas. Of course, Newton was
  not the first experimenter --- Galileo, in particular, pointed the
  way with studies of the way in which balls rolled down inclined
  planes, and with his work on pendulums. But Newton was the first
  person to express clearly the basis of what became the scientific
  method --- the combination of ideas (hypothesis), observation and
  experiment on which modern science rests.

  $B!!(BNewton's theory of colours emerged from experiments he carried
  out during his enforced sabbatical from Cambridge. By 1665, the
  fact that a ray of sunlight could be turned into a rainbow-like
  spectrum of colours by passing it through a triangular glass
  prism was well known. The standard explanation of the effect
  was based on the Aristotelian idea that white light represented
  a pure, unadulterated form, and that it became corrupted by
  passing through the glass. When the light enters the prism, it
  is bent, and then follows a straight line to the other side of
  the triangle, where it bends again as it emerges into the air.
  At the same time, the light is spread out, from a single spot
  of white light into a bar of colours. Working downwards from
  the point of the triangle, the light at the top is bent least,
  and travels the shortest distance through the glass, emerging
  as red. Lower down, where the triangular wedge of glass is wider,
  light which has been bent slightly more as it enters the prism
  travels further through the glass, and emerges into the air on
  the other side as violet. In between, there are all the colours
  of the rainbow --- red, orange, yellow, green, blue, indigo, and
  violet. Using a prism held up to the ray of light entering a
  darkened room through a small hole in the curtain (rather like
  the camera obscura set-up), the spectrum of colours can be
  displayed on the wall opposite the window.

  $B!!(BOn the Aristotelian view, white light that had travelled the
  shortest distance through the glass was modified least, and
  become red light. While light that had travelled a little further
  through the glass was modified more, and become yellow --- and
  so on all the way down to violet.

  $B!!(BNewton actually tested these ideas, using both prisms and lenses
  which he ground himself, trying to minimize the colour change
  by making lenses in different shapes. He was the first person
  to distinguish the rays of different colours, and he named the
  seven colours of the spectrum.

  $B!!(BBut the most important experiment Newton carried out at this
  time simply consisted of placing a second triangular wedge of
  glass behind the first prism but the other way up. The first
  prism, point uppermost, spread a spot of white light into a
  rainbow spectrum. The second prism, point downwards, combined
  the spread-out colours of the spectrum back into a spot of white
  light. Even though the light had passed through a further
  thickness of glass, it had not become more corrupted, but had
  returned to its former purity.

  $B!!(BAs Newton realized, this shows that white light is not `pure`
  at all, but is a mixture of all the colours of the rainbow.
  Different colours of light bend by different amounts when they
  are refracted, but all the colours are present in the original
  white spot of light. It was a revolutionary idea, both because
  it overturned a basic tenet of Aristotelian philosophy and
  because it rested upon the secure foundation of experiment.

  $B!!(B\underlineeng{``The best and safest method of philosophizing
  seems to be, first to enquire diligently into the properties of
  things, and to establish those properties by experiment and then
  to proceed more slowly to hypotheses for the explanation of them.
  For hypotheses should be employed only in explaining the
  properties of things, but not assumed in determining them;
  unless so far as they may furnish experiments.''}

  \rightline{John Gribbin(1995) : Schr\"{o}dinger's kittens and 
  the search for reality}

  unadulterated  : $B$^$<J*$N$J$$(B \quad
  camera obscura : ($B%+%a%i$N(B)$B0EH"(B \quad
  tenet          : $B655A(B \quad
  enquire        : inquire
\baselineskip=15pt
  \begin{subsubquestions}
  \SubSubQuestion
    $BGr?'8w$,%,%i%9$N;03Q%W%j%:%`$rDL2a$9$k$HFz?'$KJ,$+$l$k$3$H(B($BJ,8w(B)
    $B$K$D$$$F%"%j%9%H%F%l%93XGI$H%K%e!<%H%s$N9M$(J}$N0c$$$r=R$Y$h(B
    (200$B;z0JFb(B)$B!#(B

  \SubSubQuestion
    $B2<@~It$N%K%e!<%H%s$N8@MU$rOBLu$;$h!#(B
  \end{subsubquestions}

\SubQuestion
  $B2<$NJ8>O$r1QLu$;$h!#(B

  $B!!;d$?$A$NB@M[0J30$N@1$N$^$o$j$K$bOG@1$,$"$k$+$b$7$l$J$$$H$$$&(B
  $B9M$($O!"D9$$4V?M!9$NA[A|$r$+$-$?$F$F$-$^$7$?!#$1$l$I$b!"$3$l$^$G(B
  $B2J3X<T$O$3$N9%4q?4$rK~B-$5$;$k$[$I$N@:EY$GOG@1$r8!=P$9$k$3$H$,(B
  $B$G$-$^$;$s$G$7$?!#$?$H$(!"6a$/$N@1$K;d$?$A$NB@M[7O$K$"$k$h$&$JOG@1(B
  $B$,$"$C$?$H$7$F$b!"0E$/$F>.$5$9$.$k$N$G8+$o$1$k$N$,Fq$7$+$C$?$+$i(B
  $B$G$9!#$7$+$7!"$b$7LZ@1$N$h$&$J5pBgOG@1$r;}$D@1$,$?$/$5$s$"$k$J$i$P!"(B
  $B:#F|$"$k$5$^$6$^$JC5:wJ}K!$K$h$C$FOG@1$,8+$D$+$C$F$b$h$$;~4|$K(B
  $B$J$j$^$7$?!#$$$m$$$m$J5;=Q$r;H$C$F$N8!=P$N$?$a$N46EY$H;~4V$OOG@1(B
  $B$N50F;H>7B$K$h$C$F0[$J$j$^$9!#$9$J$o$A!"LdBj$O@1$HOG@1$N4V$N5wN%(B
  $B$,$$$/$i$+$H$$$&$3$H$G$9!#(B

  $BB@M[7O(B   : Solar System \quad
  $BLZ@1(B     : Jupiter \quad
  $B50F;H>7B(B : orbital radius

  \end{subquestions}
\end{question}
\begin{answer}{$B650i(B $B1Q8l(B}{}
\begin{subanswers}
\SubAnswer
  {\bf $BA4Lu(B}

  $B!!8=Be$N%3%s%T%e!<%?!<$OBg$-$/I92O%b%G%k$NNN0h$rH/E8$5$;$?!#(B
  \underlinejpn{$B!J(Ba$B!KJ#;($J8=<B$N4D6-$r!"I,MW$JFCD'$r;}$AJ*M}K!B'$,E,MQ2DG=$JC1=cJD:?7O$K4T85$9$k<jCJ$,%b%G%k$G$"$k!#%b%G%j%s%0$K$O(B3$B$D$NL\E*$,$"$k!#$9$J$o$A!"<B83!"@bL@!"M=8@$G$"$k!#(B}
  $B<B@$3&$G$O7h$7$F$G$-$J$$BP>]JQ?t$NCM$rJQ$($k$3$H$K$h$k8z2L$rH/8+(B
  $B$9$k$K$*$$$F!"<B83$O$7$P$7$P$b$C$H$bM-MQ$G$"$k!#@bL@$O;~!9;W$$(B
  $B0c$$$H$J$C$F$$$k!#$D$^$j!"D4@a$7$?%Q%i%a!<%?!<$K$h$k%b%G%k$,(B
  $B$b$C$H$b$i$7$$CM$r$?$/$5$s$D$/$C$F$b!"$=$N4pAC$H$J$C$F$$$k2>Dj$,(B
  $B@5$7$$$3$H$r<($7$?$3$H$K$J$i$J$$!#B?$/$N%b%G%k$OM=8@$KMQ$$$k$3$H$,(B
  $B$G$-$k$,!":G=i$O%G!<%?$KBP$9$k%F%9%H$r$9$kI,MW$,$"$k!#[#Kf$G$J$$(B
  $B%F%9%H$OFq$7$/!"$9$Y$F$N%G!<%?$rMQ$$$FD4@a$7$?%Q%i%a!<%?!<$K$h$k(B
  $B%b%G%k$rD4@a$9$k$H$$$&M6OG$OB`$1$J$1$l$P$J$i$J$$!#$3$l$i$NMn$H$77j(B
  $B$O;~!92sHr$5$l$F$$$J$$$,!":G6a$NI92OAX%b%G%k$NHO0O$O0u>]E*$J$b$N$G(B
  $B$"$k!#(B

  $B!!$3$3$G$N%"%W%m!<%A$OJ*M}$r6/D4$7!"I,MW$J$H$3$m$K?t3X$rAH$_(B
  $BF~$l$k!#?t3X$rF3F~$9$k$3$H$KBP$9$k<aL@$O$7$J$$!#Cx<T$N0U8+$K$h$k$H!"(B
  $B$[$s$N0l0.$j$NJ*M}?t3X<T$O!"H`$i$O$[$H$s$II92O$KB-$rF'$_9~$`$3$H$O(B
  $B$J$$$,!"?tI42s$N:oK`9:$NB,Dj$dI92O$N=*C<$N?JB`$N5-O?$h$j$b;v>]$N(B
  $BM}2r$X$N9W8%EY$,Bg$-$$!#$3$l$O8e<T$NJ}$,=EMW$G$J$$$3$H$r$$$&$o$1(B
  $B$G$O$J$$!#$[$+$N2J3X$HF1MM$K!"I92O3X$K$*$$$FM}O@E*$JLL$H<B83E*$J(B
  $BLL$NN>LL$,$"$k!#$7$+$7$=$NFs$D$O6(NO$r$9$Y$-$G$"$j!"<B83$OFCJL$J(B
  $BLdBj$r2r$/$?$a$K@_7W$5$l$J$1$l$P$J$i$J$$!#(B
  \underlinejpn{$B!J(Bd$B!K0JA0$O!"I92O3X$K$*$1$kB,Dj$O!"C1$J$k%G!<%?<hF@$,I92O3X$KBP$9$kM-8z$J9W8%$K$J$k!"$H$$$&A0Ds$N2<$K9T$J$o$l$k$3$H$,$[$H$s$I$@$C$?!#$7$+$7$3$N$h$&$J$3$H$O$a$C$?$K$J$$$b$N$G$"$k!#(B}

  \begin{subsubanswers}
  \SubSubAnswer
    $BA4Lu;2>H$N$3$H!#(B

  \SubSubAnswer
     $B<!$N(B3$B$D$NMn$H$77j$N$3$H!#(B\\
     $B<B83$K$D$$$F!D(B%
       $B<B:]$K9T$J$&$3$H$,$G$-$J$$!#(B\\
     $B@bL@$K$D$$$F!D(B%
       $B2DJQ%Q%i%a!<%?$r4^$`7k2L$O$"$k2>Dj$N>ZL@$K$O$J$i$J$$!#(B\\
     $BM=8@$K$D$$$F!D(B%
       $B<B83%G!<%?$NJ}$rM=8@$K9g$o$;$F$7$^$&$3$H$,$"$k!#(B

  \SubSubAnswer
  \baselineskip=12pt
    $B!!(BMost of the mathematical physicists have not visited a glacier,
    so that sometimes they do not really understand about the glacier
    and the theory they constructed might become a only brain exercise.
    In the other hand, the experimentists can use the data to assume
    what have happened in the past, so the results are quite close to
    the real world phenomenon.
  \baselineskip=15pt
  \end{subsubanswers}

\SubAnswer
  {\bf $BA4Lu(B}

  $B!!%K%e!<%H%s$N?'$NM}O@$N=EMW@-$O$=$N@5$7$5$K$"$k$@$1$G$J$/!"H`$,G!2?(B
  $B$K$7$F$=$N7kO@$KC#$7$?$+$K$b$"$k!#%K%e!<%H%s0JA0$O!"E/3X<TC#$,<+A3(B
  $B3&$K$D$$$F$N9M$($rH/E8$5$;$?J}K!$O=c?h$J;W9M$rDL$8$F$G$"$C$?!#(B
  $BNc$($P!"%G%+%k%H$O8w$OL@$k$$J*BN$+$iL\$KEA$o$C$F$/$k;EJ}$K$D$$$F(B
  $B9M$($F$$$?$,!"H`$O$=$N9M$($r;n$9<B83$r9T$o$J$+$C$?!#$b$A$m$s!"(B
  $B%K%e!<%H%s$O:G=i$N<B832H$G$O$J$+$C$?!#FC$K%,%j%l%*$O%\!<%k$,<PLL$r(B
  $BE>$,$jMn$A$k;EJ}$K$D$$$F$N8&5f$K$*$$$F$H!"?6$j;R$G$NH`$N;E;v$K$*$$$F(B
  $B<B83$9$k$H$$$&$=$NJ}K!$r;XE&$7$?!#$7$+$7%K%e!<%H%s$,=i$a$F2J3XE*$J(B
  $BJ}K!!=!=8=Be2J3X$,Mj$C$F$$$k2>@b!"4Q;!$H<B83$NAH$_9g$o$;!=!=$K$J$C$?(B
  $B$3$H$N4pAC$K$D$$$F$O$C$-$j$H<($7$??M$G$"$k!#(B

  $B!!%K%e!<%H%s$N?'$NM}O@$O%1%s%V%j%C%8$+$iM?$($i$l$?5Y2KG/EY$N4V$K(B
  $B9T$C$?<B83$+$i@8$^$l$?!#(B1665$BG/$^$G$K!"B@M[8w$,;03Q%,%i%9%W%j%:%`$r(B
  $BDL$9$3$H$GFz$_$?$$$J?'$N%9%Z%/%H%k$K$9$k;v$,$G$-$k$H$$$&;v<B$O$h$/(B
  $BCN$i$l$F$$$?!#$=$N8z2L$K$D$$$F$N4pK\E*$J@bL@$OGr?'8w$O=c?h$G:.$<(B
  $B$b$N$,$J$$J*BN$G$"$j!"%,%i%9$rDL$k$3$H$GMp$5$l$k$h$&$K$J$k$H$$$&(B
  $B%"%j%9%H%F%l%9$N9M$($K4p$E$$$F$$$?!#8w$,%W%j%:%`$KF~$k$H!"6J$,$j!"(B
  $B;03Q7A$N$[$+$N0lJU$K8~$+$C$F!"$^$C$9$0?J$_!":F$S$=$NJU$G6J$,$j!"6u5$(B
  $BCf$K$G$k!#$HF1;~$K!"8w$O!"Gr?'8w$N0lE@$+$i?'K@$KJ,;6$9$k!#;03Q7A$N(B
  $B0lE@$+$i2<8~$-$K?J$`$H$-$O!"0lHV>e$K$"$k8w$,0lHV>.$5$/6J$2$i$l!"(B
  $B%,%i%9$rDL$7$F!"$b$C$H$bC;$$5wN%$r0\F0$7$F!"@V$H$7$F=P8=$9$k!#(B
  $B2<$NJ}$[$I!"%,%i%9$N;03Q$N$/$5$S$O9-$/$J$j!"8w$O$5$i$K>/$76J$2$i$l!"(B
  $B%W%j%:%`$KF~$k$H!"%,%i%9$r$5$i$KD9$/0\F0$7$FH?BPB&$+$i6u5$Cf$K;g(B
  $B$H$7$F$G$k!#$=$N4V$K$OFz$N$9$Y$F$N?'!"@V!"\t!"2+!"NP!"@D!"Mu!";g$,(B
  $B$"$k!#%W%j%:%`$r%+!<%F%s$N>.$5$J7j$rDL$7$F0E<<$KF~$C$F$-$?8w@~$K(B
  $BBP$7$F!"D>N)$5$;$F$*$/(B($B%+%a%i$NCf$r0E$/$9$k$d$jJ}$N$h$&$K$9$k(B)$B$H!"(B
  $B?'$N%9%Z%/%H%k$r!"Ak$NH?BPB&$NJI$KI=<($9$k$3$H$,$G$-$k!#(B

  $B!!%"%j%9%H%F%l%9E*$J8+J}$K$*$$$F!"%,%i%9$rDL$7$F:GC;5wN%$r0\F0$7$F(B
  $B$-$?Gr?'8w$O!"$b$C$H$bJQ2=$r<u$1$:!"@V$$8w$H$J$k!#%,%i%9$rDL$7$F!"(B
  $B$b$&>/$7D9$/0\F0$7$F$-$?Gr?'8w$O!"$h$jJQ2=$r<u$1!"2+?'$K$J$k!"Ey!9(B
  $B$7$F;g$^$G$$$/!#(B

  $B!!%K%e!<%H%s$O<B:]$K$3$N9M$($K$D$$$F<+J,$G:n$C$?%W%j%:%`$H%l%s%:$N(B
  $BN>J}$rMQ$$$F<B83$7!"%l%s%:$r0[$J$k7A$K$9$k$3$H$G!"?'$NJQ2=$r:G>.(B
  $B$K$7$h$&$H$7$?!#H`$O!"0[$J$k?'$N8w@~$r=i$a$F6hJL$7$??M$G$"$j!"(B
  $B%9%Z%/%H%k$N<7?'$K$D$$$FL>A0$rIU$1$?!#(B

  $B!!$7$+$7!"$3$N$H$-%K%e!<%H%s$,9T$C$?<B83$G:G$b=EMW$J$N$O!"Fs$DL\$N(B
  $B;03Q$N$/$5$S$r:G=i$N%W%j%:%`$N$&$7$m$K!"$7$+$7H?BP8~$-$K$D$1$?$@$1(B
  $B$N$b$N$G$"$k!#:G=i$N!"0lHV>e8~$-$N%W%j%:%`$OGr?'8w$NE@$r!"Fz$N(B
  $B%9%Z%/%H%k$KJ,;6$5$;$?!#Fs$DL\$N2<8~$-$N%W%j%:%`$O!"%9%Z%/%H%k$N(B
  $BJ,;6$7$??'$r$b$H$NGr?'8w$NE@$K=8$a$?!#8w$r$b$C$H$"$D$$%,%i%9$rDL$7$?(B
  $B$N$K!"$=$l$O$=$l0J>eMp$l$k$3$H$O$J$/!"$7$+$70JA0$N$^$H$^$C$?>uBV$K(B
  $BLa$C$?$N$G$"$k!#(B

  $B!!%K%e!<%H%s$,5$$E$$$?$h$&$K!"$3$l$OGr?'8w$,!"$=$b$=$b!V=c?h!W$G$O(B
  $B$J$/!"Fz$N$9$Y$F$N?'$N:.9g$G$"$k$3$H$r<($7$F$$$k!#0[$J$k8w$N?'$O!"(B
  $B6~@^$5$l$k$H$-!"0[$J$kNL6J$,$k$,!"$9$Y$F$N?'$O85$NGr?'8w$NE@$NCf$K(B
  $BB8:_$9$k!#$=$l$O2h4|E*$J%"%$%G%#%"$G$"$j!"$J$<$J$i!"%"%j%9%H%F%l%9(B
  $BGIE/3X$N4pAC$rJ$$7!"$7$+$b7x8G$J<B83:,5r$K4p$E$$$F$$$k$+$i$G$"$k!#(B

  $B!!:G=i$KJ*$N@-<A$rG0F~$j$KD4$Y!"<B83$K$h$j$=$N@-<A$r3NG'$7!"$=$7$F(B
  $B$"$;$i$:$K@-<A$r@bL@$9$k2>@b$X$H?J$`$N$,!"$b$C$H$b0BA4$G$b$C$H$bNI$$(B
  $B<+A3E/3X$NJ}K!$G$"$m$&!#<B83$,2>@b$N>e$K@.$jN)$C$F$$$k$N$G$J$$8B$j!"(B
  $B2>@b$,J*$N@-<A$r7h$a$k$Y$-$G$O$J$/!"J*$N@-<A$K$h$j2>@b$r:NMQ$9$Y$-(B
  $B$J$N$G$"$k!#(B

  \begin{subsubanswers}
  \SubSubAnswer
    {\bf $B%"%j%9%H%F%l%93XGI(B}\\
    $B!!Gr?'8w$O(B``$B=c?h$J(B''$B8w$G$"$j!"%,%i%9$NCf$rDL$k$3$H$K$h$j!"(B
    ``$B1x$5$l(B''$B$F$7$^$&!#(B($B%W%j%:%`Cf$G8wO)$NC;$$8w$O$h$j>/$J$/JQ2=$7(B
    $B@V?'$K$J$j!"8wO)$ND9$$8w$OBg$-$/JQ2=$7@D?'$K$J$k!#(B)

    {\bf $B%K%e!<%H%s(B}\\
    $B!!Gr?'8w$OFz$N<7?'$N:.9g$7$?8w$G$"$k!#(B($BFs$D$N%W%j%:%`$K$h$j$^$?(B
    $BGr?'$KLa$9$3$H$,$G$-$k!#$3$l$O%"%j%9%H%F%l%93XGI$N@b$G$O@bL@(B
    $B$G$-$J$$!#(B)

  \SubSubAnswer
    $BA4Lu;2>H$N$3$H!#(B

  \end{subsubanswers}

\SubAnswer
\baselineskip=12pt
  $B!!(BThe thought that there would be another solar system have invoked
  imagination of people for a long time. But scientists still cannot
  detect any planets at the precision to satisfy their curiosity.
  Even if the close fixed stars had planets like ones of our solar
  system, it would be difficult to distinguish these planets because
  they would be too small and dark. If there are many stars that have
  a lot of big planets like Jupiter, however, it's time for us to be
  able to find out these planets with various research methods.
  Sensitivity and time on detecting them with many technique depend
  on their orbital radius, that is, the problem is how long the
  distance between fixed star and planet is.
\baselineskip=15pt

\end{subanswers}
\end{answer}



\end{document}
