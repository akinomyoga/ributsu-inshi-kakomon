\documentclass[fleqn]{jbook}
\usepackage{physpub}

\begin{document}

\begin{question}{$B650i(B $B1Q8l(B}{}

\begin{subquestions}
\SubQuestion
  $B0J2<$N1QJ8$rFI$_!"@_Ld$KEz$($h!#2rEz$O2rEzMQ;f$N=jDjMs$K5-$;!#(B
\baselineskip=12pt

  $B!!(BA future of climate warming is often described as an increase in
  the greenhouse effect. It is easy to envision a glassed-in building
  trapping the heat of the sun, creating a climate that is quite
  different from the surroundings.
  \underlineeng{(a)The Earth with its atmosphere acts very much
  like a greenhouse. Without our atmosphere, the climate on the
  surface of the earth would be very harsh with nothing to prevent the
  planet's heat from escaping into space.}
  Instead of glass, the Earth is surrounded by gases, and among them
  are gases that have an ability to trap heat.

  $B!!(BOne of the major greenhouse effect gases is water vapor. Other
  greenhouse gases are carbon dioxide, methane, nitrous oxide and
  chlorofluorocarbons(CFCs).
  \underlineeng{(b)It is these gases that are the cause for
  concern when talking about man-made climate warming and climate
  change.}
  The concentrations of all of them are increasing in the atmosphere
  and the prediction is that their accumulation is enhancing the
  natural greenhouse effect. It will become a warmer greenhouse
  because the gases will be even more effective in trapping the heat
  trying to leave the atmosphere.

  $B!!(BA planet is, however, both larger and more complex than an
  ordinary greenhouse and the atmosphere does not evenly spread an
  increase in temperature.
  \underlineeng{(c)Rather, the Earth with its atmosphere can be
  described as a playground for weather systems, where air, water and
  land interplay to determine the climatic patterns in different
  regions.}
  If North Atlantic low pressure path moves a little north or south,
  this might not seem like a big change on the global scale, but for
  the weather of Northern Europe it is what determines the number of
  clear and rainy days or if there will be real winter weather. A
  shift in the monsoon further north might bring more rain to the dry
  sub-Saharan region, but it is difficult to predict how much and when.
  And no one knows how fast the rain water will evaporate as the
  temperature may also increase. \underlineeng{(d)This complexity
  is one of the reasons scientists are still reluctant to make
  regional predictions about the effects of climate warming, even when
  they are quite confident about the global trends.}
%
  \begin{flushright}
    ---quoted, with modifications, from Annika Nilsson,\\
    \em Greenhouse Earth, \em John Wiley \& Sons, 1992.
  \end{flushright}

  carbon dioxide : $BFs;@2=C:AG(B \quad
  methane : $B%a%?%s(B \quad
  nitrous oxide : $B0l;@2=FsCbAG(B \\
  chlorofluorocarbons(CFCs) : $B%U%m%s(B
\baselineskip=15pt

  \noindent{[$B@_Ld(B]}
  \begin{subsubquestions}
  \SubSubQuestion
    $B2<@~It(B(a)$B$r(B100$B;z0JFb$GOBLu$;$h!#(B
  \SubSubQuestion
    $B2<@~It(B(b)$B$r(B100$B;z0JFb$GOBLu$;$h!#(B
  \SubSubQuestion
    $B2<@~It(B(c)$B$r(B100$B;z0JFb$GOBLu$;$h!#(B
  \SubSubQuestion
    $B2<@~It(B(d)$B$r(B100$B;z0JFb$GOBLu$;$h!#(B
  \end{subsubquestions}





\SubQuestion
  $B0J2<$N1QJ8$rFI$_!"@_Ld$KEz$($h!#2rEz$O2rEzMQ;f$N=jDjMs$K5-$;!#(B
\baselineskip=12pt

  $B!!(B\underlineeng{Ever since it was realized that an organism does
  not pass a simulacrum of itself to the next generation, but instead
  provides it with genetic material containing the information needed
  to construct a progeny organism, we have wanted to define the nature
  of this material and the manner in which its information is
  utilized.}
  Now that we know the physical structure of the genetic material,
  we may state the aim of molecular biology as defining the complexity
  of living organisms in terms of the properties of their constituent
  molecules.

  $B!!(BThe gene is the unit of genetic information. The crucial feature
  of Mendel's work, a century ago, was the realization that the gene
  is a distinct entity. The era of the molecular biology of the gene
  began in 1945 when Schr\"{o}dinger developed the view that the laws
  of physical might be inadequate to account for the properties of the
  genetic material, in particular its stability during innumerable
  generations of inheritance. The gene was expected to obey the laws
  of physics so far established, but it was thought that
  characterizing the genetic material might lead to the discovery of
  new laws of physics, a prospect that brought many physicists into
  biology.

  $B!!(BNow, of course, we know that a gene is a huge molecule, in fact
  part of a vast length of genetic material containing many genes.
  A gene does not function autonomously, but relies upon other cellular
  components for its perpetuation and function. All of these
  activities obey the known laws of physics and chemistry; and it has
  not, in the end, been necessary to invoke the new laws.
%
  \begin{flushright}
    ---quoted from Benjamin Lewin, \em Gene IV,\\
    \em Oxford University Press, 1990.
  \end{flushright}

  simulacrum : $BA|!";Q!"1F!"881F(B \quad
  progeny : $B;RB9!";R6!(B\\
\baselineskip=15pt

  \noindent{[$B@_Ld(B]}
  \begin{subsubquestions}
  \SubSubQuestion
    $B2<@~It$r(B 120 $B;z0JFb$GOBLu$;$h!#(B
  \SubSubQuestion
    $BB?$/$NJ*M}3X<T$,@8J*3X$r;O$a$?$N$O$J$<$+!"Kt!"(B
    $B$=$N7k2L$O$I$&$G$"$C$?$+!"K\J8$K4p$E$-(B 100 $B;z0JFb$G=R$Y$h!#(B
  \end{subsubquestions}




\SubQuestion
  $B0J2<$N1QJ8$rFI$_!"@_Ld$KEz$($h!#2rEz$O2rEzMQ;f$N=jDjMs$K5-$;!#(B
\baselineskip=12pt

  $B!!(BIn a democratic society, the strong support of the general public
  is needed in order to maintain a strong base in science. It is
  essential, therefore, to show the public why that support is
  important.
  \underlineeng{(a)The direct relation between science and
  technology is often difficult to discern, except by hindsight,
  even though there are many examples of observable pathways to the
  use of newly understood or newly recognized phenomena --- witness
  developments in molecular bioscience.}
  There is also a strong strand in our system, that is science, that
  ties together the gathering of all added understanding of nature's
  materials, forces, space, and time with the use of our biosphere
  for the support of all human race through technology. That strand
  is steady stream of educated scientists and engineers that our
  educational has provided over years.

  $B!!(BScience is an important part of our culture but it is vital to
  our continued existence --- hence the educated stream of scientists and
  engineers must continue apace. However, in making sure of the
  continuing success of this important task, the community has taken
  to looking at students at the beginning of the educational pipeline
  only in terms of future professionals or of future major users of
  science. This posture has led the science and technology community
  to set aside the equally important aspect of public literacy in
  science. Although this was not a deliberate decision, it had the
  effect of widening the gap between
  \underlineeng{(b)``us and them''} --- a totally undesirable effect.

  $B!!(BAt least part of the problem lies in our wish to make sure that
  every observable is understood in the best current thinking. This
  ignores the fact that 99\% of the population is not involved in
  science or engineering research, nor do they want to be. Yet many
  are likely to be interested in the observables of nature and the
  best lay explanation of them.

  $B!!(BIndeed, in our own best interest, as well as for the society as
  a whole, we should put our best creative efforts into solving the
  problem of how to fan the interest of nonscience majors in nature's
  phenomena. That means developing laboratory exercises and textbook
  designed to enhance the interest of nonprofessionals in such
  phenomena without losing them in a sea of current explanations.
  The net result of such a course or group of courses would be a
  major seeding in terms of public literacy with respect to science
  and technology with a consequent stronger foundation for public
  support of science in universities.

  $B!!(BIt is not only important that the community recognizes the problem
  but that its best talent should set about to correct it. It is, of
  course, equally important that such actions as are undertaken are
  not at the expense of the science enterprise itself.
%
  \begin{flushright}
    ---quoted, with modifications, from \em Science, \em 1992.
  \end{flushright}

  by hindsight : $B$"$H$K$J$C$F$_$k$H(B \quad
  strand : $B$h$C$?;e(B \\
  biosphere : $BCO5e$G@8L?$,B8:_$9$k$H$3$m!"@8L?7w(B \quad
  literacy : $B65M\(B \quad
  lay : $BJ?0W$J(B
\baselineskip=15pt

  \noindent{[$B@_Ld(B]}
  \begin{subsubquestions}
  \SubSubQuestion
    $B2<@~It(B(a)$B$rOBLu$;$h!#(B
  \SubSubQuestion
    $B2<@~It(B(b)$B$N(B us $B$H(B them $B$O$=$l$>$l2?$r;X$9$+F|K\8l$GEz$($h!#(B
  \SubSubQuestion
    $B$3$NJ8>O$NMW;]$r(B 200 $B;z0JFb$NF|K\8l$K$^$H$a$F5-$;!#(B
  \end{subsubquestions}





\SubQuestion
  $B<!$NJ8>O$N2<@~ItJ,$NOBJ8$r1QLu$;$h!#2rEz$O2rEzMQ;f$N=jDjMs$K5-$;!#(B

  $B!!1'Ch$NL5Ca=x$5!"$9$J$o$A%(%s%H%m%T!<$NAmNL$O>o$K;~4V$H6&$KA}Bg(B
  $B$7$F$$$k!#$"$kJ*BN$NCa=x$,A}2C$G$-$k$N$O!"$^$o$j$GL5Ca=x$NA}J,$,(B
  $BJ*BN$NCa=x$NA}J,$r>e2s$k>l9g$K8B$i$l$k!#(B
  \underlinejpn{$B!J(Ba$B!K@8L?$H$OL5Ca=x$K8~$+$&798~$K5U$i$C$F<+J,<+?H$rJ#@=$9$k$3$H$,=PMh$k$h$&$JCa=x$"$k%7%9%F%`$HDj5A$G$-$k!#(B}
  $B$9$J$o$A!"@8L?$O;w$F$O$$$k$,FHN)$7$?Ca=x$"$k%7%9%F%`$r:n$k$3$H$,(B
  $B$G$-$k!#(B
  \underlinejpn{$B!J(Bb$B!K$3$l$i$N$3$H$r9T$J$&$?$a$K!"%7%9%F%`$O?)J*!"F|8w!"EENO$J$I$NCa=x$"$k%(%M%k%.!<$rG.$N7A$GL5Ca=x$J%(%M%k%.!<$KJQ49$7$J$1$l$P$J$i$J$$!#7k2L$H$7$F$3$N%7%9%F%`$OL5Ca=x$NAmNL$OA}2C$9$k$H$$$&MW@A$rK~$?$9$3$H$,$G$-$k!#(B}
%
  \begin{flushright}
    ---quoted with modifications from a lecture(\em Life in the Universe)
    \em by S. W. Hawking(1991).
  \end{flushright}

  $BL5Ca=x$J(B : disordered




\SubQuestion
  $B<!$NJ8>O$r1QLu$;$h!#2rEz$O2rEzMQ;f$N=jDjMs$K5-$;!#(B

  $B!!2J3X$O!"$=$NK\<A$+$i9M$($F!"$=$l$^$G$N8&5f<T$?$A$K$h$C$F:n$i$l$?(B
  $BBg$-$J7zB$J*$N$F$C$Z$s$K!"?7$7$$:`NA$rIU2C$9$k$3$H$K$h$C$F@.D9$9$k(B
  $B9=B$BN$G$"$k!#0JA0$K2?$,CN$i$l$F$$$?$+$KA4$/L5CN$N8D?M$,!"0UL#$N(B
  $B$"$k?7$7$$9W8%$r$9$k5!2q$O$[$H$s$I$J$$!#$7$?$,$C$F!"?7$7$$8&5f7W2h(B
  $B$r;O$a$kA0$K$O!"$=$N8&5fJ,Ln$N8=>u$rD4$Y>e$2$F$*$/$3$H$,4pK\E*$K(B
  $B=EMW$G$"$k!#(B

\end{subquestions}
\end{question}
\begin{answer}{$B650i(B $B1Q8l(B}{}

\begin{subanswers}
\SubAnswer
  {\bf $BA4Lu(B}

  $B!!5$8u$N29CH2=$N@h9T$-$K$D$$$F!"$h$/29<<8z2L$NA}Bg$,8@$o$l$^$9!#(B
  $BB@M[$NG.$rJa$i$($k%,%i%9D%$j$N(B
  $B7zJ*$,!"$^$o$j$H$+$J$j0[$J$k5$8u$r:n$j=P$9$N$r;W$$$a$0$i$9$N$O!"(B
  $B$?$d$9$$$3$H$G$9!#(B
  \underlinejpn{$B!J(Ba$B!KCO5e$OBg5$$N$*$+$2$G!"29<<F1MM$N8z2L$r;}$A$^$9!#Bg5$$,$J$+$C$?$J$i$P!"COI=$G$N5$8u$O!"CO5e$NG.$,1'Ch$XF($2$k$N$rK8$2$k$b$N$,$J$$$3$H$+$i!"$H$F$b2W9s$J$b$N$H$J$k$G$7$g$&!#(B}(85$B;z(B)
  $B%,%i%9$NBe$o$j$KCO5e$O5$BN$K0O$^$l$F$*$j!"$=$N$$$/$D$+$OG.$rJa$i$((B
  $B$F$*$/NO$,$"$j$^$9!#(B

  $B!!<g$H$7$F29<<8z2L$r5/$3$95$BN$N$R$H$D$O?e>x5$$G$9!#B>$K$O!"Fs;@2=(B
  $BC:AG$d%a%?%s$d0l;@2=FsCbAG!"%U%m%s$,$"$j$^$9!#(B
  \underlinejpn{$B!J(Bb$B!K$3$&$$$C$?5$BN$,?MN`$K$h$k5$8u29CH2=$d5$8uJQ2=$,OCBj$H$J$k;~$N4X?4$N$b$H$K$J$j$^$9!#(B }(41$B;z(B)
  $B$3$l$iA4$F$N=89gBN$,Bg5$Cf$KA}Bg$7$D$D$"$k$3$H$GM=A[$5$l$k$N$O!"(B
  $B$=$&$$$&5$BN$,C_@Q$9$k$3$H$G<+A3$K$h$k29<<8z2L$KGo<V$,$+$+$k$3$H(B
  $B$G$9!#$=$&$J$l$P!"Bg5$$+$iF($l$h$&$H$9$kG.$r$b$C$H8z2LE*$K0z$-$H$I$a(B
  $B$k$h$&$K$J$k$N$G!"CO5e$N29CH2=$,?J$`$3$H$K$J$j$^$9!#(B

  $B!!$7$+$7!"OG@1$O29<<$h$j$:$C$HBg$-$/J#;($G$"$k$N$G!"Bg5$$O29EY$N>e>:(B
  $B$r0lMM$K?J$a$F$/$l$k$o$1$G$O$"$j$^$;$s!#(B
  \underlinejpn{$B!J(Bc$B!K$`$7$m!"Bg5$$r;}$DCO5e$O!"5$>]%7%9%F%`$N1?MQ$N>l$H$_$J$9$3$H$,$G$-$^$9!#CO5e$G$O!"Iw!"?e!"CO$,Aj8_:nMQ$7$FMM!9$JCO0h$G$N5$8u$r7h$a$F$$$k$N$G$9!#(B}(77$B;z(B)$B$b$7!"KLBg@>MNDc5$05$,!">/$7(B
  $BKL$+Fn$X$:$l$F$$$l$P!"CO5e5,LO$G$OBg$-$JJQ2=$O$_$i$l$J$$$G$7$g$&(B
  $B$1$l$I$b!"KL2$$N5$8u$K$*$$$F$O!"@2$l$d1+$NF|?t$r7h$a$k$3$H$K$J$j$^$9(B
  $B$7!"??E_JB$N5$8u$K$J$k$+$b$7$l$^$;$s!#%b%s%9!<%s$,$:$C$HKLJ}$X$:$l(B
  $B$l$P!"4%4|$N>.%5%O%iCOJ}$K1+$,9_$j$^$9$,!"$$$D$I$l$/$i$$9_$k$+$^$G(B
  $B$OM=Js$7$,$?$$$b$N$,$"$j$^$9!#$=$7$F!"29EY$b>e>:$9$k$+$b$7$l$^$;$s(B
  $B$,!"$=$l$K$D$l$F$I$l$/$i$$$O$d$/1+?e$,>xH/$9$k$+$OC/$b$o$+$j$^$;$s!#(B
  \underlinejpn{$B!J(Bd$B!K$3$&$$$C$?(B $BJ#;($5$rM}M3$N0l$D$H$7$F!"2J3X<T$?$A$O0MA3$H$7$FCO0h$K8B$k5$8u29CH2=$NM=8@$r$9$k$N$Km4m0$7$F$$$^$9!#$b$C$H$b!"CO5e5,LO$G$N0\$jJQ$o$j$K$O$+$J$j$N<+?.$,$"$k$N$G$9$,!#(B}
  (93$B;z(B)



\SubAnswer  

{\bf $BA4Lu(B}

    \underlinejpn{$B!J(Bi$B!K@8J*$O<+J,$N7A$r<!$N@$Be$KEO$9$N$G$O$J$/!"$=$N$+$o$j$K!"0dEAJ*<A$rEO$9$3$H$,H/8+$5$l$?!#$=$N0dEAJ*<A$O;RB9$N@8J*$r9=@.$9$k$?$a$KI,MW$J>pJs$r4^$s$G$$$k!#$=$NH/8+0JMh$:$C$H!";d$?$A$O$3$NJ*<A$N@-<A$d0dEA>pJs$N;H$o$lJ}$rDj$a$?$$$H9M$($F$-$?!#(B}$B!J(B119$B;z(B) $B:#$d!";d$?$A$O$=$N0dEAJ*<A$NJ*M}E*9=B$$rCN$k$h$&$K$J$C$?(B
    $B$N$G!"J,;R@8J*3X$G$O!"9=B$J*<A$NFC@-$rDL$8$F!"@8BN$NB?MM@-$rM}2r$9(B
    $B$k$3$H$rL\I8$K$7$F$$$k$H$$$&$3$H$,$G$-$k$+$b$7$l$J$$!#(B

     $B!!0dEA;R$O!"0dEA>pJs$NC10L$G$"$k!#Ls0l@$5*A0$N%a%s%G%k$N<B83$K$h$C(B
     $B$F!"0dEA;R$OL@$i$+$KB8:_$9$k$H$$$&G'<1$,7hDjE*$K$J$C$?!#0dEA;R$N(B
     $BJ,;R@8J*3X$,;O$^$C$?$N$O!"(B1945$BG/$G!"$=$N$H$-!"%7%e%l%G%#%s%,!<$O!"(B
     $BJ*M}K!B'$O!"0dEAJ*<A$NFC@-!"FC$K!"?tB?$/$N0dEA$NH/8=$N4V$N0dEA;R(B
     $B$N0BDj@-$r@bL@$9$k$N$K$U$5$o$7$/$J$$$H$$$&M=8+$rH/E8$5$;$?!#0dEA(B
     $B;R$O!"4{B8$NJ*M}K!B'$K=>$&$HM=A[$5$l$F$$$?$,!"0lJ}$G!"0dEAJ*<A$r(B
     $BFCDj$9$k$3$H$K$h$C$F!"?7$7$$J*M}K!B'$,H/8+$5$l$k$+$bCN$l$J$$$H9M(B
     $B$($i$l$F$*$j!"$=$N$?$a!"B?$/$NJ*M}3X<T$,@8J*3X$r;O$a$?!#(B

      $B!!8=:_!"L^O@!"0dEA;R$O!"Bg$-$JJ,;R$G$"$C$F!"<B:]!"0dEAJ*<A$N5pBg$J(B
      $BJ*<A$N0lIt$K!"Bt;3$N0dEA;R$r4^$s$G$$$k$3$H$rCN$C$F$$$k!#0dEA;R$O!"(B
      $B<+H/E*$K$OF/$+$:!"0BDj$7$FB8:_$7$?$j5!G=$7$?$j$9$k$N$K$O!":YK&$N(B
      $BB>$NMWAG$,I,MW$H$J$k!#$=$l$i$N3hF0$O!"4{CN$NJ*M}K!B'$K=>$$!"7k6I!"(B
      $B:#$N$H$3$m!"?7$7$$J*M}K!B'$rBG$AN)$F$kI,MW@-$O$J$$$^$^$G$"$k!#(B
%
  \begin{flushright}
    --- Benjamin Lewin, $B!X0dEA;R(BIV$B!Y(B, $B%*%C%/%9%U%)!<%IBg3X=PHG2q(B, 1990  
  \end{flushright}

  \begin{subsubanswers}
  \SubSubAnswer
    $BA4Lu$NK\J8;2>H!#(B
  \SubSubAnswer
    $B0dEAJ*<A$O4{B8$NJ*M}K!B'$NOH$K;YG[$5$l$F$$$k$H9M$($i$l$F$$$?$,!"(B
    $B$3$l$rD4$Y$k$3$H$K$h$j!"?7$7$$J*M}K!B'$rH/8+$G$-$k$HM=A[$5$l$?!#(B
    $B$7$+$7!"$=$N$h$&$J?7$7$$J*M}K!B'$NH/8+$O$J$+$C$?!#(B
  \end{subsubanswers}



\SubAnswer
  {\bf $BA4Lu(B}

  $B!!L1<g<g5A$N<R2q$G$O2J3X$N>fIW$JEZBf$r0];}$9$k$?$a$K0lHLBg=0$N6/$$(B
  $B;Y;}$,I,MW$G$"$k!#$=$N;Y;}$,$J$<=EMW$G$"$k$N$+$rBg=0$K<($9$N$O(B
  $B$=$N$?$aBg@Z$J$3$H$G$"$k!#(B
  \underlinejpn{$B!J(Ba$B!K2J3X$H5;=Q$N$"$$$@$ND>@\$N4X78$O1}!9$K$7$F!"8e$K$J$C$F$o$+$k$b$N$r=|$$$F$O8+Dj$a$k$3$H$O:$Fq$G$"$k!#$b$C$H$b!"J,;R@8J*3X$K$*$1$k?7$?$J8=>]$NM}2r$d(B $BG'<1$,L\$K8+$($F<BMQ2=$5$l$F$$$C$?F;$N$j$NB?$/$NNc$,$"$k$,!#(B}
  $B2f!9$N%7%9%F%`$K$O6/$/$h$C$?;e$b$"$k!"$=$l$O2J3X$G$"$j5;=Q$rDL$8$F!"(B
  $B?MN`$r;Y$($k$?$a$K@8L?7w$rMxMQ$7$F<+A3$NJ*<A!"NO!"6u4V!"$=$7$F;~4V$K(B
  $B4X$9$kM}2r$NB-$79g$o$;$?$b$N$r0l$D$K$^$H$a$k$b$N$G$"$k!#$=$N$h$C$?(B
  $B;e$3$=$,2f!9$N650i%7%9%F%`$,D9G/$KEO$C$F6!5k$7$F$-$?!"65M\$"$k2J3X<T(B
  $B$H5;=Q<T$N$7$C$+$j$7$?N.$l$J$N$G$"$k!#(B

  $B!!2J3X$O2f!9$NJ82=$NBg;v$JItJ,$G$"$k!"$=$7$F2f!9$,B8:_$7B3$1$k$?$a$K(B
  $BI,MWIT2D7g$J$b$N$G$"$k!#8N$K65M\$"$k2J3X<T$H5;=Q<T$NN.$l$OJbD4$r$"$o(B
  $B$;$FB8B3$7$J$/$F$O$$$1$J$$!#$7$+$7$J$,$i!"$3$N=EMW$J2]Bj$r3N<B$K@d$((B
  $B$^$J$/$9$k$?$a$K!"6&F1BN$O650i@)EY=i4|$K$*$$$F@8ELC#$r>-Mh$N%W%m%U%'%C(B
  $B%7%g%J%k$+!"L$Mh$N2J3X$N<gMW$JMxMQ<T$K$J$k$@$m$&$H$7$F$7$+8+$F$3$J$C(B
  $B$+$?!#$3$NBVEY$K$h$C$F2J3X5;=Q$N6&F1BN$O!"0lHL$N?M!9$N2J3X$N65M\$H$$(B
  $B$&F1MM$K=EMW$JB&LL$rOF$KB`$1$F$-$?!#$3$l$O8N0U$K$=$&$7$?$N$G$O$J$$$,!"(B
  ``\underlinejpn{$B!J(Bb$B!K2f!9$HH`$i(B}''$B$N$"$$$@$N%.%c%C%W$NBg$-$5$r9-$2(B
  $B$k8z2L$,$"$C$?!#(B($BA4$/K>$^$7$/$J$$8z2L$G$"$k(B)

  $B!!>/$J$/$H$bLdBj$N0lIt$O!"4QB,$7$&$k$9$Y$F$N$b$N$O8=:_:G9b$NF,G>$K(B
  $B$h$C$F3N<B$KM}2r$7$h$&$H$9$k2f!9$N4jK>$K$"$k!#$3$N$3$H$O?M8}$N$&$A(B
  $99\%$$B$,2J3X!"$^$?$O5;=Q$N8&5f$K4X$o$C$F$$$J$/!"$^$?$=$&$7$?$$$H$b(B
  $B;W$C$F$$$J$$$H$$$&;v<B$rL5;k$7$F$$$k!#$H$O$$$(B?$/$N?M$O<+A3$G4QB,(B
  $B$5$l$k$b$N$d!"$=$l$i$N:GNI$NJ?0W$J@bL@$K6=L#$r;}$C$F$$$k$i$7$$!#(B

  $B!!<B:]$N$H$3$m2f!9<+?H$N$?$a$K!"<R2qA4BN$N$?$a$K$b2f!9$O:GBg$NAOB$E*(B
  $BEXNO$r!"<+A38=>]$KBP$9$k2J3XL565M\$JBg=0$N6=L#$r$$$+$KL%N;$9$k$+$H(B
  $B$$$&$3$H$KCm$,$J$/$F$O$$$1$J$$!#$=$l$,0UL#$9$k$3$H$O<B83<<$G$N<B=,(B
  $B$NH/E8$H$H$b$K!"8=:_$N@bL@$NBg3$86$N$J$+$G<+J,$r8+<:$&$3$H$J$7$K!"(B
  $B$=$s$J8=>]$K$?$$$9$kHs%W%m%U%'%C%7%g%J%k$J?M!9$N4X?4$r9b$a$k$h$&$K(B
  $BJT=8$5$l$?652J=q$NH/E8$G$"$k!#$=$N$h$&$J2]Dx!"$^$?$OJ#?t$N2]Dx$N(B
  $B@5L#$N7k2L$OBg=0$N2J3X5;=Q$KBP$9$k65M\$K$H$C$F<gMW$J<oIU$1$H$J$j!"(B
  $B$=$N5"7k$H$7$FBg=0$K$h$kBg3X$G$N2J3X$N;Y;}$K$H$C$F$O$h$j6/$$EZBf$H(B
  $B$J$k$G$"$m$&!#(B

  $B!!6&F1BN$O$=$NLdBj$K5$IU$/$3$H$@$1$,=EMW$J$N$G$O$J$/$F!":GNI$N<jCJ$G(B
  $B$=$l$r@5$9$h$&$K$9$Y$-$@!#$b$A$m$s<B9T$5$l$k$h$&$J9TF0$,2J3X;v6H(B
  $B<+?H$NB;<:$K$J$i$J$$$h$&$K$9$k$N$bF1MM$K=EMW$G$"$k!#(B
%
  \begin{flushright}
    ---$B%5%$%(%s%9(B1992$B$+$i=$@5$r;\$7$F0zMQ(B
  \end{flushright}

  \begin{subsubanswers}
  \SubSubAnswer
    (a)$B$NOBLu$OK\J8Cf$r;2>H(B
  \SubSubAnswer
    \makebox[15mm]{us}%
      $BI.<T$N$h$&$J2J3X!&5;=Q$N@lLg2H!"%W%m%U%'%C%7%g%J%k(B\\
    \makebox[15mm]{them}%
      $B2J3X$d5;=Q$N8&5f$K7H$o$C$F$$$J$$0lHLBg=0(B

  \SubSubAnswer
    $BBg=0$K2J3X;Y;}$NI,MW@-$r<($9$N$O=EMW$G$"$k$,2J3X$N<BMQE*2ACM$r(B
    $B<($9$N$O:$Fq$G$"$k!#2J3X$O2f!9$KIT2D7g$N$b$N$@$,2J3X$N@lLg2H$r(B
    $BM\@.$9$k$3$H$K@lG0$9$k$"$^$j0lHLBg=0$H$N3J:9$,9-$,$C$F$7$^$C$?!#(B
    $B$[$H$s$I$N?M$O2J3X$K=>;v$7$F$$$J$$$K$b$+$+$o$i$:!"<+A38=>]$K$O(B
    $B6=L#$r;}$C$F$$$k!#$=$l$r<:$o$J$$$h$&$JJ}K!$G2J3X650i$rH/E8$5$;(B
    $B$J$/$F$O$J$i$J$$!#$^$?6&F1BN$O$=$NLdBj$K5$IU$/$@$1$G$O$J$/<B9T(B
    $B$K0\$5$J$/$F$O$J$i$J$$!#(B(199$B;z(B)

  \end{subsubanswers}



\SubAnswer
\baselineskip=12pt

   $B!!(BThe total amount of disorder,or entropy, in the universe always
   increases with time. The order in one body can
   increase provided that the amount of disorder in its surroundings
   increases by a greater amount. \underlineeng{(a)One can define life to be an
   ordered system that can sustain itself against the tendency to
   disorder and can reproduce itself.} That is, it can make similar,
   but independent, ordered systems. 

   $B!!(B\underlineeng{(b)To do these things, the system must convert
   energy in some ordered forms - like food, sunlight, or electric
   power - into disordered energy in the form of heat.In this way, the
   system can satisfy the requirement that the total amount of
   disorder increases.}

 {\bf [$BJL2r(B]}

  (a) Life can be defined as the ordered system that is capable of
  duplicating itself and maintaining itself contrary to the tendency
  toward disorder.

  (b) To do these things, the system has to transform the ordered energy
  such as food, the sunshine and electricpower into the disordered
  energy in the form of heat. Consequently, this system can fill the
  requirement that the total amount of the disorder increases.



\SubAnswer
  $B!!(BEssentially, science is a structure that is brought up by adding
  new material to the top of big buildings by former scholars. One
  scarcely has a chance to make contribution, who knows nothing about
  what is known at that time. So, when we begin new research plan, it
  is basically important to study completely the present situation of
  the theme.

\baselineskip=15pt
\end{subanswers}
\end{answer}


\end{document}
