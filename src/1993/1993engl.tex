\documentclass[fleqn]{jbook}
\usepackage{physpub}

\begin{document}

\begin{question}{教育 英語}{}

\begin{subquestions}
\SubQuestion
  以下の英文を読み、設問に答えよ。解答は解答用紙の所定欄に記せ。
\baselineskip=12pt

   A future of climate warming is often described as an increase in
  the greenhouse effect. It is easy to envision a glassed-in building
  trapping the heat of the sun, creating a climate that is quite
  different from the surroundings.
  \underlineeng{(a)The Earth with its atmosphere acts very much
  like a greenhouse. Without our atmosphere, the climate on the
  surface of the earth would be very harsh with nothing to prevent the
  planet's heat from escaping into space.}
  Instead of glass, the Earth is surrounded by gases, and among them
  are gases that have an ability to trap heat.

   One of the major greenhouse effect gases is water vapor. Other
  greenhouse gases are carbon dioxide, methane, nitrous oxide and
  chlorofluorocarbons(CFCs).
  \underlineeng{(b)It is these gases that are the cause for
  concern when talking about man-made climate warming and climate
  change.}
  The concentrations of all of them are increasing in the atmosphere
  and the prediction is that their accumulation is enhancing the
  natural greenhouse effect. It will become a warmer greenhouse
  because the gases will be even more effective in trapping the heat
  trying to leave the atmosphere.

   A planet is, however, both larger and more complex than an
  ordinary greenhouse and the atmosphere does not evenly spread an
  increase in temperature.
  \underlineeng{(c)Rather, the Earth with its atmosphere can be
  described as a playground for weather systems, where air, water and
  land interplay to determine the climatic patterns in different
  regions.}
  If North Atlantic low pressure path moves a little north or south,
  this might not seem like a big change on the global scale, but for
  the weather of Northern Europe it is what determines the number of
  clear and rainy days or if there will be real winter weather. A
  shift in the monsoon further north might bring more rain to the dry
  sub-Saharan region, but it is difficult to predict how much and when.
  And no one knows how fast the rain water will evaporate as the
  temperature may also increase. \underlineeng{(d)This complexity
  is one of the reasons scientists are still reluctant to make
  regional predictions about the effects of climate warming, even when
  they are quite confident about the global trends.}
%
  \begin{flushright}
    ---quoted, with modifications, from Annika Nilsson,\\
    \em Greenhouse Earth, \em John Wiley \& Sons, 1992.
  \end{flushright}

  carbon dioxide : 二酸化炭素 \quad
  methane : メタン \quad
  nitrous oxide : 一酸化二窒素 \\
  chlorofluorocarbons(CFCs) : フロン
\baselineskip=15pt

  \noindent{[設問]}
  \begin{subsubquestions}
  \SubSubQuestion
    下線部(a)を100字以内で和訳せよ。
  \SubSubQuestion
    下線部(b)を100字以内で和訳せよ。
  \SubSubQuestion
    下線部(c)を100字以内で和訳せよ。
  \SubSubQuestion
    下線部(d)を100字以内で和訳せよ。
  \end{subsubquestions}





\SubQuestion
  以下の英文を読み、設問に答えよ。解答は解答用紙の所定欄に記せ。
\baselineskip=12pt

   \underlineeng{Ever since it was realized that an organism does
  not pass a simulacrum of itself to the next generation, but instead
  provides it with genetic material containing the information needed
  to construct a progeny organism, we have wanted to define the nature
  of this material and the manner in which its information is
  utilized.}
  Now that we know the physical structure of the genetic material,
  we may state the aim of molecular biology as defining the complexity
  of living organisms in terms of the properties of their constituent
  molecules.

   The gene is the unit of genetic information. The crucial feature
  of Mendel's work, a century ago, was the realization that the gene
  is a distinct entity. The era of the molecular biology of the gene
  began in 1945 when Schr\"{o}dinger developed the view that the laws
  of physical might be inadequate to account for the properties of the
  genetic material, in particular its stability during innumerable
  generations of inheritance. The gene was expected to obey the laws
  of physics so far established, but it was thought that
  characterizing the genetic material might lead to the discovery of
  new laws of physics, a prospect that brought many physicists into
  biology.

   Now, of course, we know that a gene is a huge molecule, in fact
  part of a vast length of genetic material containing many genes.
  A gene does not function autonomously, but relies upon other cellular
  components for its perpetuation and function. All of these
  activities obey the known laws of physics and chemistry; and it has
  not, in the end, been necessary to invoke the new laws.
%
  \begin{flushright}
    ---quoted from Benjamin Lewin, \em Gene IV,\\
    \em Oxford University Press, 1990.
  \end{flushright}

  simulacrum : 像、姿、影、幻影 \quad
  progeny : 子孫、子供\\
\baselineskip=15pt

  \noindent{[設問]}
  \begin{subsubquestions}
  \SubSubQuestion
    下線部を 120 字以内で和訳せよ。
  \SubSubQuestion
    多くの物理学者が生物学を始めたのはなぜか、又、
    その結果はどうであったか、本文に基づき 100 字以内で述べよ。
  \end{subsubquestions}




\SubQuestion
  以下の英文を読み、設問に答えよ。解答は解答用紙の所定欄に記せ。
\baselineskip=12pt

   In a democratic society, the strong support of the general public
  is needed in order to maintain a strong base in science. It is
  essential, therefore, to show the public why that support is
  important.
  \underlineeng{(a)The direct relation between science and
  technology is often difficult to discern, except by hindsight,
  even though there are many examples of observable pathways to the
  use of newly understood or newly recognized phenomena --- witness
  developments in molecular bioscience.}
  There is also a strong strand in our system, that is science, that
  ties together the gathering of all added understanding of nature's
  materials, forces, space, and time with the use of our biosphere
  for the support of all human race through technology. That strand
  is steady stream of educated scientists and engineers that our
  educational has provided over years.

   Science is an important part of our culture but it is vital to
  our continued existence --- hence the educated stream of scientists and
  engineers must continue apace. However, in making sure of the
  continuing success of this important task, the community has taken
  to looking at students at the beginning of the educational pipeline
  only in terms of future professionals or of future major users of
  science. This posture has led the science and technology community
  to set aside the equally important aspect of public literacy in
  science. Although this was not a deliberate decision, it had the
  effect of widening the gap between
  \underlineeng{(b)``us and them''} --- a totally undesirable effect.

   At least part of the problem lies in our wish to make sure that
  every observable is understood in the best current thinking. This
  ignores the fact that 99\% of the population is not involved in
  science or engineering research, nor do they want to be. Yet many
  are likely to be interested in the observables of nature and the
  best lay explanation of them.

   Indeed, in our own best interest, as well as for the society as
  a whole, we should put our best creative efforts into solving the
  problem of how to fan the interest of nonscience majors in nature's
  phenomena. That means developing laboratory exercises and textbook
  designed to enhance the interest of nonprofessionals in such
  phenomena without losing them in a sea of current explanations.
  The net result of such a course or group of courses would be a
  major seeding in terms of public literacy with respect to science
  and technology with a consequent stronger foundation for public
  support of science in universities.

   It is not only important that the community recognizes the problem
  but that its best talent should set about to correct it. It is, of
  course, equally important that such actions as are undertaken are
  not at the expense of the science enterprise itself.
%
  \begin{flushright}
    ---quoted, with modifications, from \em Science, \em 1992.
  \end{flushright}

  by hindsight : あとになってみると \quad
  strand : よった糸 \\
  biosphere : 地球で生命が存在するところ、生命圏 \quad
  literacy : 教養 \quad
  lay : 平易な
\baselineskip=15pt

  \noindent{[設問]}
  \begin{subsubquestions}
  \SubSubQuestion
    下線部(a)を和訳せよ。
  \SubSubQuestion
    下線部(b)の us と them はそれぞれ何を指すか日本語で答えよ。
  \SubSubQuestion
    この文章の要旨を 200 字以内の日本語にまとめて記せ。
  \end{subsubquestions}





\SubQuestion
  次の文章の下線部分の和文を英訳せよ。解答は解答用紙の所定欄に記せ。

   宇宙の無秩序さ、すなわちエントロピーの総量は常に時間と共に増大
  している。ある物体の秩序が増加できるのは、まわりで無秩序の増分が
  物体の秩序の増分を上回る場合に限られる。
  \underlinejpn{(a)生命とは無秩序に向かう傾向に逆らって自分自身を複製することが出来るような秩序あるシステムと定義できる。}
  すなわち、生命は似てはいるが独立した秩序あるシステムを作ることが
  できる。
  \underlinejpn{(b)これらのことを行なうために、システムは食物、日光、電力などの秩序あるエネルギーを熱の形で無秩序なエネルギーに変換しなければならない。結果としてこのシステムは無秩序の総量は増加するという要請を満たすことができる。}
%
  \begin{flushright}
    ---quoted with modifications from a lecture(\em Life in the Universe)
    \em by S. W. Hawking(1991).
  \end{flushright}

  無秩序な : disordered




\SubQuestion
  次の文章を英訳せよ。解答は解答用紙の所定欄に記せ。

   科学は、その本質から考えて、それまでの研究者たちによって作られた
  大きな建造物のてっぺんに、新しい材料を付加することによって成長する
  構造体である。以前に何が知られていたかに全く無知の個人が、意味の
  ある新しい貢献をする機会はほとんどない。したがって、新しい研究計画
  を始める前には、その研究分野の現状を調べ上げておくことが基本的に
  重要である。

\end{subquestions}
\end{question}
\begin{answer}{教育 英語}{}

\begin{subanswers}
\SubAnswer
  {\bf 全訳}

   気候の温暖化の先行きについて、よく温室効果の増大が言われます。
  太陽の熱を捕らえるガラス張りの
  建物が、まわりとかなり異なる気候を作り出すのを思いめぐらすのは、
  たやすいことです。
  \underlinejpn{(a)地球は大気のおかげで、温室同様の効果を持ちます。大気がなかったならば、地表での気候は、地球の熱が宇宙へ逃げるのを妨げるものがないことから、とても苛酷なものとなるでしょう。}(85字)
  ガラスの代わりに地球は気体に囲まれており、そのいくつかは熱を捕らえ
  ておく力があります。

   主として温室効果を起こす気体のひとつは水蒸気です。他には、二酸化
  炭素やメタンや一酸化二窒素、フロンがあります。
  \underlinejpn{(b)こういった気体が人類による気候温暖化や気候変化が話題となる時の関心のもとになります。 }(41字)
  これら全ての集合体が大気中に増大しつつあることで予想されるのは、
  そういう気体が蓄積することで自然による温室効果に拍車がかかること
  です。そうなれば、大気から逃れようとする熱をもっと効果的に引きとどめ
  るようになるので、地球の温暖化が進むことになります。

   しかし、惑星は温室よりずっと大きく複雑であるので、大気は温度の上昇
  を一様に進めてくれるわけではありません。
  \underlinejpn{(c)むしろ、大気を持つ地球は、気象システムの運用の場とみなすことができます。地球では、風、水、地が相互作用して様々な地域での気候を決めているのです。}(77字)もし、北大西洋低気圧が、少し
  北か南へずれていれば、地球規模では大きな変化はみられないでしょう
  けれども、北欧の気候においては、晴れや雨の日数を決めることになります
  し、真冬並の気候になるかもしれません。モンスーンがずっと北方へずれ
  れば、乾期の小サハラ地方に雨が降りますが、いつどれくらい降るかまで
  は予報しがたいものがあります。そして、温度も上昇するかもしれません
  が、それにつれてどれくらいはやく雨水が蒸発するかは誰もわかりません。
  \underlinejpn{(d)こういった 複雑さを理由の一つとして、科学者たちは依然として地域に限る気候温暖化の予言をするのに躊躇しています。もっとも、地球規模での移り変わりにはかなりの自信があるのですが。}
  (93字)



\SubAnswer  

{\bf 全訳}

    \underlinejpn{(i)生物は自分の形を次の世代に渡すのではなく、そのかわりに、遺伝物質を渡すことが発見された。その遺伝物質は子孫の生物を構成するために必要な情報を含んでいる。その発見以来ずっと、私たちはこの物質の性質や遺伝情報の使われ方を定めたいと考えてきた。}(119字) 今や、私たちはその遺伝物質の物理的構造を知るようになった
    ので、分子生物学では、構造物質の特性を通じて、生体の多様性を理解す
    ることを目標にしているということができるかもしれない。

      遺伝子は、遺伝情報の単位である。約一世紀前のメンデルの実験によっ
     て、遺伝子は明らかに存在するという認識が決定的になった。遺伝子の
     分子生物学が始まったのは、1945年で、そのとき、シュレディンガーは、
     物理法則は、遺伝物質の特性、特に、数多くの遺伝の発現の間の遺伝子
     の安定性を説明するのにふさわしくないという予見を発展させた。遺伝
     子は、既存の物理法則に従うと予想されていたが、一方で、遺伝物質を
     特定することによって、新しい物理法則が発見されるかも知れないと考
     えられており、そのため、多くの物理学者が生物学を始めた。

       現在、勿論、遺伝子は、大きな分子であって、実際、遺伝物質の巨大な
      物質の一部に、沢山の遺伝子を含んでいることを知っている。遺伝子は、
      自発的には働かず、安定して存在したり機能したりするのには、細胞の
      他の要素が必要となる。それらの活動は、既知の物理法則に従い、結局、
      今のところ、新しい物理法則を打ち立てる必要性はないままである。
%
  \begin{flushright}
    --- Benjamin Lewin, 『遺伝子IV』, オックスフォード大学出版会, 1990  
  \end{flushright}

  \begin{subsubanswers}
  \SubSubAnswer
    全訳の本文参照。
  \SubSubAnswer
    遺伝物質は既存の物理法則の枠に支配されていると考えられていたが、
    これを調べることにより、新しい物理法則を発見できると予想された。
    しかし、そのような新しい物理法則の発見はなかった。
  \end{subsubanswers}



\SubAnswer
  {\bf 全訳}

   民主主義の社会では科学の丈夫な土台を維持するために一般大衆の強い
  支持が必要である。その支持がなぜ重要であるのかを大衆に示すのは
  そのため大切なことである。
  \underlinejpn{(a)科学と技術のあいだの直接の関係は往々にして、後になってわかるものを除いては見定めることは困難である。もっとも、分子生物学における新たな現象の理解や 認識が目に見えて実用化されていった道のりの多くの例があるが。}
  我々のシステムには強くよった糸もある、それは科学であり技術を通じて、
  人類を支えるために生命圏を利用して自然の物質、力、空間、そして時間に
  関する理解の足し合わせたものを一つにまとめるものである。そのよった
  糸こそが我々の教育システムが長年に渡って供給してきた、教養ある科学者
  と技術者のしっかりした流れなのである。

   科学は我々の文化の大事な部分である、そして我々が存在し続けるために
  必要不可欠なものである。故に教養ある科学者と技術者の流れは歩調をあわ
  せて存続しなくてはいけない。しかしながら、この重要な課題を確実に絶え
  まなくするために、共同体は教育制度初期において生徒達を将来のプロフェッ
  ショナルか、未来の科学の主要な利用者になるだろうとしてしか見てこなっ
  かた。この態度によって科学技術の共同体は、一般の人々の科学の教養とい
  う同様に重要な側面を脇に退けてきた。これは故意にそうしたのではないが、
  ``\underlinejpn{(b)我々と彼ら}''のあいだのギャップの大きさを広げ
  る効果があった。(全く望ましくない効果である)

   少なくとも問題の一部は、観測しうるすべてのものは現在最高の頭脳に
  よって確実に理解しようとする我々の願望にある。このことは人口のうち
  $99\%$が科学、または技術の研究に関わっていなく、またそうしたいとも
  思っていないという事実を無視している。とはいえ多くの人は自然で観測
  されるものや、それらの最良の平易な説明に興味を持っているらしい。

   実際のところ我々自身のために、社会全体のためにも我々は最大の創造的
  努力を、自然現象に対する科学無教養な大衆の興味をいかに魅了するかと
  いうことに注がなくてはいけない。それが意味することは実験室での実習
  の発展とともに、現在の説明の大海原のなかで自分を見失うことなしに、
  そんな現象にたいする非プロフェッショナルな人々の関心を高めるように
  編集された教科書の発展である。そのような課程、または複数の課程の
  正味の結果は大衆の科学技術に対する教養にとって主要な種付けとなり、
  その帰結として大衆による大学での科学の支持にとってはより強い土台と
  なるであろう。

   共同体はその問題に気付くことだけが重要なのではなくて、最良の手段で
  それを正すようにすべきだ。もちろん実行されるような行動が科学事業
  自身の損失にならないようにするのも同様に重要である。
%
  \begin{flushright}
    ---サイエンス1992から修正を施して引用
  \end{flushright}

  \begin{subsubanswers}
  \SubSubAnswer
    (a)の和訳は本文中を参照
  \SubSubAnswer
    \makebox[15mm]{us}%
      筆者のような科学・技術の専門家、プロフェッショナル\\
    \makebox[15mm]{them}%
      科学や技術の研究に携わっていない一般大衆

  \SubSubAnswer
    大衆に科学支持の必要性を示すのは重要であるが科学の実用的価値を
    示すのは困難である。科学は我々に不可欠のものだが科学の専門家を
    養成することに専念するあまり一般大衆との格差が広がってしまった。
    ほとんどの人は科学に従事していないにもかかわらず、自然現象には
    興味を持っている。それを失わないような方法で科学教育を発展させ
    なくてはならない。また共同体はその問題に気付くだけではなく実行
    に移さなくてはならない。(199字)

  \end{subsubanswers}



\SubAnswer
\baselineskip=12pt

    The total amount of disorder,or entropy, in the universe always
   increases with time. The order in one body can
   increase provided that the amount of disorder in its surroundings
   increases by a greater amount. \underlineeng{(a)One can define life to be an
   ordered system that can sustain itself against the tendency to
   disorder and can reproduce itself.} That is, it can make similar,
   but independent, ordered systems. 

    \underlineeng{(b)To do these things, the system must convert
   energy in some ordered forms - like food, sunlight, or electric
   power - into disordered energy in the form of heat.In this way, the
   system can satisfy the requirement that the total amount of
   disorder increases.}

 {\bf [別解]}

  (a) Life can be defined as the ordered system that is capable of
  duplicating itself and maintaining itself contrary to the tendency
  toward disorder.

  (b) To do these things, the system has to transform the ordered energy
  such as food, the sunshine and electricpower into the disordered
  energy in the form of heat. Consequently, this system can fill the
  requirement that the total amount of the disorder increases.



\SubAnswer
   Essentially, science is a structure that is brought up by adding
  new material to the top of big buildings by former scholars. One
  scarcely has a chance to make contribution, who knows nothing about
  what is known at that time. So, when we begin new research plan, it
  is basically important to study completely the present situation of
  the theme.

\baselineskip=15pt
\end{subanswers}
\end{answer}


\end{document}
