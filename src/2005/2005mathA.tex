\newcommand{\trd}[1]{{}^{\dag}\hspace{-1mm}#1}
\newcommand{\pdif}[2]{\frac{\partial #1}{\partial #2}}

\begin{answer}{第1問}{石川}
\begin{enumerate}
\item 
  \begin{eqnarray}
  P(a)P(a^{\prime} )&=&\bm{u}_a\bm{u}_{a}^{\dag}\bm{u}_{a^{\prime}}\bm{u}_{a^{\prime} }^{\dag}\\
		  &=& \bm{u}_a(\bm{u}_a,\bm{u}_{a^{\prime} })\bm{u}_{a^{\prime} }^{\dag}\\
                  &=& \delta_{a,a^{\prime} }P(a)
  \end{eqnarray}
\item 
  \begin{eqnarray}
  AP(a)\bm{x} = A\bm{u}_a\bm{u}_{a}^{\dag}\bm{x} = aP(a)\bm{x}
  \end{eqnarray}
\item 
  \begin{eqnarray}
  A\bm{x} = A\sum_a(\bm{u_a},\bm{x})\bm{u_a} = \sum_a a(\bm{u_a},\bm{x})\bm{u_a} = \sum_a aP(a)\bm{x}
  \end{eqnarray}
\item 
  \begin{eqnarray}
  f(A)\bm{u_a} &=& \sum_{a^{\prime}} f(a^{\prime} )P(a^{\prime} )\bm{u_a}\\
		  &=& \sum_{a^{\prime}} f(a^{\prime} )\bm{u}_{a^{\prime}}\delta_{a,a^{\prime}}\\
                  &=& f(a)\bm{u}_a
  \end{eqnarray}
\item 
  \begin{eqnarray}
  \bm{v} &=& \lim_{t\rightarrow \infty} \frac{e^{-At}\bm{x}}{\Vert e^{-At}\bm{x}\Vert}\\
	  &=& \lim_{t\rightarrow \infty} \frac{\sum_{a}e^{-at}P(a)\bm{x}}{\Vert \sum_{a}e^{-at}P(a)\bm{x}\Vert}\\
          &=& \lim_{t\rightarrow \infty} \frac{e^{a_0t}\sum_{a}e^{-at}P(a)\bm{x}}{e^{a_0t}\Vert \sum_{a}e^{-at}P(a)\bm{x}\Vert}\\
          &=& \lim_{t\rightarrow \infty} \frac{P(a_0)\bm{x}+\sum_{a(\neq a_0)}e^{-(a-a_0)t}P(a)\bm{x}}{\sqrt{(P(a_0)\bm{x},P(a_0)\bm{x})+\sum_{a(\neq a_0)}e^{-2(a-a_0)t}(P(a)\bm{x},P(a)\bm{x})}}\\
          &=& \frac{P(a_0)\bm{x}}{\Vert P(a_0)\bm{x}\Vert }
  \end{eqnarray}
\end{enumerate}
\end{answer}

\begin{answer}{第2問}{石川}
\begin{enumerate}
\item 
  $\delta y(x_0)=\delta y(x_1)=0$及び、$(\delta y)^{\prime}=\delta y^{\prime}$に注意して変形すると
  \begin{eqnarray}
  \delta S &=& \int_{x_0}^{x_1}\delta F(y(x),y^{\prime}(x))dx\\
  &=& \int_{x_0}^{x_1} \left( \pdif{F(y(x),y^{\prime}(x))}{y}\delta y + \pdif{F(y(x),y^{\prime}(x))}{y^{\prime}}\delta y^{\prime}\right) dx\\
  &=& \int_{x_0}^{x_1} \left( \pdif{F(y(x),y^{\prime}(x))}{y} - \frac{d}{dx}\pdif{F(y(x),y^{\prime}(x))}{y^{\prime}}\right)\delta y dx \\
  &&+ \left[\pdif{F(y(x),y^{\prime}(x))}{y^{\prime}}\delta y\right]_{x_0}^{x_1}\\
  &=& \int_{x_0}^{x_1} \left( \pdif{F(y(x),y^{\prime}(x))}{y} - \frac{d}{dx}\pdif{F(y(x),y^{\prime}(x))}{y^{\prime}}\right)\delta y dx
  \end{eqnarray}
  ここで極値を取るためには\\
  \begin{equation}
  \delta S = 0
  \end{equation}
  これは\\
  \begin{eqnarray}
  \pdif{F(y(x),y^{\prime}(x))}{y} - \frac{d}{dx}\pdif{F(y(x),y^{\prime}(x))}{y^{\prime}} &=& 0
  \end{eqnarray}
\item 
  Fの全微分の式\\
  \begin{eqnarray}
  \frac{dF}{dx} = \pdif{F}{y}y^{\prime} + \pdif{F}{y^{\prime}}\frac{dy^{\prime}}{dx}
  \end{eqnarray}
  に注意して\\
  (20)に$y^{\prime}$をかけて(21)を代入すれば\\
  \begin{eqnarray}
  0=\pdif{F}{y}y^{\prime}-y^{\prime}\frac{d}{dx}\pdif{F}{y^{\prime}} &=& \frac{dF}{dx} - \pdif{F}{y^{\prime}}\frac{dy^{\prime}}{dx}-y^{\prime}\frac{d}{dx}\pdif{F}{y^{\prime}}\\
  &=& \frac{d}{dx}\left(F-y^{\prime}\pdif{F}{y^{\prime}} \right)\\
  F-y^{\prime}\pdif{F}{y^{\prime}} &=& C
  \end{eqnarray}
\item 
  \begin{eqnarray}
  \int_{x_0}^{x_1} 2\pi y\sqrt{1+y^{\prime 2}}dx
  \end{eqnarray}
\item 
  \begin{equation}
  F = 2\pi y\sqrt{1+y^{\prime 2}}
  \end{equation}
  を(24)に代入して\\
  \begin{eqnarray}
  y\sqrt{1+y^{\prime 2}}-\frac{yy^{\prime 2}}{\sqrt{1+y^{\prime 2}}} &=& C\\
  \frac{y}{\sqrt{1+y^{\prime 2}}} &=& C\\
  1+ y^{\prime 2} &=& \left( \frac{y}{C}\right) ^2
  \end{eqnarray}
\item 
  \begin{eqnarray}
  \xi = \frac{x}{C}\\
  \eta = \frac{y}{C}
  \end{eqnarray}
  とおくと(29)は\\
  \begin{eqnarray}
  \eta^2 = 1+\eta^{\prime 2}
  \end{eqnarray}
  $y(x)$が最小値をとる座標を$(l,C)$とおく((28)より$y^{\prime}=0のときy=C(>0)$)\\
  $x \geq l(\xi \geq l/C)$の場合傾きは正だと考えられるので(32)から\\
  \begin{eqnarray}
  \sqrt{\eta^2 - 1} &=& \frac{d\eta}{d\xi}\\
  \int_1^{\eta}\frac{d\eta}{\sqrt{\eta^2-1}}&=&\int_{l/C}^{\xi}d\xi\\
  \ln(\eta+\sqrt{\eta^2-1})&=&\xi - l\\
  \eta &=& \cosh(\xi-\frac{l}{C})\\
  \end{eqnarray}
  $x \leq l(\xi \leq l/C)$の場合も同様にして\\
  \begin{eqnarray}
  -\sqrt{\eta^2 - 1} &=& \frac{d\eta}{d\xi}\\
  -\int_1^{\eta}\frac{d\eta}{\sqrt{\eta^2-1}}&=&\int_{l/C}^{\xi}d\xi\\
  \eta &=& \cosh(\xi-\frac{l}{C})\\
  \end{eqnarray}
  よって\\
  \begin{eqnarray}
  y = C\cosh(\frac{x-l}{C})
  \end{eqnarray}
\end{enumerate}
\end{answer}
