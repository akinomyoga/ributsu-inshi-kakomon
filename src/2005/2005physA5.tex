\begin{answer}{第5問 〜 Drude理論、Hall効果}{平塚淳一}
\begin{enumerate}
\def\<{\langle}
\def\>{\rangle}
\def\kB{k_{\mathrm B}}
\item 
  \begin{enumerate}
  \item 
    \begin{equation}
    \frac{d \overrightarrow{v}(t)}{dt}
    =-\frac{e}{m}(\overrightarrow{E}+\overrightarrow{v}(t)\times \overrightarrow{B})
    \end{equation}
  \item 
    \begin{equation}
    \frac{d \overrightarrow{v}(t)}{dt}
    =-\frac{e}{m}(\overrightarrow{E}+\overrightarrow{v}(t)\times \overrightarrow{B})
    -\frac{\overrightarrow{v}(t)}{\tau}
    \end{equation}
    $\overrightarrow{B}=0$のとき、電場に平行な方向について
    \begin{equation}
    \frac{dv}{dt}=-\frac{e}{m}E-\frac{v}{\tau}
    \end{equation}
    一階微分方程式の一般解から
    \begin{equation}
    v
    =e^{-\int\frac{dt}{\tau}}(\int e^{\int\frac{dt}{\tau}}(-\frac{e}{m}Edt)+const.)
    =-\frac{e\tau}{m}E +Ce^{-\frac{t}{\tau}}
    \end{equation}
    積分定数$C$を決めるため、$t=0$で$v(0)=0$より
    \begin{equation}
    0=-\frac{e\tau}{m}E+C
    \end{equation}
    よって
    \begin{align}
    v(t)&=-\frac{e\tau}{m}E(1-e^{-\frac{t}{\tau}})\\
    v_{\infty}&=-\frac{e\tau}{m}E
    \end{align}
  \item 
    \begin{equation}
    j=-nev_{\infty}=\frac{e^2\tau n}{m}E
    \end{equation}
    電気伝導度$\sigma$は
    \begin{equation}
    \sigma =\frac{j}{E}=\frac{e^2n}{m}\tau
    \end{equation}
    $\tau$の温度依存性を高温、低温の場合に考える。\\
    高温の場合、格子振動の寄与が大きい。\\
    衝突断面積$S$、電子の平均速度$\<v\>$、フォノンの振幅$a$、フォノンの質量$M$として、
    エネルギー等分配則を使うと
    \begin{align}
    \frac{1}{2}M\<a\>^2&=\frac{1}{2}\kB T\\
    \frac{1}{2}m\<v\>^2&=\frac{1}{2}\kB T
    \end{align}
    $S \propto \<a\>^2$であり、
    $\frac{1}{\tau} \propto (散乱頻度) \propto S\<v\> \propto T^{3/2}$なので
    \begin{align}
    \tau \propto T^{-3/2}
    \end{align}
    低温の場合、不純物散乱の寄与が大きい。\\
    ラザフォード散乱では
    \begin{equation}
    S \propto \<v\>^{-4}
    \end{equation}
    これを示すのはかなり面倒なので、
    力学か量子力学の何かの教科書(散乱)を見てください。
    古典でも量子でも結果は同じです。
    多分結果を記憶していたほうが良い。
    \begin{align}
    \frac{1}{2}m\<v\>^2=\frac{1}{2}\kB T
    \end{align}
    $\frac{1}{\tau} \propto \verb|(散乱頻度)| \propto S\<v\> \propto T^{-3/2}$なので
    \begin{equation}
    \tau \propto T^{3/2}
    \end{equation}
    ここではnの温度依存性は無視するようですから、
    \begin{align}
    \sigma &\propto \tau\\
    &\propto T^{-3/2} \;\; (at 高温)\\
    &\propto T^{3/2} \;\; (at 低温)
    \end{align}
  \item 
    \begin{equation}
    -e(E_y+v_xB)=0
    \end{equation}
    \begin{equation}
    v_x=v_{\infty}=-\frac{j}{ne}
    \end{equation}
    \begin{equation}
    E_H=E_y=-\frac{jB}{ne}
    \end{equation}
  \end{enumerate}
\item 
  \begin{equation}
  E_H \propto -\frac{1}{n}
  \end{equation}
  なので、$n$ の温度依存性を考えればよい。高温、中間、低温の3つの領域に分けて考える。\\
  高温領域(真性領域)\\
  ドナーは全て伝導帯へ励起しており、かつ価電子帯からも伝導帯へ励起する領域。
  \begin{equation}
  \log(n) \propto -\frac{1}{T}
  \end{equation}
  中間領域(出払い領域)\\
  ドナーが全て伝導帯へ励起して自由電子として振る舞う。価電子帯からの励起はない領域。
  \begin{equation}
  \log(n) \sim \mathrm{const}.
  \end{equation}
  低温領域(〜凍結領域)\\
  十分多数のドナーがイオン化していない(励起していない)領域。
  \begin{equation}
  \log(n) \propto -\frac{1}{T}
  \end{equation}
  図は略します。横軸は $T$ の逆数、縦軸は $\log$ スケールで描くと良いと思います。
  高温領域の傾きは低温領域の傾きよりも急です。
  詳しくは固体物理の教科書を参照(H.イバッハ、H.リュート著「固体物理学」12章など)。
  何故か$n$の表式を導くヒントが何も与えられていないので、
  これくらいの特性は覚えていろということでしょうか。
\item ブロック図は略します。図の概要のみ述べます。\\
  設問の図1の装置(ドープした試料)の$y$-$z$面については、
  片方の面に電流計を挟んで電流源を繋ぎ、
  もう片方の面は接地します。\\
  $z$-$x$面については、両端を抵抗を挟んで繋ぎ、回路にします。
  抵抗と並列に電圧計を繋ぎます。\\
  測定上の注意点としては、常に定電流、定磁場をかけて測定すること。
  ドーピングするドナーは多すぎても少なすぎてもいけない。
\end{enumerate}
\end{answer}
