%-------------------------------------------------------------------------------
% 2005年度実施 2006年度入学 数学 問題 2
%-------------------------------------------------------------------------------
\begin{question}{第2問}{村瀬}

以下の積分を考える。
\begin{equation}
S=\int_{x_0}^{x_1}F(y(x),y'(x))dx. \ilabel{eq:2005math1_1}
\end{equation}
ここで, $y'(x)\equiv dy(x)/dx$ である。また, $F(y,y')$ および $y(x)$ は,いずれの変数についても何
回でも微分可能な連続かつ一価な関数とする。さらに, 曲線 $y=y(x)$ の端点は、$y(x_0)=y_0$,
$y(x_1)=y_l$ のように固定されているものとする。以下の設問に答えよ。
\begin{enumerate}
\item 
  積分 $S$ が曲線 $y=y(x)$ の変化に対して極値をとるための十分条件となる微分方程式を,
  関数 $y(x)$ の微小変分 $\delta y(x)$ を考える事で導出せよ。
\item 
  設問(i)で求めた微分方程式は,
  \[F-y'\frac{\partial F}{\partial y'}=C,\]
  に変形できることを示せ。ここで, $C$ は定数である。
\item 
  曲線 $y=y(x)$ の両端点が $x_1>x_0>0, y_1>0, y_0>0$ を満たすとする。この曲線を $x$
  軸のまわりに $360^\circ$ 回転して得られる回転面を考え, その表面積を\ieqref{eq:2005math1_1}式の形の積分を
  用いて表せ。
\item 
  設問(iii)で得られた表面積を最小にする曲線 $y=y(x)$ が満たす微分方程式を導け。
\item 
  設問(iv)で得られた微分方程式を解き, 曲線 $y(x)$ を求めよ。
\end{enumerate}
\end{question}
