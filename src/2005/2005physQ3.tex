%-------------------------------------------------------------------------------
% 2005年度実施 2006年度入学 物理学 問題 3
%-------------------------------------------------------------------------------
\begin{question}{第3問}{村瀬}
以下に示す真空中での Maxwell の方程式を出発点として平面電磁波の
性質を導き出したい。ここで、$\vec E$ と $\vec B$ はそれぞれ電場と磁束密度、$\epsilon_0$ と
$\mu_0$ はそれぞれ真空の誘電率と透磁率である。また、 $t$ は時間であり、$\nabla$ は
空間座標 $\vec r=(x,y,z)$ についてのベクトル微分演算子である。電荷や電流
の存在しない場合を考える。

\begin{eqnarray}
\nabla\cdot\vec E &=& 0 \ilabel{eq:2005phys3-maxwell-1}\\
\nabla\times\vec B &=& \epsilon_0\mu_0\frac{\partial \vec E}{\partial t} \ilabel{eq:2005phys3-maxwell-2}\\
\nabla\cdot\vec B &=& 0 \ilabel{eq:2005phys3-maxwell-3}\\
\nabla\times\vec E &=& -\frac{\partial \vec B}{\partial t} \ilabel{eq:2005phys3-maxwell-4}
\end{eqnarray}

以下の設問に答えよ。
\begin{enumerate}
\item
  任意の三次元ベクトル $\vec V$ に対して次の公式が成り立つことを示せ。
  \[\nabla\times(\nabla\times\vec V)=\nabla(\nabla\cdot\vec V)-\nabla^2\vec V\]
\item
  前問の公式を Maxwe11 の方程式に適用して、$\vec E$ に対する波動方程式
  および $\vec B$ に対する波動方程式をそれぞれ導け。
\item
  次の式で表される $\vec E$ および $\vec B$、すなわち平面波が前問で求めた波
  動方程式の解であるための条件を求めよ。ただし、$\vec E_0$、$\vec B_0$、$\vec k$、$\vec k'$、
  は定数ベクトル、$\omega$、$\omega'$、$\alpha$、$\alpha$ はそれぞれ定数で、$0\leq<\alpha<\pi$、
  $0\leq\alpha'<\pi$ である。
  \begin{eqnarray}
  \vec E &=& \vec E_0\sin(\vec k\cdot\vec r -\omega t + \alpha) \ilabel{eq:2005phys3_1}\\
  \vec B &=& \vec B_0\sin(\vec k'\cdot\vec r -\omega' t + \alpha') \ilabel{eq:2005phys3_2}
  \end{eqnarray}
\item
  式\ieqref{eq:2005phys3_1}と式\ieqref{eq:2005phys3_2}で表される $\vec E$ および $\vec B$ がそれぞれの波数ベクトルに
  垂直であること、すなわち横波であることを示せ。
\item
  式\ieqref{eq:2005phys3_1}と式\ieqref{eq:2005phys3_2}において、$\vec k=\vec k'$、$\omega=\omega'$ および $\alpha=\alpha'$ であること
  を示せ。
\item
  式\ieqref{eq:2005phys3_1}と式\ieqref{eq:2005phys3_2}で表される $\vec E$ と $\vec B$ は互いに直交していることを示せ。
\item
  平面電磁波の位相速度を求めよ。
\item
  式\ieqref{eq:2005phys3_1}と式\ieqref{eq:2005phys3_2}で電場 $\vec E$ と磁場 $\vec H (=\vec B/\mu_0)$ の大きさの比 $|\vec E|/|\vec H|$ が
  常に一定であること、およびこの比は抵抗の次元をもつことを示せ。
\end{enumerate}
\end{question}
