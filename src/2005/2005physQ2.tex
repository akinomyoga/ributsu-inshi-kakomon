%-------------------------------------------------------------------------------
% 2005年度実施 2006年度入学 物理学 問題 2
%-------------------------------------------------------------------------------
\begin{question}{第2問}{村瀬}
一辺の長さが $L$ の立方体の箱があって、絶対温度 $T$ の熱浴に接しているとする。箱の中に存
在する電磁波が熱浴と熱平衡になっている場合を考える。ただし、ボルツマン定数を $k_\mathrm{B}$ とし、
$\hbar=h/2\pi$ とする。

\begin{enumerate}
\item
  箱の中の異なる角振動数を持つ電磁波は、お互いに独立に振る舞うと考えられる。さら
  に、ある角振動数 $\omega$ を持つ電磁波は、同じ角振動数を持つ量子力学的な調和振動子として
  扱うことができる。絶対温度 $T$ の場合に、角振動数 $\omega$ の調和振動子の分配関数を求めよ。
\item
  設問1の調和振動子の平均エネルギーを求めよ。さらに、非常に低温の場合と非常に高
  温の場合に、平均エネルギーはどのような温度依存性を持つか調べ、その結果について
  物理的内容を簡潔に説明せよ。
\item
  電磁波は箱の中で定在波を作る。壁の位置で節を作るとして、電磁波の波数 $\vec k=(k_x,k_y,k_z)$
  のとり得る値を求めよ。
\item
  波数 $\vec k$ を持つ電磁波の角振動数は、$k=|\vec k|$ として $\omega=ck$ で与えられる。ここで $c$ は光
  速度である。箱内の電磁波の持つ全エネルギーは、すべてのとり得る角振動数について
  の和として書ける。このとき、金工ネルギー $E$ を
  \[E=\int_0^\infty\varepsilon(k)\mathrm dk\]
  という形に表したときの $\varepsilon(k)$ を求めよ。ただし、箱の一辺の長さ $L$ は十分長いとして、
  固有振動についての和を積分の形にしてよい。また、零点振動のエネルギーは無視せよ。
  (電磁波は横波であり、横波の偏光の方向が2通りあることも考慮せよ。)
\item
  全エネルギー $E$ が、どのような温度のぺき乗則に従うか求めよ。
\item
  非常に低温の場合に、設問4で得られた $\varepsilon(k)$ が極大になる $k_\mathrm{max}$ の表式を求めよ。絶対
  温度が 2.7K のとき、$k_\mathrm{max}$ の波数を持つ電磁波の波長を求めよ。ただし $\hbar=1.0\times10^{-34}$
  Js、 $c=3.0\times10^8$ m/s、 $k_\mathrm{B}= 1.4 \times 10^{-23}$ J/Kを用いて、有効数字2桁で答えよ。
\end{enumerate}
\end{question}
