%-------------------------------------------------------------------------------
% 2005年度実施 2006年度入学 物理学 問題 4
%-------------------------------------------------------------------------------
\begin{question}{第4問}{村瀬}
\begin{enumerate}
\item
  $\pi^0$ は、静止質量 $m_0$、平均寿命 $\gamma$ を持つ粒子で、主に2個の光子に崩壊する。\\
  $\pi^0$ の静止系での4元座標の成分を $ct_0$、$x_0$、$y_0$、$z_0$ とする。これらは、$\pi^0$ が $z$ 軸方向に
  速度 $v=c\beta$ で運動している系での座標の成分 $ct$、$x$、$y$、$z$ と、次の式 (ローレンツ変換)
  で関係付けられる。ここで、$c$ は光速、$\gamma=1/\sqrt{1-\beta^2}$ である。
  \[
    \left(\begin{array}{c}ct\\x\\y\\z\end{array}\right)
    =\left(\begin{array}{cccc}
      \gamma&0&0&\gamma\beta\\
      0&1&0&0\\
      0&0&1&0\\
      \gamma\beta&0&0&\gamma
    \end{array}\right)
    \left(\begin{array}{c}ct_0\\x_0\\y_0\\z_0\end{array}\right)
  \]
  \begin{enumerate}
  \item $z$軸方向に速度 $v=c\beta$ で運動している系での、$\pi^0$ の4元運動量を求めよ。
  \item この系での、$\pi^0$ の平均寿命と平均飛距離を求めよ。
  \item 
    運助量$p$について$pc$=135 GeVを持つ $\pi^0$ の、生成から崩壊までの平均飛距離
    を有効数字2桁で求めよ。その際、光速 $c=3.0\times10^8$ m/s、 $m_0c^2$=135 MeV、
    $\tau=8.4\times10^{-17}$ sを用いよ。また、1MeV=$10^6$ eV、1GeV=$10^9$ eV である。
  \end{enumerate}
\item
  高いエネルギーを持つ光子を鉛に入射したときの電磁シャワー過程について考察する。電
  磁シャワーとは、高いエネルギーを持つ光子からの電子・陽電子対生成過程 (図\iref{fig:2005phys4_1}左) と、
  高いエネルギーを持つ電子または陽電子からの制動放射過程 (図\iref{fig:2005phys4_1}右) が、原子核近傍
  の強い電磁場を受けて次々に起こり、雪崩(なだれ)的に粒子数が増大する過程である。
  ここでは、以下のように単純化した模型を考える。上述の電子・陽電子対生成過程と制
  動放射過程のみを考慮する。これらの過程はそれぞれの粒子が物質中を距離 $X_0$ 進む毎に
  起き、各ステップでの原子核が受ける反跳は無視し、終状態の残りの2粒子は、始状態
  の粒子の持つエネルギーを等分する。生成粒子は進行方向に放出され、横方向の拡がり
  は無いものとする。
  \begin{figure}[b]
    \begin{center}
    \includegraphics[width=14cm]{2005phys4_1.eps}
    \end{center}
    \caption{(左) 光子の電子・陽電子対生成過程。 (右) 電子及び陽電子の制動放射過程。}
    \ilabel{fig:2005phys4_1}
  \end{figure}
  \begin{enumerate}
  \item
  高いエネルギーの光子を入射する。$n$ ステップ後の光子数を $G(n)$、電子数と陽電子
  数の和を $E(n)$ として、これららの漸化式を示せ。次に、$G(n)$ と $E(n)$ の解を求めよ。
  この際、$S(n)=G(n)+E(n)$ と $D(n)=2G(n)-E(n)$ の漸化式を考えてもよい。
  \item
  エネルギー $E_0$ の光子を入射したとき、$n$ ステップ終了時に個々の粒子が持つエネル
  ギーを求めよ。また、物質への入射点から距離 $nX_0$ までの電子と陽電子の走行距離
  の総和を求めよ。
  \item
  電磁シャワー過程は、反応で生成される電子・陽電子のエネルギーが小さくなり、電
  離損失過程が優勢になると、終焉(しゅうえん)に向う。ここでは、簡単のために、
  生成粒子のエネルギーが $E_\mathrm{C}$ 以下となった時点で突然終了するものとする。何ステッ
  プで終了するか、エネルギー 1 GeV の光子について計算せよ。次に、電子と陽電子
  の走行距離の総和を求めよ。その際、$E_\mathrm{C}$=7MeV、$X_0$=5.6 mmを用いよ。また、
  1 MeV=$10^6$ eV、1 GeV=$10^9$ eV である。
  \item
  電磁力ロリメーターは、電磁シャワーを利用して、光子等のエネルギーを測定する
  検出器である。実際のエネルギー測定では、電子・陽電子の電離損失過程やチェレ
  ンコフ光放出過程を利用して、近似的に電予と陽電子の走行距離の総和を測定する。
  光子の入射エネルギー 28 MeV 〜 500 MeVの範囲について、入射エネルギーと電子
  と陽電子の走行距離の総和の関係を略図で示せ。この結果をもとに、単純化した模
  型の問題点を指摘し、その原因について考察せよ。
  \end{enumerate}
\end{enumerate}
\end{question}
