%-------------------------------------------------------------------------------
% 2005年度実施 2006年度入学 物理学 問題 5
%-------------------------------------------------------------------------------
\begin{question}{第5問}{村瀬}
\begin{enumerate}
\item
  \begin{enumerate}
  \item
    金属中の電子の運動を古典的に自由電子モデルで考え、電気伝導を考察する。外部
    電場 $\vec E$ および外部磁場(磁束密度$\vec B$)中の自由電子の運動方程式を、時刻 $t$ での速度を
    $\vec v(t)$ として、$\mathrm d\vec v(t)/\mathrm dt=\cdots$ の形式で記せ。ただし、電子の質量を $m$、電荷を$-\epsilon$とし、相
    対論的効果は無視できるとする。
  \item
    金属中では、伝導電子は格子欠陥、不純物、格子振動などによって散乱される。この
    散乱の効果を、平均の電子散乱時間 $\tau$ を用いて以下の運動方程式で表す。
    \begin{equation}
    \mathrm d\vec v(t)/\mathrm dt=-\vec v/\tau \ilabel{eq:2005phys5_1}
    \end{equation}
    金属中の伝導電子の運動方程式は、設問 1(a) で求めた運動方程式の右辺に\ieqref{eq:2005phys5_1}式の右
    辺を加えたものとする。外部磁場がないときに、電場に平行方向の電子の1次元運動方
    程式を解け。速度 $v$ は時刻 $t=0$ で $v(0)=0$ とする。ここで、簡単のため金属中の伝導
    電子の有効質量は $m$ のままとし、電荷は $-e$ とする。また、時刻 $t=\infty$ での定常速度 $v_\infty$。
    を求めよ。
  \item
    電流密度 $j$ は定常速度 $v_\infty$ と伝導電子密度 $n$ を用いて、$i=-n\epsilon v\infty$ で与えられる。電
    気伝導度 $\sigma$ と散乱時間 $\tau$ の関係を求めよ。また、散乱時間の温度依存性を考察し、この
    モデルでは金属の電気伝導度はどのように温度に依存するかを定性的に説明せよ。
  \item
    直方体の試料を図 \iref{fig:2005phys5_1} のようにおき、電流密度 $j$ の電流を $x$ 軸方向に流し、磁束密度 $B$
    の外部磁場をそれと垂直な $z$ 軸方向にかけると、双方に垂直な $y$ 軸方向に電位差が生じ
    る。この電位差のために生じる $y$ 軸方向の電場 $E_\mathrm{H}$(ホール電場)を求めよ。ここで、定常
    状態では、磁場によって電子にかかるローレンツ力が $y$ 軸方向の電場による力とつり合っ
    ているとしてよい。
  \end{enumerate}
  \begin{figure}[b]
  \begin{center}
  \includegraphics[width=10cm]{2005phys5_1.eps}
  \end{center}
  \caption{}
  \ilabel{fig:2005phys5_1}
  \end{figure}
\item
  純粋なシリコン(Si)は真性半導体であり、そのバンドギャップは、 1.1 eV である。これ
  に電子を供給するドナー不純物としてリン(P)をドープする。リンが供給した電子は基
  底状態でリン原子に束縛され、そのエネルギー準位は伝導帯の底から 44 meV 下にある。
  電流密度と磁場を一定にして温度を変化させたとき、リンをドープしたシリコンのホー
  ル電場はどのように温度に依存するかを図に示して定性的に説明せよ。この場合は、伝
  導帯に熱励起された電子にっいて設問1(d)と同じ考え方ができるとしてよい。
\item
  ホール電場の大きさを測定するための実験装置及びその配置をブロック図で示し、測定
  上注意すべき点を簡潔に説明せよ。
\end{enumerate}
\end{question}
