%-------------------------------------------------------------------------------
% 2005年度実施 2006年度入学 数学 問題 1
%-------------------------------------------------------------------------------
\begin{question}{第1問}{村瀬}

複素数を成分とする $n$ 行1列 ($n\geq1$) の縦ベクトル全体の集合 $\mathbb{C}^n$ は、$n$ 次元の複素線形空
間をなす。2つのベクトル $\bm x, \bm y\in\mathbb C^n$ について、その内積 $(\bm x,\bm y)$ を、$(\bm x,\bm y)\equiv \bm x^\dag\bm y$ にて定義
する。ただし、$\bm x^\dag$ は $\bm x$ のエルミート共役 (転置の複素共役) を表す。$\|\bm x\|\equiv\sqrt{(\bm x,\bm x)}$ を $\bm x$
のノルムと言う。$A$ を$n$行$n$列のエルミート行列とし、その固有値はどれも縮退がないとす
る。固有値 $a$ に対応する(属する)規格化された固有ベクトルを $\bm u_a$ と記す。即ち、
\[
  A\bm u_a=a\bm u_a,
  \quad (\bm u_a,\bm u_{a'})=\delta_{a,a'}.
\]
固有値 $a$ は全て実数である。また、全ての $a$ について $\bm u_a$ を集めた集合 $\{\bm u_a\}$ は、$\mathbb C^n$ の正規
直交完全系を成す。以下の設問に答えよ。
\begin{enumerate}
\item
  各固有値 $a$ ごとに、次の行列を定義する:
  \[P(a)\equiv\bm u_a\bm u_a^\dag.\]
  このとき、次の等式が成り立っことを示せ:
  \[P(a)P(a')\equiv\delta_{a,a'}P(a).\]
\item 
  適当に選んだベクトル $\bm x$ について、もしも $P(a)\bm x$ がゼロベクトルでなければ、$P(a)\bm x$
  は固有値 $a$ に対する $A$ の固有ベクトルであることを示せ。
\item 
  次の等式を示せ:
  \[A\equiv\sum_a aP(a).\]
  即ち、任意の $\bm x\in\mathbb C^n$ について、$A\bm x=\sum_a aP(a)\bm x$ を示せ。
\item 
  固有値 $a$ の全てについて定義された関数 $f(a)$ を用いて、次の行列 $f(A)$ を定義する:
  \[f(A)\equiv\sum_af(a)P(a).\]
  $f(A)$ の固有値と固有ベクトルが、それぞれ $f(a), \bm u_a$ で与えられることを示せ。
\item 
$A$ の最小固有値を $a_0$ とする。適当に選んだベクトル $\bm x$ について、$P(a_0)\bm x$ がゼロベク
トルでなかったとする。このとき、
\[
  \bm v\equiv\lim_{t\to\infty}\frac{e^{-At}\bm x}{\|e^{-At}\bm x\|}
\]
とおくと、$n$ が $a_0$ に対応する固有ベクトルであることを示せ。ただし、$t$ を任意の実
数として、$f(a)=e^{-at}$ と選んだ場合の $f(A)$ を $e^{-At}$ と記した。
\end{enumerate}
\end{question}

