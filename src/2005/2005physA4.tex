\begin{answer}{第4問 〜 特殊相対論、電磁シャワー}{平塚淳一}
\begin{enumerate}
\def\<{\langle}
\def\>{\rangle}
\item 
  \begin{enumerate}
  \item $\pi^0$の静止系の運動量$p^{\mu}_0$は
    \begin{equation}
    p^{\mu}_0=\left(
    \begin{array}{c}
    m_0c\\
    0\\
    0\\
    0\\
    \end{array} \right)
    \end{equation}
    $v=c\beta$にローレンツブーストすると、
    \begin{align}
    p^{\mu}&=\left(
    \begin{array}{cccc}
    \gamma & 0 & 0 & \gamma\beta \\
    0 & 1 & 0 & 0 \\
    0 & 0 & 1 & 0 \\
    \gamma\beta & 0 & 0 & \gamma \\
    \end{array} \right)
    \left(
    \begin{array}{c}
    m_0c\\
    0\\
    0\\
    0\\
    \end{array} \right)\\
    &=\left(
    \begin{array}{c}
    \gamma m_0c \\
    0 \\
    0 \\
    \gamma\beta m_0c \\
    \end{array} \right)
    \end{align}
  \item $t_0$を静止系として
    \begin{equation}
    ds^2=c^2dt^2-dx^2-dy^2-dz^2=c^2dt^2_0
    \end{equation}
    より
    \begin{equation}
    dt_0=dt\sqrt{1-\frac{dx^2+dy^2+dz^2}{c^2dt^2}}
    =dt\sqrt{1-\beta}
    =dt/\gamma
    \end{equation}
    よってvで運動している系での寿命$\tau^{\prime}$は
    \begin{equation}
    \tau^{\prime}=\gamma \tau
    \end{equation}
    平均飛距離$l^{\prime}$は
    \begin{equation}
    l^{\prime}=\gamma\tau v =\gamma\tau c\beta
    \end{equation}
  \item $p^{\mu}_0p_{0\mu}=p^{\mu}p_{\mu}$から
    \begin{equation}
    (m_0c^2)^2=(\gamma m_0c^2)^2 -(pc)^2
    \end{equation}
    \begin{equation}
    \gamma^2=1+\frac{(pc)^2}{(m_0c^2)^2}=1+10^6
    \end{equation}
    \begin{equation}
    \gamma \simeq 10^3
    \end{equation}
    \begin{align}
    \beta^2&=1-\frac{1}{1+\frac{(pc)^2}{(m_0c^2)^2}}\\
    &\simeq 1-10^{-6}\\
    &\simeq 1
    \end{align}
    以上から
    \begin{align}
    l^{\prime} &\simeq10^3 \times 8.4\times 10^{-17} \times 3.0 \times 10^8 \times 1\\
    &\simeq 2.5 \times 10^{-5} [m]
    \end{align}
  \end{enumerate}
\item 
  \begin{enumerate}
  \item 具体的に自分で図を描いて調べてみましょう。すると表\iref{tbl:2005phys4a_1}のようになります。
    \begin{table}
      \begin{center}
        \caption{$n$ ステップでの$G(n)$と$E(n)$}
        \ilabel{tbl:2005phys4a_1}
        \begin{tabular}{ccccccccc}
          \hline
           $n$ステップ数 & 0 & 1 & 2 & 3 & 4 & 5 & 6 & 7  \\
          \hline
           $G(n)$ & 1 & 0 & 2 & 2 & 6 & 10 & 22 & 42 \\
          \hline
           $E(n)$ & 0 & 2 & 2 & 6 & 10 & 22 & 42 & 86 \\
          \hline
        \end{tabular}
      \end{center}
    \end{table}
    1ステップ前に存在する全ての陽電子と電子は1ステップで全て光子になり、光子の発生源は
    これだけなので
    \begin{equation}
    G(n)= E(n-1)
    \end{equation}
    陽電子と電子の発生源は、1ステップ前の全ての光子と、1ステップ前の陽電子、電子なので
    \begin{equation}
    E(n)= 2G(n-1)+E(n-1)
    \end{equation}
    設問通りS(n)とD(n)を考えると、上の漸化式を用いて
    \begin{align}
    S(n)&=G(n)+E(n)
    =2(G(n-1)+E(n-1))
    =2S(n-1)\\
    D(n)&=2G(n)-E(n)
    =-(2G(n-1)-E(n-1))
    =-D(n-1)
    \end{align}
    これらはすぐ解けて、
    \begin{align}
    S(n)&=2^nS(0)=2^n\\
    D(n)&=(-1)^nD(0)=2(-1)^n
    \end{align}
    これらからG(n)、E(n)を構成すると
    \begin{align}
    G(n)&=\frac{1}{3}(S(n)+D(n))=\frac{1}{3}(2^n+2(-1)^n)\\
    E(n)&=\frac{1}{3}(2S(n)-D(n))=\frac{1}{3}(2^{n+1}+2(-1)^{n+1})
    \end{align}
  \item 全粒子数は$G(n)E(n)=2^n$で、崩壊で常にエネルギーを2つの粒子に当分するから、
    個々の粒子のエネルギーは
    \begin{equation}
    \frac{1}{2^n}E_0
    \end{equation}
    陽電子と電子の走行距離の総和は
    \begin{equation}
    \sum^{n-1}_{i=0}E(i)X_0
    =\frac{1}{3}\sum^{n-1}_{i=0}(2^{i+1}+2(-1)^{i+1})X_0
    \end{equation}
  \item 反応は
    \begin{equation}
    \frac{1}{2^n}E_0 \le E_c
    \end{equation}
    で終了する。よって
    \begin{equation}
    1000 \le 7\times 2^n
    \end{equation}
    を初めに満たす整数である8ステップで反応は終了する。
    電子と陽電子の走行距離は
    \begin{equation}
    \sum^{8-1}_{i=0}E(i)X_0
    =170X_0
    \sim 0.95\mathrm{[m]}
    \end{equation}
  \item 入射エネルギー$E_0$に対して電子、陽電子の走行距離は
    \begin{equation}
    2^n \ge \frac{E_0}{7}
    \end{equation}
    を初めに満たす整数nを用いて
    \begin{equation}
    \sum^{n-1}_{i=0}E(i)X_0
    \end{equation}
    だから、走行距離は表\iref{tbl:2005phys4a_2}のようになる。
    \begin{table}
      \begin{center}
        \caption{入射エネルギーと走行距離との関係}
        \ilabel{tbl:2005phys4a_2}
        \begin{tabular}{ccccccc}
          \hline
           $E_0$ & 28 & 28-56 & 56-112 & 112-224 & 224-448 & 448-500 \\
          \hline
           $n$ & 2 & 3 & 4 & 5 & 6 & 7 \\
          \hline
           走行距離(概数)[mm] & 11 & 22 & 56 & 112 & 240 & 470 \\
          \hline
        \end{tabular}
      \end{center}
    \end{table}
    図は略。ステップ関数がいくつも繋がった図(バスの料金表みたいなの)になる。\\
    単純化した模型の問題点は図から、走行距離が離散的であるため
    特に高い入射エネルギーにおいて、入射エネルギーを増やしても走行距離がなかなか増えない
    (入射エネルギーに対する走行距離の応答が遅い)。
    これは1ステップで走る粒子の飛距離を一定の$X_0$にしたためと考えられる。
    もし入射エネルギーが高いほど1ステップの飛距離が長い
    (速い粒子は同じ時間で遠くまで飛ぶことに相当)
    とすれば、エネルギーを高くすると同じステップ数でも走行距離がより長くなるので、
    図を連続的にとることが可能になる。
  \end{enumerate}
\end{enumerate}
\end{answer}
