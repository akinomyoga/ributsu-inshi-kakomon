%-------------------------------------------------------------------------------
% 2005年度実施 2006年度入学 物理学 問題 6
%-------------------------------------------------------------------------------
\begin{question}{第6問}{村瀬}
図 \iref{fig:2005phys6_1} のような円柱状の真空容器内に閉じ込められている、電子および水素イオンか
らなるプラズマの電子温度 $T_\mathrm{e}$を、レーザー光の散乱を用いて計測することを考える。座標系は
原点を真空容器の中心とし、$z$ 軸を円柱の対称軸とする。レーザー光は原点を通り $y$ 軸に沿っ
て入射し、プラズマにより散乱される。散乱光のうち $x$ 軸方向に進む光が集光され、検出され
るものとする。電子はレーザーの電場に加速されて振動する。その結果、電磁波が放射され、
これが散乱光として観測される。ここでは電子の速度は十分遅く、相対論的効果を無視できる
場合を考える。以下の設問に答えよ。
\begin{figure}[b]
  \begin{center}
  \includegraphics[width=10cm]{2005phys6_1.eps}
  \end{center}
  \caption{}
  \ilabel{fig:2005phys6_1}
\end{figure}
\begin{enumerate}
\item
  なぜ電子による散乱の方がイオンによる散乱より重要か説明せよ。
\item
  まず1個の電子による散乱を考える。散乱光は電子の運動によりドップラーシフトを受
  ける。入射光および散乱光の角振動数をそれぞれ $\omega_\mathrm{i}$、$\omega_\mathrm{s}$、波数ベクトルを $\vec k_\mathrm{i}$、$\vec k_\mathrm{s}$、電子
  の速度を $\vec v$ とする。ただし、電子の静止系では入射光と反射光の角振動数は変化しない
  とせよ (Thomson 散乱)。プラズマ中の光速は $3\times10^8$ m/sとしてよい。
  \begin{enumerate}
  \item
    入射光は、運動している電子からはどのような角振動数 $\omega_\mathrm{e}$ で振動しているように見
    えるか、
  \item
    レーザー光の電場に加速された電子により放射された光は、どのような角振動数 $\omega_\mathrm{s}$
    の散乱光として観測されるか、$\omega_\mathrm{s}$ を $\omega_\mathrm{i}$、$\vec k_\mathrm{i}$、$\vec k_\mathrm{s}$、$\vec v$ を用いて表せ。
  \item
    入射光として YAG レーザー(波長1.06$\mu$m)を用いた場合を考える。電子速度の $(x, y)$
    成分が ($3\times10^6$ m/s, $-3\times10^6$ m/s)のとき、散乱光は入射光からどちらにどれだ
    け波長のシフトを受けるか。有効数字1桁で求めよ。
  \end{enumerate}
\item
  実際観測される散乱光は多数の電子による散乱の寄与を足し合わせて得られる。粒子間
  の相関は無視できる場合を考えるので、散乱光の干渉は考えなくてよい。電子の速度分
  布が温度 $T_\mathrm{e}$ の Maxwell 分布
  $\displaystyle f(\vec v)\propto \exp\left(-\frac{m_\mathrm{e}|\vec v|^2}{2k_\mathrm{B}T_\mathrm{e}}\right)$
  で表されるとき、散乱光の中心角振
  動数 $\omega_0$ および角振動数ひろがりの幅 $\Delta\omega$ (散乱光強度がピーク値の $1/e$ になるところま
  での半幅)を求めよ。ただし $m_\mathrm{e}$ は電子の質量、$k_\mathrm{B}$ は Boltzmann 定数である。散乱光の
  角振動数スペクトルを図示し、$\omega_0$ およぴ $\Delta\omega$ を図中に示せ。
\item
  入射光、散乱光とも $z$ 方向に偏光している場合、$x$-$y$ 平面上における微分断面積は $\mathrm d\sigma/\mathrm d\Omega_\mathrm{s}=r_\mathrm{s}^2$
  である。但し $r_\mathrm{e}=3\times10^{-15}$ m は電子の古典半径である。レーザー光路に沿って $L$=1 cm
  の長さの領域から立体角 $\Omega_\mathrm{s}=10^{-2}$ sr で散乱光を集める。電子密度が $n_\mathrm{e}=10^{20}$ m${}^{-3}$の
  場合、検出器に到達する光子数は入射光子数のどのくらいの割合か。有効数字1桁で答
  えよ。
\item
  散乱光の波長スペクトルを測る方法を具体的な例をあげて説明せよ。前問で求めたよう
  に、プラズマによる散乱光は極めて微弱なので、その検出には注意を要する。真空窓や
  真空容器壁により散乱または反射された光(迷光という)を低減する方法を一つ述べよ。
\end{enumerate}
\end{question}
