\begin{answer}{第6問 〜 ドップラー効果}{平塚淳一}
\begin{enumerate}
\def\<{\langle}
\def\>{\rangle}
\def\kB{k_{\mathrm B}}
\item イオンは電子に比べて十分重いため、イオンの振動は電子に比べ遥かに緩やかで、無視できるから。
\item 
  \begin{enumerate}
  \item 入射光を感じる電子は、$y$方向に
    \begin{equation}
    v_y=\frac{\overrightarrow{k_i}\cdot \overrightarrow{v}}{|\overrightarrow{k_i}|}
    \end{equation}
    の速度で動くため、ドップラー効果を考えると
    \begin{align}
    \omega_e&=\frac{c-v_y}{c}\omega_i\\
    &=\frac{c-(\overrightarrow{k_i}\cdot \overrightarrow{v})/|\overrightarrow{k_i}|}{c}\omega_i
    \end{align}
  \item 
    同様に電子が放射する際は、光源が
    \begin{equation}
    v_x=\frac{\overrightarrow{k_s}\cdot \overrightarrow{v}}{|\overrightarrow{k_s}|}
    \end{equation}
    で動くので、ドップラー効果を考えて、
    \begin{align}
    \omega_s&=\frac{c}{c-v_x}\omega_e\\
    &=\frac{c}{c-v_x}\frac{c-v_y}{c}\omega_i\\
    &=\frac{c-(\overrightarrow{k_i}\cdot \overrightarrow{v})/|\overrightarrow{k_i}|}{c-(\overrightarrow{k_s}\cdot \overrightarrow{v})/|\overrightarrow{k_s}|}\omega_i
    \end{align}
  \item 
    \begin{equation}
    \omega_s=\frac{3\times 10^8+3\times 10^6}{3\times 10^8-3\times 10^6}\omega_i
    \end{equation}
    $\omega=2\pi c/\lambda$から
    \begin{equation}
    \lambda_s=\frac{2.97}{3.03}\lambda_i
    \end{equation}
    散乱光は入射光より0.02$\mu$m 短くなる。
  \end{enumerate}
\item 電子の平均速度$\<\overrightarrow{v}\>=0$より$\omega_0=\omega_i$
  (ドップラーシフトを受けないことに相当)。
  \begin{equation}
  \omega_e=\frac{c-v_y}{c}\omega_0
  \end{equation}
  より
  \begin{equation}
  v^2_y=\frac{(\omega_e-\omega_0)^2}{\omega^2_0}c^2
  \end{equation}
  Maxwell分布で考えるので、電子が$v_y$で動いているとき
  (電子に入射する光の角振動数としては$\omega_e$)の
  入射光の強度Iは
  \begin{align}
  I &\propto \exp(-\frac{m_e v^2_y}{2\kB T_e})\\
  &= \exp(-\frac{m_e c^2(\omega_e-\omega_0)^2}{2\kB T_e \omega^2_0})
  \end{align}
  よって入射光の強度が$1/e$になるときの$\omega_e$は
  \begin{equation}
  \frac{1}{e}=\exp(-\frac{m_e c^2(\omega_e-\omega_0)^2}{2\kB T_e \omega^2_0})
  \end{equation}
  を満たす。
  よって入射光の広がりの幅$\Delta \omega_1$は上の式で$\Delta \omega_1= \omega_e-\omega_0$として
  \begin{equation}
  \Delta \omega_1 = \omega_0\sqrt{\frac{2\kB T_e}{m_ec^2}}
  \end{equation}
  光が電子に入射する際と、光が電子から放射される際のドップラー広がりは同程度
  と考えられるので、散乱光の広がりの幅$\Delta \omega_2=\Delta \omega_1$である。\\
  最終的に検出される散乱光のスペクトルは、光が電子に入射した際のドップラー広がりと
  光が電子から放射される際のドップラー広がりを合成したものと見なすことができる。
  2つのドップラー広がりを合成したスペクトル$G(\omega)$は
  入射、放射のスペクトル$G_1(\omega),G_2(\omega)$を用いて
  \begin{equation}
  G(\omega)=\int^{\infty}_{-\infty}G_1(\nu)G_2(\omega+\omega_0-\nu)d\nu
  \end{equation}
  と書くことができる。今回の場合
  \begin{align}
  G_1(\omega)&=G_2(\omega)\\
  &\propto \exp(-\frac{m_e c^2(\omega-\omega_0)^2}{2\kB T_e \omega^2_0})
  \end{align}
  だから、
  \begin{equation}
  G(\omega) \propto \exp(-\frac{m_e c^2((\omega-\nu)^2+(\omega_0-\nu)^2)}{2\kB T_e \omega^2_0})
  \end{equation}
  となる。ガウス積分して整理すると、
  \begin{equation}
  G(\omega) \propto \exp(-\frac{m_e c^2(\omega-\omega_0)^2}{4\kB T_e \omega^2_0})
  \end{equation}
  となるので、この広がりの幅$\Delta \omega = \omega-\omega_0$は
  \begin{equation}
  \frac{1}{e} = \exp(-\frac{m_e c^2(\Delta \omega)^2}{4\kB T_e \omega^2_0})
  \end{equation}
  から
  \begin{equation}
  \Delta \omega =2\omega_0\sqrt{\frac{\kB T_e}{m_ec^2}}
  \end{equation}
  2つのガウス型曲線を合成した曲線の幅$\Delta \omega$は
  各々のガウス型曲線の幅$\Delta \omega_1,\Delta \omega_2$を用いて
  \begin{equation}
  (\Delta \omega)^2= (\Delta \omega_1)^2+(\Delta \omega_2)^2
  \end{equation}
  となることを知っているとより簡単に答えられることに注意。\\
  スペクトルの図は略します。いつものガウス曲線(縦軸に強度、横軸に$\omega$)を描いて、
  ピークにあたる$\omega$を$\omega_0$、強度が$1/e$になる$\omega$と$\omega_0$との幅を$\Delta \omega$とすればよい。
\item 散乱される光子の割合$\sigma_\mathrm{total}$は
  \begin{align}
  \sigma_\mathrm{total}
  &=\int n_e \times L\times 10^{-2}\times \frac{d\sigma}{d\Omega_s} d\Omega_s\\
  &=10^{20} \times 1 \times 10^{-2} \times r^2_e \times 10^{-2}\\
  &=9 \times 10^{-14}
  \end{align}
\item モノクロメータのような分光器で特定の波長の光子を選択し、
  光電子増倍管などで強度を測定すればよい。\\
  迷光対策:レーザーの太さを絞る、真空容器のサイズを大きくする、真空容器壁を
  光の反射率が低い材質のもので覆う、など。
\end{enumerate}
\end{answer}
