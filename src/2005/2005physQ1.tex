%-------------------------------------------------------------------------------
% 2005年度実施 2006年度入学 物理学 問題 1
%-------------------------------------------------------------------------------
\begin{question}{第1問}{村瀬}
1次元系で無限に高い2つの壁にはさまれた領域を考える。1次元の座標を $x$ で表すこと
とし、壁の位置は $x=-a$、及び、$x=a$ とする ($a>0$)。そこに質量 $m$ の粒子がある。二つ
の壁の間の領域でのポテンシャルエネルギーを $U(x)$ と表す。以下の設問に答えよ。ただし、
$\hbar=h/(2\pi)$ ($h$ はプランク定数) とする。
\begin{enumerate}
\item
この粒子の従う、時間に依存しないシュレーディンガー方程式を示せ。ただし、基底状態
から見て第$n$励起状態のエネルギー固有値を $E_n$ とせよ。また、$n=0$ は基底状態を表わ
すとせよ。次に、このシュレーディンガー方程式を解く際の境界条件を述べよ。さらに、
\begin{equation}
U(x)=0,\quad -a<x<a
\end{equation}
の場合 (図1参照) について、エネルギー固有値 $E_n$ $(n=0,1,2,\ldots)$ を、解を得る概略とともに示せ。
固有は導関数 $\phi_n(x)$ を、$n=0,1$ の場合について、具体的に示せ。
%
\item
設問 1 の書く固有関数 $\phi_n(x)$ は、座標系の $x\to-x$ なる変換によってどのように変化するか述べよ。
波動関数のそのような性質は、ハミルトニアンのどの普遍性の帰結として起こるのか説明せよ。
%
\item
ポテンシャル $U(x)$ をゼロからわずかにずらす事にする。まず、図2のように、
\begin{equation}
U(x)=0,\quad -a<x\le0
\end{equation}
\begin{equation}
U(x)=-w, \quad 0<x<a
\end{equation}
とする。$w$ は所為の実数であるが、絶対値は十分小さいものとする。
この時の、$a=0$ の状態と $n=1$ の状態のエネルギー固有値をそれぞれ1次の摂動論で求めよ。
%
\item
設問3の場合、設問1に比べてポテンシャルがわずかに変わっている。
ここで、$n=0$の状態の波動関数を $\psi_0(x)$ とすると、
$\psi_0(x)$ を設問1の $\phi_n(x)$ ($n=0,1,2,\ldots$) で展開した時の
$\phi_0(x)$ と $\phi_1(x)$ の確率振幅を、$w$ に関して1次の項まで求めよ。
%
\item
設問1 の波動関数 $\phi_0(x)$ に対して座標 $x$ の期待値 $\langle x\rangle$ がどのようになるか、
簡単な説明とともに示せ。それは、設問2とはどのような関係にあるのか述べよ。
%
\item
設問4の波動関数 $\psi_0(x)$ を $\phi_0(x)$ と $\phi_1(x)$ までで展開したものに対して、
期待値 $\langle x\rangle$ を $w$ の1次まで示せ。
それは、設問2とはどのような関係にあるか述べよ。
%
\item
十分小さい正の実数 $u$ に対して、次のポテンシャルを考える (図3)。
\begin{equation}
U(x)=-ux/a,\quad-a<x<a
\end{equation}
この時の、$n=0$ の状態と $n=1$ の状態のエネルギー固有値はそれぞれどのようになるか、
$u$ に関して1次までの効果を述べよ。
\end{enumerate}

\begin{center}
\includegraphics[width=15em]{2005phys1_1.eps}
\end{center}
\begin{center}
図 1: ポテンシャル (設問1)
\end{center}
\vspace{1em}

\begin{center}
\includegraphics[width=15em]{2005phys1_2.eps}
\end{center}
\begin{center}
図 2: ポテンシャル (設問3)
\end{center}
\vspace{1em}

\begin{center}
\includegraphics[width=15em]{2005phys1_3.eps}
\end{center}
\begin{center}
図 3: ポテンシャル (設問7)
\end{center}
\end{question}
