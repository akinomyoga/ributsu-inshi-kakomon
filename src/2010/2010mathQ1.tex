%% -*- coding:sjis -*-
%%
%% ChangeLog
%% 2013-07-14, Koichi Murase, 入力
%%
\begin{question}{第1問}{村瀬}

$A, B$ は$n$行$n$列の複素行列であり, $A^2=B^2=E,AB+BA=O$ を満たすものとする。
ただし, $E$は単位行列, $O$ は全ての成分が0に等しい零行列である。また, 行列$C$, $D$をそ
れぞれ $C=-iAB,B=A+iB$ によって定義する。$i$ は虚数単位である。このとき, 以下
の設問に答えよ。
\begin{enumerate}
\item 行列$A$の固有値の取り得る値を全て求めよ。
\item $C^2=E$ となること, また $BC+CB=CA+AC=O$ となることを示せ。
\item $D^2=0$ かつ $D\ne O$ となることを示せ。
\item 設問(3)の結果により, $D\bm{r}\ne\bm{0}$
  (ただし$\bm{0}$は零ベクトル) となる縦ベクトル$\bm{r}$が存在する。
  ここで $\bm{p}=D\bm{r}$ によりベクトル $\bm{p}$ を定義すると,
  $\bm{p}\ne\bm{0}$ かつ $D\bm{p}=D^2\bm{r}=\bm{0}$ をみたす。
  また $\bm{p}$ を用いて, ベクトル $\bm{q}$ を,  $\bm{q}=\frac12(A-iB)\bm{p}$ によって定義する。
  \begin{enumerate}
  \item ベクトル $\bm{p}, \bm{q}$ は, ともに行列$C$の固有ベクトルになっていることを示し,
    それぞれの固有値を求めよ。
  \item ベクトル $\bm{p}, \bm{q}$ は互いに線形独立であることを証明せよ。
  \end{enumerate}
\item 以下では $n=2$ とし, 説明(4)で定義した $\bm{p}, \bm{q}$ が2次元縦ベクトルとなる場合を考える。
  \begin{enumerate}
  \item $\bm{p}, \bm{q}$を並べて作った2次元正方行列 $P=(\bm{p}\;\bm{q})$ は正則であることを, 設問(4)の
    結果を用いることによって示せ。
  \item 行列 $P^{-1}C P$ を計算せよ。
  \item 行列 $P^{-1}(A\pm iB)P$ を計算することによって, $P^{-1}AP$ およぴ $P^{-1}BP$ を求めよ。
  \end{enumerate}
\item $n=2$のとき.行列 $A, B, C$ の具体形を一組与えよ。
\end{enumerate}

\end{question}
