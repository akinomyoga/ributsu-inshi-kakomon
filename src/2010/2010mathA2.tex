%% -*- coding:sjis -*-
%%
%% ChangeLog
%% 2013-07-14, Koichi Murase, 作成
%%
\begin{answer}{第2問}{村瀬}
\begin{enumerate}
\item
  \begin{enumerate}
  \item
    先ず区間 $0\le x<y$ で考える.
    この区間では $u'' = 0$ が成立するので, これを積分して
    $u = Ax + B$ ($A$, $B$ は積分定数) を得る.
    境界条件より, $0=u(1)=B$, つまり $B=0$ である.

    また区間 $y< x \le 1$ でも同様にして,
    $u = A'x +B'$ ($A'$, $B'$ は積分定数) を得る.
    境界条件から, $0=u(1)=A'+B'$, つまり $B'=-A'$ である.

    次に、$\epsilon$を微少量 ($\epsilon>0$)として,
    微分方程式を区間 $(y-\epsilon, y+\epsilon)$ で積分する。
    \begin{align*}
      \left.\frac{du}{dx}\right|_{y-\epsilon}^{y+\epsilon}
        &= -\int_{y-\epsilon}^{y+\epsilon} dx\delta(x-y),\\
      u'(y+\epsilon) - u'(y-\epsilon) &= -1,\\
      A' - A &= -1,\\
      A' = A - 1.
    \end{align*}

    また、$x=y$ での連続性から、
    \begin{align*}
      u(y+\epsilon) &= u(y-\epsilon),\\
      (A-1)(y-1) &= A y,\\
      -y -A +1 &= 0,\\
      A &=1-y.
    \end{align*}

    よって、
    \begin{align*}
      u(x) &= \begin{cases}
        (1-y) x   & 0 \le x \le y,\\
        -y (x -1) & y < x \le 1.
      \end{cases}
    \end{align*}

  \item
    \begin{align*}
      \frac{d^2v}{dx^2} &= -\int_0^1 dy \rho(y)(-\delta(x-y)),\\
      v(x) &= - \int_0^1 dy \rho(y)G(x,y).\\
    \end{align*}
  \end{enumerate}

\item
  \begin{enumerate}
  \item
    \begin{align*}
      u_n(x,y) &= z^n + \bar z^n = 2 \Re z^n\\
        &= 2\Re (re^{i\theta})^n = 2 r^n \cos n\theta,\\
      \left(\frac{\partial^2}{\partial x^2} + \frac{\partial^2}{\partial y^2}\right) u_n(x,y)
        &= \left(\frac{\partial^2}{\partial r^2}
          +\frac1r\frac{\partial}{\partial r}
          +\frac1{r^2}\frac{\partial^2}{\partial \theta^2}\right) 2 r^n \cos n\theta\\
        &= \frac{n(n-1)}{r^2} u_n + \frac{n}{r^2} u_n + \frac{-n^2}{r^2} u_n\\
        &= \frac{n^2 -n + n -n^2}{r^2} u_n = 0\text{■}.
    \end{align*}
    また, $u_n|_{r=1} = 2\cos n\theta$ である.

  \item
    \begin{align*}
      |\theta| &= C + \sum_{n=1}^\infty B_n \cos n\theta,\\
      C
       &= \frac1{2\pi} \int_{-\pi}^\pi d\theta |\theta|
        = \frac1\pi \int_0^\pi \theta d\theta = \frac\pi2,\\
      B_n
       &= \frac1{\pi}\int_{-\pi}^\pi d\theta |\theta|\cos n\theta
        = \frac2{\pi} \int_0^\pi d\theta\, \theta\cos n\theta\\
       &= \left. \frac2{n\pi} \theta\sin n\theta \right|_0^\pi
          - \frac2{n\pi}\int_0^\pi d\theta \sin n\theta\\
       &= \left. \frac2{n^2\pi}\cos n\theta \right|_0^\pi
        = \begin{cases}
          \frac4{n^2\pi} & \text{$n$: even},\\
          0 & \text{$n$: odd}.
        \end{cases}
    \end{align*}
    よって,
    \begin{align*}
      |\theta|
        &= \frac\pi2 + \sum_{m=1}^\infty \frac{\cos 2m\theta}{m^2\pi}\\
        &= \frac\pi2 \left[ 1 + \sum_{m=1}^\infty \frac{2\cos 2m\theta}{(m\pi)^2} \right],\\
      u(x,y)
        &= \frac\pi2 \left[ 1 + \sum_{m=1}^\infty \frac{u_{2m}(x,y) }{(m\pi)^2} \right].
    \end{align*}
    
  \end{enumerate}

\item{}
  \paragraph{球対称な解}
  (H1) $u(\bm{x})$ を球対称な関数と仮定する: $u(\bm{x}) =u(r);\; r = |\bm{x}|$。
  この時, $V$ として半径 $R$ の球を考えると,
  \begin{align*}
    \int_{\partial V} dS\,\bm{n}\cdot\nabla u(r) &= \int_{V} d^dx \nabla\cdot\nabla u(\bm{x}),\\
    \int_{\partial V} R^{d-1} d\Omega \left. \frac{\partial}{\partial r} u(r) \right|_{r=R}
      &= -\int_{V} \prod_{a=1}^d dx_a \delta(x_a),\\
    \frac{2\pi^{d/2}R^{d-1}}{\Gamma(d/2)}\frac{\partial}{\partial R} u(R) &= -1,\\
    u(R)
      &= -\frac{\Gamma(d/2)}{2\pi^{d/2}} \int dR \frac1{R^{d-1}}\\
      &= -\frac{\Gamma(d/2)}{2\pi^{d/2}}\cdot \frac{-1}{(d-2)R^{d-2}} + C \quad\text{($C$ 積分定数)}.\\
  \end{align*}
  ここで、境界条件 $0 = \lim_{|\bm{x}|\to\infty} u(\bm{x}) = \lim_{r\to\infty} u(r)$ より, $C=0$.
  \begin{align*}
    u(r)
      &= \frac{\Gamma(d/2)}{2(d-2)\pi^{d/2} r^{d-2}}.
  \end{align*}

  \paragraph{解の一意性}
  解として $u_1(\bm x)$, $u_2(\bm x)$ が存在する時,
  $f(\bm x) :=u_1(\bm x) - u_2(\bm x)$ とする.
  \begin{align*}
  \nabla\cdot\nabla f
    &= \nabla\cdot\nabla u_1 - \nabla\cdot\nabla u_2\\
    &= -(\prod_{a=1}^d \delta(x_a) - \prod_{a=1}^d \delta(x_a)) = 0,\\
  f &= \bm A \cdot \bm x  + B,\\
  \end{align*}
  ここで, 境界条件より,
  $\lim_{|\bm{x}|\to\infty} f(\bm x)
  = \lim_{|\bm{x}|\to\infty} (u_1(\bm x) - u_2(\bm{x})) = 0$ より $\bm{A} = B = 0$.
  従って, $f\equiv0$, つまり $u_1(\bm x) \equiv u_2(\bm{x})$ であり, 解は一意に存在する.
  この事から先に求めた球対称な解がそれであると分かり,
  仮定 (H1) が妥当であった事が分かる.
  \begin{align*}
    u(\bm{x}) &= \frac{\Gamma(d/2)}{2(d-2)\pi^{d/2} |\bm{x}|^{d-2}}.
  \end{align*}

\end{enumerate}
\end{answer}
