\documentclass[fleqn]{jbook}
\usepackage{physpub}
\def\t{\times}
\def\lam#1{\lambda_{#1}}
\def\w#1{\vec{w}_{#1}}


\begin{document}

%%%%%【数学 [第1問] (問)】%%%%%%%%%%%%%%%%%%%%%%%%%%%%%%%%%%%%%%%%%%%%%%%%%%%%%%%

\begin{question}{数学}{岡村圭祐}
\setcounter{equation}{0}

    %%%%% 第1問 %%%%%
[第1問]\\
 $2\t 2$の実行列$:A =
     \left( {\begin{array}{*{20}c}
   {1 - \gamma \cos 2\phi } & { - \gamma \sin 2\phi }  \\
   { - \gamma \sin 2\phi } & {1 + \gamma \cos 2\phi }  \\
     \end{array}} \right)$に関する以下の設問に答えよ。ただし,$0<\gamma<1$であるものとする。\\
     
    \begin{description}
        \item[{\rm (i)}]  行列$A$の固有値$\lam+,\lam-$を求めよ($\lam+>\lam-$とする)。\\
        
        \item[{\rm (ii)}]  $\lam+$と$\lam-$に対応する規格化された固有ベクトル$\w+,\w-$を求めよ。\\
        
        \item[{\rm (iii)}]  $UAU^{-1}=
        \left( {\begin{array}{*{20}c}
   {\lam+} & {0}  \\
   {0} & {\lam-}  \\
     \end{array}} \right)$のように$A$を対角化する行列$U$を求めよ。また,この行列$U$が表す変換はどのような操作に対応しているか簡単に述べよ。\\
     
        \item[{\rm (iv)}]  下図のように,単位円C上の点P$(\cos\phi,\sin\phi)$において上述の行列$A$が定義されているものとし,この点Pを原点とした任意のベクトル$\vec{w}$を,点Pを原点としたベクトル$A\vec{w}$へ変換する。この写像によって,図に点線で示された8つの円はどのように変形するか。答案用紙に簡単に図示して説明せよ。\\

%%%%%%%%%%%%%%%%%%%%%%%%%%%%%%%%%%%%%%%%%%%%%%%%%%%%%%%%%%%%
\begin{figure}[hbtp]
\begin{center}
%\caption[]{\ilabel{}}
\vspace{1.0cm}
\input{2001math1-1.tpc}
\end{center}
\end{figure}
%%%%%%%%%%%%%%%%%%%%%%%%%%%%%%%%%%%%%%%%%%%%%%%%%%%%%%%%%%%%

    \end{description}

%\end{question}

\newpage

%%%%%【数学 [第2問] (問)】%%%%%%%%%%%%%%%%%%%%%%%%%%%%%%%%%%%%%%%%%%%%%%%%%%%%%%%

%\begin{question}{数学}{岡村圭祐}
\setcounter{equation}{0}

    %%%%% 第2問 %%%%%

[第2問]\\

 三次元ラプラス方程式は,極座標を用いて
    \begin{eqnarray}
  &&\left( {\frac{1}{r}\frac{{\partial ^2 }}{{\partial r^2 }}r - \frac{{\hat L^2 }}{{r^2 }}} \right)U\left( {r,\theta ,\varphi } \right) = 0, \\ 
  && \cr
  &&\hat L^2  \equiv  - \left( {\frac{1}{{\sin \theta }}\frac{\partial }{{\partial \theta }}\sin \theta \frac{\partial }{{\partial \theta }} + \frac{1}{{\sin ^2 \theta }}\frac{{\partial ^2 }}{{\partial \varphi ^2 }}} \right) 
    \end{eqnarray}
    と表せる。また,球面調和関数$Y_{\ell m}\left( \theta,\varphi \right)$は,
    \begin{equation}
    \hat L^2Y_{\ell m}\left( \theta,\varphi \right)=\ell \left( \ell +1 \right)Y_{\ell m}\left( \theta,\varphi \right)
    \end{equation}
    を満たす$\left( \ell =0,1,2,\dots;~m=-\ell,-\ell+1,\dots,\ell \right)$。\\
    
    
    \begin{description}
    
        \item[{\rm (i)}]  ラプラス方程式の解$U\left( r,\theta,\varphi \right)$を$R_\ell\left( r \right)Y_{\ell m}\left( \theta,\varphi \right)$と変数分離したとき,$R_\ell\left( r \right)$の満たす微分方程式を求めよ。\\
        
        \item[{\rm (ii)}]  設問(i)で得られた微分方程式から$R_\ell\left( r \right)$の独立な二つの基本解$A_\ell\left( r \right),B_\ell\left( r \right)$を求めよ。\\
    \end{description}
    
    $U\left( r,\theta,\varphi \right)$に対する境界条件が球面上で与えられれば,設問(ii)で得られた$A_\ell\left( r \right)$と$B_\ell\left( r \right)$を用いて,
    \begin{equation}
U\left( {r,\theta ,\varphi } \right) = \sum\limits_{\ell = 0}^\infty  {\sum\limits_{m =  - \ell }^{\ell } {\left[ {\alpha _{\ell m} A_\ell \left( r \right) + \beta _{\ell m} B_\ell \left( r \right)} \right]Y_{\ell m} \left( {\theta ,\varphi } \right)} } 
    \end{equation}
    と展開し,境界条件に合うような係数$\alpha _{\ell m},\beta _{\ell m}$を持つ解を求めることができる。具体的に,$r=a$及び$b$(ただし$b>a>0$とする)での境界条件が
    \begin{equation}
    U\left( {a,\theta ,\varphi } \right)=0,\quad U\left( {b,\theta ,\varphi } \right)=\sin\theta\cos\varphi
    \end{equation}
    と与えられているとき,以下の問いに答えよ。\\
    
    \begin{description}
    
        \item[{\rm (iii)}]  $\sin\theta\cos\varphi$を$Y_{\ell m}\left( \theta,\varphi \right)$を用いて表せ。%% 原文ではここで改行あり。
        ただし,$\ds Y_{00}\left( \theta,\varphi \right)=\sqrt{\frac{1}{4\pi}}$,$\ds Y_{1 \pm 1}\left( \theta,\varphi \right)=\mp\sqrt{\frac{3}{8\pi}}\sin\theta e^{\pm i\varphi}$(複号同順),$\ds Y_{10}\left( \theta,\varphi \right)=\sqrt{\frac{3}{4\pi}}\cos\theta$を用いてよい。\\
        
        \item[{\rm (iv)}]  境界条件(5)と$Y_{\ell m}\left( \theta,\varphi \right)$の直交条件
        \begin{equation}
\int {\d\Omega Y_{\ell'm'} ^ *  \left( {\theta ,\varphi } \right)Y_{\ell m} \left( {\theta ,\varphi } \right)}  = \delta _{\ell'\ell} \delta _{m'm} 
        \end{equation}
        から係数$\alpha _{\ell m},\beta _{\ell m}$を決定せよ。
        
    \end{description}
    

\end{question}


%%%%%【数学 [第1問] (答)】%%%%%%%%%%%%%%%%%%%%%%%%%%%%%%%%%%%%%%%%%%%%%%%%%%%%%%%

\begin{answer}{数学}{山本哲朗}
\setcounter{equation}{0}

[第1問]\\

    %%%%% 第1問 %%%%%

    \begin{description}
        \item[{\rm (i)}]  $A$の固有値方程式は
\begin{equation}
\lambda^2-\lambda+(1-\gamma^2)=0 
~\iff~ (\lambda-1-\gamma)(\lambda-1+\gamma)=0 
~\iff~ \lambda=1\pm \gamma\nonumber
\end{equation}
よって\qquad $\lambda_+=1+\gamma,~\lambda_-=1-\gamma .$

        \item[{\rm (ii)}]  今,$\cos\phi\neq 0$としておく。$A\left(
\begin{array}{c}
x \\
y \\
\end{array}
\right)
=\lambda_+\left(
\begin{array}{c}
x \\
y \\
\end{array}
\right)$とおくと,

\begin{equation}
\left\{ {\begin{array}{*{20}c}
   {(1-\gamma\cos 2\phi)x-\gamma\sin 2\phi\cdot y=(1+\gamma)x}  \\
   {-\gamma\sin 2\phi\cdot x+(1+\gamma\cos 2\phi)y=(1+\gamma)y}  \\
\end{array}} \right.~\Longrightarrow~x:y=(-\sin 2\phi):(1+\cos 2\phi)\nonumber
\end{equation}

よって
\begin{equation*}
\vec{w}_+ = \frac{1}{\sqrt{2(1+\cos 2\phi)}}
\left(
\begin{array}{c}
-\sin 2\phi \\
1+\cos 2\phi \\
\end{array}
\right) = \frac{1}{2\cos \phi}
\left(
\begin{array}{c}
-2\sin\phi\cos\phi \\
2\cos^2\phi \\
\end{array}
\right) = \left(
\begin{array}{c}
-\sin\phi \\
\cos\phi \\
\end{array}
\right) .
\end{equation*}
(途中で根号には絶対値をつけるべきだが,後の議論に支障はない。)同様にして,
%\begin{equation}
$\ds \vec{w}_-=\left(
\begin{array}{c}
\cos\phi \\
\sin\phi \\
\end{array}
\right) .$\\
%\end{equation}
一方で,${\cos\phi=0}$のときも上で求めたベクトルが固有ベクトルになっている。\\

        
        \item[{\rm (iii)}]  (ii)で求めた$\vec{w}_+,\vec{w}_-$を用いて,
%\begin{equation}
$\ds U = ^t(\vec{w}_-,\vec{w}_+) 
=
\left(
\begin{array}{cc}
\cos\phi & \sin\phi \\
-\sin\phi & \cos\phi \\
\end{array}
\right) .$\\
%\end{equation}
この行列が表す変換は,
%% 
位置ベクトルを原点周りに
%% 
$-\phi$だけ回転させる操作に対応している。\\

        \item[{\rm (iv)}]  次の図のように,各々の円はCの動径方向に$(1-\gamma)$倍,接線方向に$(1+\gamma)$倍される。\\
%%%%%%%%%%%%%%%%%%%%%%%%%%%%%%%%%%%%%%%%%%%%%%%%%%%%%%%%%%%%
\begin{figure}[hbtp]
\begin{center}
%\caption[]{\ilabel{}}
%\vspace{1.0cm}
\input{2001math1-2.tpc}
\end{center}
\end{figure}
%%%%%%%%%%%%%%%%%%%%%%%%%%%%%%%%%%%%%%%%%%%%%%%%%%%%%%%%%%%%

    \end{description}
%\end{answer}

\newpage

%%%%%【数学 [第2問] (答)】%%%%%%%%%%%%%%%%%%%%%%%%%%%%%%%%%%%%%%%%%%%%%%%%%%%%%%%

%\begin{answer}{数学}{山本哲朗}
\setcounter{equation}{0}

[第2問]\\

    %%%%% 第2問 %%%%%

    \begin{description}
        \item[{\rm (i)}]  $R_\ell\left( r \right)Y_{\ell m}\left( \theta,\varphi \right)$を$\ds \left( {\frac{1}{r}\frac{{\partial ^2 }}{{\partial r^2 }}r - \frac{{\hat L^2 }}{{r^2 }}} \right)U\left( {r,\theta ,\varphi } \right) = 0$に代入すると,
\begin{equation}
\left\{ {\begin{array}{*{20}l}
   {\ds Y_{\ell m}\left( \theta,\varphi \right)\left\{ \frac{1}{r}\frac{\partial^2}{\partial r^2}rR_\ell\left( r \right)\right\} -R_\ell\left( r \right)\frac{\hat{L}^2}{r^2}Y_{\ell m}\left( \theta,\varphi \right)=0}  \\
   \cr
   {\ds \frac{1}{R_\ell\left( r \right)}\left\{\frac{1}{r}\frac{\partial^2}{\partial r^2}rR_\ell\left( r \right)\right\}-\frac{1}{Y_{\ell m}\left( \theta,\varphi \right)}\frac{\ell\left(\ell+1\right)}{r^2}Y_{\ell m}\left( \theta,\varphi \right)=0}  \\
   \cr
   {\ds \left\{ r^2\frac{\partial^2}{\partial r^2}+2r\frac{\partial}{\partial r}-\ell\left(\ell +1\right)\right\}R_\ell\left( r \right)=0}  \\
\end{array}} \right.
\end{equation}
これが$R_\ell\left( r \right)$の満たす微分方程式である。\\
        
        \item[{\rm (ii)}]  $\ds R_\ell\left( r \right)=r^\lambda\sum^\infty_{n=0}c_nr^n~(c_0\neq 0)$とおいて(i)の答の式に代入すると,$r^\lambda$の係数について
\begin{eqnarray}
\left[\lambda\left(\lambda-1\right)+2\lambda-\ell\left(\ell+1\right)\right]c_0=0 \nonumber \\
\lambda^2+\lambda-\ell\left(\ell+1\right)=0 \nonumber \\
\left(\lambda-\ell\right)\left(\lambda+\ell+1\right)=0 \nonumber \\
\lambda=\ell,-\ell-1 \nonumber
\end{eqnarray}
また,$r^{\lambda+n}$の係数について
\begin{eqnarray}
\left(\lambda+n\right)\left(\lambda+n-1\right)c_n+2\left(\lambda+n\right)c_n-\ell\left(\ell+1\right)c_n=0 \nonumber \\
\left(\lambda+n-l\right)\left(\lambda+n+l+1\right)c_n=0 \nonumber
\end{eqnarray}
よって
\begin{description}
    \item[{\rm a)}] ~$\lambda=\ell$のとき,$n\left(n+2\ell+1\right)c_n=0$ ,ゆえに$n\neq 0,-2\ell-1$ならば$c_n=0$ 
    \item[{\rm b)}] ~$\lambda=-\ell-1$のとき,$\left(n-2\ell-1\right)nc_n=0$ ,ゆえに$n\neq 0,2\ell+1$ならば$c_n=0$ 
\end{description}
このことから独立な2つの基本解として
\begin{eqnarray}
A_\ell\left(r\right)=r^\ell,~B_\ell\left(r\right)=r^{-\ell-1} \nonumber
\end{eqnarray}
が求められる。\\
        
        \item[{\rm (iii)}]  次のように表される。
\begin{eqnarray}
\sin\theta\cos\phi &=& \frac{1}{2}\sin\theta\cdot e^{i\phi}+\frac{1}{2}\sin\cdot e^{-i\phi} \nonumber \\
&=& -\sqrt{\frac{2\pi}{3}}Y_{1,1}\left( \theta,\varphi \right)+\sqrt{\frac{2\pi}{3}}Y_{1,-1}\left( \theta,\varphi \right) \nonumber \\
&=& \sqrt{\frac{2\pi}{3}}\left\{-Y_{11}\left( \theta,\varphi \right)+Y_{1,-1}\left( \theta,\varphi \right)\right\} \nonumber
\end{eqnarray}
        
        \item[{\rm (iv)}]  直交条件より
\begin{equation}
\int {\d\Omega Y_{lm}\left( \theta,\varphi \right)U\left( b,\theta,\varphi \right)}=
\left\{ {\begin{array}{*{20}r}
   {\ds -\sqrt{\frac{2\pi}{3}}} & {\left( \ell=1,m=1 \right)}  \\
   \cr
   {\ds \sqrt{\frac{2\pi}{3}}} & {\left( \ell=1,m=-1 \right)}  \\
   \cr
   {\ds 0} & {\left( {\rm otherwise} \right)}  \\
\end{array}} \right.\nonumber
\end{equation}
よって$U\left( {r,\theta ,\varphi } \right) = \sum\limits_{\ell = 0}^\infty  {\sum\limits_{m =  - \ell }^{\ell } {\left[ {\alpha _{\ell m} A_\ell \left( r \right) + \beta _{\ell m} B_\ell \left( r \right)} \right]Y_{\ell m} \left( {\theta ,\varphi } \right)} } $を代入して,
\begin{eqnarray}
\left\{
\begin{array}{rrr}
\ds \alpha_{11}\cdot a+\beta_{11}\cdot \frac{1}{a^2} &=& \ds 0 \\
\cr
\ds \alpha_{11}\cdot b+\beta_{11}\cdot \frac{1}{b^2} &=& \ds -\sqrt{\frac{2\pi}{3}} \\
\end{array}
\right.\nonumber\\
\Longrightarrow~\alpha_{1,1}=-\frac{b^2}{b^3-a^3}\sqrt{\frac{2\pi}{3}},\quad 
\beta_{1,1}=\frac{a^3b^2}{b^3-a^3}\sqrt{\frac{2\pi}{3}}. \nonumber
\end{eqnarray}
同様にして
\begin{eqnarray}
\alpha_{1,-1}=\frac{b^2}{b^3-a^3}\sqrt{\frac{2\pi}{3}},\quad 
\beta_{1,-1}=-\frac{a^3b^2}{b^3-a^3}\sqrt{\frac{2\pi}{3}}. \nonumber
\end{eqnarray}
その他の$\alpha_{\ell m},\beta_{\ell m}$については$0$である。
        
    \end{description}
    

\end{answer}

\end{document}

