\documentclass[fleqn]{jbook}
\usepackage{physpub}
\def\bm{\boldsymbol}
\def\ds{\displaystyle}



%% Defined by 

\begin{document}

%%%%%【問題7(問)】%%%%%%%%%%%%%%%%%%%%%%%%%%%%%%%%%%%%%%%%%%%%%%%%%%%%%%%

\begin{question}{問題7}{岡村}
\setcounter{equation}{0}

 電子と同じ電荷$e$を持ち運動量$p~[{\rm GeV}/c]$を持つ素粒子が一様磁場$B\Unit{[Tesla]}$のもとで円運動をする。円運動の半径$R\Unit{[m]}$は
  \begin{equation}
  R=\frac{p}{0.3B}
  \end{equation}
  に従う。\\
   図のような%% 原文では「様な」になっているが一般的でない。
  装置を考える。磁場$\Vec{B}$は紙面に垂直方向を向いており,3枚の平板状の位置検出器を,図の紙面では垂直に,等間隔$L$で設置してある。矢印のついている円弧は荷電素粒子の軌跡である。なお,$L$は円運動の半径$R$に比べて十分小さい場合を考える。このような装置をマグネティック・スペクトロメータと呼んでいる。以下で,各位置検出器は有限な位置分解能を持つ%% 原文では「持つ」になっているが,先の「持ち」との統一のため。
  として運動量測定について考えるが,運動量の測定誤差を$\delta p$とした場合に$\delta p/p$を運動量分解能と定義する。以下の問に,解答に至る筋道を添えて答えよ。\\
%%%%%%%%%%%%%%%%%%%%%%%%%%%%%%%%%%%%%%%%%%%%%%%%%%%%%%%%%%%%
\begin{figure}[hbt]
\begin{center}
%\caption[]{\ilabel{}}
\vspace{-0.2cm}
\hspace{-1.0cm}\documentclass[fleqn]{jbook}
\usepackage{physpub}
\def\bm{\boldsymbol}
\def\ds{\displaystyle}



%% Defined by 

\begin{document}

%%%%%【問題7(問)】%%%%%%%%%%%%%%%%%%%%%%%%%%%%%%%%%%%%%%%%%%%%%%%%%%%%%%%

\begin{question}{問題7}{岡村}
\setcounter{equation}{0}

 電子と同じ電荷$e$を持ち運動量$p~[{\rm GeV}/c]$を持つ素粒子が一様磁場$B\Unit{[Tesla]}$のもとで円運動をする。円運動の半径$R\Unit{[m]}$は
  \begin{equation}
  R=\frac{p}{0.3B}
  \end{equation}
  に従う。\\
   図のような%% 原文では「様な」になっているが一般的でない。
  装置を考える。磁場$\Vec{B}$は紙面に垂直方向を向いており,3枚の平板状の位置検出器を,図の紙面では垂直に,等間隔$L$で設置してある。矢印のついている円弧は荷電素粒子の軌跡である。なお,$L$は円運動の半径$R$に比べて十分小さい場合を考える。このような装置をマグネティック・スペクトロメータと呼んでいる。以下で,各位置検出器は有限な位置分解能を持つ%% 原文では「持つ」になっているが,先の「持ち」との統一のため。
  として運動量測定について考えるが,運動量の測定誤差を$\delta p$とした場合に$\delta p/p$を運動量分解能と定義する。以下の問に,解答に至る筋道を添えて答えよ。\\
%%%%%%%%%%%%%%%%%%%%%%%%%%%%%%%%%%%%%%%%%%%%%%%%%%%%%%%%%%%%
\begin{figure}[hbt]
\begin{center}
%\caption[]{\ilabel{}}
\vspace{-0.2cm}
\hspace{-1.0cm}\documentclass[fleqn]{jbook}
\usepackage{physpub}
\def\bm{\boldsymbol}
\def\ds{\displaystyle}



%% Defined by 

\begin{document}

%%%%%【問題7(問)】%%%%%%%%%%%%%%%%%%%%%%%%%%%%%%%%%%%%%%%%%%%%%%%%%%%%%%%

\begin{question}{問題7}{岡村}
\setcounter{equation}{0}

 電子と同じ電荷$e$を持ち運動量$p~[{\rm GeV}/c]$を持つ素粒子が一様磁場$B\Unit{[Tesla]}$のもとで円運動をする。円運動の半径$R\Unit{[m]}$は
  \begin{equation}
  R=\frac{p}{0.3B}
  \end{equation}
  に従う。\\
   図のような%% 原文では「様な」になっているが一般的でない。
  装置を考える。磁場$\Vec{B}$は紙面に垂直方向を向いており,3枚の平板状の位置検出器を,図の紙面では垂直に,等間隔$L$で設置してある。矢印のついている円弧は荷電素粒子の軌跡である。なお,$L$は円運動の半径$R$に比べて十分小さい場合を考える。このような装置をマグネティック・スペクトロメータと呼んでいる。以下で,各位置検出器は有限な位置分解能を持つ%% 原文では「持つ」になっているが,先の「持ち」との統一のため。
  として運動量測定について考えるが,運動量の測定誤差を$\delta p$とした場合に$\delta p/p$を運動量分解能と定義する。以下の問に,解答に至る筋道を添えて答えよ。\\
%%%%%%%%%%%%%%%%%%%%%%%%%%%%%%%%%%%%%%%%%%%%%%%%%%%%%%%%%%%%
\begin{figure}[hbt]
\begin{center}
%\caption[]{\ilabel{}}
\vspace{-0.2cm}
\hspace{-1.0cm}\documentclass[fleqn]{jbook}
\usepackage{physpub}
\def\bm{\boldsymbol}
\def\ds{\displaystyle}



%% Defined by 

\begin{document}

%%%%%【問題7(問)】%%%%%%%%%%%%%%%%%%%%%%%%%%%%%%%%%%%%%%%%%%%%%%%%%%%%%%%

\begin{question}{問題7}{岡村}
\setcounter{equation}{0}

 電子と同じ電荷$e$を持ち運動量$p~[{\rm GeV}/c]$を持つ素粒子が一様磁場$B\Unit{[Tesla]}$のもとで円運動をする。円運動の半径$R\Unit{[m]}$は
  \begin{equation}
  R=\frac{p}{0.3B}
  \end{equation}
  に従う。\\
   図のような%% 原文では「様な」になっているが一般的でない。
  装置を考える。磁場$\Vec{B}$は紙面に垂直方向を向いており,3枚の平板状の位置検出器を,図の紙面では垂直に,等間隔$L$で設置してある。矢印のついている円弧は荷電素粒子の軌跡である。なお,$L$は円運動の半径$R$に比べて十分小さい場合を考える。このような装置をマグネティック・スペクトロメータと呼んでいる。以下で,各位置検出器は有限な位置分解能を持つ%% 原文では「持つ」になっているが,先の「持ち」との統一のため。
  として運動量測定について考えるが,運動量の測定誤差を$\delta p$とした場合に$\delta p/p$を運動量分解能と定義する。以下の問に,解答に至る筋道を添えて答えよ。\\
%%%%%%%%%%%%%%%%%%%%%%%%%%%%%%%%%%%%%%%%%%%%%%%%%%%%%%%%%%%%
\begin{figure}[hbt]
\begin{center}
%\caption[]{\ilabel{}}
\vspace{-0.2cm}
\hspace{-1.0cm}\input{2001phy7.tpc}
\vspace{-0.5cm}
\end{center}
\end{figure}
%%%%%%%%%%%%%%%%%%%%%%%%%%%%%%%%%%%%%%%%%%%%%%%%%%%%%%%%%%%%


\begin{enumerate}

%%%%%%【1.】%%%%%%
  \item  よりよい運動量分解能を得るためには強い磁場,測定器のより広い間隔,よりよい位置分解能が必要であることを直観的に説明せよ。また,運動量が小さい場合と大きい場合とどちらが運動量分解能が高いか答えよ。\\
  
  
%%%%%%【2.】%%%%%%
  \item  以上の傾向を定量的に調べるために以下の設問に答えよ。なおここでは簡略化のため,荷電粒子は$x$-$y$面内に入射し,円運動の中心は図のように2の測定器面上にあるとする。まず,図の$S$をサジッタと呼ぶ。\\
  
    \begin{enumerate}
    
%%%% a %%%%
    \item  サジッタ$S$を$L$と$R$の関数で表せ。$L\ll R$であることを考慮して近似を用いる%% 原文では「もちいる」と平仮名。ここも統一のため漢字にした。
    こと。さらに,$(1)$式を用いてサジッタ$S$を$L$と$p$の関数で表せ。
    
%%%% b %%%%
    \item  サジッタに誤差$\delta S$があると考え,$\delta p$を$L,B,p,\delta S$の関数で表せ。
    
%%%% c %%%%
    \item  この誤差の原因は測定器による誤差である。測定器1,2および3は,等しい$z$座標を持つ%% 原文では「もつ」と平仮名。
    面上において$x$方向のみ測定するとし,各測定器の位置分解能($x$方向)は同じ値$\sigma$を持つと仮定して,サジッタ$S$の誤差$\delta S$を$\sigma$の関数で表せ。
    
%%%% d %%%%
    \item  以上により,$\delta p/p$を$L,B,p,\sigma$の関数で表せ。\\
    
\end{enumerate}

%%%%%%【3.】%%%%%%
  \item  現実に,マグネティック・スペクトロメータを設計する際に,上記以外の要因も含めて,どのような点に留意せねばならないかを論ぜよ。\\
  
\end{enumerate}

\end{question}

%%%%%【問題7(答)】%%%%%%%%%%%%%%%%%%%%%%%%%%%%%%%%%%%%%%%%%%%%%%%%%%%%%%%

\begin{answer}{問題7}{新原}
\setcounter{equation}{0}



\begin{enumerate}

%%%%%%【1.】%%%%%%
  \item  よりよい運動量分解能を得るには,観測点3点により作られる三角形の外接円の半径を精度良く観測できればよく,これは作られる三角形を大きくすれば誤差が相対的に小さくなり達成することができる。この前提をもとに諸パラメータの大きさをどのようにすればよいかを以下論ずることにする。
  
  \begin{itemize}
    \item ~位置分解能$\sigma$ について:\\
    これは明らかに,三角形の大きさを正確に決めるには小さい方が良く議論するまでも無い。
    \item ~$B,L,p$について:\\
    三角形の辺の長さが大きくなるときの条件としてLが大きく,サジッタ$S$が大きいほうが良いと思われる。そのためR$\gg$Lの時には$R$が小さいほうが良いことになり,
\begin{equation}
R = \frac{p}{0.3B}
\end{equation}
より$B$は大きくしたほうが良く,$p$が小さいときのほうが精度良く測定することができる。

  \end{itemize}
  

%%%%%%【2.】%%%%%%
  \item  
  \begin{enumerate}
    
%%%% a %%%%
    \item  図より$S = R -R\cos\theta$,$\ds \cos\theta \simeq 1 - \frac{1}{2} \left(\frac{L}{R}\right)^2$。以上より
\begin{equation}
S = \frac{L^2}{2R} = 0.15 \frac{BL^2}{p}
\end{equation}
    
    
%%%% b %%%%
    \item  (a)より$\ds p = 0.15 \times \frac{BL^2}{S}$。よって
\begin{equation}
\delta p = 0.15 \times \frac{BL^2}{S^2} \delta S = \frac{p^2}{0.15 \times BL^2} \delta S
\end{equation}
    
%%%% c %%%%
    \item  $\delta S = x_2 - x_1$。よって
\begin{equation}
\delta S = \sqrt{\sigma ^2 + \sigma ^2} 
= \sqrt{2} \sigma
\end{equation}

%%%% d %%%%
    \item  $\ds \frac{\delta p}{p} = \frac{\sqrt{2} \sigma p}{0.15 BL^2}$\\
    
\end{enumerate}

%%%%%%【3.】%%%%%%
  \item  これまで論じてきた測定計画を忠実に再現できれば,上に述べた測定精度で測定することができるが,実際にはこの実験を阻害する要因がある。\\
 次にその改善策を示していきたいと思う。(ただしここでは粒子の種類の同定などの機構については考えず,$p$を精度良く測定することについて論ずることにする。)\\
 この実験では$(1)$式が成り立つという仮定のもとに考えており,換言すれば粒子がこの式に従って運動すれば理想通りに行くものと思われる。そのためには

\begin{itemize}
    \item 測定器1〜3の間では運動量一定
    \item 測定器のなかでは磁場一定で電場は存在しない(ドリフトを起こさせないため)
\end{itemize}
の二つの条件を満たすようにすることが重要になり,
\begin{itemize}
    \item 装置内は十分な真空度を保つ
    \item 乾板を薄くする
    \item 磁場の大きさが変化しない程度に装置を小さくするか,磁場を一様に保つ工夫を考える
\end{itemize}
といったことに留意すべきであると思われる。\\

\end{enumerate}


\end{answer}

\end{document}

\vspace{-0.5cm}
\end{center}
\end{figure}
%%%%%%%%%%%%%%%%%%%%%%%%%%%%%%%%%%%%%%%%%%%%%%%%%%%%%%%%%%%%


\begin{enumerate}

%%%%%%【1.】%%%%%%
  \item  よりよい運動量分解能を得るためには強い磁場,測定器のより広い間隔,よりよい位置分解能が必要であることを直観的に説明せよ。また,運動量が小さい場合と大きい場合とどちらが運動量分解能が高いか答えよ。\\
  
  
%%%%%%【2.】%%%%%%
  \item  以上の傾向を定量的に調べるために以下の設問に答えよ。なおここでは簡略化のため,荷電粒子は$x$-$y$面内に入射し,円運動の中心は図のように2の測定器面上にあるとする。まず,図の$S$をサジッタと呼ぶ。\\
  
    \begin{enumerate}
    
%%%% a %%%%
    \item  サジッタ$S$を$L$と$R$の関数で表せ。$L\ll R$であることを考慮して近似を用いる%% 原文では「もちいる」と平仮名。ここも統一のため漢字にした。
    こと。さらに,$(1)$式を用いてサジッタ$S$を$L$と$p$の関数で表せ。
    
%%%% b %%%%
    \item  サジッタに誤差$\delta S$があると考え,$\delta p$を$L,B,p,\delta S$の関数で表せ。
    
%%%% c %%%%
    \item  この誤差の原因は測定器による誤差である。測定器1,2および3は,等しい$z$座標を持つ%% 原文では「もつ」と平仮名。
    面上において$x$方向のみ測定するとし,各測定器の位置分解能($x$方向)は同じ値$\sigma$を持つと仮定して,サジッタ$S$の誤差$\delta S$を$\sigma$の関数で表せ。
    
%%%% d %%%%
    \item  以上により,$\delta p/p$を$L,B,p,\sigma$の関数で表せ。\\
    
\end{enumerate}

%%%%%%【3.】%%%%%%
  \item  現実に,マグネティック・スペクトロメータを設計する際に,上記以外の要因も含めて,どのような点に留意せねばならないかを論ぜよ。\\
  
\end{enumerate}

\end{question}

%%%%%【問題7(答)】%%%%%%%%%%%%%%%%%%%%%%%%%%%%%%%%%%%%%%%%%%%%%%%%%%%%%%%

\begin{answer}{問題7}{新原}
\setcounter{equation}{0}



\begin{enumerate}

%%%%%%【1.】%%%%%%
  \item  よりよい運動量分解能を得るには,観測点3点により作られる三角形の外接円の半径を精度良く観測できればよく,これは作られる三角形を大きくすれば誤差が相対的に小さくなり達成することができる。この前提をもとに諸パラメータの大きさをどのようにすればよいかを以下論ずることにする。
  
  \begin{itemize}
    \item ~位置分解能$\sigma$ について:\\
    これは明らかに,三角形の大きさを正確に決めるには小さい方が良く議論するまでも無い。
    \item ~$B,L,p$について:\\
    三角形の辺の長さが大きくなるときの条件としてLが大きく,サジッタ$S$が大きいほうが良いと思われる。そのためR$\gg$Lの時には$R$が小さいほうが良いことになり,
\begin{equation}
R = \frac{p}{0.3B}
\end{equation}
より$B$は大きくしたほうが良く,$p$が小さいときのほうが精度良く測定することができる。

  \end{itemize}
  

%%%%%%【2.】%%%%%%
  \item  
  \begin{enumerate}
    
%%%% a %%%%
    \item  図より$S = R -R\cos\theta$,$\ds \cos\theta \simeq 1 - \frac{1}{2} \left(\frac{L}{R}\right)^2$。以上より
\begin{equation}
S = \frac{L^2}{2R} = 0.15 \frac{BL^2}{p}
\end{equation}
    
    
%%%% b %%%%
    \item  (a)より$\ds p = 0.15 \times \frac{BL^2}{S}$。よって
\begin{equation}
\delta p = 0.15 \times \frac{BL^2}{S^2} \delta S = \frac{p^2}{0.15 \times BL^2} \delta S
\end{equation}
    
%%%% c %%%%
    \item  $\delta S = x_2 - x_1$。よって
\begin{equation}
\delta S = \sqrt{\sigma ^2 + \sigma ^2} 
= \sqrt{2} \sigma
\end{equation}

%%%% d %%%%
    \item  $\ds \frac{\delta p}{p} = \frac{\sqrt{2} \sigma p}{0.15 BL^2}$\\
    
\end{enumerate}

%%%%%%【3.】%%%%%%
  \item  これまで論じてきた測定計画を忠実に再現できれば,上に述べた測定精度で測定することができるが,実際にはこの実験を阻害する要因がある。\\
 次にその改善策を示していきたいと思う。(ただしここでは粒子の種類の同定などの機構については考えず,$p$を精度良く測定することについて論ずることにする。)\\
 この実験では$(1)$式が成り立つという仮定のもとに考えており,換言すれば粒子がこの式に従って運動すれば理想通りに行くものと思われる。そのためには

\begin{itemize}
    \item 測定器1〜3の間では運動量一定
    \item 測定器のなかでは磁場一定で電場は存在しない(ドリフトを起こさせないため)
\end{itemize}
の二つの条件を満たすようにすることが重要になり,
\begin{itemize}
    \item 装置内は十分な真空度を保つ
    \item 乾板を薄くする
    \item 磁場の大きさが変化しない程度に装置を小さくするか,磁場を一様に保つ工夫を考える
\end{itemize}
といったことに留意すべきであると思われる。\\

\end{enumerate}


\end{answer}

\end{document}

\vspace{-0.5cm}
\end{center}
\end{figure}
%%%%%%%%%%%%%%%%%%%%%%%%%%%%%%%%%%%%%%%%%%%%%%%%%%%%%%%%%%%%


\begin{enumerate}

%%%%%%【1.】%%%%%%
  \item  よりよい運動量分解能を得るためには強い磁場,測定器のより広い間隔,よりよい位置分解能が必要であることを直観的に説明せよ。また,運動量が小さい場合と大きい場合とどちらが運動量分解能が高いか答えよ。\\
  
  
%%%%%%【2.】%%%%%%
  \item  以上の傾向を定量的に調べるために以下の設問に答えよ。なおここでは簡略化のため,荷電粒子は$x$-$y$面内に入射し,円運動の中心は図のように2の測定器面上にあるとする。まず,図の$S$をサジッタと呼ぶ。\\
  
    \begin{enumerate}
    
%%%% a %%%%
    \item  サジッタ$S$を$L$と$R$の関数で表せ。$L\ll R$であることを考慮して近似を用いる%% 原文では「もちいる」と平仮名。ここも統一のため漢字にした。
    こと。さらに,$(1)$式を用いてサジッタ$S$を$L$と$p$の関数で表せ。
    
%%%% b %%%%
    \item  サジッタに誤差$\delta S$があると考え,$\delta p$を$L,B,p,\delta S$の関数で表せ。
    
%%%% c %%%%
    \item  この誤差の原因は測定器による誤差である。測定器1,2および3は,等しい$z$座標を持つ%% 原文では「もつ」と平仮名。
    面上において$x$方向のみ測定するとし,各測定器の位置分解能($x$方向)は同じ値$\sigma$を持つと仮定して,サジッタ$S$の誤差$\delta S$を$\sigma$の関数で表せ。
    
%%%% d %%%%
    \item  以上により,$\delta p/p$を$L,B,p,\sigma$の関数で表せ。\\
    
\end{enumerate}

%%%%%%【3.】%%%%%%
  \item  現実に,マグネティック・スペクトロメータを設計する際に,上記以外の要因も含めて,どのような点に留意せねばならないかを論ぜよ。\\
  
\end{enumerate}

\end{question}

%%%%%【問題7(答)】%%%%%%%%%%%%%%%%%%%%%%%%%%%%%%%%%%%%%%%%%%%%%%%%%%%%%%%

\begin{answer}{問題7}{新原}
\setcounter{equation}{0}



\begin{enumerate}

%%%%%%【1.】%%%%%%
  \item  よりよい運動量分解能を得るには,観測点3点により作られる三角形の外接円の半径を精度良く観測できればよく,これは作られる三角形を大きくすれば誤差が相対的に小さくなり達成することができる。この前提をもとに諸パラメータの大きさをどのようにすればよいかを以下論ずることにする。
  
  \begin{itemize}
    \item ~位置分解能$\sigma$ について:\\
    これは明らかに,三角形の大きさを正確に決めるには小さい方が良く議論するまでも無い。
    \item ~$B,L,p$について:\\
    三角形の辺の長さが大きくなるときの条件としてLが大きく,サジッタ$S$が大きいほうが良いと思われる。そのためR$\gg$Lの時には$R$が小さいほうが良いことになり,
\begin{equation}
R = \frac{p}{0.3B}
\end{equation}
より$B$は大きくしたほうが良く,$p$が小さいときのほうが精度良く測定することができる。

  \end{itemize}
  

%%%%%%【2.】%%%%%%
  \item  
  \begin{enumerate}
    
%%%% a %%%%
    \item  図より$S = R -R\cos\theta$,$\ds \cos\theta \simeq 1 - \frac{1}{2} \left(\frac{L}{R}\right)^2$。以上より
\begin{equation}
S = \frac{L^2}{2R} = 0.15 \frac{BL^2}{p}
\end{equation}
    
    
%%%% b %%%%
    \item  (a)より$\ds p = 0.15 \times \frac{BL^2}{S}$。よって
\begin{equation}
\delta p = 0.15 \times \frac{BL^2}{S^2} \delta S = \frac{p^2}{0.15 \times BL^2} \delta S
\end{equation}
    
%%%% c %%%%
    \item  $\delta S = x_2 - x_1$。よって
\begin{equation}
\delta S = \sqrt{\sigma ^2 + \sigma ^2} 
= \sqrt{2} \sigma
\end{equation}

%%%% d %%%%
    \item  $\ds \frac{\delta p}{p} = \frac{\sqrt{2} \sigma p}{0.15 BL^2}$\\
    
\end{enumerate}

%%%%%%【3.】%%%%%%
  \item  これまで論じてきた測定計画を忠実に再現できれば,上に述べた測定精度で測定することができるが,実際にはこの実験を阻害する要因がある。\\
 次にその改善策を示していきたいと思う。(ただしここでは粒子の種類の同定などの機構については考えず,$p$を精度良く測定することについて論ずることにする。)\\
 この実験では$(1)$式が成り立つという仮定のもとに考えており,換言すれば粒子がこの式に従って運動すれば理想通りに行くものと思われる。そのためには

\begin{itemize}
    \item 測定器1〜3の間では運動量一定
    \item 測定器のなかでは磁場一定で電場は存在しない(ドリフトを起こさせないため)
\end{itemize}
の二つの条件を満たすようにすることが重要になり,
\begin{itemize}
    \item 装置内は十分な真空度を保つ
    \item 乾板を薄くする
    \item 磁場の大きさが変化しない程度に装置を小さくするか,磁場を一様に保つ工夫を考える
\end{itemize}
といったことに留意すべきであると思われる。\\

\end{enumerate}


\end{answer}

\end{document}

\vspace{-0.5cm}
\end{center}
\end{figure}
%%%%%%%%%%%%%%%%%%%%%%%%%%%%%%%%%%%%%%%%%%%%%%%%%%%%%%%%%%%%


\begin{enumerate}

%%%%%%【1.】%%%%%%
  \item  よりよい運動量分解能を得るためには強い磁場,測定器のより広い間隔,よりよい位置分解能が必要であることを直観的に説明せよ。また,運動量が小さい場合と大きい場合とどちらが運動量分解能が高いか答えよ。\\
  
  
%%%%%%【2.】%%%%%%
  \item  以上の傾向を定量的に調べるために以下の設問に答えよ。なおここでは簡略化のため,荷電粒子は$x$-$y$面内に入射し,円運動の中心は図のように2の測定器面上にあるとする。まず,図の$S$をサジッタと呼ぶ。\\
  
    \begin{enumerate}
    
%%%% a %%%%
    \item  サジッタ$S$を$L$と$R$の関数で表せ。$L\ll R$であることを考慮して近似を用いる%% 原文では「もちいる」と平仮名。ここも統一のため漢字にした。
    こと。さらに,$(1)$式を用いてサジッタ$S$を$L$と$p$の関数で表せ。
    
%%%% b %%%%
    \item  サジッタに誤差$\delta S$があると考え,$\delta p$を$L,B,p,\delta S$の関数で表せ。
    
%%%% c %%%%
    \item  この誤差の原因は測定器による誤差である。測定器1,2および3は,等しい$z$座標を持つ%% 原文では「もつ」と平仮名。
    面上において$x$方向のみ測定するとし,各測定器の位置分解能($x$方向)は同じ値$\sigma$を持つと仮定して,サジッタ$S$の誤差$\delta S$を$\sigma$の関数で表せ。
    
%%%% d %%%%
    \item  以上により,$\delta p/p$を$L,B,p,\sigma$の関数で表せ。\\
    
\end{enumerate}

%%%%%%【3.】%%%%%%
  \item  現実に,マグネティック・スペクトロメータを設計する際に,上記以外の要因も含めて,どのような点に留意せねばならないかを論ぜよ。\\
  
\end{enumerate}

\end{question}

%%%%%【問題7(答)】%%%%%%%%%%%%%%%%%%%%%%%%%%%%%%%%%%%%%%%%%%%%%%%%%%%%%%%

\begin{answer}{問題7}{新原}
\setcounter{equation}{0}



\begin{enumerate}

%%%%%%【1.】%%%%%%
  \item  よりよい運動量分解能を得るには,観測点3点により作られる三角形の外接円の半径を精度良く観測できればよく,これは作られる三角形を大きくすれば誤差が相対的に小さくなり達成することができる。この前提をもとに諸パラメータの大きさをどのようにすればよいかを以下論ずることにする。
  
  \begin{itemize}
    \item ~位置分解能$\sigma$ について:\\
    これは明らかに,三角形の大きさを正確に決めるには小さい方が良く議論するまでも無い。
    \item ~$B,L,p$について:\\
    三角形の辺の長さが大きくなるときの条件としてLが大きく,サジッタ$S$が大きいほうが良いと思われる。そのためR$\gg$Lの時には$R$が小さいほうが良いことになり,
\begin{equation}
R = \frac{p}{0.3B}
\end{equation}
より$B$は大きくしたほうが良く,$p$が小さいときのほうが精度良く測定することができる。

  \end{itemize}
  

%%%%%%【2.】%%%%%%
  \item  
  \begin{enumerate}
    
%%%% a %%%%
    \item  図より$S = R -R\cos\theta$,$\ds \cos\theta \simeq 1 - \frac{1}{2} \left(\frac{L}{R}\right)^2$。以上より
\begin{equation}
S = \frac{L^2}{2R} = 0.15 \frac{BL^2}{p}
\end{equation}
    
    
%%%% b %%%%
    \item  (a)より$\ds p = 0.15 \times \frac{BL^2}{S}$。よって
\begin{equation}
\delta p = 0.15 \times \frac{BL^2}{S^2} \delta S = \frac{p^2}{0.15 \times BL^2} \delta S
\end{equation}
    
%%%% c %%%%
    \item  $\delta S = x_2 - x_1$。よって
\begin{equation}
\delta S = \sqrt{\sigma ^2 + \sigma ^2} 
= \sqrt{2} \sigma
\end{equation}

%%%% d %%%%
    \item  $\ds \frac{\delta p}{p} = \frac{\sqrt{2} \sigma p}{0.15 BL^2}$\\
    
\end{enumerate}

%%%%%%【3.】%%%%%%
  \item  これまで論じてきた測定計画を忠実に再現できれば,上に述べた測定精度で測定することができるが,実際にはこの実験を阻害する要因がある。\\
 次にその改善策を示していきたいと思う。(ただしここでは粒子の種類の同定などの機構については考えず,$p$を精度良く測定することについて論ずることにする。)\\
 この実験では$(1)$式が成り立つという仮定のもとに考えており,換言すれば粒子がこの式に従って運動すれば理想通りに行くものと思われる。そのためには

\begin{itemize}
    \item 測定器1〜3の間では運動量一定
    \item 測定器のなかでは磁場一定で電場は存在しない(ドリフトを起こさせないため)
\end{itemize}
の二つの条件を満たすようにすることが重要になり,
\begin{itemize}
    \item 装置内は十分な真空度を保つ
    \item 乾板を薄くする
    \item 磁場の大きさが変化しない程度に装置を小さくするか,磁場を一様に保つ工夫を考える
\end{itemize}
といったことに留意すべきであると思われる。\\

\end{enumerate}


\end{answer}

\end{document}
