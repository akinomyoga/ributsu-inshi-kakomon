\documentclass[fleqn]{jbook}
\usepackage{physpub}
\def\bm{\boldsymbol}
\def\ds{\displaystyle}
\def\B{\mathrm B}

\begin{document}

%%%%%【問題2(問)】%%%%%%%%%%%%%%%%%%%%%%%%%%%%%%%%%%%%%%%%%%%%%%%%%%%%%%%

\begin{question}{問題2}{藤岡}
\setcounter{equation}{0}

 図のように,矢印で表される要素$N$個が$x$方向に繋がった鎖状分子を考える。各要素はそれぞれ長さ$b$で,矢印の先端に電荷$Q~(>0)$をもつ。各要素は,矢印が$+x$方向に向くか,$-x$方向に向くかの二つの状態のみ取りうるものとする。これを,$i$番目要素の状態変数$\mu_i$を導入し,$\mu_i=+1$または$-1$で表す。1番目要素の矢印の始点が点O~$(x=0)$に固定されていて,大きさ$E$の電場$(E\geq 0)$が$+x$の向きにかけられているものとする。また,この鎖状分子は温度$T$の熱浴に接しているものとする。\\
 電荷は外部電場$E$とだけ相互作用し,電荷間の相互作用は無視できるものとする。また,$x$と垂直方向への分子の広がりは無視できるものとし,ボルツマン定数を$k_\B$として,以下の問に,解答に至る道筋を添えて答えよ。

%%%%%%%%%%%%%%%%%%%%%%%%%%%%%%%%%%%%%%%%%%%%%%%%%%%%%%%%%%%%
\begin{figure}[hbt]
\begin{center}
%\caption[]{\ilabel{}}
\vspace{0.5cm}
\input{2001phy2-2.tpc}
\vfill
\end{center}
\end{figure}
%%%%%%%%%%%%%%%%%%%%%%%%%%%%%%%%%%%%%%%%%%%%%%%%%%%%%%%%%%%%

\begin{enumerate}
    \item  まず,$E=0$の場合を考える。\\
%% 原文にはないが,ここで一行空けてある。

    \begin{enumerate}
        \item  分子の端点の位置$x_N$の関数としてこの分子のエントロピー$S$を求めよ。ただし,$L=bN$とし,$N,x_N/b\gg 1$であるとして,必要であればStirlingの公式,すなわち,$x\gg 1$のとき
        $$
        \log x!\sim x\log x-x+\dots
        $$
        を用いよ。\\
        
        \item  端点位置を$x_N$に保つために,この端点に加えなければならない力$X$と$x_N$との関係を導け。\\
        
        \item  $N\gg x_N/b~(\gg 1)$の場合,$X$と$x_N$との関係式は近似的にどのように与えられるか。$x_N$を外力$X$に対する分子の伸びとみたとき,この分子はどのような弾性体と言えるか。\\
\end{enumerate}

    \item  次に,$E>0$の場合を考える。\\
    
    \begin{enumerate}
        \item  $E$が与えられたもとでのこの鎖状分子の分配関数$Z_N$を,変数$A\equiv bQE/k_\B T$を用いて表せ。\\
        
        \item  $i$番目要素の$\mu_i$の平均値$\langle \mu_i\rangle$を求めよ。\\
        
        \item  この鎖状分子の長さの平均値$\langle x_N\rangle$について,電場が十分小さい極限,$NA\ll 1$,での表式を求めよ。$\langle x_N\rangle$を外力の総和$NQE$に対する分子の伸びとみたとき,前問1~(c)との関連を論ぜよ。\\
\end{enumerate}

\end{enumerate}



\end{question}

%%%%%【問題2(答)】%%%%%%%%%%%%%%%%%%%%%%%%%%%%%%%%%%%%%%%%%%%%%%%%%%%%%%%

\begin{answer}{問題2}{湯淺吉晴}
\setcounter{equation}{0}

$+x$方向を向く要素の個数を$N_+$,$-x$方向を向く要素の個数を$N_-$とする。
\begin{align}
N &= N_+ + N_- , & x_N &= b(N_+ - N_-) 
\end{align}
\begin{align}
より\quad N_+ = \frac{bN + x_N}{2b} &= \frac{N}{2}\left(1 + \frac{x_N}{L}\right), & N_+ = \frac{bN - x_N}{2b} &= \frac{N}{2}\left(1 - \frac{x_N}{L}\right)
\end{align}
\vspace{1mm}
\begin{enumerate}
    \item  $E = 0$の場合

    \begin{enumerate}
        \item  $N,\;|x_N| / b \gg 1$より$N_+,\;N_- \gg 1$としてStirlingの公式を用いると
\begin{align}
S &= k_B\log W, \qquad W = \frac{N!}{N_+!\,N_-!} &&  \\
S &= k_B\left(\log N!\, - \log N_+!\, - \log N_-!\,\right) && \nonumber \\
  &\simeq k_B\left(N\log N - N - N_+\log N_+ + N_+ - N_-\log N_- + N_-\right) && \nonumber \\
  &= k_B\left(N\log N - N_+\log N_+ - N_-\log N_- \right) && \nonumber \\
  &= k_B\left[N\log N - \frac{N}{2}\left(1 + \frac{x_N}{L}\right)\log \left\{\frac{N}{2}\left(1 + \frac{x_N}{L}\right)\right\} - \frac{N}{2}\left(1 - \frac{x_N}{L}\right)\log \left\{\frac{N}{2}\left(1 - \frac{x_N}{L}\right)\right\} \right] && \nonumber \\
  \begin{split}
  &= k_B\left[N\log N - \frac{N}{2}\left(1 + \frac{x_N}{L}\right)\log \frac{N}{2} - \frac{N}{2}\left(1 + \frac{x_N}{L}\right)\log \left(1 + \frac{x_N}{L}\right)\right.  \\
  &\qquad\qquad\qquad\;\; \left. - \,\frac{N}{2}\left(1 - \frac{x_N}{L}\right)\log \frac{N}{2} - \frac{N}{2}\left(1 - \frac{x_N}{L}\right)\log \left(1 - \frac{x_N}{L}\right) \right]
  \end{split}
  \nonumber && \\
  &= k_BN\left[\log 2 - \frac{1}{2}\left(1 + \frac{x_N}{L}\right)\log \left(1 + \frac{x_N}{L}\right) - \frac{1}{2}\left(1 - \frac{x_N}{L}\right)\log \left(1 - \frac{x_N}{L}\right)\right] && 
\end{align}
        \vspace{1mm}
        \item  $E = 0$なので,内部エネルギー$U$は分子の形状によらない。Helmholtzの自由エネルギー$F = U - TS$,$dF = -SdT + Xdx_N$より外力$X$を求めると次のようになる。
\begin{align}
X = \left.\frac{\partial F}{\partial x_N}\right|_T 
 &= - T\left.\frac{\partial S}{\partial x_N}\right|_T  \\
 &= k_BTN\left[\frac{1}{2L}\log \left(1 + \frac{x_N}{L}\right) + \frac{1}{2L} - \frac{1}{2L}\log \left(1 - \frac{x_N}{L}\right) - \frac{1}{2L}\right] \nonumber \\
 &= \frac{k_BT}{2b}\log \dfrac{1 + x_N / L}{1 - x_N / L}
\end{align}
        
        \item  $x_N / L \ll 1$のとき
\begin{align}
X &= \frac{k_BT}{2b}\left[\log \left(1 + \frac{x_N}{L}\right) - \log \left(1 - \frac{x_N}{L}\right)\right] \simeq \frac{k_BT}{2b}\left(\frac{x_N}{L} + \frac{x_N}{L}\right) = \frac{k_BT}{Nb^2}x_N \ilabel{1c}
\end{align}
関係式$(7)$はこの分子がHookeの法則(外力$X$が変位$x_N$に比例する)に従うゴム状の弾性体であることを表している。\\
\end{enumerate}

    \item  $E > 0$の場合
    
    \begin{enumerate}
        \item  $x_i = bm_i$とする。
\begin{equation}
\sum_{i=1}^{N}m_i = \sum_{i=1}^{N}\sum_{j=1}^{i}\mu_j = \sum_{i=1}^{N}(N + 1 - i)\mu_i
\end{equation}
より,エネルギーを$H( = U)$として
\begin{equation}
-\beta H = \frac{bQE}{k_BT}\sum_{i=1}^{N}m_i = A\sum_{i=1}^{N}(N + 1 - i)\mu_i
\end{equation}
したがって,分配関数$Z_N$は
\begin{align}
Z_N &= \sum_{\mu_i = \pm 1}e^{- \beta H} \\
 &= \bigl(e^{NA} + e^{- NA}\bigr)\bigl(e^{(N - 1)A} + e^{- (N - 1)A}\bigr)\cdots \bigl(e^{A} + e^{- A}\bigr) \nonumber \\
 &= \prod_{l = 1}^{N}\left(e^{lA} + e^{- lA}\right) \ilabel{z1}
\end{align}
        
        \item  $(11)$より$\mu_i$の平均値は
\begin{equation}
\left\langle \mu_i \right\rangle = \dfrac{e^{A(N + 1 - i)} - e^{- A(N + 1 - i)}}{e^{A(N + 1 - i)} + e^{- A(N + 1 - i)}} = \tanh A(N + 1 - i)
\end{equation}
となる。\\
        
        \item  $|x| \ll 1$のとき,$\tanh x \simeq x - x^3/3$なので,電場が十分小さい極限$NA \ll 1$で
\begin{align}
\left\langle x_N \right\rangle &= b\biggl\langle \sum_{i = 1}^{N}\mu_i \biggr\rangle = b\sum_{i = 1}^{N}\left\langle \mu_i \right\rangle = b\sum_{i = 1}^{N}\tanh A(N + 1 - i) \nonumber \\
 &\simeq b\sum_{i = 1}^{N}A(N + 1 - i) = \frac12N(N + 1)Ab = \frac{(N + 1)b^2}{2k_BT}(NQE) \\
NQE &\simeq \frac{2k_BT}{Nb^2}\left\langle x_N \right\rangle \qquad \text{($N \gg 1$)}
\ilabel{2c}
\end{align}
$(14)$の外力$NQE$は微視的な力の合力である。一方,$(7)$の外力$X$は,$dF = -SdT + Xdx_N$で定まる巨視的な熱,統計力学的力である。$(7)$と$(14)$を比較すると,外力の総和$NQE$は端点に働く$X = NQE/2$の張力に相当することが分かる。$1/2$の因子がついたのは,始点に近い要素$i$の電荷に働く力が,実効的な張力$X$にほとんど寄与しないためである。 
\end{enumerate}
%%
\paragraph{補足}前問1~(c)~$E = 0$で,一定の外力$X$が端点に働いている場合,$X$一定のカノニカル分布(このとき,エネルギー$H$($U$)は一定なので考えなくて良い)により,鎖状分子の長さ$\left\langle x_N \right\rangle$を求めると次のようになる。
\begin{align}
Z_N &= \sum_{x_N}We^{\beta Xx_N} \qquad (\text{$W$は端点位置が$x_N$となる状態数})\nonumber \\
 &= \sum_{\mu_i = \pm 1}\exp \biggl(\beta Xb\sum_i\mu_i\biggr) = {\left(e^{\beta Xb} + e^{-\beta Xb}\right)}^N = 2^N{\left(\cosh \beta Xb\right)}^N \\
\left\langle x_N \right\rangle &= \frac{1}{\beta}\frac{\partial \log Z_N}{\partial X} = Nb\tanh (\beta Xb) \simeq \frac{Nb^2}{k_BT}X \ilabel{a1}
\end{align}
$(16)$の最後で$\beta Xb \ll 1$とした。これは1(c)の$N \gg x_N/b$に対応する。よって
\begin{equation}
X = \frac{k_BT}{Nb^2}\left\langle x_N \right\rangle
\ilabel{a2}
\end{equation}
となり,$(7)$と同じ結果が得られる。\\
\end{enumerate}










\end{answer}

\end{document}
