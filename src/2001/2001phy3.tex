\documentclass[fleqn]{jbook}
\usepackage{physpub}
\usepackage{graphicx}
\usepackage{amsmath}
\usepackage{amssymb}
\usepackage{multicol}

\def\ds{\displaystyle}
\def\t{\rm t}
\def\Vec#1{\mbox{\boldmath $#1$}}

\begin{document}

%%%%%【問題3(問)】%%%%%%%%%%%%%%%%%%%%%%%%%%%%%%%%%%%%%%%%%%%%%%%%%%%%%%%

\begin{question}{問題3}{藤岡}
\setcounter{equation}{0}


 角振動数$\omega$の単色電磁波が真空から誘電体に入射する場合を考える。Maxwellの方程式は次のように与えられる。
$$
\begin{array}{*{20}l}
   \ds{\Vec{\nabla} \cdot (\epsilon \Vec{E})=0,} 
   &\quad \ds{\Vec{\nabla}\cdot\Vec{B}=0,}  \\
   & & \cr
   \ds{\Vec{\nabla} \times \Vec{E}=-\frac{\partial \Vec{B}}{\partial t},} 
   &\quad \ds{\Vec{\nabla} \times\left(\frac{\Vec{B}}{\mu}\right)=
\frac{\partial(\epsilon \Vec{E})}{\partial t}.}  \\
\end{array}
$$
ここで$\Vec{E}$は電場,$\Vec{B}$は磁場,$\epsilon$は誘電率,$\mu$は透磁率を表す。図のように,3次元空間において$x<0$が真空,$x>0$が誘電体であるとする。真空中の誘電率,透磁率をそれぞれ$\epsilon_0,\mu_0$で表し,誘電体の誘電率は$\epsilon_1(>\epsilon_0)$,透磁率は$\mu_0$で与えられるものとする。電場,磁場はそれぞれ,入射波に対しては$\Vec{E}_0(\Vec{r},t)$,$\Vec{B}_0(\Vec{r},t)$,屈折波に対しては$\Vec{E}_1(\Vec{r},t)$,$\Vec{B}_1(\Vec{r},t)$,反射波に対しては$\Vec{E}_2(\Vec{r},t)$,$\Vec{B}_2(\Vec{r},t)$とする。以下の問に,解答に至る筋道を添えて答えよ。

%%%%%%%%%%%%%%%%%%%%%%%%%%%%%%%%%%%%%%%%%%%%%%%%%%%%%%%%%%%%
\begin{figure}[hbtp]
\begin{center}
%\caption[]{\ilabel{}}
%\vspace{1.0cm}
\input{2001phy3-2.tpc}
\end{center}
\end{figure}
%%%%%%%%%%%%%%%%%%%%%%%%%%%%%%%%%%%%%%%%%%%%%%%%%%%%%%%%%%%%

\begin{enumerate}
    \item 真空と誘電体の境界$x=0$において,
    \begin{eqnarray}
    &&E_{0,\t}(0,y,z,t)+E_{2,\t}(0,y,z,t)=E_{1,\t}(0,y,z,t)\\
    &&\epsilon_0[E_{0,x}(0,y,z,t)+E_{2,x}(0,y,z,t)]=\epsilon_1E_{1,x}(0,y,z,t)\\
    &&B_{0,\t}(0,y,z,t)+B_{2,\t}(0,y,z,t)=B_{1,\t}(0,y,z,t)
    \end{eqnarray}
    が成立する。ここで添字の$\t$は各ベクトルの真空と誘電体の境界に平行な成分,添字の$x$は$x$成分を表す。\\

    \begin{enumerate}
        \item  $(1)$式が成立することを示せ。\\
        
        \item  誘導電荷密度$\rho$と電場$\Vec{E}$の間には関係式$\epsilon_0\Vec{\nabla}\cdot \Vec{E}=\rho$が成立する。誘電体の表面($x=+0$)における電荷の面密度$\sigma(y,z,t)$を$E_{1,x}(0,y,z,t)$,$\epsilon_0$,$\epsilon_1$を用いて表せ。\\
    \end{enumerate}
    
    \item  入射波の方向が$xy$面にあるように座標軸をとり,$x,y,z$軸方向の単位ベクトルをそれぞれ$\hat{\Vec{x}},\hat{\Vec{y}},\hat{\Vec{z}}$で表す。入射波の進行方向の単位ベクトルを$\Vec{n}_0=\hat{\Vec{x}}\cos\theta_0+\hat{\Vec{y}}\sin\theta_0~(0<\theta_0<\pi/2)$とし,入射波の磁場$\Vec{B}_0(\Vec{r},t)$が$z$方向に直線偏光している場合,
    $$
    \Vec{B}_0(\Vec{r},t)=\hat{\Vec{z}}B_0\exp[i(k_0\Vec{n}_0\cdot\Vec{r}-\omega t)]
    $$
    を考える。入射波の電場は,
    $$
    \Vec{E}_0(\Vec{r},t)=\Vec{E}_0\exp[i(k_0\Vec{n}_0\cdot\Vec{r}-\omega t)]
    $$
    と表す。\\

    \begin{enumerate}
        \item  $k_0$を$\omega$,$\epsilon_0$,$\mu_0$を用いて表せ。\\
        
        \item  入射波の電場の振幅$\Vec{E}_0$を$B_0$,$\hat{\Vec{z}}$,$k_0$,$\Vec{n}_0$,$\omega$を用いて表せ。\\
        
    \end{enumerate}
    
    \item  屈折波,反射波の磁場をそれぞれ,
    \begin{eqnarray}
    &&\Vec{B}_1(\Vec{r},t)=\hat{\Vec{z}}B_1\exp[i(k_0\Vec{n}_1\cdot\Vec{r}-\omega t)] ,\cr
    &&\Vec{B}_2(\Vec{r},t)=\hat{\Vec{z}}B_2\exp[i(k_0\Vec{n}_2\cdot\Vec{r}-\omega t)] \nonumber
    \end{eqnarray}
    と表す。$\Vec{n}_1$,$\Vec{n}_2$はそれぞれ屈折波,反射波の進行方向の単位ベクトルであり,これらを
    \begin{eqnarray}
    &&\Vec{n}_1=\hat{\Vec{x}}\cos\theta_1+\hat{\Vec{y}}\sin\theta_1 ,\cr
    &&\Vec{n}_2=-\hat{\Vec{x}}\cos\theta_0+\hat{\Vec{y}}\sin\theta_0 \nonumber
    \end{eqnarray}
    と表す。\\
    
    \begin{enumerate}

        \item  (3)式を用いて入射角$\theta_0$と屈折角$\theta_1$の関係式を求めよ。\\
        
        \item  (1)式と(3)式を用いて$B_1$,$B_2$を求めよ。答は$k_0$,$k_1$,$\theta_0$,$\theta_1$,$B_0$を用いて表せ。\\
%% 原文では「答え」となっているが,「答」に直した。

        \item   $\theta_0+\theta_1=\pi/2$のとき,何が起こるか。\\
\end{enumerate}
\end{enumerate}

\end{question}

%%%%%【問題3(答)】%%%%%%%%%%%%%%%%%%%%%%%%%%%%%%%%%%%%%%%%%%%%%%%%%%%%%%%

\begin{answer}{問題3}{湯淺吉晴}
\setcounter{equation}{0}

真電荷と伝導電流のない等方一様な媒質を考えているので,Maxwell方程式は次のようになる。
%\begin{multicols}{2}
\begin{align}
\nabla \cdot (\epsilon \boldsymbol{E}) &= 0 \ilabel{m1} \\
\nabla \cdot \boldsymbol{B} &= 0 \ilabel{m2} \\
\nabla \times \boldsymbol{E} &= -\frac{\partial \boldsymbol{B}}{\partial t} \ilabel{m3} \\
\nabla \times \left(\frac{\boldsymbol{B}}{\mu}\right) &= \frac{\partial \,(\epsilon \boldsymbol{E})}{\partial t} \ilabel{m4}
\end{align}

\begin{align}
\text{誘電率} && \epsilon &= 
\begin{cases}
\epsilon_1\,( > \epsilon_0) & \text{誘電体}x > 0 \\
\epsilon_0 & \text{真空}x < 0
\end{cases} \nonumber \\
\text{透磁率} && \mu &= \mu_0 \nonumber
\end{align}
%\end{multicols}
以下,真空($x < 0$)の電場,磁場を$\boldsymbol{E}_v$,$\boldsymbol{B}_v$,誘電体($x > 0$)の電場,磁場を$\boldsymbol{E}_d$,$\boldsymbol{B}_d$で表す。
\begin{enumerate}
%%%%%%【1.】%%%%%%
\item  
%%%%%%%%%%%%%%%%%%%%%%%%%%%%%%%%%%%%%%%%%%%%%%%%%%%%%%%%%%%%
\vspace{-1.0cm}
\begin{figure}[hbtp]
\begin{center}
\caption[境界面$x = 0$における電磁場の接線成分(左),法線成分(右)の連続性]{
境界面$x = 0$における電磁場の接線成分(左),法線成分(右)の連続性\ilabel{fig1}}
\vspace{2mm}
\input{2001phy3-1.tpc}
\end{center}
\end{figure}
%%%%%%%%%%%%%%%%%%%%%%%%%%%%%%%%%%%%%%%%%%%%%%%%%%%%%%%%%%%%

\begin{enumerate}


  %%%%%% 1.-(a) %%%%%%
\item  境界面に平行な任意の単位ベクトルを$\hat{\boldsymbol t}$とし,$x$軸方向(法線方向)の単位ベクトルを$\hat{\boldsymbol x}$とする。図\iref{fig1}(左)のような長方形の積分路ABCD($\overline{\text{AB}} = \Delta l$,$\overline{\text{BC}} = \delta$)をとり,これを縁とする平面$S$上で(3)の両辺を面積分する。
\begin{equation}
\int_{S}\left(\nabla \times \boldsymbol{E}\right)\cdot d\boldsymbol{S} = -\frac{\partial}{\partial t}\int_{S}\boldsymbol{B}\cdot d\boldsymbol{S}   
\ilabel{bc1}
\end{equation}
磁場$\boldsymbol{B}$は有限なので,$\delta \to 0$とすると$\text{(右辺)} \to 0$となる。(左辺)にStokesの定理を適用すると
\begin{align}
\int_{S}\left(\nabla \times \boldsymbol{E}\right)\cdot d\boldsymbol{S}
 &= \oint_{\text{ABCD}}\boldsymbol{E}\cdot d\boldsymbol{l} && \text{Stokesの定理} \nonumber \\
 &\to \int_{\text{AB}}\boldsymbol{E}\cdot d\boldsymbol{l} + \int_{\text{CD}}\boldsymbol{E}\cdot d\boldsymbol{l} && (\delta \to 0) \nonumber \\
 &= (E_{d,t} - E_{v,t})\Delta l = 0 && \nonumber \\
\therefore E_{v,t}(0,\,y,\,z,\,t) &= E_{d,t}(0,\,y,\,z,\,t) && (\Delta l \to 0) \ilabel{bc12}
\end{align}
(4)についても同様にして
\begin{align}
\frac{1}{\mu_0}\int_{S}\left(\nabla \times \boldsymbol{B}\right)\cdot d\boldsymbol{S}
 &= \frac{\partial}{\partial t}\int_{S}\epsilon \boldsymbol{E}\cdot d\boldsymbol{S} \to 0 \qquad (\delta \to 0) \nonumber \\
\oint_{\text{ABCD}}\boldsymbol{B}\cdot d\boldsymbol{l}
 &\to (B_{d,t} - B_{v,t})\Delta l = 0 \ilabel{bc2}
\end{align}
次に,図\iref{fig1}(右)のような底面積$\Delta S$,高さ$\delta$の円柱$V$を考え,その表面を$S$とする。(1)の両辺を$V$で積分するとGaussの定理より
\begin{align}
\int_{V}\nabla \cdot (\epsilon \boldsymbol{E}) dV
 &= \int_{S}\epsilon \boldsymbol{E}\cdot d\boldsymbol{S} && \text{Gaussの定理} \ilabel{bc31} \\
 &\to (\epsilon_1E_{d,x} - \epsilon_0E_{v,x})\Delta S = 0 && (\delta \to 0) \nonumber \\
\therefore \epsilon_0E_{v,x}(0,\,y,\,z,\,t) &= \epsilon_1E_{d,x}(0,\,y,\,z,\,t) && (\Delta S \to 0) \ilabel{bc3}
\end{align}
以上(6),(9),(7)より境界$x = 0$において,(10),(11),(12)がそれぞれ成り立つ。
\begin{gather}
E_{0,t}(0,\,y,\,z,\,t) + E_{2,t}(0,\,y,\,z,\,t) = E_{1,t}(0,\,y,\,z,\,t) \ilabel{bc101} \\
\epsilon_0\left[E_{0,x}(0,\,y,\,z,\,t) + E_{2,x}(0,\,y,\,z,\,t)\right] = \epsilon_1E_{1,x}(0,\,y,\,z,\,t) \ilabel{bc102} \\
B_{0,t}(0,\,y,\,z,\,t) + B_{2,t}(0,\,y,\,z,\,t) = B_{1,t}(0,\,y,\,z,\,t) \ilabel{bc103}
\end{gather}

  %%%%%% 1.-(b) %%%%%%
\item 
\begin{equation}
\epsilon_0\nabla \cdot \boldsymbol{E} = \rho 
\ilabel{m0}
\end{equation}
(13)を(8)$\sim$(9)と同様にして$V$で積分すると
\begin{equation}
\epsilon_0E_{1,x}(0,\,y,\,z,\,t) - \epsilon_0\left[E_{0,x}(0,\,y,\,z,\,t) + E_{2,x}(0,\,y,\,z,\,t)\right] = \sigma (y,\,z,\,t)
\end{equation}
境界条件(11)より
\begin{equation}
\sigma (y,\,z,\,t) = (\epsilon_0 - \epsilon_1)E_{1,x}(0,\,y,\,z,\,t)
\end{equation}

\end{enumerate}

%%%%%%【2.】%%%%%%
\item  各波数成分について独立にMaxwell方程式は成り立つ。真空中$x < 0$では
\begin{gather}
\boldsymbol{k}\cdot \boldsymbol{E}(\boldsymbol{k},\omega) = 0,\qquad \boldsymbol{k}\cdot \boldsymbol{B}(\boldsymbol{k},\omega) = 0 \ilabel{mk2} \\
\boldsymbol{k}\times \boldsymbol{E}(\boldsymbol{k},\omega) = \omega\boldsymbol{B}(\boldsymbol{k},\omega),\qquad \boldsymbol{k}\times \boldsymbol{B}(\boldsymbol{k},\omega) = -\epsilon_0\mu_0\omega\boldsymbol{E}(\boldsymbol{k},\omega) \ilabel{mk4}
\end{gather}
となる。

\begin{enumerate}

  %%%%%% 2.-(a) %%%%%%
\item $k_0\boldsymbol{n}_0$成分($\boldsymbol{E}_0$)について考えると(17)より
\begin{align}
k_0\boldsymbol{n}_0 \times \left(k_0\boldsymbol{n}_0 \times \boldsymbol{E}_0\right) &= -\epsilon_0\mu_0{\omega}^2\boldsymbol{E}_0 \\
(k_0\boldsymbol{n}_0\cdot \boldsymbol{E}_0)k_0\boldsymbol{n}_0 - (k_0\boldsymbol{n}_0\cdot k_0\boldsymbol{n}_0)\boldsymbol{E}_0 &= -\epsilon_0\mu_0{\omega}^2\boldsymbol{E}_0 \ilabel{k02}
\end{align}
(19)の(左辺)第1項は(16)より0なので
\begin{align}
{k_0}^2 &= \epsilon_0\mu_0{\omega}^2 &
\therefore k_0 &= \omega\sqrt{\epsilon_0\mu_0} \ilabel{k03}
\end{align}

  %%%%%% 2.-(b) %%%%%%
\item  (17),(20)より
\begin{equation}
\boldsymbol{E}_0 = \frac{- k_0B_0}{\epsilon_0\mu_0\omega}\boldsymbol{n}_0 \times \hat{\boldsymbol{z}} = \frac{\omega B_0}{k_0}\hat{\boldsymbol{z}}\times \boldsymbol{n}_0
\ilabel{k04}
\end{equation}
\end{enumerate}

%%%%%%【3.】%%%%%%
\item  屈折波と反射波の電場振幅$\boldsymbol{E}_1$,$\boldsymbol{E}_2$についても(21)と同様にして
\begin{align}
\boldsymbol{E}_1 &= \frac{\omega B_1}{k_1}\hat{\boldsymbol{z}}\times \boldsymbol{n}_1, & \boldsymbol{E}_2 &= \frac{\omega B_2}{k_0}\hat{\boldsymbol{z}}\times \boldsymbol{n}_2
\ilabel{f2}
\end{align}
が成り立つ。
\begin{enumerate}

  %%%%%% 3.-(a) %%%%%%
\item  (12)より
\begin{equation}
B_0\exp \,[i(k_0\boldsymbol{n}_0 \cdot \boldsymbol{r} - \omega t)] + B_2\exp \,[i(k_0\boldsymbol{n}_2 \cdot \boldsymbol{r} - \omega t)] = B_1\exp \,[i(k_1\boldsymbol{n}_1 \cdot \boldsymbol{r} - \omega t)]
\ilabel{s1}
\end{equation}
これが任意の$y,\,z,\,t$で成り立つためには
\begin{align}
k_0n_{0,y} = k_0n_{2,y} &= k_1n_{1,y} &&\Longrightarrow &&k_0\sin \theta_0 = k_1\sin \theta_1
\ilabel{s2}
\end{align}

  %%%%%% 3.-(b) %%%%%%
\item 
\begin{align}
&\text{(23),(24)より} && B_0 + B_2 = B_1 \ilabel{f3} \\
&\text{(10),(22),(24)より} && \frac{\omega B_0}{k_0}\cos \theta_0 - \frac{\omega B_2}{k_0}\cos \theta_0 = \frac{\omega B_1}{k_1}\cos \theta_1 \ilabel{f4}
\end{align}
(25),(26)を解いて
\begin{align}
B_1 &= \frac{2k_1\cos \theta_0}{k_1\cos \theta_0 + k_0\cos \theta_1}B_0, & B_2 &= \frac{k_1\cos \theta_0 - k_0\cos \theta_1}{k_1\cos \theta_0 + k_0\cos \theta_1}B_0 \ilabel{f5}
\end{align}

  %%%%%% 3.-(c) %%%%%%
\item  $\theta_0 + \theta_1 = \pi/2$のとき,$\sin \theta_0 = \cos \theta_1,\;\sin \theta_1 = \cos \theta_0$が成り立つ。(24)を用いると(27)は
\begin{align}
\frac{B_1}{B_0} &= \frac{2\cos \theta_0\sin \theta_0}{\cos \theta_0\sin \theta_0 + \sin \theta_1\cos \theta_1} = 1 && (\because \cos \theta_0\sin \theta_0 = \sin \theta_1\cos \theta_1) \\
&\cr
\frac{B_2}{B_0} &= \frac{\sin \theta_0\cos \theta_0 - \sin \theta_1\cos \theta_1}{\sin \theta_0\cos \theta_0 + \sin \theta_1\cos \theta_1} = 0 &&
\end{align}
となる。また,電場振幅は(21),(22)より
\begin{align}
\frac{k_1E_1}{k_0E_0} &= \frac{B_1}{B_0} = 1, & \frac{E_2}{E_0} &= \frac{B_2}{B_0} = 0
\end{align}
となる。したがって,磁場が入射面($xy$平面)に垂直に偏光している場合(このとき電場は入射面と平行),$\theta_0 + \theta_1 = \pi/2$で反射がなくなる。このときの入射角$\theta_0 = \theta_B$(Brewster角)。\\
\end{enumerate}
\end{enumerate}


\end{answer}

\end{document}
