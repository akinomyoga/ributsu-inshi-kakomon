\documentclass[fleqn]{jbook}
\usepackage{physpub}
\def\bm{\boldsymbol}
\def\F{{\rm F}}
\def\L{{\rm L}}
\def\R{{\rm R}}
\def\ds{\displaystyle}



%% Defined by 

\begin{document}

%%%%%【問題8(問)】%%%%%%%%%%%%%%%%%%%%%%%%%%%%%%%%%%%%%%%%%%%%%%%%%%%%%%%
%\'ecrit par Ryo Suzuki 2002.3.9.
%minipage environnement arrang\'e pour a4.
%alignez les phrases soulign\'e quand vous compilez.

\begin{question}{問題8}{鈴木了}
\setcounter{equation}{0}


\begin{enumerate}

%%%%%%【1.】%%%%%%
  \item  図(a)のような幅$L$,高さ$U_0$の箱形ポテンシャル障壁に,左側から入射するエネルギー$E~(U_0>E>0)$を持つ粒子の1次元的な運動を考える。以下の問に,解答に至る道筋を添えて答えよ。なお,入射粒子の波動関数を$e^{i kx}$(ただし,$\hbar^2k^2/2m=E,~k>0$,$m$は粒子の質量)とする。電子間相互作用は
無視してよい。\\

%%%%%%%%%%%%%%%%%%%%%%%%%%%%%%%%%%%%%%%%%%%%%%%%%%%%%%%%%%%%
\begin{figure}[hbt]
\begin{center}
%\caption[]{\ilabel{}}
%\vspace{1.0cm}
\documentclass[fleqn]{jbook}
\usepackage{physpub}
\def\bm{\boldsymbol}
\def\F{{\rm F}}
\def\L{{\rm L}}
\def\R{{\rm R}}
\def\ds{\displaystyle}



%% Defined by 

\begin{document}

%%%%%�y���8�i��j�z%%%%%%%%%%%%%%%%%%%%%%%%%%%%%%%%%%%%%%%%%%%%%%%%%%%%%%%
%\'ecrit par Ryo Suzuki 2002.3.9.
%minipage environnement arrang\'e pour a4.
%alignez les phrases soulign\'e quand vous compilez.

\begin{question}{���8}{��ؗ�}
\setcounter{equation}{0}


\begin{enumerate}

%%%%%%�y1.�z%%%%%%
  \item �@�}(a)�̂悤�ȕ�$L$�C����$U_0$�̔��`�|�e���V������ǂɁC����������˂���G�l���M�[$E~(U_0>E>0)$�����—��q��1�����I�ȉ^�����l����B�ȉ��̖�ɁC�𓚂Ɏ��铹�؂�Y���ē�����B�Ȃ��C���˗��q�̔g���֐���$e^{i kx}$�i�������C$\hbar^2k^2/2m=E,~k>0$�C$m$�͗��q�̎��ʁj�Ƃ���B�d�q�ԑ��ݍ�p��
�������Ă悢�B\\

%%%%%%%%%%%%%%%%%%%%%%%%%%%%%%%%%%%%%%%%%%%%%%%%%%%%%%%%%%%%
\begin{figure}[hbt]
\begin{center}
%\caption[]{\label{}}
%\vspace{1.0cm}
\documentclass[fleqn]{jbook}
\usepackage{physpub}
\def\bm{\boldsymbol}
\def\F{{\rm F}}
\def\L{{\rm L}}
\def\R{{\rm R}}
\def\ds{\displaystyle}



%% Defined by 

\begin{document}

%%%%%�y���8�i��j�z%%%%%%%%%%%%%%%%%%%%%%%%%%%%%%%%%%%%%%%%%%%%%%%%%%%%%%%
%\'ecrit par Ryo Suzuki 2002.3.9.
%minipage environnement arrang\'e pour a4.
%alignez les phrases soulign\'e quand vous compilez.

\begin{question}{���8}{��ؗ�}
\setcounter{equation}{0}


\begin{enumerate}

%%%%%%�y1.�z%%%%%%
  \item �@�}(a)�̂悤�ȕ�$L$�C����$U_0$�̔��`�|�e���V������ǂɁC����������˂���G�l���M�[$E~(U_0>E>0)$�����—��q��1�����I�ȉ^�����l����B�ȉ��̖�ɁC�𓚂Ɏ��铹�؂�Y���ē�����B�Ȃ��C���˗��q�̔g���֐���$e^{i kx}$�i�������C$\hbar^2k^2/2m=E,~k>0$�C$m$�͗��q�̎��ʁj�Ƃ���B�d�q�ԑ��ݍ�p��
�������Ă悢�B\\

%%%%%%%%%%%%%%%%%%%%%%%%%%%%%%%%%%%%%%%%%%%%%%%%%%%%%%%%%%%%
\begin{figure}[hbt]
\begin{center}
%\caption[]{\label{}}
%\vspace{1.0cm}
\documentclass[fleqn]{jbook}
\usepackage{physpub}
\def\bm{\boldsymbol}
\def\F{{\rm F}}
\def\L{{\rm L}}
\def\R{{\rm R}}
\def\ds{\displaystyle}



%% Defined by 

\begin{document}

%%%%%�y���8�i��j�z%%%%%%%%%%%%%%%%%%%%%%%%%%%%%%%%%%%%%%%%%%%%%%%%%%%%%%%
%\'ecrit par Ryo Suzuki 2002.3.9.
%minipage environnement arrang\'e pour a4.
%alignez les phrases soulign\'e quand vous compilez.

\begin{question}{���8}{��ؗ�}
\setcounter{equation}{0}


\begin{enumerate}

%%%%%%�y1.�z%%%%%%
  \item �@�}(a)�̂悤�ȕ�$L$�C����$U_0$�̔��`�|�e���V������ǂɁC����������˂���G�l���M�[$E~(U_0>E>0)$�����—��q��1�����I�ȉ^�����l����B�ȉ��̖�ɁC�𓚂Ɏ��铹�؂�Y���ē�����B�Ȃ��C���˗��q�̔g���֐���$e^{i kx}$�i�������C$\hbar^2k^2/2m=E,~k>0$�C$m$�͗��q�̎��ʁj�Ƃ���B�d�q�ԑ��ݍ�p��
�������Ă悢�B\\

%%%%%%%%%%%%%%%%%%%%%%%%%%%%%%%%%%%%%%%%%%%%%%%%%%%%%%%%%%%%
\begin{figure}[hbt]
\begin{center}
%\caption[]{\label{}}
%\vspace{1.0cm}
\input{2001phy8.tpc}
\vfill
\end{center}
\end{figure}
%%%%%%%%%%%%%%%%%%%%%%%%%%%%%%%%%%%%%%%%%%%%%%%%%%%%%%%%%%%%

  \begin{enumerate}
  
%%%% a %%%%
    \item �@$x<0$�C$0\le x\le L$�C$x>L$ �̊e�̈�ɂ����闱�q�̔g���֐������ꂼ��$\phi_�T(x),\phi_�U(x),\phi_�V(x)$�Ƃ��āC���̈�ʓI�Ȋ֐��`�������B�Ⴆ�΁C$Ae^{i kx}+ ...$�̂悤�ɁC�K���ȌW��$A, ...$ ���g���ċL�q���Ă悢�B\\
    
%%%% b %%%%
    \item �@�֐�$\phi_�T(x),\phi_�U(x),\phi_�V(x)$���������ׂ����E�����������B\\
    
%%%% c %%%%
    \item �@�d��$-e$�����“d�q���E���֓��˂���Ƃ��ɁC��ǂ̓��ߗ���$T$�Ƃ��āC�P�ʎ��ԓ�����ɏ�ǂ𓧉߂���d�q�̓d�ׂ̑��ʁi�d���j$j$�����߂�B�A���C�G�l���M�[$E$�����“d�q�͒P�ʒ���������$n$�̖��x�����‚Ƃ���B\\
    
\end{enumerate}

%%%%%%�y2.�z%%%%%%
  \item �@���̌n�ɓd�q���l�߂��ꍇ�̓d�C�`�����l���悤�B�Ȃ��C�S�d�q�̖��x��$N$�C�d�q�̃X�s���k�d�x��2�Ƃ���B\\
  
    \begin{enumerate}
    
%%%% a %%%%
    \item �@�悸�C1�������R�d�q�C�̂́C�G�l���M�[$E$�ɑ΂���P�ʒ���������̏�Ԗ��x$D(E)$�C�y�ѐ�Η�x�ɂ�����t�F���~���x$v_\F$���C$m$�C$h$�i�v�����N�萔�j�C$N$��p���ĕ\���B\\
    
%%%% b %%%%
    \item �@�}(a)�Ɏ������C��ǂ�����n�ɂ�����G�l���M�[���z��}(b)�̂悤�ɕ\���B���Ȃ킿�C��ǂ̍����ƉE���̓d�q�̃t�F���~�E�G�l���M�[���C���ꂼ��$E_{\F\L},E_{\F\R}$�C���x$T$�ɂ����镪�z�֐������ꂼ��$f_\L(E,T),f_\R(E,T)$�Ƃ���B�Ȃ��C�d�q�C�̂͒ቷ�ŏk�ނ��Ă�����̂Ƃ���B���̂Ƃ��C1�������̂�������E�֗����S�d��$J$�́C��ǂ����Ƃ��č�����E�ɗ��ꍞ�ޓd�q�ɂ��S�d��$J_{\L\R}$�ƁC�E���獶�ɗ��ꍞ�ޓd�q�ɂ��S�d��$J_{\R\L}$�̍�$(J=J_{\L\R}-J_{\R\L})$�Ƃ��ė^������B$J_{\L\R}$��
\begin{equation}
 J_{\L\R}=-\frac{e}{2} \int_0^{\infty}\sqrt{\frac{2E}{m}}D(E)f_\L(E,T)
   [1-f_\R(E,T)]T(E)\d E
\end{equation}
�ŗ^�����邱�Ƃ��������B\\

%%%% c %%%%
    \item �@�}(c)�Ɏ����悤�ɁC��ǂ̗��[�Ԃɋ͂��̓d��$\Delta V$�������č����̓d�q�̃t�F���~�E�G�l���M�[���E�����$\Delta\mu~(\ll E_{\F\L} \approx E_{\F\R})$������������ƁC�S�d����$\Delta J$������������B���x���Η�x�Ƃ���$\Delta J$���v�Z���CG=$\ds \mathop {\lim }\limits_{\Delta V \to 0}\frac{\Delta J}{\Delta V}$�ŗ^������d�C�`���x��
$$
G=\left(\frac{2e^2}{h}\right)T(E_{\F\L})
$$
�ƕ\���邱�Ƃ������B�Ȃ��C$\Delta\mu$��$\Delta V$�̊Ԃɂ�$\Delta\mu=-e\Delta V$�̊֌W������B\\

\end{enumerate}

%%%%%%�y3.�z%%%%%%
  \item �@��L�̂悤�ȓ��̌n����邽�߂ɁC�ޗ��Ƃ���Fe��n�`Si�����������B�������C���ޗ��͌����������Ă���̂ŋ�ʂ��‚��Ȃ��Ȃ��Ă��܂����B�����I�ȑ���ɂ���āC���҂������������BFe�Cn�`Si����ʂ��邽�߂ɍs���ׂ�������2�‹�����B(���R���Y���āC�e40���ȓ��Łj�B�������C��d�C�d�x�C����C���w�����̈Ⴂ�͏����B\\
  
\end{enumerate}


\end{question}

%%%%%�y���8�i���j�z%%%%%%%%%%%%%%%%%%%%%%%%%%%%%%%%%%%%%%%%%%%%%%%%%%%%%%%

\begin{answer}{���8}{�n�Ӂ{��ؗ�}
\setcounter{equation}{0}


\begin{enumerate}

%%%%%%�y1.�z%%%%%%
  \item �@
  
  \begin{enumerate}
  
%%%% a %%%%
    \item �@
\begin{equation}
\begin{cases}
       ~\phi_�T(x) & = \mathrm{e}^{i kx} + A \mathrm{e}^{-i kx} \\
       ~\phi_�U(x) & = B \mathrm{e}^{\rho x} + C \mathrm{e}^{ - \rho x} \\
       ~\phi_�V(x) & = D \mathrm{e}^{i kx} 
\end{cases}
\end{equation}
�ƕ\����B������
\begin{equation}
k = \frac{\sqrt {2mE}}{\hbar} , \quad \rho = \frac{\sqrt {2m(U_0-E)}}{\hbar}\nonumber
\end{equation}
�ł���B\\
    
%%%% b %%%%
    \item �@$x=0$�őΐ����������,
\begin{equation}
k\frac{1-A}{1+A} = \rho\frac{B-C}{B+C}
\end{equation}
$x=L$�őΐ����������,
\begin{equation}
\rho\frac{Be^{\rho L}+Ce^{-\rho L}}{Be^{\rho L}-Ce^{-\rho L}} = i k
\end{equation}
�ȏオ�������ׂ����E�����ł���B\\

%%%% c %%%%
    \item �@�^����ꂽ���ߗ������߂�$T_{\rm r}$�Ƃ���\footnote{�ݖ�2~(b)�ł̉��x$T$�ƕ���킵�����߁B}�C�E���ɓ��߂���Flux��$j_{\rm trans}$ �Ƃ����
\begin{equation*}
j_{\rm trans} = \frac{\hbar k}{m}|D|^2 = \frac{\hbar k}{m}T_{\rm r} = \sqrt{\frac{2E}{m}}T_{\rm r}
\end{equation*}
%�G�l���M�[$E$�����“d�q�̖��x��$n(E)$�Ə����邩��C�S�d����
%\begin{equation}
%j = - e \int _0 ^{\infty} \sqrt{\frac{2E}{m}}n(E) T_{\rm r}(E) \d E \label{j}
%\end{equation}
�d�q�̖��x��$n$�Ȃ̂ŁA�C�S�d����
\begin{equation}
j = - e \sqrt{\frac{2E}{m}}nT_{\rm r}(E) \label{j}
\end{equation}
    
\end{enumerate}

%%%%%%�y2.�z%%%%%%
  \item �@
  
    \begin{enumerate}
    
%%%% a %%%%
    \item �@1�������R�d�q�C�̂̒P�ʒ���������̏�Ԗ��x�́C�X�s�����l�������
\vspace{2mm}
\begin{eqnarray}
\frac{\d p}{h} &=& 2\cdot \frac{1}{h} \sqrt{\frac{m}{2E}}\d E = D(E)\d E  \cr
& & \cr
\therefore D(E) &=& \frac{1}{h} \sqrt{\frac{2m}{E}}
\end{eqnarray}
\vspace{2mm}

��Η�x�ł͓d�q�͏k�ނ��Ă���̂�
\begin{equation}
N = \int _0^{\infty} D(E)f(E)\d E = \int _0^{E_\F} D(E)\d E = \frac{2\sqrt{2mE_\F}}{h} \nonumber
\end{equation}
�����Fermi���x��
\begin{equation}
v_\F = \sqrt{\frac{2E_\F}{m}} = \frac{Nh}{2m}
\end{equation}

    
%%%% b %%%%
    \item �@���̌n���C�d���������ɕۂ��ꂽ����Ԃɂ��邱�Ƃɒ��ӂ���B��$(4)$���C�G�l���M�[$E$������$f_{\L}(E)D(E)\d E$�̓d�q�����̌n����E�̌n�ɒ��I�ɗ��ꍞ�ނ��C���̂���$f_{\R}(E)$�̊����ŁC���ɂ����d�q�͉E�ɂ����d�q�������o�������ł���B�䂦��,
\begin{equation}
J_{\L\R} = -e\int_0^{\infty} \sqrt{\frac{2E}{m}} D(E) 
f_\L(E)\left[1-f_\R(E)\right]T(E)\d E
\end{equation}
���l�̗��ꂪ�E���獶�Ɍ������Ă����݂���̂�,
\begin{equation}
J_{\R\L} = -e\int_0^{\infty} \sqrt{\frac{2E}{m}} D(E)
f_\R(E)\left[1-f_\L(E)\right]T(E)\d E
\end{equation}

%%%% c %%%%
    \item �@
\begin{eqnarray}
\Delta J & =& -e\int_0^{\infty} \sqrt{\frac{2E}{m}} D(E)\left[f_\L(E)-f_\R(E)\right]T(E)\d E \cr
         & & \cr
         & =& -e\int_{E_{\F\R}}^{E_{\F\L}} \sqrt{\frac{2E}{m}} D(E)T(E)\d E \cr
         & & \cr
         & =& -e\int_{E_{\F\R}}^{E_{\F\L}} \sqrt{\frac{2E}{m}} \frac{1}{h} \sqrt{\frac{2m}{E}} T(E)\d E \cr
         & & \cr
         & =& -\frac{2e}{h} \int_{E_{\F\R}}^{E_{\F\L}} T(E)\d E \cr
         & & \cr
         & \approx& -\frac{2e}{h} T(E_{\F\L})\Delta \mu \cr
         & & \cr
         & =& -\frac{2e}{h} T(E_{\F\L})(-e\Delta V)
\end{eqnarray}
�����
\begin{equation}
G = \mathop {\lim }\limits_{\Delta V \to 0} \frac{\Delta J}{\Delta V} = \left(\frac{2e^2}{h}\right)T(E_{\F\L})
\end{equation}
    
\end{enumerate}

%%%%%%�y3.�z%%%%%%
  \item �@
\begin{itemize}
    \item Fe�͋��������������΂ɂ����‚����Cn�`Si�͂����‚��Ȃ��B
    (30��)
    \item Fe�͉��x�㏸�ƂƂ��ɒ�R�����傷�邪�C
    n�`Si�̓L�����A�[���������ߌ�������B(40��)
\end{itemize}
�ȂǁB\\
  
\end{enumerate}

\end{answer}

\end{document}

\vfill
\end{center}
\end{figure}
%%%%%%%%%%%%%%%%%%%%%%%%%%%%%%%%%%%%%%%%%%%%%%%%%%%%%%%%%%%%

  \begin{enumerate}
  
%%%% a %%%%
    \item �@$x<0$�C$0\le x\le L$�C$x>L$ �̊e�̈�ɂ����闱�q�̔g���֐������ꂼ��$\phi_�T(x),\phi_�U(x),\phi_�V(x)$�Ƃ��āC���̈�ʓI�Ȋ֐��`�������B�Ⴆ�΁C$Ae^{i kx}+ ...$�̂悤�ɁC�K���ȌW��$A, ...$ ���g���ċL�q���Ă悢�B\\
    
%%%% b %%%%
    \item �@�֐�$\phi_�T(x),\phi_�U(x),\phi_�V(x)$���������ׂ����E�����������B\\
    
%%%% c %%%%
    \item �@�d��$-e$�����“d�q���E���֓��˂���Ƃ��ɁC��ǂ̓��ߗ���$T$�Ƃ��āC�P�ʎ��ԓ�����ɏ�ǂ𓧉߂���d�q�̓d�ׂ̑��ʁi�d���j$j$�����߂�B�A���C�G�l���M�[$E$�����“d�q�͒P�ʒ���������$n$�̖��x�����‚Ƃ���B\\
    
\end{enumerate}

%%%%%%�y2.�z%%%%%%
  \item �@���̌n�ɓd�q���l�߂��ꍇ�̓d�C�`�����l���悤�B�Ȃ��C�S�d�q�̖��x��$N$�C�d�q�̃X�s���k�d�x��2�Ƃ���B\\
  
    \begin{enumerate}
    
%%%% a %%%%
    \item �@�悸�C1�������R�d�q�C�̂́C�G�l���M�[$E$�ɑ΂���P�ʒ���������̏�Ԗ��x$D(E)$�C�y�ѐ�Η�x�ɂ�����t�F���~���x$v_\F$���C$m$�C$h$�i�v�����N�萔�j�C$N$��p���ĕ\���B\\
    
%%%% b %%%%
    \item �@�}(a)�Ɏ������C��ǂ�����n�ɂ�����G�l���M�[���z��}(b)�̂悤�ɕ\���B���Ȃ킿�C��ǂ̍����ƉE���̓d�q�̃t�F���~�E�G�l���M�[���C���ꂼ��$E_{\F\L},E_{\F\R}$�C���x$T$�ɂ����镪�z�֐������ꂼ��$f_\L(E,T),f_\R(E,T)$�Ƃ���B�Ȃ��C�d�q�C�̂͒ቷ�ŏk�ނ��Ă�����̂Ƃ���B���̂Ƃ��C1�������̂�������E�֗����S�d��$J$�́C��ǂ����Ƃ��č�����E�ɗ��ꍞ�ޓd�q�ɂ��S�d��$J_{\L\R}$�ƁC�E���獶�ɗ��ꍞ�ޓd�q�ɂ��S�d��$J_{\R\L}$�̍�$(J=J_{\L\R}-J_{\R\L})$�Ƃ��ė^������B$J_{\L\R}$��
\begin{equation}
 J_{\L\R}=-\frac{e}{2} \int_0^{\infty}\sqrt{\frac{2E}{m}}D(E)f_\L(E,T)
   [1-f_\R(E,T)]T(E)\d E
\end{equation}
�ŗ^�����邱�Ƃ��������B\\

%%%% c %%%%
    \item �@�}(c)�Ɏ����悤�ɁC��ǂ̗��[�Ԃɋ͂��̓d��$\Delta V$�������č����̓d�q�̃t�F���~�E�G�l���M�[���E�����$\Delta\mu~(\ll E_{\F\L} \approx E_{\F\R})$������������ƁC�S�d����$\Delta J$������������B���x���Η�x�Ƃ���$\Delta J$���v�Z���CG=$\ds \mathop {\lim }\limits_{\Delta V \to 0}\frac{\Delta J}{\Delta V}$�ŗ^������d�C�`���x��
$$
G=\left(\frac{2e^2}{h}\right)T(E_{\F\L})
$$
�ƕ\���邱�Ƃ������B�Ȃ��C$\Delta\mu$��$\Delta V$�̊Ԃɂ�$\Delta\mu=-e\Delta V$�̊֌W������B\\

\end{enumerate}

%%%%%%�y3.�z%%%%%%
  \item �@��L�̂悤�ȓ��̌n����邽�߂ɁC�ޗ��Ƃ���Fe��n�`Si�����������B�������C���ޗ��͌����������Ă���̂ŋ�ʂ��‚��Ȃ��Ȃ��Ă��܂����B�����I�ȑ���ɂ���āC���҂������������BFe�Cn�`Si����ʂ��邽�߂ɍs���ׂ�������2�‹�����B(���R���Y���āC�e40���ȓ��Łj�B�������C��d�C�d�x�C����C���w�����̈Ⴂ�͏����B\\
  
\end{enumerate}


\end{question}

%%%%%�y���8�i���j�z%%%%%%%%%%%%%%%%%%%%%%%%%%%%%%%%%%%%%%%%%%%%%%%%%%%%%%%

\begin{answer}{���8}{�n�Ӂ{��ؗ�}
\setcounter{equation}{0}


\begin{enumerate}

%%%%%%�y1.�z%%%%%%
  \item �@
  
  \begin{enumerate}
  
%%%% a %%%%
    \item �@
\begin{equation}
\begin{cases}
       ~\phi_�T(x) & = \mathrm{e}^{i kx} + A \mathrm{e}^{-i kx} \\
       ~\phi_�U(x) & = B \mathrm{e}^{\rho x} + C \mathrm{e}^{ - \rho x} \\
       ~\phi_�V(x) & = D \mathrm{e}^{i kx} 
\end{cases}
\end{equation}
�ƕ\����B������
\begin{equation}
k = \frac{\sqrt {2mE}}{\hbar} , \quad \rho = \frac{\sqrt {2m(U_0-E)}}{\hbar}\nonumber
\end{equation}
�ł���B\\
    
%%%% b %%%%
    \item �@$x=0$�őΐ����������,
\begin{equation}
k\frac{1-A}{1+A} = \rho\frac{B-C}{B+C}
\end{equation}
$x=L$�őΐ����������,
\begin{equation}
\rho\frac{Be^{\rho L}+Ce^{-\rho L}}{Be^{\rho L}-Ce^{-\rho L}} = i k
\end{equation}
�ȏオ�������ׂ����E�����ł���B\\

%%%% c %%%%
    \item �@�^����ꂽ���ߗ������߂�$T_{\rm r}$�Ƃ���\footnote{�ݖ�2~(b)�ł̉��x$T$�ƕ���킵�����߁B}�C�E���ɓ��߂���Flux��$j_{\rm trans}$ �Ƃ����
\begin{equation*}
j_{\rm trans} = \frac{\hbar k}{m}|D|^2 = \frac{\hbar k}{m}T_{\rm r} = \sqrt{\frac{2E}{m}}T_{\rm r}
\end{equation*}
%�G�l���M�[$E$�����“d�q�̖��x��$n(E)$�Ə����邩��C�S�d����
%\begin{equation}
%j = - e \int _0 ^{\infty} \sqrt{\frac{2E}{m}}n(E) T_{\rm r}(E) \d E \label{j}
%\end{equation}
�d�q�̖��x��$n$�Ȃ̂ŁA�C�S�d����
\begin{equation}
j = - e \sqrt{\frac{2E}{m}}nT_{\rm r}(E) \label{j}
\end{equation}
    
\end{enumerate}

%%%%%%�y2.�z%%%%%%
  \item �@
  
    \begin{enumerate}
    
%%%% a %%%%
    \item �@1�������R�d�q�C�̂̒P�ʒ���������̏�Ԗ��x�́C�X�s�����l�������
\vspace{2mm}
\begin{eqnarray}
\frac{\d p}{h} &=& 2\cdot \frac{1}{h} \sqrt{\frac{m}{2E}}\d E = D(E)\d E  \cr
& & \cr
\therefore D(E) &=& \frac{1}{h} \sqrt{\frac{2m}{E}}
\end{eqnarray}
\vspace{2mm}

��Η�x�ł͓d�q�͏k�ނ��Ă���̂�
\begin{equation}
N = \int _0^{\infty} D(E)f(E)\d E = \int _0^{E_\F} D(E)\d E = \frac{2\sqrt{2mE_\F}}{h} \nonumber
\end{equation}
�����Fermi���x��
\begin{equation}
v_\F = \sqrt{\frac{2E_\F}{m}} = \frac{Nh}{2m}
\end{equation}

    
%%%% b %%%%
    \item �@���̌n���C�d���������ɕۂ��ꂽ����Ԃɂ��邱�Ƃɒ��ӂ���B��$(4)$���C�G�l���M�[$E$������$f_{\L}(E)D(E)\d E$�̓d�q�����̌n����E�̌n�ɒ��I�ɗ��ꍞ�ނ��C���̂���$f_{\R}(E)$�̊����ŁC���ɂ����d�q�͉E�ɂ����d�q�������o�������ł���B�䂦��,
\begin{equation}
J_{\L\R} = -e\int_0^{\infty} \sqrt{\frac{2E}{m}} D(E) 
f_\L(E)\left[1-f_\R(E)\right]T(E)\d E
\end{equation}
���l�̗��ꂪ�E���獶�Ɍ������Ă����݂���̂�,
\begin{equation}
J_{\R\L} = -e\int_0^{\infty} \sqrt{\frac{2E}{m}} D(E)
f_\R(E)\left[1-f_\L(E)\right]T(E)\d E
\end{equation}

%%%% c %%%%
    \item �@
\begin{eqnarray}
\Delta J & =& -e\int_0^{\infty} \sqrt{\frac{2E}{m}} D(E)\left[f_\L(E)-f_\R(E)\right]T(E)\d E \cr
         & & \cr
         & =& -e\int_{E_{\F\R}}^{E_{\F\L}} \sqrt{\frac{2E}{m}} D(E)T(E)\d E \cr
         & & \cr
         & =& -e\int_{E_{\F\R}}^{E_{\F\L}} \sqrt{\frac{2E}{m}} \frac{1}{h} \sqrt{\frac{2m}{E}} T(E)\d E \cr
         & & \cr
         & =& -\frac{2e}{h} \int_{E_{\F\R}}^{E_{\F\L}} T(E)\d E \cr
         & & \cr
         & \approx& -\frac{2e}{h} T(E_{\F\L})\Delta \mu \cr
         & & \cr
         & =& -\frac{2e}{h} T(E_{\F\L})(-e\Delta V)
\end{eqnarray}
�����
\begin{equation}
G = \mathop {\lim }\limits_{\Delta V \to 0} \frac{\Delta J}{\Delta V} = \left(\frac{2e^2}{h}\right)T(E_{\F\L})
\end{equation}
    
\end{enumerate}

%%%%%%�y3.�z%%%%%%
  \item �@
\begin{itemize}
    \item Fe�͋��������������΂ɂ����‚����Cn�`Si�͂����‚��Ȃ��B
    (30��)
    \item Fe�͉��x�㏸�ƂƂ��ɒ�R�����傷�邪�C
    n�`Si�̓L�����A�[���������ߌ�������B(40��)
\end{itemize}
�ȂǁB\\
  
\end{enumerate}

\end{answer}

\end{document}

\vfill
\end{center}
\end{figure}
%%%%%%%%%%%%%%%%%%%%%%%%%%%%%%%%%%%%%%%%%%%%%%%%%%%%%%%%%%%%

  \begin{enumerate}
  
%%%% a %%%%
    \item �@$x<0$�C$0\le x\le L$�C$x>L$ �̊e�̈�ɂ����闱�q�̔g���֐������ꂼ��$\phi_�T(x),\phi_�U(x),\phi_�V(x)$�Ƃ��āC���̈�ʓI�Ȋ֐��`�������B�Ⴆ�΁C$Ae^{i kx}+ ...$�̂悤�ɁC�K���ȌW��$A, ...$ ���g���ċL�q���Ă悢�B\\
    
%%%% b %%%%
    \item �@�֐�$\phi_�T(x),\phi_�U(x),\phi_�V(x)$���������ׂ����E�����������B\\
    
%%%% c %%%%
    \item �@�d��$-e$�����“d�q���E���֓��˂���Ƃ��ɁC��ǂ̓��ߗ���$T$�Ƃ��āC�P�ʎ��ԓ�����ɏ�ǂ𓧉߂���d�q�̓d�ׂ̑��ʁi�d���j$j$�����߂�B�A���C�G�l���M�[$E$�����“d�q�͒P�ʒ���������$n$�̖��x�����‚Ƃ���B\\
    
\end{enumerate}

%%%%%%�y2.�z%%%%%%
  \item �@���̌n�ɓd�q���l�߂��ꍇ�̓d�C�`�����l���悤�B�Ȃ��C�S�d�q�̖��x��$N$�C�d�q�̃X�s���k�d�x��2�Ƃ���B\\
  
    \begin{enumerate}
    
%%%% a %%%%
    \item �@�悸�C1�������R�d�q�C�̂́C�G�l���M�[$E$�ɑ΂���P�ʒ���������̏�Ԗ��x$D(E)$�C�y�ѐ�Η�x�ɂ�����t�F���~���x$v_\F$���C$m$�C$h$�i�v�����N�萔�j�C$N$��p���ĕ\���B\\
    
%%%% b %%%%
    \item �@�}(a)�Ɏ������C��ǂ�����n�ɂ�����G�l���M�[���z��}(b)�̂悤�ɕ\���B���Ȃ킿�C��ǂ̍����ƉE���̓d�q�̃t�F���~�E�G�l���M�[���C���ꂼ��$E_{\F\L},E_{\F\R}$�C���x$T$�ɂ����镪�z�֐������ꂼ��$f_\L(E,T),f_\R(E,T)$�Ƃ���B�Ȃ��C�d�q�C�̂͒ቷ�ŏk�ނ��Ă�����̂Ƃ���B���̂Ƃ��C1�������̂�������E�֗����S�d��$J$�́C��ǂ����Ƃ��č�����E�ɗ��ꍞ�ޓd�q�ɂ��S�d��$J_{\L\R}$�ƁC�E���獶�ɗ��ꍞ�ޓd�q�ɂ��S�d��$J_{\R\L}$�̍�$(J=J_{\L\R}-J_{\R\L})$�Ƃ��ė^������B$J_{\L\R}$��
\begin{equation}
 J_{\L\R}=-\frac{e}{2} \int_0^{\infty}\sqrt{\frac{2E}{m}}D(E)f_\L(E,T)
   [1-f_\R(E,T)]T(E)\d E
\end{equation}
�ŗ^�����邱�Ƃ��������B\\

%%%% c %%%%
    \item �@�}(c)�Ɏ����悤�ɁC��ǂ̗��[�Ԃɋ͂��̓d��$\Delta V$�������č����̓d�q�̃t�F���~�E�G�l���M�[���E�����$\Delta\mu~(\ll E_{\F\L} \approx E_{\F\R})$������������ƁC�S�d����$\Delta J$������������B���x���Η�x�Ƃ���$\Delta J$���v�Z���CG=$\ds \mathop {\lim }\limits_{\Delta V \to 0}\frac{\Delta J}{\Delta V}$�ŗ^������d�C�`���x��
$$
G=\left(\frac{2e^2}{h}\right)T(E_{\F\L})
$$
�ƕ\���邱�Ƃ������B�Ȃ��C$\Delta\mu$��$\Delta V$�̊Ԃɂ�$\Delta\mu=-e\Delta V$�̊֌W������B\\

\end{enumerate}

%%%%%%�y3.�z%%%%%%
  \item �@��L�̂悤�ȓ��̌n����邽�߂ɁC�ޗ��Ƃ���Fe��n�`Si�����������B�������C���ޗ��͌����������Ă���̂ŋ�ʂ��‚��Ȃ��Ȃ��Ă��܂����B�����I�ȑ���ɂ���āC���҂������������BFe�Cn�`Si����ʂ��邽�߂ɍs���ׂ�������2�‹�����B(���R���Y���āC�e40���ȓ��Łj�B�������C��d�C�d�x�C����C���w�����̈Ⴂ�͏����B\\
  
\end{enumerate}


\end{question}

%%%%%�y���8�i���j�z%%%%%%%%%%%%%%%%%%%%%%%%%%%%%%%%%%%%%%%%%%%%%%%%%%%%%%%

\begin{answer}{���8}{�n�Ӂ{��ؗ�}
\setcounter{equation}{0}


\begin{enumerate}

%%%%%%�y1.�z%%%%%%
  \item �@
  
  \begin{enumerate}
  
%%%% a %%%%
    \item �@
\begin{equation}
\begin{cases}
       ~\phi_�T(x) & = \mathrm{e}^{i kx} + A \mathrm{e}^{-i kx} \\
       ~\phi_�U(x) & = B \mathrm{e}^{\rho x} + C \mathrm{e}^{ - \rho x} \\
       ~\phi_�V(x) & = D \mathrm{e}^{i kx} 
\end{cases}
\end{equation}
�ƕ\����B������
\begin{equation}
k = \frac{\sqrt {2mE}}{\hbar} , \quad \rho = \frac{\sqrt {2m(U_0-E)}}{\hbar}\nonumber
\end{equation}
�ł���B\\
    
%%%% b %%%%
    \item �@$x=0$�őΐ����������,
\begin{equation}
k\frac{1-A}{1+A} = \rho\frac{B-C}{B+C}
\end{equation}
$x=L$�őΐ����������,
\begin{equation}
\rho\frac{Be^{\rho L}+Ce^{-\rho L}}{Be^{\rho L}-Ce^{-\rho L}} = i k
\end{equation}
�ȏオ�������ׂ����E�����ł���B\\

%%%% c %%%%
    \item �@�^����ꂽ���ߗ������߂�$T_{\rm r}$�Ƃ���\footnote{�ݖ�2~(b)�ł̉��x$T$�ƕ���킵�����߁B}�C�E���ɓ��߂���Flux��$j_{\rm trans}$ �Ƃ����
\begin{equation*}
j_{\rm trans} = \frac{\hbar k}{m}|D|^2 = \frac{\hbar k}{m}T_{\rm r} = \sqrt{\frac{2E}{m}}T_{\rm r}
\end{equation*}
%�G�l���M�[$E$�����“d�q�̖��x��$n(E)$�Ə����邩��C�S�d����
%\begin{equation}
%j = - e \int _0 ^{\infty} \sqrt{\frac{2E}{m}}n(E) T_{\rm r}(E) \d E \label{j}
%\end{equation}
�d�q�̖��x��$n$�Ȃ̂ŁA�C�S�d����
\begin{equation}
j = - e \sqrt{\frac{2E}{m}}nT_{\rm r}(E) \label{j}
\end{equation}
    
\end{enumerate}

%%%%%%�y2.�z%%%%%%
  \item �@
  
    \begin{enumerate}
    
%%%% a %%%%
    \item �@1�������R�d�q�C�̂̒P�ʒ���������̏�Ԗ��x�́C�X�s�����l�������
\vspace{2mm}
\begin{eqnarray}
\frac{\d p}{h} &=& 2\cdot \frac{1}{h} \sqrt{\frac{m}{2E}}\d E = D(E)\d E  \cr
& & \cr
\therefore D(E) &=& \frac{1}{h} \sqrt{\frac{2m}{E}}
\end{eqnarray}
\vspace{2mm}

��Η�x�ł͓d�q�͏k�ނ��Ă���̂�
\begin{equation}
N = \int _0^{\infty} D(E)f(E)\d E = \int _0^{E_\F} D(E)\d E = \frac{2\sqrt{2mE_\F}}{h} \nonumber
\end{equation}
�����Fermi���x��
\begin{equation}
v_\F = \sqrt{\frac{2E_\F}{m}} = \frac{Nh}{2m}
\end{equation}

    
%%%% b %%%%
    \item �@���̌n���C�d���������ɕۂ��ꂽ����Ԃɂ��邱�Ƃɒ��ӂ���B��$(4)$���C�G�l���M�[$E$������$f_{\L}(E)D(E)\d E$�̓d�q�����̌n����E�̌n�ɒ��I�ɗ��ꍞ�ނ��C���̂���$f_{\R}(E)$�̊����ŁC���ɂ����d�q�͉E�ɂ����d�q�������o�������ł���B�䂦��,
\begin{equation}
J_{\L\R} = -e\int_0^{\infty} \sqrt{\frac{2E}{m}} D(E) 
f_\L(E)\left[1-f_\R(E)\right]T(E)\d E
\end{equation}
���l�̗��ꂪ�E���獶�Ɍ������Ă����݂���̂�,
\begin{equation}
J_{\R\L} = -e\int_0^{\infty} \sqrt{\frac{2E}{m}} D(E)
f_\R(E)\left[1-f_\L(E)\right]T(E)\d E
\end{equation}

%%%% c %%%%
    \item �@
\begin{eqnarray}
\Delta J & =& -e\int_0^{\infty} \sqrt{\frac{2E}{m}} D(E)\left[f_\L(E)-f_\R(E)\right]T(E)\d E \cr
         & & \cr
         & =& -e\int_{E_{\F\R}}^{E_{\F\L}} \sqrt{\frac{2E}{m}} D(E)T(E)\d E \cr
         & & \cr
         & =& -e\int_{E_{\F\R}}^{E_{\F\L}} \sqrt{\frac{2E}{m}} \frac{1}{h} \sqrt{\frac{2m}{E}} T(E)\d E \cr
         & & \cr
         & =& -\frac{2e}{h} \int_{E_{\F\R}}^{E_{\F\L}} T(E)\d E \cr
         & & \cr
         & \approx& -\frac{2e}{h} T(E_{\F\L})\Delta \mu \cr
         & & \cr
         & =& -\frac{2e}{h} T(E_{\F\L})(-e\Delta V)
\end{eqnarray}
�����
\begin{equation}
G = \mathop {\lim }\limits_{\Delta V \to 0} \frac{\Delta J}{\Delta V} = \left(\frac{2e^2}{h}\right)T(E_{\F\L})
\end{equation}
    
\end{enumerate}

%%%%%%�y3.�z%%%%%%
  \item �@
\begin{itemize}
    \item Fe�͋��������������΂ɂ����‚����Cn�`Si�͂����‚��Ȃ��B
    (30��)
    \item Fe�͉��x�㏸�ƂƂ��ɒ�R�����傷�邪�C
    n�`Si�̓L�����A�[���������ߌ�������B(40��)
\end{itemize}
�ȂǁB\\
  
\end{enumerate}

\end{answer}

\end{document}

\vfill
\end{center}
\end{figure}
%%%%%%%%%%%%%%%%%%%%%%%%%%%%%%%%%%%%%%%%%%%%%%%%%%%%%%%%%%%%

  \begin{enumerate}
  
%%%% a %%%%
    \item  $x<0$,$0\le x\le L$,$x>L$ の各領域における粒子の波動関数をそれぞれ$\phi_Ⅰ(x),\phi_Ⅱ(x),\phi_Ⅲ(x)$として,その一般的な関数形を書け。例えば,$Ae^{i kx}+ ...$のように,適当な係数$A, ...$ を使って記述してよい。\\
    
%%%% b %%%%
    \item  関数$\phi_Ⅰ(x),\phi_Ⅱ(x),\phi_Ⅲ(x)$が満たすべき境界条件を書け。\\
    
%%%% c %%%%
    \item  電荷$-e$を持つ電子が右方へ入射するときに,障壁の透過率を$T$として,単位時間当たりに障壁を透過する電子の電荷の総量(電流)$j$を求めよ。但し,エネルギー$E$を持つ電子は単位長さ当たり$n$の密度を持つとする。\\
    
\end{enumerate}

%%%%%%【2.】%%%%%%
  \item  この系に電子を詰めた場合の電気伝導を考えよう。なお,全電子の密度を$N$,電子のスピン縮重度を2とする。\\
  
    \begin{enumerate}
    
%%%% a %%%%
    \item  先ず,1次元自由電子気体の,エネルギー$E$に対する単位長さ当たりの状態密度$D(E)$,及び絶対零度におけるフェルミ速度$v_\F$を,$m$,$h$(プランク定数),$N$を用いて表せ。\\
    
%%%% b %%%%
    \item  図(a)に示した,障壁がある系におけるエネルギー分布を図(b)のように表す。すなわち,障壁の左側と右側の電子のフェルミ・エネルギーを,それぞれ$E_{\F\L},E_{\F\R}$,温度$T$における分布関数をそれぞれ$f_\L(E,T),f_\R(E,T)$とする。なお,電子気体は低温で縮退しているものとする。このとき,1次元導体を左から右へ流れる全電流$J$は,障壁を境として左から右に流れ込む電子による全電流$J_{\L\R}$と,右から左に流れ込む電子による全電流$J_{\R\L}$の差$(J=J_{\L\R}-J_{\R\L})$として与えられる。$J_{\L\R}$は
\begin{equation}
 J_{\L\R}=-\frac{e}{2} \int_0^{\infty}\sqrt{\frac{2E}{m}}D(E)f_\L(E,T)
   [1-f_\R(E,T)]T(E)\d E
\end{equation}
で与えられることを説明せよ。\\

%%%% c %%%%
    \item  図(c)に示すように,障壁の両端間に僅かの電圧$\Delta V$を加えて左側の電子のフェルミ・エネルギーを右側より$\Delta\mu~(\ll E_{\F\L} \approx E_{\F\R})$だけ高くすると,全電流が$\Delta J$だけ発生する。温度を絶対零度として$\Delta J$を計算し,G=$\ds \mathop {\lim }\limits_{\Delta V \to 0}\frac{\Delta J}{\Delta V}$で与えられる電気伝導度が
$$
G=\left(\frac{2e^2}{h}\right)T(E_{\F\L})
$$
と表せることを示せ。なお,$\Delta\mu$と$\Delta V$の間には$\Delta\mu=-e\Delta V$の関係がある。\\

\end{enumerate}

%%%%%%【3.】%%%%%%
  \item  上記のような導体系を作るために,材料としてFeとn形Siを準備した。しかし,両材料は見かけが似ているので区別がつかなくなってしまった。物理的な測定によって,両者を見分けたい。Fe,n形Siを区別するために行うべき実験を2つ挙げよ。(理由も添えて,各40字以内で)。ただし,比重,硬度,光沢,化学反応の違いは除く。\\
  
\end{enumerate}


\end{question}

%%%%%【問題8(答)】%%%%%%%%%%%%%%%%%%%%%%%%%%%%%%%%%%%%%%%%%%%%%%%%%%%%%%%

\begin{answer}{問題8}{渡辺+鈴木了}
\setcounter{equation}{0}


\begin{enumerate}

%%%%%%【1.】%%%%%%
  \item  
  
  \begin{enumerate}
  
%%%% a %%%%
    \item  
\begin{equation}
\begin{cases}
       ~\phi_Ⅰ(x) & = \mathrm{e}^{i kx} + A \mathrm{e}^{-i kx} \\
       ~\phi_Ⅱ(x) & = B \mathrm{e}^{\rho x} + C \mathrm{e}^{ - \rho x} \\
       ~\phi_Ⅲ(x) & = D \mathrm{e}^{i kx} 
\end{cases}
\end{equation}
と表せる。ここで
\begin{equation}
k = \frac{\sqrt {2mE}}{\hbar} , \quad \rho = \frac{\sqrt {2m(U_0-E)}}{\hbar}\nonumber
\end{equation}
である。\\
    
%%%% b %%%%
    \item  $x=0$で対数微分すると,
\begin{equation}
k\frac{1-A}{1+A} = \rho\frac{B-C}{B+C}
\end{equation}
$x=L$で対数微分すると,
\begin{equation}
\rho\frac{Be^{\rho L}+Ce^{-\rho L}}{Be^{\rho L}-Ce^{-\rho L}} = i k
\end{equation}
以上が満たすべき境界条件である。\\

%%%% c %%%%
    \item  与えられた透過率を改めて$T_{\rm r}$とおき\footnote{設問2~(b)での温度$T$と紛らわしいため。},右側に透過するFluxを$j_{\rm trans}$ とすると
\begin{equation*}
j_{\rm trans} = \frac{\hbar k}{m}|D|^2 = \frac{\hbar k}{m}T_{\rm r} = \sqrt{\frac{2E}{m}}T_{\rm r}
\end{equation*}
%エネルギー$E$を持つ電子の密度は$n(E)$と書けるから,全電流は
%\begin{equation}
%j = - e \int _0 ^{\infty} \sqrt{\frac{2E}{m}}n(E) T_{\rm r}(E) \d E \ilabel{j}
%\end{equation}
電子の密度は$n$なので、,全電流は
\begin{equation}
j = - e \sqrt{\frac{2E}{m}}nT_{\rm r}(E) \ilabel{j}
\end{equation}
    
\end{enumerate}

%%%%%%【2.】%%%%%%
  \item  
  
    \begin{enumerate}
    
%%%% a %%%%
    \item  1次元自由電子気体の単位長さ当たりの状態密度は,スピンを考慮すると
\vspace{2mm}
\begin{eqnarray}
\frac{\d p}{h} &=& 2\cdot \frac{1}{h} \sqrt{\frac{m}{2E}}\d E = D(E)\d E  \cr
& & \cr
\therefore D(E) &=& \frac{1}{h} \sqrt{\frac{2m}{E}}
\end{eqnarray}
\vspace{2mm}

絶対零度では電子は縮退しているので
\begin{equation}
N = \int _0^{\infty} D(E)f(E)\d E = \int _0^{E_\F} D(E)\d E = \frac{2\sqrt{2mE_\F}}{h} \nonumber
\end{equation}
よってFermi速度は
\begin{equation}
v_\F = \sqrt{\frac{2E_\F}{m}} = \frac{Nh}{2m}
\end{equation}

    
%%%% b %%%%
    \item  この系が,電圧差を一定に保たれた定常状態にあることに注意する。式$(4)$より,エネルギー$E$を持つ$f_{\L}(E)D(E)\d E$の電子が左の系から右の系に定常的に流れ込むが,そのうち$f_{\R}(E)$の割合で,左にいた電子は右にいた電子を押し出すだけである。ゆえに,
\begin{equation}
J_{\L\R} = -e\int_0^{\infty} \sqrt{\frac{2E}{m}} D(E) 
f_\L(E)\left[1-f_\R(E)\right]T(E)\d E
\end{equation}
同様の流れが右から左に向かっても存在するので,
\begin{equation}
J_{\R\L} = -e\int_0^{\infty} \sqrt{\frac{2E}{m}} D(E)
f_\R(E)\left[1-f_\L(E)\right]T(E)\d E
\end{equation}

%%%% c %%%%
    \item  
\begin{eqnarray}
\Delta J & =& -e\int_0^{\infty} \sqrt{\frac{2E}{m}} D(E)\left[f_\L(E)-f_\R(E)\right]T(E)\d E \cr
         & & \cr
         & =& -e\int_{E_{\F\R}}^{E_{\F\L}} \sqrt{\frac{2E}{m}} D(E)T(E)\d E \cr
         & & \cr
         & =& -e\int_{E_{\F\R}}^{E_{\F\L}} \sqrt{\frac{2E}{m}} \frac{1}{h} \sqrt{\frac{2m}{E}} T(E)\d E \cr
         & & \cr
         & =& -\frac{2e}{h} \int_{E_{\F\R}}^{E_{\F\L}} T(E)\d E \cr
         & & \cr
         & \approx& -\frac{2e}{h} T(E_{\F\L})\Delta \mu \cr
         & & \cr
         & =& -\frac{2e}{h} T(E_{\F\L})(-e\Delta V)
\end{eqnarray}
よって
\begin{equation}
G = \mathop {\lim }\limits_{\Delta V \to 0} \frac{\Delta J}{\Delta V} = \left(\frac{2e^2}{h}\right)T(E_{\F\L})
\end{equation}
    
\end{enumerate}

%%%%%%【3.】%%%%%%
  \item  
\begin{itemize}
    \item Feは強磁性を持ち磁石にくっつくが,n形Siはくっつかない。
    (30字)
    \item Feは温度上昇とともに抵抗が増大するが,
    n形Siはキャリアーが増すため減少する。(40字)
\end{itemize}
など。\\
  
\end{enumerate}

\end{answer}

\end{document}
