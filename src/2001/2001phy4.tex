\documentclass[fleqn]{jbook}
\usepackage{physpub}
\usepackage{txfonts}
\def\bm{\boldsymbol}
\def\ds{\displaystyle}

\begin{document}

%%%%%【問題4(問)】%%%%%%%%%%%%%%%%%%%%%%%%%%%%%%%%%%%%%%%%%%%%%%%%%%%%%%%

\begin{question}{問題4}{岡村}
\setcounter{equation}{0}

 図のように頂角が$\alpha$の円錐を切り,鉛直軸に沿って逆さにして作った漏斗状の滑らかな面がある。切り口の高さを$h$として,この斜面上の質量$m$の質点の運動を考える。以下の問に,解答に至る筋道を添えて答えよ。\\


\begin{enumerate}


  \item  鉛直方向に$z$軸をとり,質点の座標を円柱座標$\left(z,\theta\right)$で表したとき,質点の斜面上の運動のラグランジアンを求めよ。ただし,重力の加速度定数を$g$とする。\\
  
  \item  この系の保存量を求めよ。\\
  
  \item  質点が高さ$2h$の斜面上で水平に初速度$v_0$で運動を始めたとする。初速度の違いによる質点の運動の違いについて述べよ。\\

\end{enumerate}

%%%%%%%%%%%%%%%%%%%%%%%%%%%%%%%%%%%%%%%%%%%%%%%%%%%%%%%%%%%%
\begin{figure}[hbtp]
\begin{center}
%\caption[]{\ilabel{}}
\vspace{1.0cm}
\input{2001phy4-2.tpc}
\end{center}
\end{figure}
%%%%%%%%%%%%%%%%%%%%%%%%%%%%%%%%%%%%%%%%%%%%%%%%%%%%%%%%%%%%

\end{question}

%%%%%【問題4(答)】%%%%%%%%%%%%%%%%%%%%%%%%%%%%%%%%%%%%%%%%%%%%%%%%%%%%%%%

\begin{answer}{問題4}{大河原斉揚}
\setcounter{equation}{0}

\begin{enumerate}

%%%%%%【1.】%%%%%%
  \item  重力エネルギーの原点をOとする。$\left(z,\theta\right)$を$xyz$座標で表すと$\vect{z\tan\alpha \cos\theta}{z\tan\alpha \sin\theta}{z}$なので,速度は\\
$\vect{\dot{z}\tan\alpha \cos\theta -z\dot{\theta}\tan\alpha \sin\theta}{\dot{z}\tan\alpha \sin\theta+z\dot{\theta}\tan\alpha \cos\theta}{\dot{z}}$,その大きさは
$$v^{2}=\left(\dot{z}\tan\alpha\right)^{2}+\left(z\dot{\theta}\tan\alpha\right)^{2}+{\dot{z}}^{2}$$
である。これより,求めるラグランジアン$\mathcal L$は
\begin{eqnarray}
{\mathcal L}&=&\frac{1}{2}m\left[\left({\dot{z}}^{2}+z^{2}{\dot{\theta}}^{2}\right)\tan^{2}\alpha+{\dot{z}}^{2}\right]-mgz\\
&=&\frac{1}{2}m\left(\frac{{\dot{z}}^{2}}{\cos^{2}\alpha}+z^{2}{\dot{\theta}}^{2}\tan^{2}\alpha\right)-mgz .
\end{eqnarray}
  
%%%%%%【2.】%%%%%%
  \item  $\theta$は循環座標であるから$\ds \frac{{\partial}{\mathcal L}}{{\partial}\dot{\theta}}$が保存し,また${\mathcal L}$の式は$t$を含まないので全エネルギーも保存する。
$$\frac{{\partial}{\mathcal L}}{{\partial}\dot{\theta}}=mz^{2}\dot{\theta}\tan^{2}\alpha$$
よって,$z^{2}\dot{\theta}$及び全エネルギー$E$が保存する。\\
  
  
%%%%%%【3.】%%%%%%
  \item  \begin{eqnarray}
z^{2}\dot{\theta}=4h^{2}{\cdot}\frac{v_{0}}{2h\tan\alpha}=\frac{2hv_{0}}{\tan\alpha} .
\end{eqnarray}
またエネルギーは,
\begin{eqnarray}
E&=&\frac{1}{2}m\left[\left({\dot{z}}^{2}+z^{2}{\dot{\theta}}^{2}\right)\tan^{2}\alpha+{\dot{z}}^{2}\right]+mgz\\
&=&\frac{1}{2}m\left[4h^{2}\cdot\left(\frac{v_{0}}{2h\tan\alpha}\right)^{2}\tan^{2}\alpha\right]+mg{\cdot}2h\\
&=&\frac{1}{2}mv_{0}^{2}+2mgh
\end{eqnarray}
(3)式を代入して整理すると,
\begin{eqnarray}
&&\frac{1}{\cos^2{\alpha}}{\dot{z}}^{2}+\left(2hv_{0}\right)^{2}\frac{1}{z^{2}}+2gz=v_{0}^{2}+4gh ,\\
&&\frac{dt}{dz}=\cdots=\pm\sqrt{\frac{-2g\cos^{2}\alpha}{z^{2}}\left(z-2h\right)\left(z^{2}-\frac{v_{0}^{2}}{2g}z-\frac{hv_{0}^{2}}{g}\right)}
\end{eqnarray}
また(7)式を時間微分すると,
\begin{eqnarray}
\frac{d^{2}z}{dt^{2}}=\left(4h^{2}v_{0}^{2}\frac{1}{z^{3}}-g\right)\cos^{2}\alpha
\end{eqnarray}
となることが分かる。これより,$\ds \frac{v_{0}^{2}}{2h}-g
\left\{ {\begin{array}{*{20}c}
   {>0}  \\
   {<0}  \\
\end{array}} \right\}$の時,開始状態から質点は$
\left\{ {\begin{array}{*{20}c}
   {上}  \\
   {下}  \\
\end{array}} \right\}$向きに進み始める。\\

\begin{enumerate}
    \item  $|v_{0}|>\sqrt{2hg}$のとき\\
 質点は上向きに進み始める。その後$\ds z=\left[\frac{\left(2hv_{0}\right)^{2}}{g}\right]^{\frac{1}{3}}$の位置で$z=z(t)$の函数は変曲点を迎え,$\ds z=\frac{v_{0}\left(v_{0}+\sqrt{v_{0}^{2}+16gh}\right)}{4g}$において$\ds \frac{dz}{dt}=0$となって,再び$z$は減少し始める$\Big($$\ds \frac{dz}{dt}$の符号が$+$から$-$に変わる$\Big)$。そしてまた$\ds z=\left[\frac{\left(2hv_{0}\right)^{2}}{g}\right]^{\frac{1}{3}}$で変曲点を迎え,$z=2h$で$\ds \frac{dz}{dt}=0$となって,再び$z$は増加し始め,これを繰り返す。つまり,
$$2h\leq z \leq \frac{v_{0}\left(v_{0}+\sqrt{v_{0}^{2}+16gh}\right)}{4g}$$
で振動する。\\

    \item  $|v_{0}|=\sqrt{2hg}$のとき$\cdots$この時は$z=2h$で円軌道を保つ。\\
    
    \item  $|v_{0}|<\sqrt{2hg}$のとき\\
 質点は下向きに進み始め,$\ds z=\left[\frac{\left(2hv_{0}\right)^{2}}{g}\right]^{\frac{1}{3}}$で変曲点を迎え,$\ds z=\frac{v_{0}\left(v_{0}+\sqrt{v_{0}^{2}+16gh}\right)}{4g}$で$\ds \frac{dz}{dt}=0$となり$\cdots$と同様に考えるが,$z=h$に底があるので少し注意が必要である。具体的には$\ds z=\frac{v_{0}\left(v_{0}+\sqrt{v_{0}^{2}+16gh}\right)}{4g}$と$z=h$の大小を見ればよい。\\

\begin{enumerate}
    \item  $\ds \sqrt{\frac{2hg}{3}} \leq |v_{0}| < \sqrt{2hg}$のとき\\
 このときは質点は底に触れることがなく(等号のときは,触れても影響を受けない)\\
$$\frac{v_{0}\left(v_{0}+\sqrt{v_{0}^{2}+16gh}\right)}{4g} \leq z \leq 2h$$
で滑らかに振動する。\\

    \item  $\ds |v_{0}| < \sqrt{\frac{2hg}{3}}$のとき\\
 このときは$z=h$で下向きの速度を持っているので、開口から落下してしまう。\\

    \end{enumerate}
    
  \end{enumerate}

\end{enumerate}

\end{answer}

\end{document}
