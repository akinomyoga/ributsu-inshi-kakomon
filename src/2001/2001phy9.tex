\documentclass[fleqn]{jbook}
\usepackage{physpub}
\def\bm{\boldsymbol}
\def\ds{\displaystyle}
\def\o{{\rm o}}
\def\i{{\rm i}}
\def\millimol{{\rm ~[millimol/\ell]}}

%% Defined by 

\begin{document}

%%%%%【問題9(問)】%%%%%%%%%%%%%%%%%%%%%%%%%%%%%%%%%%%%%%%%%%%%%%%%%%%%%%%

\begin{question}{問題9}{岡村}
\setcounter{equation}{0}

\begin{enumerate}

%%%%%%【1.】%%%%%%
  \item  神経細胞の膜電位は,細胞膜をはさむ細胞内外のイオン濃度差および細胞膜の選択的イオン透過性に依存している。\\
  
    \begin{enumerate}
    
%%%% a %%%%
    \item  1molの溶質が細胞内へと細胞膜を通過する際の自由エネルギー変化$\Delta G$は
    $$
    \Delta G = -2.3RT\log_{10}\left( C_\o/C_\i \right)+zFV
    $$
    で表される。ここで$R=2.0\times10^{-3}$kcal/K$\cdot$molは気体定数,$T$はKで表した絶対温度,$C_\o$は細胞外部での溶質の濃度,$C_\i$は細胞内部での溶質の濃度,$z$は溶質のもつ電荷の価数,$F=23$kcal/V$\cdot$molはファラデー定数,$V$はボルトで表した膜電位(細胞内電位)である。この式より,イオンの平衡電位を表すネルンストの式を導け。
    
%%%% b %%%%
    \item  静止状態において,神経細胞の細胞内電位は細胞外に対して負である。これは静止時の細胞膜が,或るイオン種に対して特に高い透過性をもつからである。下の表は,ヤリイカの巨大軸索における細胞内外の主なイオン種の濃度を示したものであるが,これを参照し,細胞膜が上記のイオン種のみを通すとした場合,その膜電位(細胞内電位)をネルンストの式を用い求めよ。ただし温度は$20\degC$とせよ。\\
%%%%%%%%%%%%%%%%%%%%%%%%%%%%%%%%%%%%%%%%%%%%%%%%%%%%%%%%%%%%
\begin{table}[hbtp]
\begin{center}
%\caption[]{ \ilabel{ }}
\vspace{-2mm}
\begin{tabular}[h]{c|r|r}
イオン種    &細胞内~(millimol/$\ell$)   &細胞外~(millimol/$\ell$)\\
\hline
K${}^+$     &400                        &20\\
Na${}^+$    &50                         &440\\
Cl${}^-$    &51                         &560\\
\end{tabular}
\end{center}
\end{table}
%%%%%%%%%%%%%%%%%%%%%%%%%%%%%%%%%%%%%%%%%%%%%%%%%%%%%%%%%%%%
    
%%%% c %%%%
    \item  活動電位(神経インパルス)の発生時においては,膜電位は一過性に正となる。この際起こる膜のイオンの透過性の一連の変化について,その分子機構を含め説明せよ。\\
    
%%%% d %%%%
    \item  神経インパルスは通常軸索基部において発生し軸索先端部へ向かい一方向に伝播する。途中から反対方向へと逆行しない理由を述べよ。\\
    
\end{enumerate}


%%%%%%【2.】%%%%%%
  \item  ショウジョウバエのeyeless遺伝子の機能欠失変異体では眼の形成が起こらなくなる。\\
  
    \begin{enumerate}
    
%%%% a %%%%
    \item  このことはeyeless遺伝子が眼の形成に必要であることを示している。さらにこの遺伝子が眼の形成に十分な働きをもつかを調べるにはどのような実験を行えばよいかを述べよ。\\

%%%% b %%%%
    \item  上記の実験からeyeless遺伝子は眼の形成に必要かつ十分な活性をもつことが知られている。またこの遺伝子はある種の転写因子をコードすることが分かっている。以上のことからeyeless遺伝子の働きについて推察せよ。\\
    
\end{enumerate}
  
\end{enumerate}



\end{question}

%%%%%【問題9(答)】%%%%%%%%%%%%%%%%%%%%%%%%%%%%%%%%%%%%%%%%%%%%%%%%%%%%%%%

\begin{answer}{問題9}{石田良}
\setcounter{equation}{0}



\begin{enumerate}

%%%%%%【1.】%%%%%%
  \item  
  
    \begin{enumerate}
    
%%%% a %%%%
    \item  イオンが平衡電位を達成しているときは,自由エネルギー変化はどちらへのイオンの移動に対しても$0$でなければならない。よって
$$
zFV=2.3RT\log_{10}\left(C_\o/C_\i\right)
$$
これから
$$
V=2.3\frac{RT}{zF}\log_{10}\left(C_\o/C_\i\right)
$$
が導かれる。尚,実はこれは$\ds V=\frac{RT}{zF}\ln\left(C_\o/C_\i\right)$と書き直せる。\\
    
%%%% b %%%%
    \item  静止状態での細胞膜は主にK$^{+}$を透過させることが知られている。ここで,細胞内と細胞外でのK$^{+}$の濃度の差から,膜電位を求めよう。
$$
zFV=2.3RT\log_{10}\left(C_\o/C_\i\right)
$$
を使えば
$$
1\times23[\Unit{kcal/V\!\cdot\! mol}]\times V[\Unit{V}]=2.3\times 2.0\times 10^{-3}[\Unit{kcal/K\!\cdot\! mol}]\times 293[\Unit{K}]\log_{10}\left(\frac{20\millimol}{400\millimol}\right)
$$
これより$V = 0.1\times 2\times 0.001\times 293\times \log\left[1/20\right]=-0.0762$,よって$V_{\rm K^+}=-76[\Unit{mV}]$。\\
    
%%%% c %%%%
    \item  静止状態での神経細胞細胞膜は前問で述べたように主にK$^{+}$を透過させ,更にNa$^{+}$K$^{+}$channelによって細胞内のNa$^{+}$濃度を低く保つことによって結果的に細胞内静止電位をK$^{+}$の平衡電位にほぼ等しい$-70[\Unit{mV}]$付近に保っている。しかし,活動電位の発生時においては次のような変化がおきる。\\
 活動電位とは,脱分極の閾値を越えた電位であるが,細胞膜が閾値を越えて脱分極すると,活動電位が発生する。その際,まず,脱分極地点の電位依存性Na$^{+}$チャンネルが開き,細胞膜のNa$^{+}$透過性が激増する。これにより,電位の勾配によりNa$^{+}$の細胞内への流入が起こり,それが周辺に脱分極を引き起こす。\\
 しかしこの脱分極も次の変化が引き続いて起こることにより終焉する。まず,電位依存性K$^{+}$チャンネルが開くことによりK$^{+}$の細胞外への流出が起こる。このことを再分極というが,これにより細胞内の電位が低下し,電位依存性のK$^{+}$,Na$^{+}$チャンネルは両方とも閉じる。これらの変化は僅か$1[\Unit{msec}]$程度で終了する。\\
    
%%%% d %%%%
    \item  神経インパルスが周辺部に広がっていく過程は上の問題で述べた。さて,ここではその伝達が一方向である理由を述べよう。\\
 神経インパルスが伝達する際,細胞膜は脱分極,再分極を経てもとの状態に戻るが,膜が次の活動電位を発生させうる状態になるのには脱分極,再分極を経るだけの時間が必要となる。膜が次の活動電位を発生させうる状態になる前の状態を不活性状態と呼ぶ。\\
 このことにより,今まさに脱分極が起こっている部位は片方向にしかその神経インパルスを伝達できないのである。なぜならほんの少し前に脱分極が起こっていた側は不活性状態になっているので脱分極が引き起こされないからである。
    
\end{enumerate}


%%%%%%【2.】%%%%%%
  \item  
  
    \begin{enumerate}
    
%%%% a %%%%
    \item  eyeless遺伝子が目の形成に必要であることは問題文で与えられたように示すことが
できる。しかしこれが目の形成に十分であることはどのように示したらよいだろうか。\\
 そのためにはこの遺伝子だけにより,本来目が形成されない所にも複眼が形成されることを示せばよいだろう。\\
 具体的には次のようにすればよい。任意のプロモーターにより発現可能なGAL4を持つ個体と,UAS配列(GAL4にのみ反応し,活性化する領域)にコントロールされるeyelessをもつ個体を交配させ,結果として,GAL4のプロモーターに対応した転写活性化因子によりeyelessが発現する仕組みを持つ子供ができる。その転写活性化因子が存在する場所で眼がつくられることが示されれば良い。\\
 余談ではあるが,このことは実際に示されている。

\paragraph{註:}実際にはGAL4などの知識は要求されないと考えられる。「本来は目が形成されない所を形作ることになるDNAの部位(正確には成虫原基)にeyeless遺伝子を挿入することによって複眼が形成される」旨を書けば良い。\\
    
%%%% b %%%%
    \item  複眼のような複雑な組織を形成するのに,高々ひとつの遺伝子が作る蛋白質のみで対応できるとは考えにくい。また,この遺伝子が転写因子(あるプロモータを活性化させる分子)をコードしていることから,このeyeless遺伝子は複眼を形成するための様々な遺伝子の転写活性をも活性化させるのであると考えられる。つまり,eyeless遺伝子は複眼形成のためのマスター遺伝子であると考えられる。\\
    
\end{enumerate}
  
\end{enumerate}



\end{answer}

\end{document}


