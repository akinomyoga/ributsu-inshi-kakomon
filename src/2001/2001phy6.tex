\documentclass[fleqn]{jbook}
\usepackage{physpub}
\def\bm{\boldsymbol}
\def\ds{\displaystyle}



%% Defined by 

\begin{document}

%%%%%【問題6(問)】%%%%%%%%%%%%%%%%%%%%%%%%%%%%%%%%%%%%%%%%%%%%%%%%%%%%%%%

\begin{question}{問題6}{岡村}
\setcounter{equation}{0}

\begin{enumerate}

%%%%%%【1.】%%%%%%
  \item  以下の問に,解答に至る筋道を添えて答えよ。\\
  
  \begin{enumerate}
  
%%%% a %%%%
    \item  ある気体の圧力が$1.0\times10^{-5}$Pa,温度は$0\degC$の状態にある。この気体$1.0$cm$^3$中に気体分子は何個あるか。ここで,気体は理想気体とみなせるとし,$1$気圧を$1.0\times10^{-5}$Pa,アボガドロ数を$6.0\times10^{23}$個,標準状態($0\degC$,$1$気圧)の理想気体$1$モルの体積を$22l$とする。\\
    
%%%% b %%%%
    \item  加速された電子がこの気体分子に衝突すると,電子の運動エネルギーによって分子は電離してイオンとなる。この気体中で長さ$1.0$cmを持つ領域に電流密度$1.0$mA/cm$^2$の電流が流れているとき,単位体積あたり$1$秒間に何個のイオンが生成されるか。ここで,この電子の衝突による分子の電離断面積を$3.0\times10^{-16}$cm$^2$とし,素電荷を$1.6\times10^{-19}$Cとする。また,電流の流れている体積内の圧力は常に一定で,生成されたイオンは直ちに遠方に飛び去り,単位体積内では電子と気体分子の衝突は最大$1$回までとする。\\
    
\end{enumerate}

%%%%%%【2.】%%%%%%
  \item  高真空で使用する電離真空計では,電離させた残留気体分子によるイオン電流を測定し,その電流値を真空度に換算している。\\
  
    \begin{enumerate}
    
%%%% a %%%%
    \item  次の図は,%% 原文では「次のページの図」となっていた。
    電離真空計の模式図である。この真空計の動作原理を図中のフィラメント,グリッド,コレクターという言葉を使用して説明せよ。
%%%%%%%%%%%%%%%%%%%%%%%%%%%%%%%%%%%%%%%%%%%%%%%%%%%%%%%%%%%%
\begin{figure}[hbt]
\begin{center}
%\caption[]{\ilabel{}}
%\vspace{1.0cm}
\documentclass[fleqn]{jbook}
\usepackage{physpub}
\def\bm{\boldsymbol}
\def\ds{\displaystyle}



%% Defined by 

\begin{document}

%%%%%�y���6�i��j�z%%%%%%%%%%%%%%%%%%%%%%%%%%%%%%%%%%%%%%%%%%%%%%%%%%%%%%%

\begin{question}{���6}{����}
\setcounter{equation}{0}

\begin{enumerate}

%%%%%%�y1.�z%%%%%%
  \item �@�ȉ��̖�ɁC�𓚂Ɏ���ؓ���Y���ē�����B\\
  
  \begin{enumerate}
  
%%%% a %%%%
    \item �@����C�̂̈��͂�$1.0\times10^{-5}$Pa�C���x��$0\degC$�̏�Ԃɂ���B���̋C��$1.0$cm$^3$���ɋC�̕��q�͉��‚��邩�B�����ŁC�C�̂͗��z�C�̂Ƃ݂Ȃ���Ƃ��C$1$�C����$1.0\times10^{-5}$Pa�C�A�{�K�h������$6.0\times10^{23}$�C�W����ԁi$0\degC$�C$1$�C���j�̗��z�C��$1$�����̑̐ς�$22l$�Ƃ���B\\
    
%%%% b %%%%
    \item �@�������ꂽ�d�q�����̋C�̕��q�ɏՓ˂���ƁC�d�q�̉^���G�l���M�[�ɂ���ĕ��q�͓d�����ăC�I���ƂȂ�B���̋C�̒��Œ���$1.0$cm�����—̈�ɓd�����x$1.0$mA/cm$^2$�̓d��������Ă���Ƃ��C�P�ʑ̐ς�����$1$�b�Ԃɉ��‚̃C�I������������邩�B�����ŁC���̓d�q�̏Փ˂ɂ�镪�q�̓d���f�ʐς�$3.0\times10^{-16}$cm$^2$�Ƃ��C�f�d�ׂ�$1.6\times10^{-19}$C�Ƃ���B�܂��C�d���̗���Ă���̐ϓ��̈��͂͏�Ɉ��ŁC�������ꂽ�C�I���͒����ɉ����ɔ�ы���C�P�ʑ̐ϓ��ł͓d�q�ƋC�̕��q�̏Փ˂͍ő�$1$��܂łƂ���B\\
    
\end{enumerate}

%%%%%%�y2.�z%%%%%%
  \item �@���^��Ŏg�p����d���^��v�ł́C�d���������c���C�̕��q�ɂ��C�I���d���𑪒肵�C���̓d���l��^��x�Ɋ��Z���Ă���B\\
  
    \begin{enumerate}
    
%%%% a %%%%
    \item �@���̐}�́C%% �����ł́u���̃y�[�W�̐}�v�ƂȂ��Ă����B
    �d���^��v�̖͎��}�ł���B���̐^��v�̓��쌴����}���̃t�B�������g�C�O���b�h�C�R���N�^�[�Ƃ������t���g�p���Đ�������B
%%%%%%%%%%%%%%%%%%%%%%%%%%%%%%%%%%%%%%%%%%%%%%%%%%%%%%%%%%%%
\begin{figure}[hbt]
\begin{center}
%\caption[]{\label{}}
%\vspace{1.0cm}
\documentclass[fleqn]{jbook}
\usepackage{physpub}
\def\bm{\boldsymbol}
\def\ds{\displaystyle}



%% Defined by 

\begin{document}

%%%%%�y���6�i��j�z%%%%%%%%%%%%%%%%%%%%%%%%%%%%%%%%%%%%%%%%%%%%%%%%%%%%%%%

\begin{question}{���6}{����}
\setcounter{equation}{0}

\begin{enumerate}

%%%%%%�y1.�z%%%%%%
  \item �@�ȉ��̖�ɁC�𓚂Ɏ���ؓ���Y���ē�����B\\
  
  \begin{enumerate}
  
%%%% a %%%%
    \item �@����C�̂̈��͂�$1.0\times10^{-5}$Pa�C���x��$0\degC$�̏�Ԃɂ���B���̋C��$1.0$cm$^3$���ɋC�̕��q�͉��‚��邩�B�����ŁC�C�̂͗��z�C�̂Ƃ݂Ȃ���Ƃ��C$1$�C����$1.0\times10^{-5}$Pa�C�A�{�K�h������$6.0\times10^{23}$�C�W����ԁi$0\degC$�C$1$�C���j�̗��z�C��$1$�����̑̐ς�$22l$�Ƃ���B\\
    
%%%% b %%%%
    \item �@�������ꂽ�d�q�����̋C�̕��q�ɏՓ˂���ƁC�d�q�̉^���G�l���M�[�ɂ���ĕ��q�͓d�����ăC�I���ƂȂ�B���̋C�̒��Œ���$1.0$cm�����—̈�ɓd�����x$1.0$mA/cm$^2$�̓d��������Ă���Ƃ��C�P�ʑ̐ς�����$1$�b�Ԃɉ��‚̃C�I������������邩�B�����ŁC���̓d�q�̏Փ˂ɂ�镪�q�̓d���f�ʐς�$3.0\times10^{-16}$cm$^2$�Ƃ��C�f�d�ׂ�$1.6\times10^{-19}$C�Ƃ���B�܂��C�d���̗���Ă���̐ϓ��̈��͂͏�Ɉ��ŁC�������ꂽ�C�I���͒����ɉ����ɔ�ы���C�P�ʑ̐ϓ��ł͓d�q�ƋC�̕��q�̏Փ˂͍ő�$1$��܂łƂ���B\\
    
\end{enumerate}

%%%%%%�y2.�z%%%%%%
  \item �@���^��Ŏg�p����d���^��v�ł́C�d���������c���C�̕��q�ɂ��C�I���d���𑪒肵�C���̓d���l��^��x�Ɋ��Z���Ă���B\\
  
    \begin{enumerate}
    
%%%% a %%%%
    \item �@���̐}�́C%% �����ł́u���̃y�[�W�̐}�v�ƂȂ��Ă����B
    �d���^��v�̖͎��}�ł���B���̐^��v�̓��쌴����}���̃t�B�������g�C�O���b�h�C�R���N�^�[�Ƃ������t���g�p���Đ�������B
%%%%%%%%%%%%%%%%%%%%%%%%%%%%%%%%%%%%%%%%%%%%%%%%%%%%%%%%%%%%
\begin{figure}[hbt]
\begin{center}
%\caption[]{\label{}}
%\vspace{1.0cm}
\documentclass[fleqn]{jbook}
\usepackage{physpub}
\def\bm{\boldsymbol}
\def\ds{\displaystyle}



%% Defined by 

\begin{document}

%%%%%�y���6�i��j�z%%%%%%%%%%%%%%%%%%%%%%%%%%%%%%%%%%%%%%%%%%%%%%%%%%%%%%%

\begin{question}{���6}{����}
\setcounter{equation}{0}

\begin{enumerate}

%%%%%%�y1.�z%%%%%%
  \item �@�ȉ��̖�ɁC�𓚂Ɏ���ؓ���Y���ē�����B\\
  
  \begin{enumerate}
  
%%%% a %%%%
    \item �@����C�̂̈��͂�$1.0\times10^{-5}$Pa�C���x��$0\degC$�̏�Ԃɂ���B���̋C��$1.0$cm$^3$���ɋC�̕��q�͉��‚��邩�B�����ŁC�C�̂͗��z�C�̂Ƃ݂Ȃ���Ƃ��C$1$�C����$1.0\times10^{-5}$Pa�C�A�{�K�h������$6.0\times10^{23}$�C�W����ԁi$0\degC$�C$1$�C���j�̗��z�C��$1$�����̑̐ς�$22l$�Ƃ���B\\
    
%%%% b %%%%
    \item �@�������ꂽ�d�q�����̋C�̕��q�ɏՓ˂���ƁC�d�q�̉^���G�l���M�[�ɂ���ĕ��q�͓d�����ăC�I���ƂȂ�B���̋C�̒��Œ���$1.0$cm�����—̈�ɓd�����x$1.0$mA/cm$^2$�̓d��������Ă���Ƃ��C�P�ʑ̐ς�����$1$�b�Ԃɉ��‚̃C�I������������邩�B�����ŁC���̓d�q�̏Փ˂ɂ�镪�q�̓d���f�ʐς�$3.0\times10^{-16}$cm$^2$�Ƃ��C�f�d�ׂ�$1.6\times10^{-19}$C�Ƃ���B�܂��C�d���̗���Ă���̐ϓ��̈��͂͏�Ɉ��ŁC�������ꂽ�C�I���͒����ɉ����ɔ�ы���C�P�ʑ̐ϓ��ł͓d�q�ƋC�̕��q�̏Փ˂͍ő�$1$��܂łƂ���B\\
    
\end{enumerate}

%%%%%%�y2.�z%%%%%%
  \item �@���^��Ŏg�p����d���^��v�ł́C�d���������c���C�̕��q�ɂ��C�I���d���𑪒肵�C���̓d���l��^��x�Ɋ��Z���Ă���B\\
  
    \begin{enumerate}
    
%%%% a %%%%
    \item �@���̐}�́C%% �����ł́u���̃y�[�W�̐}�v�ƂȂ��Ă����B
    �d���^��v�̖͎��}�ł���B���̐^��v�̓��쌴����}���̃t�B�������g�C�O���b�h�C�R���N�^�[�Ƃ������t���g�p���Đ�������B
%%%%%%%%%%%%%%%%%%%%%%%%%%%%%%%%%%%%%%%%%%%%%%%%%%%%%%%%%%%%
\begin{figure}[hbt]
\begin{center}
%\caption[]{\label{}}
%\vspace{1.0cm}
\input{2001phy6.tpc}
\vfill
\end{center}
\end{figure}
%%%%%%%%%%%%%%%%%%%%%%%%%%%%%%%%%%%%%%%%%%%%%%%%%%%%%%%%%%%%
    
%%%% b %%%%
    \item �@�\�ɂ́C����d���^��v�ɂ����āC��K�X�̐^��x$1.0\times10^{-6}$Pa�̂Ƃ����o�����C�I���d���l���C��K�X���q�̎�ޕʂɎ�����Ă���B���̂悤�Ɋ�K�X���q�̎�ނɂ���ăC�I���d���l���قȂ闝�R��萫�I�ɐ�������B\\
%%%%%%%%%%%%%%%%%%%%%%%%%%%%%%%%%%%%%%%%%%%%%%%%%%%%%%%%%%%%
\begin{table}[hbtp]
\begin{center}
%\caption[]{ \label{ }}
\vspace{-1.0cm}
\begin{tabular}[h]{c||c|c|c|c|c}
�C��                        &He     &Ne     &Ar     &Kr     &Xe     \\
\hline
�C�I���d���l~(nA)  &0.1    &0.3    &1.3    &1.7    &2.4    \\
\end{tabular}
\end{center}
\end{table}
%%%%%%%%%%%%%%%%%%%%%%%%%%%%%%%%%%%%%%%%%%%%%%%%%%%%%%%%%%%%
    
%%%% c %%%%
    \item �@�d���^��v�ł́C�t�B�������g������o���ꂽ�d�q���O���b�h�ɏՓ˂���ۂɔ��������X���������ƂȂ��ăC�I���d���ȊO�̓d��������C���m�Ȑ^��x������ł��Ȃ��B�ǂ̂悤�ȓd�����������邩���q�ׂ�B�܂��C���̌��ʂ����������C�����^��x�܂ő���ł���悤�ɂ��邽�߂ɂ́C�t�B�������g�C�O���b�h�C�R���N�^�[�̂����ǂꂩ���H�v����΂悢�B�ǂ̂悤�ɂ���΂��̌��ʂ����������邱�Ƃ��ł��邩���q�ׂ�B\\
\end{enumerate}

%%%%%%�y3.�z%%%%%%
  \item �@�d���^��v�̓��[�^���[�|���v�Ŕr�C�ł�����x�̐^��ł͒ʏ�g�p���Ȃ��B���̐^��x�Ŏg�p�ł���^��v���ЂƂ—�ɂ����āC���̓��쌴�����ȒP�ɐ�������B\\
  
\end{enumerate}



\end{question}

%%%%%�y���6�i���j�z%%%%%%%%%%%%%%%%%%%%%%%%%%%%%%%%%%%%%%%%%%%%%%%%%%%%%%%

\begin{answer}{���6}{�C�ȑn��}
\setcounter{equation}{0}



\begin{enumerate}

%%%%%%�y1.�z%%%%%%
  \item �@
  \begin{enumerate}
  
%%%% a %%%%
    \item �@��ʂ�$pV=nRT$�Ƃ����֌W�ɂ��C�̂̈��́C�̐ρC�������i�������q���j���x�͊֌W�t�����Ă���B�����ő�ӂ��툳�C$0\degC$�ɂ�����C�̕��q���̊֌W�����^�����Ă��邩��C�����p����$0\degC$�C$1.0\times10^{-5}$Pa�ɂ�����$1.0$cm$^3$���̋C�̕��q�̑����́C
\begin{equation}
    \frac{6.0\times 10^{23} \times 1.0 \times 10^{-5} \times 1.0}{1.0\times 10^5 \times 22 \times 10^3 } = 2.73 \times 10^9\nonumber
\end{equation}
���C$2.73 \times 10^9$�‚Ƌ��܂����B\\
    
%%%% b %%%%
    \item �@�d���f�ʐςƁC���q�̑����y�ѓd�q�̑����̐ς��S�d�����q����^����B���̂��Ƃ��C(a)�̌��ʂƁC��ɗ^�����Ă��錋�ʂ�p���āC
\begin{equation}
    \frac{1.0\times 10^{-3}}{1.6 \times 10^{-19}} \times 3.0\times 10^{-16} \times 2.73 \times 10^9 = 5.12 \times 10^9\nonumber
\end{equation}
�ƂȂ邱�Ƃ���C�����͖��b$5.12 \times 10^9$�‚Ƌ��܂����B\\
    
\end{enumerate}

%%%%%%�y2.�z%%%%%%
  \item �@
  \begin{enumerate}
    
%%%% a %%%%
    \item �@�t�B�������g�ɓd���𗬂����Ƃɂ��C�t�B�������g�͔M�����C���̕\�ʂ���M�d�q�𔭐�������B���̔M�d�q���C�t�B�������g�O���b�h�Ԃɓd���������C�������Ă��B����ƁC���̎��ɂ��C�M�d�q�͋C�̕��q�ɏՓ˂��C�z�C�I���𔭐�������B�����Ŕ��������z�C�I���͗z�ɂƂȂ�R���N�^�[�ɍׂ�����C�d���ƂȂ��Ċϑ������B���̎��C�t�B�������g-�O���b�h�Ԃɗ����d�q�d���ƁC�t�B�������g-�R���N�^�[�Ԃɗ����C�I���d���̊Ԃɂ͒P���Ȕ��֌W�����藧���C�܂����l�ɃC�I���d���͐^��v�����̈��͂ɔ�Ⴗ��B�����̂��Ƃ��C�d�q�d�������ɕۂ��đ�������{����΁C�C�I���d���𑪒肷�邱�Ƃ͑������͂𑪒肷�邱�Ƃɂ‚Ȃ���C���̎��𗘗p���Đ^��v�̈��͂𑪒肷��̂ł���B\\
    
%%%% b %%%%
    \item �@�C�I���d���̒l�͂��̋C�̂̏Փ˒f�ʐςɍ��E�����C�Փ˒f�ʐς͓d�q�̉^���G�l���M�[�ƋC�̂̎�ނɈˑ����邪�C��`�I�ɂ͋C�̕��q�̑傫���ɂ��ƍl���Ă悢�B���ɂ���͊�K�X���f�̏ꍇ���̔������̖R��������t�Ɍ����ł���ƍl���Ă悢�B���̂��Ƃ͑傫�ȕ��q�قǏՓ˒f�ʐς��傫���Ȃ�ƍl���Ă悭�C����ɂ�蕪�q�ʂ�������قǃC�I���d�����傫���Ȃ邱�Ƃ��\�z�ł��C���ۂɂ����Ȃ��Ă��邱�Ƃ��m���߂��Ă���B\\
    
%%%% c %%%%
    \item �@�O���b�h�ɏՓ˂���ہC�������˂ɂ����o�����X���̓R���N�^�[�ɓ��B������Ɍ��d���ʂ��N�����ăR���N�^�[�̕\�ʂ���d�q���яo������B���̂��Ƃ��n�ɗ]�v�ȓd���𗬂����ƂƂȂ�C�덷�̌����ƂȂ��Ă���B���̂悤�Ȍ덷������������ɂ̓R���N�^�[�̍ގ��Ƃ��ēd�q���яo�����邽�߂̎d���֐����o���邾���傫������悤�Ȃ��̂�I�ׂ΂悢�B���̎��ɂ��C���d���ʂɂ�蔭������d�q�̗ʂ�}���邱�Ƃ��o���C���^��x�܂ł̑��肪�”\�ƂȂ�̂ł���B\\

\end{enumerate}

%%%%%%�y3.�z%%%%%%
  \item �@�d���^��v���g���Ȃ��悤�ȍ������͂Ŏg�p�ł��鈳�͌v�Ƃ��ẮA�s���j�Q�[�W������B�s���j�Q�[�W�̑���q�͋����t�B�������g�ł���A�d����������ĉ��M����Ă���B�C�̕��q�̏Փ˂ɂ���ĔM�G�l���M�[���^�ы����邪�A���͂������قǎ����M�G�l���M�[�������A�t�B�������g�̉��x��������B���x�̒ቺ�ɂ���ēd�C��R��������̂ŁA���̕ω��𑪒肷�邱�Ƃň��͂�m�邱�Ƃ��ł���B
\end{enumerate}


\end{answer}

\end{document}

\vfill
\end{center}
\end{figure}
%%%%%%%%%%%%%%%%%%%%%%%%%%%%%%%%%%%%%%%%%%%%%%%%%%%%%%%%%%%%
    
%%%% b %%%%
    \item �@�\�ɂ́C����d���^��v�ɂ����āC��K�X�̐^��x$1.0\times10^{-6}$Pa�̂Ƃ����o�����C�I���d���l���C��K�X���q�̎�ޕʂɎ�����Ă���B���̂悤�Ɋ�K�X���q�̎�ނɂ���ăC�I���d���l���قȂ闝�R��萫�I�ɐ�������B\\
%%%%%%%%%%%%%%%%%%%%%%%%%%%%%%%%%%%%%%%%%%%%%%%%%%%%%%%%%%%%
\begin{table}[hbtp]
\begin{center}
%\caption[]{ \label{ }}
\vspace{-1.0cm}
\begin{tabular}[h]{c||c|c|c|c|c}
�C��                        &He     &Ne     &Ar     &Kr     &Xe     \\
\hline
�C�I���d���l~(nA)  &0.1    &0.3    &1.3    &1.7    &2.4    \\
\end{tabular}
\end{center}
\end{table}
%%%%%%%%%%%%%%%%%%%%%%%%%%%%%%%%%%%%%%%%%%%%%%%%%%%%%%%%%%%%
    
%%%% c %%%%
    \item �@�d���^��v�ł́C�t�B�������g������o���ꂽ�d�q���O���b�h�ɏՓ˂���ۂɔ��������X���������ƂȂ��ăC�I���d���ȊO�̓d��������C���m�Ȑ^��x������ł��Ȃ��B�ǂ̂悤�ȓd�����������邩���q�ׂ�B�܂��C���̌��ʂ����������C�����^��x�܂ő���ł���悤�ɂ��邽�߂ɂ́C�t�B�������g�C�O���b�h�C�R���N�^�[�̂����ǂꂩ���H�v����΂悢�B�ǂ̂悤�ɂ���΂��̌��ʂ����������邱�Ƃ��ł��邩���q�ׂ�B\\
\end{enumerate}

%%%%%%�y3.�z%%%%%%
  \item �@�d���^��v�̓��[�^���[�|���v�Ŕr�C�ł�����x�̐^��ł͒ʏ�g�p���Ȃ��B���̐^��x�Ŏg�p�ł���^��v���ЂƂ—�ɂ����āC���̓��쌴�����ȒP�ɐ�������B\\
  
\end{enumerate}



\end{question}

%%%%%�y���6�i���j�z%%%%%%%%%%%%%%%%%%%%%%%%%%%%%%%%%%%%%%%%%%%%%%%%%%%%%%%

\begin{answer}{���6}{�C�ȑn��}
\setcounter{equation}{0}



\begin{enumerate}

%%%%%%�y1.�z%%%%%%
  \item �@
  \begin{enumerate}
  
%%%% a %%%%
    \item �@��ʂ�$pV=nRT$�Ƃ����֌W�ɂ��C�̂̈��́C�̐ρC�������i�������q���j���x�͊֌W�t�����Ă���B�����ő�ӂ��툳�C$0\degC$�ɂ�����C�̕��q���̊֌W�����^�����Ă��邩��C�����p����$0\degC$�C$1.0\times10^{-5}$Pa�ɂ�����$1.0$cm$^3$���̋C�̕��q�̑����́C
\begin{equation}
    \frac{6.0\times 10^{23} \times 1.0 \times 10^{-5} \times 1.0}{1.0\times 10^5 \times 22 \times 10^3 } = 2.73 \times 10^9\nonumber
\end{equation}
���C$2.73 \times 10^9$�‚Ƌ��܂����B\\
    
%%%% b %%%%
    \item �@�d���f�ʐςƁC���q�̑����y�ѓd�q�̑����̐ς��S�d�����q����^����B���̂��Ƃ��C(a)�̌��ʂƁC��ɗ^�����Ă��錋�ʂ�p���āC
\begin{equation}
    \frac{1.0\times 10^{-3}}{1.6 \times 10^{-19}} \times 3.0\times 10^{-16} \times 2.73 \times 10^9 = 5.12 \times 10^9\nonumber
\end{equation}
�ƂȂ邱�Ƃ���C�����͖��b$5.12 \times 10^9$�‚Ƌ��܂����B\\
    
\end{enumerate}

%%%%%%�y2.�z%%%%%%
  \item �@
  \begin{enumerate}
    
%%%% a %%%%
    \item �@�t�B�������g�ɓd���𗬂����Ƃɂ��C�t�B�������g�͔M�����C���̕\�ʂ���M�d�q�𔭐�������B���̔M�d�q���C�t�B�������g�O���b�h�Ԃɓd���������C�������Ă��B����ƁC���̎��ɂ��C�M�d�q�͋C�̕��q�ɏՓ˂��C�z�C�I���𔭐�������B�����Ŕ��������z�C�I���͗z�ɂƂȂ�R���N�^�[�ɍׂ�����C�d���ƂȂ��Ċϑ������B���̎��C�t�B�������g-�O���b�h�Ԃɗ����d�q�d���ƁC�t�B�������g-�R���N�^�[�Ԃɗ����C�I���d���̊Ԃɂ͒P���Ȕ��֌W�����藧���C�܂����l�ɃC�I���d���͐^��v�����̈��͂ɔ�Ⴗ��B�����̂��Ƃ��C�d�q�d�������ɕۂ��đ�������{����΁C�C�I���d���𑪒肷�邱�Ƃ͑������͂𑪒肷�邱�Ƃɂ‚Ȃ���C���̎��𗘗p���Đ^��v�̈��͂𑪒肷��̂ł���B\\
    
%%%% b %%%%
    \item �@�C�I���d���̒l�͂��̋C�̂̏Փ˒f�ʐςɍ��E�����C�Փ˒f�ʐς͓d�q�̉^���G�l���M�[�ƋC�̂̎�ނɈˑ����邪�C��`�I�ɂ͋C�̕��q�̑傫���ɂ��ƍl���Ă悢�B���ɂ���͊�K�X���f�̏ꍇ���̔������̖R��������t�Ɍ����ł���ƍl���Ă悢�B���̂��Ƃ͑傫�ȕ��q�قǏՓ˒f�ʐς��傫���Ȃ�ƍl���Ă悭�C����ɂ�蕪�q�ʂ�������قǃC�I���d�����傫���Ȃ邱�Ƃ��\�z�ł��C���ۂɂ����Ȃ��Ă��邱�Ƃ��m���߂��Ă���B\\
    
%%%% c %%%%
    \item �@�O���b�h�ɏՓ˂���ہC�������˂ɂ����o�����X���̓R���N�^�[�ɓ��B������Ɍ��d���ʂ��N�����ăR���N�^�[�̕\�ʂ���d�q���яo������B���̂��Ƃ��n�ɗ]�v�ȓd���𗬂����ƂƂȂ�C�덷�̌����ƂȂ��Ă���B���̂悤�Ȍ덷������������ɂ̓R���N�^�[�̍ގ��Ƃ��ēd�q���яo�����邽�߂̎d���֐����o���邾���傫������悤�Ȃ��̂�I�ׂ΂悢�B���̎��ɂ��C���d���ʂɂ�蔭������d�q�̗ʂ�}���邱�Ƃ��o���C���^��x�܂ł̑��肪�”\�ƂȂ�̂ł���B\\

\end{enumerate}

%%%%%%�y3.�z%%%%%%
  \item �@�d���^��v���g���Ȃ��悤�ȍ������͂Ŏg�p�ł��鈳�͌v�Ƃ��ẮA�s���j�Q�[�W������B�s���j�Q�[�W�̑���q�͋����t�B�������g�ł���A�d����������ĉ��M����Ă���B�C�̕��q�̏Փ˂ɂ���ĔM�G�l���M�[���^�ы����邪�A���͂������قǎ����M�G�l���M�[�������A�t�B�������g�̉��x��������B���x�̒ቺ�ɂ���ēd�C��R��������̂ŁA���̕ω��𑪒肷�邱�Ƃň��͂�m�邱�Ƃ��ł���B
\end{enumerate}


\end{answer}

\end{document}

\vfill
\end{center}
\end{figure}
%%%%%%%%%%%%%%%%%%%%%%%%%%%%%%%%%%%%%%%%%%%%%%%%%%%%%%%%%%%%
    
%%%% b %%%%
    \item �@�\�ɂ́C����d���^��v�ɂ����āC��K�X�̐^��x$1.0\times10^{-6}$Pa�̂Ƃ����o�����C�I���d���l���C��K�X���q�̎�ޕʂɎ�����Ă���B���̂悤�Ɋ�K�X���q�̎�ނɂ���ăC�I���d���l���قȂ闝�R��萫�I�ɐ�������B\\
%%%%%%%%%%%%%%%%%%%%%%%%%%%%%%%%%%%%%%%%%%%%%%%%%%%%%%%%%%%%
\begin{table}[hbtp]
\begin{center}
%\caption[]{ \label{ }}
\vspace{-1.0cm}
\begin{tabular}[h]{c||c|c|c|c|c}
�C��                        &He     &Ne     &Ar     &Kr     &Xe     \\
\hline
�C�I���d���l~(nA)  &0.1    &0.3    &1.3    &1.7    &2.4    \\
\end{tabular}
\end{center}
\end{table}
%%%%%%%%%%%%%%%%%%%%%%%%%%%%%%%%%%%%%%%%%%%%%%%%%%%%%%%%%%%%
    
%%%% c %%%%
    \item �@�d���^��v�ł́C�t�B�������g������o���ꂽ�d�q���O���b�h�ɏՓ˂���ۂɔ��������X���������ƂȂ��ăC�I���d���ȊO�̓d��������C���m�Ȑ^��x������ł��Ȃ��B�ǂ̂悤�ȓd�����������邩���q�ׂ�B�܂��C���̌��ʂ����������C�����^��x�܂ő���ł���悤�ɂ��邽�߂ɂ́C�t�B�������g�C�O���b�h�C�R���N�^�[�̂����ǂꂩ���H�v����΂悢�B�ǂ̂悤�ɂ���΂��̌��ʂ����������邱�Ƃ��ł��邩���q�ׂ�B\\
\end{enumerate}

%%%%%%�y3.�z%%%%%%
  \item �@�d���^��v�̓��[�^���[�|���v�Ŕr�C�ł�����x�̐^��ł͒ʏ�g�p���Ȃ��B���̐^��x�Ŏg�p�ł���^��v���ЂƂ—�ɂ����āC���̓��쌴�����ȒP�ɐ�������B\\
  
\end{enumerate}



\end{question}

%%%%%�y���6�i���j�z%%%%%%%%%%%%%%%%%%%%%%%%%%%%%%%%%%%%%%%%%%%%%%%%%%%%%%%

\begin{answer}{���6}{�C�ȑn��}
\setcounter{equation}{0}



\begin{enumerate}

%%%%%%�y1.�z%%%%%%
  \item �@
  \begin{enumerate}
  
%%%% a %%%%
    \item �@��ʂ�$pV=nRT$�Ƃ����֌W�ɂ��C�̂̈��́C�̐ρC�������i�������q���j���x�͊֌W�t�����Ă���B�����ő�ӂ��툳�C$0\degC$�ɂ�����C�̕��q���̊֌W�����^�����Ă��邩��C�����p����$0\degC$�C$1.0\times10^{-5}$Pa�ɂ�����$1.0$cm$^3$���̋C�̕��q�̑����́C
\begin{equation}
    \frac{6.0\times 10^{23} \times 1.0 \times 10^{-5} \times 1.0}{1.0\times 10^5 \times 22 \times 10^3 } = 2.73 \times 10^9\nonumber
\end{equation}
���C$2.73 \times 10^9$�‚Ƌ��܂����B\\
    
%%%% b %%%%
    \item �@�d���f�ʐςƁC���q�̑����y�ѓd�q�̑����̐ς��S�d�����q����^����B���̂��Ƃ��C(a)�̌��ʂƁC��ɗ^�����Ă��錋�ʂ�p���āC
\begin{equation}
    \frac{1.0\times 10^{-3}}{1.6 \times 10^{-19}} \times 3.0\times 10^{-16} \times 2.73 \times 10^9 = 5.12 \times 10^9\nonumber
\end{equation}
�ƂȂ邱�Ƃ���C�����͖��b$5.12 \times 10^9$�‚Ƌ��܂����B\\
    
\end{enumerate}

%%%%%%�y2.�z%%%%%%
  \item �@
  \begin{enumerate}
    
%%%% a %%%%
    \item �@�t�B�������g�ɓd���𗬂����Ƃɂ��C�t�B�������g�͔M�����C���̕\�ʂ���M�d�q�𔭐�������B���̔M�d�q���C�t�B�������g�O���b�h�Ԃɓd���������C�������Ă��B����ƁC���̎��ɂ��C�M�d�q�͋C�̕��q�ɏՓ˂��C�z�C�I���𔭐�������B�����Ŕ��������z�C�I���͗z�ɂƂȂ�R���N�^�[�ɍׂ�����C�d���ƂȂ��Ċϑ������B���̎��C�t�B�������g-�O���b�h�Ԃɗ����d�q�d���ƁC�t�B�������g-�R���N�^�[�Ԃɗ����C�I���d���̊Ԃɂ͒P���Ȕ��֌W�����藧���C�܂����l�ɃC�I���d���͐^��v�����̈��͂ɔ�Ⴗ��B�����̂��Ƃ��C�d�q�d�������ɕۂ��đ�������{����΁C�C�I���d���𑪒肷�邱�Ƃ͑������͂𑪒肷�邱�Ƃɂ‚Ȃ���C���̎��𗘗p���Đ^��v�̈��͂𑪒肷��̂ł���B\\
    
%%%% b %%%%
    \item �@�C�I���d���̒l�͂��̋C�̂̏Փ˒f�ʐςɍ��E�����C�Փ˒f�ʐς͓d�q�̉^���G�l���M�[�ƋC�̂̎�ނɈˑ����邪�C��`�I�ɂ͋C�̕��q�̑傫���ɂ��ƍl���Ă悢�B���ɂ���͊�K�X���f�̏ꍇ���̔������̖R��������t�Ɍ����ł���ƍl���Ă悢�B���̂��Ƃ͑傫�ȕ��q�قǏՓ˒f�ʐς��傫���Ȃ�ƍl���Ă悭�C����ɂ�蕪�q�ʂ�������قǃC�I���d�����傫���Ȃ邱�Ƃ��\�z�ł��C���ۂɂ����Ȃ��Ă��邱�Ƃ��m���߂��Ă���B\\
    
%%%% c %%%%
    \item �@�O���b�h�ɏՓ˂���ہC�������˂ɂ����o�����X���̓R���N�^�[�ɓ��B������Ɍ��d���ʂ��N�����ăR���N�^�[�̕\�ʂ���d�q���яo������B���̂��Ƃ��n�ɗ]�v�ȓd���𗬂����ƂƂȂ�C�덷�̌����ƂȂ��Ă���B���̂悤�Ȍ덷������������ɂ̓R���N�^�[�̍ގ��Ƃ��ēd�q���яo�����邽�߂̎d���֐����o���邾���傫������悤�Ȃ��̂�I�ׂ΂悢�B���̎��ɂ��C���d���ʂɂ�蔭������d�q�̗ʂ�}���邱�Ƃ��o���C���^��x�܂ł̑��肪�”\�ƂȂ�̂ł���B\\

\end{enumerate}

%%%%%%�y3.�z%%%%%%
  \item �@�d���^��v���g���Ȃ��悤�ȍ������͂Ŏg�p�ł��鈳�͌v�Ƃ��ẮA�s���j�Q�[�W������B�s���j�Q�[�W�̑���q�͋����t�B�������g�ł���A�d����������ĉ��M����Ă���B�C�̕��q�̏Փ˂ɂ���ĔM�G�l���M�[���^�ы����邪�A���͂������قǎ����M�G�l���M�[�������A�t�B�������g�̉��x��������B���x�̒ቺ�ɂ���ēd�C��R��������̂ŁA���̕ω��𑪒肷�邱�Ƃň��͂�m�邱�Ƃ��ł���B
\end{enumerate}


\end{answer}

\end{document}

\vfill
\end{center}
\end{figure}
%%%%%%%%%%%%%%%%%%%%%%%%%%%%%%%%%%%%%%%%%%%%%%%%%%%%%%%%%%%%
    
%%%% b %%%%
    \item  表には,ある電離真空計において,希ガスの真空度$1.0\times10^{-6}$Paのとき検出されるイオン電流値が,希ガス原子の種類別に示されている。このように希ガス原子の種類によってイオン電流値が異なる理由を定性的に説明せよ。\\
%%%%%%%%%%%%%%%%%%%%%%%%%%%%%%%%%%%%%%%%%%%%%%%%%%%%%%%%%%%%
\begin{table}[hbtp]
\begin{center}
%\caption[]{ \ilabel{ }}
\vspace{-1.0cm}
\begin{tabular}[h]{c||c|c|c|c|c}
気体                        &He     &Ne     &Ar     &Kr     &Xe     \\
\hline
イオン電流値~(nA)  &0.1    &0.3    &1.3    &1.7    &2.4    \\
\end{tabular}
\end{center}
\end{table}
%%%%%%%%%%%%%%%%%%%%%%%%%%%%%%%%%%%%%%%%%%%%%%%%%%%%%%%%%%%%
    
%%%% c %%%%
    \item  電離真空計では,フィラメントから放出された電子がグリッドに衝突する際に発生する軟X線が原因となってイオン電流以外の電流が流れ,正確な真空度が測定できない。どのような電流が発生するかを述べよ。また,この効果を減少させ,高い真空度まで測定できるようにするためには,フィラメント,グリッド,コレクターのうちどれかを工夫すればよい。どのようにすればこの効果を減少させることができるかを述べよ。\\
\end{enumerate}

%%%%%%【3.】%%%%%%
  \item  電離真空計はロータリーポンプで排気できる程度の真空では通常使用しない。その真空度で使用できる真空計をひとつ例にあげて,その動作原理を簡単に説明せよ。\\
  
\end{enumerate}



\end{question}

%%%%%【問題6(答)】%%%%%%%%%%%%%%%%%%%%%%%%%%%%%%%%%%%%%%%%%%%%%%%%%%%%%%%

\begin{answer}{問題6}{辰己創一}
\setcounter{equation}{0}



\begin{enumerate}

%%%%%%【1.】%%%%%%
  \item  
  \begin{enumerate}
  
%%%% a %%%%
    \item  一般に$pV=nRT$という関係により気体の圧力,体積,モル数(即ち分子数)温度は関係付けられている。ここで題意より常圧,$0\degC$における気体分子数の関係式が与えられているから,これを用いて$0\degC$,$1.0\times10^{-5}$Paにおける$1.0$cm$^3$中の気体分子の総数は,
\begin{equation}
    \frac{6.0\times 10^{23} \times 1.0 \times 10^{-5} \times 1.0}{1.0\times 10^5 \times 22 \times 10^3 } = 2.73 \times 10^9\nonumber
\end{equation}
より,$2.73 \times 10^9$個と求まった。\\
    
%%%% b %%%%
    \item  電離断面積と,分子の総数及び電子の総数の積が全電離分子数を与える。このことより,(a)の結果と,題に与えられている結果を用いて,
\begin{equation}
    \frac{1.0\times 10^{-3}}{1.6 \times 10^{-19}} \times 3.0\times 10^{-16} \times 2.73 \times 10^9 = 5.12 \times 10^9\nonumber
\end{equation}
となることから,答えは毎秒$5.12 \times 10^9$個と求まった。\\
    
\end{enumerate}

%%%%%%【2.】%%%%%%
  \item  
  \begin{enumerate}
    
%%%% a %%%%
    \item  フィラメントに電流を流すことにより,フィラメントは熱せられ,その表面から熱電子を発生させる。その熱電子を,フィラメントグリッド間に電圧をかけ,加速してやる。すると,その事により,熱電子は気体分子に衝突し,陽イオンを発生させる。ここで発生した陽イオンは陽極となるコレクターに細くされ,電流となって観測される。この時,フィラメント-グリッド間に流れる電子電流と,フィラメント-コレクター間に流れるイオン電流の間には単純な比例関係が成り立ち,また同様にイオン電流は真空計内部の圧力に比例する。これらのことより,電子電流を一定に保って測定を実施すれば,イオン電流を測定することは即ち圧力を測定することにつながり,この事を利用して真空計の圧力を測定するのである。\\
    
%%%% b %%%%
    \item  イオン電流の値はその気体の衝突断面積に左右される,衝突断面積は電子の運動エネルギーと気体の種類に依存するが,一義的には気体分子の大きさによると考えてよい。特にそれは希ガス元素の場合その反応性の乏しさから逆に顕著であると考えてよい。このことは大きな分子ほど衝突断面積が大きくなると考えてよく,これにより分子量が増えるほどイオン電流が大きくなることが予想でき,実際にそうなっていることが確かめられている。\\
    
%%%% c %%%%
    \item  グリッドに衝突する際,制動放射により放出されるX線はコレクターに到達した後に光電効果を起こしてコレクターの表面から電子を飛び出させる。このことが系に余計な電流を流すこととなり,誤差の原因となっている。このような誤差を小さくするにはコレクターの材質として電子を飛び出させるための仕事関数を出来るだけ大きくするようなものを選べばよい。その事により,光電効果により発生する電子の量を抑えることが出来,高真空度までの測定が可能となるのである。\\

\end{enumerate}

%%%%%%【3.】%%%%%%
  \item  電離真空計が使えないような高い圧力で使用できる圧力計としては、ピラニゲージがある。ピラニゲージの測定子は金属フィラメントであり、電流が流されて加熱されている。気体分子の衝突によって熱エネルギーが運び去られるが、圧力が高いほど失う熱エネルギーが多く、フィラメントの温度が下がる。温度の低下によって電気抵抗が下がるので、この変化を測定することで圧力を知ることができる。
\end{enumerate}


\end{answer}

\end{document}
