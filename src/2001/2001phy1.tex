\documentclass[fleqn]{jbook}
\usepackage{physpub}
\def\bm{\boldsymbol}
\def\ds{\displaystyle}

\begin{document}

%%%%%【問題1(問)】%%%%%%%%%%%%%%%%%%%%%%%%%%%%%%%%%%%%%%%%%%%%%%%%%%%%%%%

\begin{question}{問題1}{岡村}
\setcounter{equation}{0}

 ハミルトニアン
\begin{eqnarray}\ilabel{(1)}
H = \frac{1}{{2m}}\hat p^2  + \frac{{m\omega ^2 }}{2}\hat x^2 
\end{eqnarray}
で記述される1次元調和振動子を考えよう。
%% 原文では「1次元調和振動子」と全角の1で書いてあるが,半角に直した。
ここで$\hat p$は運動量演算子,$\hat x$は位置演算子。この系は演算子
\begin{eqnarray}
  a &=& \sqrt {\frac{{m\omega }}{{2\hbar }}} \hat x + \frac{{\rm i}}{{\sqrt {2m\hbar \omega } }}\hat p , \ilabel{(2)}\\ 
%%原文にはないが,ここに","を入れた。また虚数単位にはsolidのiを使った。
  & & \cr
  a^{\dagger}  &=& \sqrt {\frac{{m\omega }}{{2\hbar }}} \hat x - \frac{{\rm i}}{{\sqrt {2m\hbar \omega } }}\hat p \ilabel{(3)}
\end{eqnarray}
を用いて調べることが出来る。以下の問に,解答に至る筋道を添えて答えよ。\\

\begin{enumerate}
%%%%%%【1.】%%%%%%
  \item  $(\iref{(1)})$式のハミルトニアンを$a,a^{\dagger}$を用いて表せ。また,$a,a^{\dagger}$の物理的意味を述べよ。\\
  
%%%%%%【2.】%%%%%%
  \item  基底状態$\left| 0 \right\rangle$は関係$a\left| 0 \right\rangle = 0$を満たす。これを用いて座標表示の基底状態波動関数$\psi _0 \left( x \right)=\left\langle {x}|{0} \right\rangle $を求めよ。また,第一励起状態の波動関数$\psi _1 \left( x \right)$を求めよ。ここでは規格化は考えなくてよい。\\
  
%%%%%%【3.】%%%%%%
  \item  基底状態$\left| 0 \right\rangle$に演算子$a^{\dagger}$の指数関数をほどこして得られる状態
  \begin{equation}
\left| \alpha  \right\rangle  = \exp \left( {\alpha a^{\dagger} } \right)\left| 0 \right\rangle ,\qquad \alpha は任意の複素数
  \end{equation}
  はコヒーレント状態と呼ばれる。\\
  
  \begin{enumerate}
    %%%% a %%%%
    \item  コヒーレント状態が演算子$a$の固有状態になっていることを示せ。\\
    
    %%%% b %%%%
    \item  コヒーレント状態の間の内積$\left\langle {\beta ^ *}|{\alpha} \right\rangle $を計算せよ。
%% 原文ではここで改行が入っているが,ここでは改行しないことにした。
 ただし$\left\langle {\beta ^ *  } \right| = \left\langle 0 \right|\exp \left( {\beta ^ *  a} \right)$である。\\
    
    %%%% c %%%%
    \item  コヒーレント状態が$n$番目の励起状態を含む確率を求めよ。\\
    
    %%%% d %%%%
    \item  コヒーレント状態について不確定性関係を調べよう。座標の期待値$\left\langle {\hat x} \right\rangle  \equiv \left\langle {{\alpha ^ *}\left| {\hat x} \right|{\alpha}} \right\rangle / \left\langle {\alpha ^ *}|{\alpha} \right\rangle $と座標の二乗の期待値$\left\langle {\hat x} \right\rangle  \equiv \left\langle {{\alpha ^ *}\left| {\hat x^2} \right|{\alpha}} \right\rangle / \left\langle {\alpha ^ *}|{\alpha} \right\rangle $を計算し,座標の不確定さの二乗
 \begin{equation}
\left( {\Delta x} \right)^2  = \left\langle {\left( {\hat x - \left\langle {\hat x} \right\rangle } \right)^2 } \right\rangle 
 \end{equation}
 を求めよ。\\
    
    %%%% e %%%%
    \item  同様に運動量の不確定さの二乗$\left( {\Delta p} \right)^2  = \left\langle {\left( {\hat p - \left\langle {\hat p} \right\rangle } \right)^2 } \right\rangle $を計算し,不確定性関係を確かめよ。\\
  \end{enumerate}
  
\end{enumerate}

\end{question}

%%%%%【問題1(答)】%%%%%%%%%%%%%%%%%%%%%%%%%%%%%%%%%%%%%%%%%%%%%%%%%%%%%%%

\begin{answer}{問題1}{岡村}
\setcounter{equation}{0}
\begin{enumerate}

%%%%%%【1.】%%%%%%
  \item  $\ds a = \sqrt {\frac{{m\omega }}{{2\hbar }}} \hat x + \frac{{\rm i}}{{\sqrt {2m\hbar \omega } }}\hat p$,$\ds a^{\dagger} = \sqrt {\frac{{m\omega }}{{2\hbar }}} \hat x - \frac{{\rm i}}{{\sqrt {2m\hbar \omega } }}\hat p$を$\hat x$,$\hat p$について解くと
  \begin{eqnarray}
  \hat x &=& \sqrt {\frac{\hbar }{{2m\omega }}} \left( {a^{\dagger}  + a} \right) ,  \\ 
  & & \cr
  \hat p &=& {\rm i}\sqrt {\frac{{m\hbar \omega }}{2}} \left( {a^{\dagger}  - a} \right)  
  \end{eqnarray}
  
  であるから,これらを与えられたハミルトニアンに代入して,
  \begin{eqnarray}
  H &=& \frac{1}{{2m}}\hat p^2  + \frac{{m\omega ^2 }}{2}\hat x^2  \cr 
  & & \cr
   &=& \frac{1}{{2m}}\left[ {{\rm i}\sqrt {\frac{{m\hbar \omega }}{2}} \left( {a^{\dagger}  - a} \right)} \right]^2  + \frac{{m\omega ^2 }}{2}\left[ {\sqrt {\frac{\hbar }{{2m\omega }}} \left( {a^{\dagger}  + a} \right)} \right]^2  \cr 
   & & \cr
%%   &=& \frac{{\hbar \omega }}{4}\left[ { - \left( {a^{\dagger}  - a} \right)^2  + \left( {a^{\dagger}  + a} \right)^2 } \right] \cr 
%%   & & \cr
   &=& \frac{{\hbar \omega }}{2}\left( {aa^{\dagger}  + a^{\dagger} a} \right) . 
  \end{eqnarray}
  
  交換関係$\left[ {\hat x,\hat p} \right] = {\rm i}\hbar $より従う$a$と$a^{\dagger}$の間の交換関係:
  \begin{equation}
\left[ {a,a^{\dagger} } \right] = 1 \iff aa^{\dagger}  = 1 + a^{\dagger} a 
  \end{equation}
  
  を使えば,ハミルトニアンは
  \begin{equation}
H = \hbar \omega \left( {\frac{1}{2} + a^{\dagger} a} \right) 
  \end{equation}
  
  の形に書かれる。\\
   $a,a^{\dagger}$の物理的意味については次の通り。調和振動子の$n~(=1,2,\dots)$番目のエネルギー固有値を$E_n$,これに属する固有状態を$\left| n \right\rangle$としたとき,$\left| n \right\rangle$に$a$を作用させた状態$a\left| n \right\rangle$は,固有値$E_n-\hbar\omega$を持った調和振動子のエネルギー固有状態となり,また$\left| n \right\rangle$に$a^{\dagger}$を作用させた状態$a^{\dagger}\left| n \right\rangle$は,固有値$E_n+\hbar\omega$を持った調和振動子のエネルギー固有状態となる。このように調和振動子のエネルギー固有状態に作用したとき,エネルギー$\hbar\omega$を持つ量子をそれぞれ消滅,生成する意味で,$a,a^{\dagger}$はそれぞれ消滅演算子,生成演算子と呼ばれる。また,ハミルトニアンの表式$(5)$に現れた$a^{\dagger}a \equiv \hat N$は個数演算子と呼ばれ,$\hat N \left(a^{\dagger}\right)^n \left| 0 \right\rangle= n\left(a^{\dagger}\right)^n \left| 0 \right\rangle$が成り立つことから,$\hat N$の固有値は調和振動子の基底状態の上に励起しているエネルギー量子$\hbar\omega$の数を表す。\\
  
%% $a,a^{\dagger}$にそれぞれ$\sqrt{\hbar}$を掛けて得られる$a\sqrt{\hbar},a^{\dagger}\sqrt{\hbar}$は,1次元調和振動子のエネルギー固有関数に作用させたとき,$a\sqrt{\hbar}$の場合は元のエネルギー固有値よりも$\hbar \omega$だけ低い値を,また$a^{\dagger}\sqrt{\hbar}$の場合は$\hbar \omega$だけ高い値を,それぞれエネルギー固有値として持つ固有関数へと変換する演算子であり,$a,a^{\dagger}$はそれぞれ消滅演算子,生成演算子と呼ばれる。
  
  
%%%%%%【2.】%%%%%%
  \item  $a\left| 0 \right\rangle  = 0$より$\left\langle {x\left| a \right|0} \right\rangle  = 0$。これを計算して,
  \begin{eqnarray}
   & & \left\langle {x\left| {\sqrt {\frac{{m\omega }}{{2\hbar }}} \hat x + \frac{{\rm i}}{{\sqrt {2m\hbar \omega } }}\hat p} \right|0} \right\rangle  = 0 \cr 
   & & \cr
   & & \iff \left[ {\sqrt {\frac{{m\omega }}{{2\hbar }}} x + \frac{{\rm i}}{{\sqrt {2m\hbar \omega } }}\left( {\frac{\hbar }{{\rm i}}\frac{{\rm d}}{{{\rm d}x}}} \right)} \right]\psi _0 \left( x \right) = 0 \cr 
   & & \cr
   & & \iff \left( {\frac{{m\omega }}{\hbar }x + \frac{{\rm d}}{{{\rm d}x}}} \right)\psi _0 \left( x \right) = 0. 
  \end{eqnarray}
  
  これより基底状態の波動関数は
  \begin{equation}
  \psi _0 \left( x \right) = {\rm Const.} \exp \left( { - \frac{{m\omega }}{{2\hbar }}x^2 } \right)  
  \end{equation}
  
  と書ける\footnote{規格化するならば,$\left\langle 0|0\right\rangle = \int \left| \psi_0\left(x\right) \right|^2=1$より{\rm Const.}を求めればよい。結果は
 \begin{equation}
 \psi_0 \left(x\right) = \left( \pi^{1/4}\sqrt{\eta} \right)^{-1}\exp\left[ -x^2/\left( 2\eta^2 \right) \right] 
 \end{equation}
 と書ける。ここに調和振動子の長さのスケールを決めるパラメータ$\eta  \equiv \sqrt {\hbar /\left( {m\omega } \right)} $を導入した。\\
 
 }。\\
 また第一励起状態は$\left| 1 \right\rangle  \propto a^{\dagger} \left| 0 \right\rangle $として得られるから,規格化を考えなければ第一励起状態の波動関数は
  \begin{eqnarray}
  \psi _1 \left( x \right) &=& \left\langle {x}|{1} \right\rangle \propto \left\langle {x\left| {\sqrt {\frac{{m\omega }}{{2\hbar }}} \hat x - \frac{{\rm i}}{{\sqrt {2m\hbar \omega } }}\hat p} \right|0} \right\rangle  \cr 
  & & \cr
   &=& \left[ {\sqrt {\frac{{m\omega }}{{2\hbar }}} x - \frac{{\rm i}}{{\sqrt {2m\hbar \omega } }}\left( {\frac{\hbar }{{\rm i}}\frac{{\rm d}}{{{\rm d}x}}} \right)} \right]\psi _0 \left( x \right) \cr 
   & & \cr
   &=& \sqrt {\frac{{2m\omega }}{\hbar }} x\exp \left( { - \frac{{m\omega }}{{2\hbar }}x^2 } \right) = {\rm Const.}x\exp \left( { - \frac{{m\omega }}{{2\hbar }}x^2 } \right) 
  \end{eqnarray}
  
  と求められる\footnote{先程の$\eta$を用いれば規格化した形で
  \begin{equation}
  \psi _1 \left( x \right) = \left( \pi^{1/4}\sqrt{\eta}^{3/2}/\sqrt{2} \right)^{-1}x\exp\left[ -x^2/\left( 2\eta^2 \right) \right] 
  \end{equation}
  
    と書ける。一般に第$n$励起状態の波動関数$\psi_n\left( x \right)$は,$\xi\equiv x/\eta$,$n$次のエルミート多項式$H_n\left( \xi \right)$を用いて,規格化した形で
    \begin{equation}
    \psi\left( \xi \right) = \left( 2^n n! \right)^{-1/2}\left[ m\omega/\left( \pi\hbar \right) \right]^{1/4}\exp\left( -\xi^2/2 \right)H_n\left( \xi \right) 
    \end{equation}
    
    と書ける。$H_0\left( \xi \right)=1$,$H_1\left( \xi \right)=2\xi$,$H_2\left( \xi \right)=4\xi^2-2$,...。}。\\
  
%%%%%%【3.】%%%%%%
  \item  
  \begin{enumerate}
  %%%%%% 3.-(a) %%%%%%
    \item  $\left| \alpha  \right\rangle  = \exp \left( {\alpha a^{\dagger} } \right)\left| 0 \right\rangle$の両辺に左から$a$を作用させる:
    \begin{eqnarray}
  a\left| \alpha  \right\rangle  &=& a\exp \left( {\alpha a^{\dagger} } \right)\left| 0 \right\rangle  = a\sum\limits_{n = 0}^\infty  {\frac{{\left( {\alpha a^{\dagger} } \right)^n }}{{n!}}} \left| 0 \right\rangle  \cr 
  & & \cr
   &=& \sum\limits_{n = 0}^\infty  {\frac{{\alpha ^n }}{{n!}}\left[ {\left( {a^{\dagger} } \right)^n a + n\left( {a^{\dagger} } \right)^{n - 1} } \right]} \left| 0 \right\rangle  \cr 
   & & \cr
   &=& \alpha \sum\limits_{n = 1}^\infty  {\frac{{\left( {\alpha a^{\dagger} } \right)^{n - 1} }}{{\left( {n - 1} \right)!}}} \left| 0 \right\rangle  = \alpha \exp \left( {\alpha a^{\dagger} } \right)\left| 0 \right\rangle  \cr 
   & & \cr
   &=& \alpha \left| \alpha  \right\rangle .
    \end{eqnarray}
    
    従ってコヒーレント状態は消滅演算子$a$の固有値$\alpha$に属する固有状態である。\\
    
  %%%%%% 3.-(b) %%%%%%
    \item  $a\left| \alpha  \right\rangle  = \alpha \left| \alpha  \right\rangle $の両辺に第$n$励起状態のケットを作用させて,
    \begin{equation}
\left\langle {n\left| a \right|\alpha } \right\rangle  = \left\langle {n\left| \alpha  \right|\alpha } \right\rangle  \iff \sqrt {n + 1} \left\langle {{n + 1}}|{\alpha } \right\rangle  = \alpha \left\langle {n}|{\alpha } \right\rangle . 
    \end{equation}
    
 これより,
 \begin{equation}
\left\langle {n}|{\alpha } \right\rangle  = \frac{\alpha }{{\sqrt n }}\frac{\alpha }{{\sqrt {n - 1} }}\frac{\alpha }{{\sqrt {n - 2} }} \cdots \frac{\alpha }{{\sqrt 1 }}\left\langle {0}|{\alpha } \right\rangle  = \frac{{\alpha ^n }}{{\sqrt {n!} }}\left\langle {0}|{\alpha } \right\rangle  
 \end{equation}
 
 であるが,ここに
 \begin{equation}
 \left\langle {0}|{\alpha } \right\rangle  = \left\langle {0\left| {\exp \left( {\alpha a^{\dagger} } \right)} \right|0} \right\rangle  = \left\langle {0\left| {\sum\limits_{n = 0}^\infty  {\frac{{\left( {\alpha a^{\dagger} } \right)^n }}{{n!}}} } \right|0} \right\rangle  = \left\langle {0\left| 1 \right|0} \right\rangle  = 1 
 \end{equation}
 
 であるから,$\ds \left\langle {n}|{\alpha } \right\rangle  = \frac{{\alpha ^n }}{{\sqrt {n!} }}$となる。従ってフォック状態の基底$\{\left| n  \right\rangle\}_{n=0,1,2,\dots}$による$\left| \alpha  \right\rangle $の展開は,
 
 \begin{equation}
\left| \alpha  \right\rangle  = \sum\limits_n {\left| n \right\rangle \left\langle {n}|{\alpha } \right\rangle }  = \sum\limits_n {\frac{{\alpha ^n }}{{\sqrt {n!} }}} \left| n \right\rangle  
 \end{equation}
 
 となる。同様にして$\ds \left\langle {\beta ^ *  } \right| = \sum\limits_m {\frac{{\left( {\beta ^ *  } \right)^m }}{{\sqrt {m!} }}} \left\langle m \right|$であるから,これらの内積は,
 
 \begin{equation}
\left\langle {{\beta ^ *  }}|{\alpha } \right\rangle  = \sum\limits_{n,m} {\frac{{\left( {\beta ^ *  } \right)^m }}{{\sqrt {m!} }}\frac{{\alpha ^n }}{{\sqrt {n!} }}} \left\langle {m}|{n} \right\rangle  = \sum\limits_n {\frac{{\left( {\beta ^ *  \alpha } \right)^n }}{{n!}}}  = \exp \left( {\beta ^ *  \alpha } \right) 
 \end{equation}
 
 と計算される。\\
 
  %%%%%% 3.-(c) %%%%%%
    \item $|\alpha\rangle$の規格化を考えると、3-(b)の解答で$\beta=\alpha$として、$\langle\alpha|\alpha\rangle=\exp(|\alpha|^2)$であるから、規格化されたフォック状態は
    \begin{eqnarray*}
    |\alpha_{normal}\rangle=\frac{1}{\sqrt{\exp(|\alpha|^2)}}\sum_n {\frac{{\alpha ^n }}{{\sqrt {n!} }}} \left| n \right\rangle
    \end{eqnarray*}
    と表される。従って、コヒーレント状態が第$n$励起状態を含む確率$\left| P_n \right|$は,
    \begin{equation}
P_n  = \left| {\left\langle {n}|{\alpha_{normal} } \right\rangle } \right|^2  = \frac{1}{\exp(|\alpha|^2)} \frac{{\left| \alpha  \right|^{2n} }}{{n!}} 
    \end{equation}
    で与えられる。\\
%% $\left\langle n \right\rangle  \equiv \left\langle {\alpha ^ *  \left| {\hat N} \right|\alpha } \right\rangle  = \left\langle {\alpha ^ *  \left| {a^{\dagger} a} \right|\alpha } \right\rangle  = \left| \alpha  \right|^2 $と書ける。

  %%%%%% 3.-(d) %%%%%%
    \item  座標の期待値$\left\langle {\hat x} \right\rangle$,座標の二乗の期待値$\left\langle {\hat x^2} \right\rangle$はそれぞれ,
    
    \begin{eqnarray}
\left\langle {\hat x} \right\rangle &=& \left\langle {\alpha ^ *  \left| {\hat x} \right|\alpha } \right\rangle / \left\langle {\alpha ^ *}|{\alpha} \right\rangle = \left\langle {\alpha ^ *  \left| {\sqrt {\frac{\hbar }{{2m\omega }}} \left( {a^{\dagger}  + a} \right)} \right|\alpha } \right\rangle  / \left\langle {\alpha ^ *}|{\alpha} \right\rangle \cr
 & & \cr
 &=& \sqrt {\frac{\hbar }{{2m\omega }}} \left( {\alpha ^ *   + \alpha } \right) ,\\
  & & \cr
  \left\langle {\hat x^2} \right\rangle &=& \left\langle {\alpha ^ *  \left| {\hat x^2 } \right|\alpha } \right\rangle  / \left\langle {\alpha ^ *}|{\alpha} \right\rangle  = \left\langle {\alpha ^ *  \left| {\frac{\hbar }{{2m\omega }}\left( {a^{\dagger}  + a} \right)^2 } \right|\alpha } \right\rangle  / \left\langle {\alpha ^ *}|{\alpha} \right\rangle  \cr 
  & & \cr
   &=& \frac{\hbar }{{2m\omega }}\left\langle {\alpha ^ *  \left| {\left\{ {\left( {a^{\dagger} } \right)^2  + a^{\dagger} a + \left( {1 + a^{\dagger} a} \right) + a^2 } \right\}} \right|\alpha } \right\rangle  / \left\langle {\alpha ^ *}|{\alpha} \right\rangle  \cr 
   & & \cr
   &=& \frac{\hbar }{{2m\omega }}\left[ {\left( {\alpha ^ *  } \right)^2  + 2\left| \alpha  \right|^2  + \alpha ^2  + 1} \right] \cr
   & & \cr
   &=& \frac{\hbar }{{2m\omega }}\left[ {\left( {\alpha ^ *   + \alpha } \right)^2  + 1} \right] \\\nonumber
 \end{eqnarray}
 
 と計算される。$(20)$の計算途中で交換関係$(4)$を使った。これらより座標の不確定さ${\Delta x}$の二乗は
 
 \begin{eqnarray}
 \left( {\Delta x} \right)^2  &=& \left\langle {\left( {\hat x - \left\langle {\hat x} \right\rangle } \right)^2 } \right\rangle = \left\langle {\hat x^2 } \right\rangle  - \left\langle {\hat x} \right\rangle ^2 \cr
 & & \cr
 &=& \frac{\hbar }{{2m\omega }}\left[ {\left( {\alpha ^ *   + \alpha } \right)^2  + 1} \right] - \left[ \sqrt {\frac{\hbar }{{2m\omega }}} \left( {\alpha ^ *   + \alpha } \right) \right]^2 \cr
 & & \cr
 &=&\frac{\hbar }{{2m\omega }} 
 \end{eqnarray}
 と求められる。\\
 
  %%%%%% 3.-(e) %%%%%%
    \item  前問(d)と同様にして,運動量の期待値$\left\langle {\hat p} \right\rangle$,運動量の二乗の期待値$\left\langle {\hat p^2} \right\rangle$はそれぞれ,
    
    \begin{eqnarray}
  \left\langle {\hat p} \right\rangle  &=& \left\langle {\alpha ^ *  \left| {\hat p} \right|\alpha } \right\rangle /\left\langle {{\alpha ^ *  }}|{\alpha } \right\rangle  = \left\langle {\alpha ^ *  \left| {{\rm i}\sqrt {\frac{{m\hbar \omega }}{2}} \left( {a^{\dagger}  - a} \right)} \right|\alpha } \right\rangle /\left\langle {{\alpha ^ *  }}|{\alpha } \right\rangle  \cr 
  & & \cr
   &=& {\rm i}\sqrt {\frac{{m\hbar \omega }}{2}} \left( {\alpha ^ *   - \alpha } \right), \\ 
   & & \cr
  \left\langle {\hat p^2 } \right\rangle  &=& \left\langle {\alpha ^ *  \left| {\hat p^2 } \right|\alpha } \right\rangle /\left\langle {{\alpha ^ *  }}|{\alpha } \right\rangle  = \left\langle {\alpha ^ *  \left| { - \frac{{m\hbar \omega }}{2}\left( {a^{\dagger}  - a} \right)^2 } \right|\alpha } \right\rangle /\left\langle {{\alpha ^ *  }}|{\alpha } \right\rangle  \cr 
  & & \cr
   &=&  - \frac{{m\hbar \omega }}{2}\left\langle {\alpha ^ *  \left| {\left\{ {\left( {a^{\dagger} } \right)^2  - a^{\dagger} a - \left( {1 + a^{\dagger} a} \right) + a^2 } \right\}} \right|\alpha } \right\rangle /\left\langle {{\alpha ^ *  }}|{\alpha } \right\rangle  \cr 
   & & \cr
   &=&  - \frac{{m\hbar \omega }}{2}\left[ {\left( {\alpha ^ *  } \right)^2  - 2\left| \alpha  \right|^2  + \alpha ^2  - 1} \right] \cr 
   & & \cr
   &=& \frac{{m\hbar \omega }}{2}\left[ {1 - \left( {\alpha ^ *   - \alpha } \right)^2 } \right] \\\nonumber
    \end{eqnarray}
    
    と計算される。これらより,運動量の不確定さ${\Delta p}$の二乗は
    
    \begin{eqnarray}
 \left( {\Delta p} \right)^2  &=& \left\langle {\left( {\hat p - \left\langle {\hat p} \right\rangle } \right)^2 } \right\rangle = \left\langle {\hat p^2 } \right\rangle  - \left\langle {\hat p} \right\rangle ^2 \cr
 & & \cr
 &=& \frac{{m\hbar \omega }}{2}\left[ {1 - \left( {\alpha ^ *   - \alpha } \right)^2 } \right] - \left[ {\rm i}\sqrt {\frac{{m\hbar \omega }}{2}} \left( {\alpha ^ *   - \alpha } \right) \right]^2 \cr
 & & \cr
 &=&\frac{{m\hbar \omega }}{2}\\\nonumber
 \end{eqnarray}
 
 と求められる。以上より,不確定性関係は
 \begin{equation}
\left( {\Delta x} \right)^2 \left( {\Delta p} \right)^2  = \frac{{\hbar ^2 }}{4} 
\iff \Delta x\Delta p = \frac{\hbar }{2} 
 \end{equation}
 
 となり,コヒーレント状態は,座標とその正準共役な運動量とについて,両者の不確定性の積を最小にする状態であることが判る。(つまり古典的描像に最も近い。)\\
 
  \end{enumerate}
  
\end{enumerate}

\end{answer}

\end{document}
