\documentclass[fleqn]{jbook}
\usepackage{physpub}

\begin{document}

\begin{question}{専攻 問題6}{}
真空中では点電荷の間の静電相互作用はクーロンの法則に従うので、
互いに$3\mbox{\AA}$離れた一価の電荷の間には、約$460\Unit{kJ/mol}(=110\Unit{kcal/mol})$
もの静電エネルギーが働く。しかし核酸やタンパク質などの
生体分子は水中に存在するので、$(\hspace {10mm})$ のために、
静電エネルギーは真空中の約$1/80$に減少する。また生理的条件の下では、
水中に多くの電解質イオンが存在するため、
静電ポテンシャルはこれらの遮蔽効果により、さらに減少する。
Debye-H\"ukelの理論によれば、このようなイオンの遮蔽効果を考慮したとき、
位置$\Vec r$における静電ポテンシャル$\phi(\Vec r)$は
\begin{equation}
\nabla^2\phi(\Vec r) 
= -\frac{1}{\varepsilon}\sum_{i}n_i Z_i q\cdot
\exp\left[- \frac{Z_i q \phi(\Vec r)}{kT} \right] \eqname{Q1}
\end{equation}
で与えられる。ここに和は異なるイオン種についてとるものとし、
$\varepsilon$は電解質溶液の誘電率、$q$は素電荷、$k$はBoltzmann定数、
$T$は絶対温度、そして$n_i$と$Z_i$はそれぞれ第$i$種イオンの濃度(数密度)
と電荷数(イオン価)を表す。以下の問いに答えよ。

\begin{subquestions}

\SubQuestion
上の$(\hspace {10mm})$の中を埋めよ。

\SubQuestion
上の式\eqhref{Q1}は、Poisson-Boltzmannの方程式と呼ばれる。
この式はどのようにして導かれるかを説明せよ。

\SubQuestion
静電エネルギーが熱エネルギーよりも十分に小さいときは、
式\eqhref{Q1}は線形微分方程式
\begin{equation}
\nabla^2\phi(\Vec r) = \kappa^2\phi(\Vec r) \eqname{Q2}
\end{equation}
で表されることを示せ。
ただし$\kappa^2\equiv(q^2/\varepsilon kT)\sum_{i}n_i Z_i^2$とする。

\SubQuestion
原点に電荷数$Z_0$の中心イオンをおき、
他のイオンはその周りに球対称に分布していると仮定したとき、
式\eqhref{Q2}の解を求めよ。またイオン濃度をゼロに近づけたとき、
その解はどのようになるか。
ただし、各イオンの大きさは無視できるほど小さいとする。

\SubQuestion
設問4で、イオン濃度がゼロのときの静電ポテンシャルを$\phi_0(\Vec r)$
とする。$\phi(\Vec r)$が$\phi_0(\Vec r)$の$1/e(e = 2.7183..)$
となるときの$r$をDebye長(またはイオン雰囲気の厚さ)という。
温度が25${}^{\circ}$Cで、イオン強度$(= \frac{1}{2}\sum_{i}n_iZ_i^2)$
が生理的条件下の値$8.4\times10^{25}\mathrm{m^{-3}(=0.14 mol/liter)}$
に等しい電解質のDebye長を求めよ。ただし水溶液の誘電率を
$\varepsilon = 7.0\times10^{-10}\mathrm{Fm^{-1}}$とし、
$q = 1.6\times 10^{-19}$Cおよび
$k= 1.38\times10^{-23}\mathrm{JK^{-1}}$を、用いてよい。

\SubQuestion
二本の鎖よりなるDNAは、負に帯電したリン酸基をもち、
リン酸基どうしの間隔は約$10\mbox{\AA}$である。
これらの電荷が水溶液中のイオンにより遮蔽されることは、
DNAの二重螺旋構造を安定化する上で重要である。
いまDNA水溶液中のイオン強度を生理的条件下の値より
さらに小さくしていったとき、DNAが熱融解する温度は、
どのように変化すると期待されるか。理由を付して述べよ。

\SubQuestion
簡単なDebye-h\"ukelの理論だけでは、
核酸やタンパク質など生体分子のもつ立体構造の安定性を、
定量的に説明できないことが知られている。
これらの生体分子の立体構造安定性に
大きく寄与している他の要因を二つ以上あげ、
それらについて知るところを述べよ。

\end{subquestions}
\end{question}
\begin{answer}{専攻 問題6}{}

\begin{subanswers}
\SubAnswer

水の誘電分極

\SubAnswer

位置$\Vec r$の静電ポテンシャルは$\phi(\Vec r)$だから、この点で
イオンが電荷$Z_i q$を持つ確率はBoltzmann分布
\[ e^{-Z_i q \phi(\Vec r)/kT} \]
に従う。
それゆえ、位置$\Vec r$での第i種イオンの密度は
\[ n_i \exp\left[- \frac{Z_i q \phi(\Vec r)}{kT} \right] \]
となる。それゆえ、電荷密度は
\[ \rho(\Vec r) = \sum_i n_i Z_i q \exp \left[ - \frac{Z_i q \phi(\Vec r)}
{kT} \right]. \]
これとPoisson方程式
\[ \nabla^2 \phi( \Vec r) = - \frac{\rho(\Vec r)}{\varepsilon} \]
より、件のPoisson-Boltzmann方程式
\[
\nabla^2 \phi(\Vec r) = -\frac{1}{\varepsilon} \sum_i n_i Z_i q
\exp \left[ - \frac{Z_i q \phi(\Vec r)}{kT} \right]
\]
を得る。

\SubAnswer

 溶液全体は電気的に中性だから、
\[ \sum_i n_i Z_i q = 0.\]
となる。さらに、
\[\left| \frac{Z_i q \phi(\Vec r)}{kT} \right|  << 1\]
となることに注意して、
\begin{eqnarray*}
\nabla^2 \phi(\Vec r) & \approx &-\frac{1}{\varepsilon} \sum_i
n_i Z_i q \left[ 1 - \frac{Z_i q \phi(\Vec r)}{kT} \right] \\
&=& -\frac{1}{\varepsilon} \sum_i n_i Z_i q + \frac{q^2}{\varepsilon kT}
\phi(\Vec r) \sum_i n_i {Z_i}^2 \\
&=&  \frac{q^2}{\varepsilon kT} \sum_i n_i {Z_i}^2 \phi(\Vec r) \\
&=& \kappa^2 \phi(\Vec r)
\end{eqnarray*}
を得る。

\SubAnswer

球対称であるから、$ \phi(\Vec r) = \phi(r).$
この時、方程式は
\[
\nabla^2 \phi(\Vec r) = \frac{1}{r^2} \frac{\d}{\d r} \left(
r^2 \frac{\d \phi(r)}{\d r} \right)
\]
を使って
\[
\frac{1}{r^2} \frac{\d}{\d r} \left( r^2 \frac{\d \phi(r)}
{\d r} \right) = \kappa^2 \phi(r).
\]

一般に中性ではあっても媒質中を荷電粒子が運動する場合、周囲の電荷を
引きずっているために余分に質量を纏っているように見える。これは
電磁相互作用を運ぶ光子についても同様で、ポテンシャルは質量を持った
光子によって媒介されるように見える。
これはYukawa型相互作用だから、方程式を最初から解かずに、
\[ \phi(r) = a \frac{e^{-\alpha r}}{r} \]
とおくことができる。ここで、$ a \propto Z_0, \alpha > 0$である。
これを方程式に代入すると、
\begin{eqnarray*}
a \alpha^2 \frac{e^{-\alpha r}}{r} &=& \kappa^2 a \frac{e^{-\alpha r}}{r} \\
\alpha^2 &=& \kappa^2 \\
\alpha &=& \kappa. 
\end{eqnarray*}

を得る。更に、上の述べた考察により、$n_i \rightarrow 0$のとき、
イオンの衣による遮蔽効果は小さくなり、$\phi(r)$はCoulomb相互作用に
近づく。つまり、
\[ \phi(r) \rightarrow \frac{1}{4 \pi \varepsilon} \frac{Z_0}{r}. \]
であるから、これにより、
\[ a = \frac{Z_0}{4 \pi \varepsilon} \]
がわかる。まとめると、
\[ \phi(r) = \frac{Z_0}{4 \pi \varepsilon} \frac{e^{-\kappa r}}{r}.\]
$n_i \rightarrow 0$のとき、$\kappa \rightarrow 0$であって、
\[ \phi(r) \rightarrow \frac{1}{4 \pi \varepsilon} \frac{Z_0}{r}\]
となり、Coulomb相互作用になる。

\SubAnswer

上の考察から、
\begin{eqnarray*}
\phi_0(r) &=& \frac{Z_0}{4 \pi \varepsilon} \frac{1}{r} \\
\phi(r) &=& \frac{Z_0}{4 \pi \varepsilon} \frac{e^{-\kappa r}}{r} \\
\frac{\phi(r)}{\phi_0(r)} &=& e^{-\kappa r}.
\end{eqnarray*}
それゆえ、Debye長は
\begin{eqnarray*}
r_D &=& \frac{1}{\kappa} 
= \frac{1}{\sqrt{\displaystyle{\frac{q^2}{\varepsilon kT}} \sum_i n_i {Z_i}^2} }\\
&=& \frac{1}{\sqrt{\displaystyle{\frac{2 q^2}{\varepsilon kT}}}} \frac{1}{\sqrt{
\displaystyle{\frac{1}{2}}{\
\sum_i n_i {Z_i}^2}}} \\
&\approx& 8.18 \times 10^{-10} \approx 8.2 \mbox{\AA}
\end{eqnarray*}

\SubAnswer
イオン強度を下げていくと
\[ r_D \propto \frac{1}{\sqrt{イオン強度}} \]
であるから、$r_D$は大きくなる。ここで、生理条件下では$r_D \approx 8.2$\AA
で、DNAのリン酸基の間隔は10\AA 程度であるから、イオン強度を下げていくと
容易にDebye長は10\AA よりも大きくなり、互いに反発力を及ぼし、DNAは
不安定になる。それゆえ、DNAが熱融解する温度は下がると期待される。

\SubAnswer
\begin{itemize}
\item{イオンの大きさ}

Debye-H\"uckelの理論ではイオンの大きさを無視して方程式を解いたが、
実際はイオンは数\AA 程度の大きさを持つのでイオンの占めることができる
は空間的領域は制限される。

\item{熱力学的要素}

Debye-H\"uckelの理論はイオン同士の静電エネルギーのみを考慮したが、
実際はエントロピーとエネルギーの兼ね合いによって構造が決まる。
つまり、Gibbs自由エネルギーが最小になるように原子が配置される。

\item{水素結合、van der Waals相互作用}

分子を構成している原子間の電気陰性度の差により偏極がおこり、
そのために水素結合やvan der Waals力が働き立体構造を変化させる。
なお、この力は分子を構成する原子のうち、互いに接近した組に対して
働く。

\end{itemize}


\end{subanswers}
\end{answer}

\end{document}