%% -*- coding:sjis -*-
%%
%% 2013-07-17, Koichi Murase, inshi20phys-ochi.tex より転載、編集。
%%
\begin{answer}{第1問}{ochi}
※使い方の注意\\
脚注は気になった人のためのものなので、基本的には本文だけ読めば事足ります。
\begin{enumerate}
\item
  順にT,F,T,F,T(T=True, F=False)\\

  解説は略します。今考えているSO(3)の定義が「直交かつdet=1」なことに
  注意すれば大丈夫です。(b)は回転軸変えればだいたい非可換になるし、
  (d)の反例としては鏡映など。\\

\item
  z軸を中心とした回転なので、普通に2次元の回転行列を考えます。これを$\Omega_0$とおいときます。
  \begin{equation}
  \Omega_0(\theta)=
  \begin{pmatrix}
  \cos \theta & -\sin \theta & 0\\
  \sin \theta & \cos \theta & 0\\
  0 & 0 & 1\\
  \end{pmatrix}
  \end{equation}
  固有値の出し方はいいでしょう。exp($\pm$i$\theta$),$1$。あと$(0,0,1)=\mathbf{n_0}$としときます。\\

\item
  固有値は上と同じです。(任意の$\mathbf{n}$に対して、その方向をz軸の方向として
  基底を変換すれば(1)の話に帰着するので。
  基底の変換(もとい座標の向きの決め方)に対して固有値は
  不変です。\footnote{もしもっと形式的に議論をしたければ、例えば$(0,0,1)=\mathbf{n_0}$
  を$\mathbf{n}$に
  回転させてもっていく行列を$R$とします。すると(1)の行列を$\Omega_0$,(2)の行列を$\Omega$と
  したとき、$\Omega=R\Omega_0 R^{-1}$となることがわかります。(右辺は右から順に、
  「回転軸を$\mathbf{n}$から$(0,0,1)$に動かした」「まわした」「戻した」(結局、二つの回転軸
  のまわりの世界を$R$や$R^{-1}$で行ったり来たりできると思えば、わかるはずです。))あとは
  すぐわかるでしょう。例えば$\Omega_0 v=\lambda v$なら$R\Omega_0 R^{-1} (R v)=\lambda (R v)$と
  なるので、$\Omega$も固有値$\lambda$を持ち、逆も同様なので二つの行列の固有値の集合は一致。
  $det[ \Omega-\lambda I]=det[R]det[\Omega-\lambda I]det[R^{-1}]=
  det[ \Omega_0 - \lambda I]$でもいいでしょう。})

\item
  単純計算でも絵を描いても出来ますが、面倒なので楽に。
  まず、求めたい式の両辺に$\Omega(\mathbf{n},-\theta)$を掛けて
  \begin{equation}
    (\Omega(\mathbf{n}, \Delta \theta) -I)\mathbf{v_0}=
    \mathbf{n}\times \mathbf{v_0} \Delta \theta
  \end{equation}
  とします。\footnote{右辺では、ベクトルの外積が回転行列$\Omega$に対して$\Omega
  (\mathbf{a}\times\mathbf{b})
  =(\Omega\mathbf{a})\times(\Omega\mathbf{b})$のように、要はベクトル的に変換することを使いました。
  ($\mathbf{n}$
  は$\mathbf{n}$まわりでいくら回転させても$\mathbf{n}$のままなのに注意。)
  ベクトルの外積の意味
  を考えても当たり前ですが、数式的にこれを証明するには、
  テンソルの記法で$\Omega_{ij}a_kb_l\epsilon_{klj}=\epsilon_{imn}\Omega_{mp}a_p\Omega_{nq}b_q$を
  示せばできます。(テンソル苦手なんで、全部下つきの添え字で書いちゃいました。)これを示すには、
  右辺に$\epsilon_{imn}\Omega_{mp}\Omega_{nq}=\Omega_{ir}\epsilon_{pqr}$
  (直交行列の各列ベクトルは正規直交基底を
  成すので、第p列と第q列のベクトルの外積の第i成分を左辺が示すと思えば。例えば$p=1,q=2$なら
  それは第3列のベクトルの第i成分ですよね。)を代入します。}
  さらに脚注1で書いた$R^{-1}$を左から掛けて(その脚注にあるように$\Omega=R\Omega_0 R^{-1}$に注意。)
  \begin{equation}
    (\Omega_0(\Delta \theta)-I)(R^{-1}\mathbf{v_0})
    =\mathbf{n_0} \times (R^{-1}\mathbf{v_0}) \Delta \theta
  \end{equation}
  を示せばいいことになります。(ここでも脚注2の性質を使いました。)
  $R^{-1}\mathbf{v_0}=(x,y,z)$とおいて、式(1)を元に、この式(4)の成分を$O(\Delta \theta)$
  まであらわに書くと
  \begin{equation}
    \begin{pmatrix}
      0 & -\Delta \theta & 0\\
      \Delta \theta & 0 & 0\\
      0 & 0 & 0
    \end{pmatrix}
    \begin{pmatrix}
      x\\
      y\\
      z
    \end{pmatrix}=
    \begin{pmatrix}
      0\\
      0\\
      1
    \end{pmatrix}\times
    \begin{pmatrix}
      x\\
      y\\
      z
    \end{pmatrix}
    \Delta \theta
  \end{equation}
  となりますが、これは明らかでしょう。\\

  (式変形が少しややこしかったかもしれませんが、この解答は本質的には「一般の回転軸のまわりで
  $\theta \to \theta+\Delta \theta$」だとめんどいから、$z$軸まわりで$\theta=0$にしちゃえ、って
  してるだけです。)\\

\item
  設問(3)で与えられた式の両辺を$\Delta \theta$で割ることによって得られる微分方程式は
  \begin{equation}
    \frac{d \Omega}{d \theta}(\mathbf{n}, \theta) \mathbf{v_0}
    =\mathbf{n}\times \mathbf{v_0}
  \end{equation}
  です。ここで設問に与えられた$A$を使うと\footnote{このような$A$の存在証明を一応しときます。
  $\mathbf{u}\mapsto \mathbf{a}\times\mathbf{u}$は$\mathbf{u}$について線形なので、線形変換
  を行列表示したものを$A(\mathbf{a})$とすれば$A(\mathbf{a})\mathbf{u}=\mathbf{a}\times \mathbf{u}$と
  書ける。これが実の行列であることは、任意のベクトル$\mathbf{u}$に対して右辺が実だから。}
  右辺は$A(\mathbf{n})\mathbf{v_0}$と書けるので
  \begin{equation}
    \frac{d \Omega}{d \theta}(\mathbf{n}, \theta)=A(\mathbf{n})
  \end{equation}
  となります。あとはいいでしょう。($\theta=0$を考慮して積分定数を決めるのも忘れずに。)

\item
  $X\mathbf{u}=\mathbf{a_0}\times\mathbf{u}$。$\mathbf{a_0}=(x,y,z)$とおきます。
  $\mathbf{u}=(0,0,1),(0,1,0)$を代入して
  \begin{equation}
    \begin{pmatrix} x_{13} \\ x_{23} \\ 0 \end{pmatrix}
      = \begin{pmatrix} y \\ -x \\ 0 \end{pmatrix},\quad
    \begin{pmatrix} x_{12} \\ 0 \\ x_{32} \end{pmatrix}
      = \begin{pmatrix} -z \\ 0 \\ x \end{pmatrix}
  \end{equation}
  を得ます。よって$\mathbf{a_0}=(x_{32},x_{13},x_{21})$。

\item
  $X$が実交代$\Leftrightarrow {}^tX=-X$。いま、$Y$が「直交、かつdet=1」
  ならいいので、順に見てくと
  \begin{equation}
  {}^tYY={}^t(e^X)e^X=e^{{}^tX}e^X=e^{-X}e^X=I, 同様にY{}^tY=I
  \end{equation}
  \begin{equation}
  det[Y]=det [e^X]=e^{\lambda_1+\lambda_2+\lambda_3}=e^{tr X}=e^0=1
  \end{equation}
  より示せました。ただし$\lambda$は$X$の固有値。\\

  下については少し補足すると、
  「detは固有値全ての積、trは固有値全ての和」という性質を使いました。(これは
  対角化すれば明らかです。対角化不可能な行列についても、固有値出すための
  方程式に解と係数の関係使えばわかります。)$tr X=0$
  は、交代行列なら定義から対角成分が$0$になるからです。(例えば(5)の設問の式。)
  exponentialの扱いで気になったら、テイラー展開すれば多分わかります。
\end{enumerate}
\end{answer}

\begin{answer}{第2問}{ochi}
\begin{enumerate}
\item
  $n$次。これは帰納法で。$n=0$なら明らか。あとは以下の関係式に注意。
  \begin{equation}
    f_n(x)
      =e^{x^2}\frac{d^n}{dx^n}(e^{-x^2})=e^{x^2}\frac{d}{dx}(e^{-x^2}f_{n-1}(x))
      =-2xf_{n-1}(x)+\frac{d}{dx}f_{n-1}(x).
  \end{equation}

\item
  $n\geq 2$を使うと、部分積分などから以下がわかります。
  \begin{equation}
    \int_{-\infty}^{\infty}xf_n(x)e^{-x^2}dx=\int_{-\infty}^{\infty}x\frac{d^n}{dx^n}(e^{-x^2})dx=
      -\int_{\infty}^{\infty}\frac{d^{n-1}}{dx^{n-1}}(e^{-x^2})dx=0.
  \end{equation}

\item
  上と同様に部分積分を繰り返すとわかります。
  \begin{equation}
    \int_{-\infty}^{\infty}x^nf_n(x)e^{-x^2}dx=\int_{-\infty}^{\infty}x^n\frac{d^n}{dx^n}(e^{-x^2})dx
      =\dots=(-1)^n n! \int_{-\infty}^{\infty}e^{-x^2}dx=(-1)^nn!\sqrt{\pi}.
  \end{equation}

\item
  与えられた式に$f_n(x)$の定義式を代入すると、結局
  \begin{equation}
    \sum_{n=0}^{\infty}\frac{t^n}{n!}\frac{d^n}{dz^n}(e^{-z^2})=e^{-(t+z)^2}
  \end{equation}
  を示せればいいことがわかります。ここで与えられた積分の式を左辺に入れて
  \begin{align}
    (l.h.s)=\sum_{n=0}^{\infty}\frac{t^n}{n!}\frac{d^n}{dz^n}\frac{1}{2\pi i}\oint_C
      \frac{e^{-\omega ^2}}{\omega -z}d\omega
      &=\sum_{n=0}^{\infty}\frac{t^n}{2\pi i}\oint_C
        \frac{e^{-\omega ^2}}{(\omega -z)^{n+1}}d\omega \\
      &=\frac{1}{2\pi i}\oint_C \sum_{n=0}^{\infty} t^n
        \frac{e^{-\omega ^2}}{(\omega -z)^{n+1}}d\omega \\
      &=\frac{1}{2\pi i}\oint_C \frac{e^{-\omega ^2}}{\omega -z-t}d\omega \\
      &=e^{-(t+z)^2}
  \end{align}
  となるので示せました。ただし最後から二番目の等号は$\sum z^n=1/(1-z)$という普通の幾何級数
  を、最後の等号は$|t|<1$より(poleがCの内側にあるので)
  普通の留数計算(あるいは、本文で与えられた公式)です。

\item
  この設問で与えられた微分方程式に、(iv)で示した式をそのまま代入して
  \begin{equation}
    \sum_{n=0}^{\infty}\frac{t^n}{n!}(f_n''(z)-2zf_n'(z)+2nf_n(z))=0
  \end{equation}
  を得ます。これが恒等式なので、各$n$について、sumされている中身は0にならなくては
  いけません。(tの多項式だとでも思って。。)よって$\lambda=2n$。
\end{enumerate}
\end{answer}
