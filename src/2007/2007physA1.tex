%% -*- coding:sjis -*-
%%
%% 2013-07-17, Koichi Murase, inshi20phys-ochi.tex より転載、編集。
%%
\begin{answer}{第1問}{ochi}
※使い方の注意\\
脚注は気になった人のためのものなので、基本的には本文だけ読めば事足ります。
\begin{enumerate}
\item
  $[ a_k^\dag, a_l^\dag ]=[a_k, a_l]=0, [a_k, a_l^\dag]=\delta_{kl} (k,l=1,2), 
  \mathcal{H}=\sum_{k=1}^2 \hbar \omega(a_k^\dag a_k+\frac{1}{2})$.

\item
  $p_k\to(\hbar/i)(\partial/\partial x_k)$に換えた以下が微分方程式。
  \begin{equation}
    (x_k+\frac{\hbar}{m\omega}\frac{\partial}{\partial x_k} )\psi (x)=0
  \end{equation}
  解は$exp(-(m\omega/2\hbar)r^2)$。(変数分離法から、各座標成分
  $x_k$ごとに$exp(-(m\omega/2\hbar)x_k^2)$が解になるので。)

\item
  固有状態は$(1/\sqrt{n!m!})(a_1^\dag )^{n_1}(a_2^\dag)^{n_2}
  | 0\rangle$, $n_1,n_2=0,1,2\dots$。規格化は$\| (a^\dag)^n |0 \rangle\|^2
  =\langle 0|a^n(a^\dag)^n |0\rangle=\langle 0|a^{n-1}(a(a^\dag)^n)|0\rangle
  =\langle 0|a^{n-1}(a^\dag)^{n-1}|0\rangle \times n
  =\| (a^\dag)^{n-1} |0\rangle \|^2 \times n$
  のように帰納的にわかります。\footnote{途中、$a(a^\dag)^n=n(a^\dag)^{n-1}+(a^\dag)^na$
  を使いました。証明は以下。まず左辺を$b_n$とおくと
  $b_n=(aa^\dag)(a^\dag)^{n-1}=(1+a^\dag a)(a^\dag)^{n-1}=(a^\dag)^{n-1}+a^\dag b_{n-1}$
  と帰納法。特に$a(a^\dag)^n|0\rangle=n(a^\dag)^{n-1}|0\rangle$。なお、本質的には
  $a$を交換関係使って右に動かしていくときに、1つ動かすにつき$+1$の因子が出るというだけなので、
  わざわざこう式の形で示さなくても、言葉で説明すれば十分でしょう。}\\

  固有値を出すために、1番で求めた$\mathcal{H}$をこれに作用させます。脚注4の式を使って
  $\hbar \omega (n_1+n_2+1)$。縮退度は$n=n_1+n_2$に対して$n+1$。

\item
  気合です。$n$は、波動関数の一価性のために$\Theta(\theta)$は周期$2\pi$の関数でなくては
  いけないので、整数でなくてはいけません。(ちゃんと$\Theta(\theta)=exp(in\theta)$となることも
  式で示しましょう。)

\item
  条件より$\alpha\geq 0$。
  $r\to 0$で$f(r)\sim r^{\alpha}$。微分方程式は$f''+(1/r)f'-(n^2/r^2)f\sim 0$。
  代入して$\alpha(\alpha -1)+\alpha -n^2=0$。解いて$\alpha=n$。

\item
  べき級数を代入して気合で整理します。すると
  $f_1(2n+1)r^{n-1}+\sum_{s=0}^\infty [
  f_{s+2}\{ (n+s+2)(n+s)-n^2 \} +2f_s\{
  E-(1+n+s) \} ] r^{n+s}=0$となります。(多分。違うかもしれません。)
  すると各$r^{n+s}$の係数は0になるはずですが、
  ある$s'$以上で$f_{s'}=0$となるので、その境目あたりで
  $E=1+n+s$となるはずです。\footnote{もう少し議論を精緻化すると、
  全ての$f_s$が0でなく、かつべき級数が有限の項からなるのならば、
  $f_s=0(s\geq s'),f_s\neq 0(s=s'-1)$なる$s'$が存在するはずで、
  そのとき$r^{n+s'-1}$の係数に着目すれば。}$n,s$ともに0以上の整数より設問3と一致。
  \footnote{さらに、角度方向の量子数が、動径方向の量子数以下であることもわかります。}
\end{enumerate}
\end{answer}

\begin{answer}{第2問}{ochi}
\begin{enumerate}
\item
  $Z=(2+exp(-\beta \epsilon))^N$より、以下のように求まります。
  \begin{equation}
    E=-\frac{\partial}{\partial \beta}(\ln Z)
      =-N\frac{-\epsilon e^{-\beta \epsilon}}{2+e^{-\beta \epsilon}}
      =\frac{N\epsilon e^{-\beta \epsilon}}{2+e^{-\beta \epsilon}}.
  \end{equation}

\item
  \begin{equation}
    S=\frac{E-F}{T}=k_B \beta (E+\frac{1}{\beta}\ln Z)
    =Nk_B \left( \ln{(2+e^{-\beta \epsilon})} +\frac{\beta \epsilon e^{-\beta \epsilon}}
    {2+e^{-\beta \epsilon}} \right).
  \end{equation}

  ここで$\beta = \infty(T = 0)$とすると
  \begin{equation}
    S(T=0)=Nk_B\begin{cases}
      \ln 2 & \epsilon >0\\
      \ln 3 & \epsilon =0\\
      0(=\ln 1) & \epsilon <0
    \end{cases}
  \end{equation}
  と求まります。説明は、
  それぞれ上から、各分子の最低のエネルギーが$2,3,1$重縮退をしていること、$S=k_B\ln W$という
  表現、絶対零度で系は基底状態を取ることに触れればいいでしょう。

  以下、$p=exp(-\beta \epsilon)/(2+exp(-\beta \epsilon))$とおきます。いまの系は、
  各分子のエネルギーが確率$p$で$\epsilon$、$1-p$で$0$にあるものと考えられます。

\item
  $P(N_{\epsilon})={}_NC_{N_\epsilon} p^{N_\epsilon} (1-p)^{N-N_\epsilon}$.
  二項分布そのまま。

\item
  以下の変形がわからなかったら、${}_nC_m=n!/((n-m)!m!)$を思い出してください。
  \begin{equation}
    \langle N_\epsilon\rangle =\sum_{n=0}^N {}_NC_m p^m (1-p)^{N-m}m
      =Np\sum_{n=1}^N {}_{N-1}C_{m-1} p^{m-1} (1-p)^{(N-1)-(m-1)}
      =Np(p+(1-p))^{N-1}=Np
  \end{equation}
  なお設問で与えられた公式を使いました。
  ここに$p$の定義式を代入してグラフを描くのは略します。$T=0$だと
  $\langle N_\epsilon\rangle =N(\epsilon <0),\ N/3(\epsilon=0),\ 0(\epsilon >0)$に注意して、
  有限温度ならそれを緩めればOKです。$\epsilon\sim 0$で
  一次関数的に振る舞い(テイラー展開より)、$\epsilon\to \pm \infty$
  では$T=0$での値にexponentialに近づきます。\footnote{
  これは、エネルギーギャップ大きい=飛び移りにくい
  =極限では$T=0$と同じ状況に、ということ。さらに飛び移りにくさがエネルギーに対して
  exponentialになるのはカノニカル分布から。}温度が2倍になれば、
  (pは$\beta \epsilon$の関数なので)グラフは横に2倍に伸びるだけです。

\item
  \begin{equation}
    \langle N_\epsilon^2-N_\epsilon\rangle =\sum_{n=0}^N {}_NC_m p^m (1-p)^{N-m}m(m-1)
    =N(N-1)p^2,\quad\text{(示し方は上と全く同じ)}
  \end{equation}
  \begin{equation}
    \langle N_\epsilon^2\rangle -\langle N_\epsilon\rangle ^2=Np(1-p).\quad\text{(式(23)も使って)}
  \end{equation}
  例えば$p=0,1$で分散0は見ればわかりますね。

\item
  単に微分して$\chi=-\beta(\langle 
  N_\epsilon^2\rangle -\langle N_\epsilon \rangle^2)$。ちゃんと$\epsilon$とかで
  表すのは省略します。
\end{enumerate}
\end{answer}

\begin{answer}{第3問}{ochi}
\begin{enumerate}
\item
  マクスウェル方程式は以下の通り。
  \begin{align}
    \mathbf{\nabla}\cdot \mathbf{\tilde{E}}&=0\\
    \mathbf{\nabla}\times \mathbf{\tilde{E}}&=-\frac{\partial \mathbf{\tilde{B}}}{\partial t}\\
    \mathbf{\nabla}\cdot \mathbf{\tilde{B}}&=0\\
    \mathbf{\nabla}\times \mathbf{\tilde{B}}&=\frac{1}{c^2}\frac{\partial \mathbf{\tilde{E}}}{\partial t}
  \end{align}
  $(\partial/\partial t)$(式29)$-\mathbf{\nabla}\times$(式27)で$\mathbf{\tilde{E}}$の式。ここで
  $\mathbf{\nabla}\times(\mathbf{\nabla}\times \mathbf{A})=\mathbf{\nabla}
  (\mathbf{\nabla}\cdot \mathbf{A})-\nabla^2 \mathbf{A}$
  を使いました。

\item
  これは波動方程式に代入するだけ。$\gamma^2=(\omega/c)^2-k^2$.\\

  問題とは少しズレますが、せっかくだしマクスウェル方程式全部に、いまの
  電磁場の関数系を代入してそのまんま計算すると、以下の方程式系を得ます。(実際に計算すれば
  わかりますが、これ以上の式は出てこないので、これらは(今の関数系に対する仮定の下で)
  マクスウェル方程式全体と等価になります。)\footnote{方針としては、まずrotの第1,2成分の計4式から
  $E_x,E_y,B_x,B_y$を$E_z,B_z$のみで表し、その後はそれら4式を残り全てに代入。
  なお$\gamma=0$でもこれらの式は成り立ちますが、そのときは等価にはなりません。
  これは必要になったら考えます。(その場合でも、これらの式が成り立つのなら、これからの
  議論に支障はないので。)}
  \begin{align}
    \gamma^2 E_x&=
      i(\omega \frac{\partial B_z}{\partial y}+k\frac{\partial E_z}{\partial x}),\\
    \gamma^2 E_y&=
      i(-\omega \frac{\partial B_z}{\partial x}+k\frac{\partial E_z}{\partial y}),\\
    \gamma ^2B_x&=
      i(-\frac{\omega}{c^2} \frac{\partial E_z}{\partial y}+k\frac{\partial B_z}{\partial x}),\\
    \gamma ^2B_y&=
      i(\frac{\omega}{c^2}\frac{\partial E_z}{\partial x}+k\frac{\partial B_z}{\partial y}),\\
    \left( \frac{\partial^2}{\partial x^2}+\frac{\partial^2}{\partial y^2}+\gamma^2 \right)E_z&=0,\\
    \left( \frac{\partial^2}{\partial x^2}+\frac{\partial^2}{\partial y^2}+\gamma^2 \right)B_z&=0.
  \end{align}

  さて、これを見ると下の二つの方程式を解いて$E_z,B_z$を求めてしまえば、
  残りは全部わかるようです。境界条件も、上の式を使えば全て$E_z,B_z$に対するものとして
  表すことができます。(例えば$E_x$に対する境界条件があれば、1番目の式を使って。)これ以降
  、$E_z,B_z$に絞って解析を進めているのは、そのためです。

\item
  $E_z(x,y)=f(x)g(y)$とおいて代入すると、$(f''/f)+(g''/g)+\gamma^2=0$より、
  $f,g$ともに三角関数で表されることがわかります。\footnote{細かいこと言うと、
  これだけだと指数関数や一次関数になる可能性もあります。
  が、そのときは$E_z$の境界条件を満たすことができません。
  (指数関数については、
  $f=ae^{\alpha x}+be^{-\alpha x}$とでもおいてみてください。$x=0$で$f=0$だと$f\propto \sinh$より
  単調関数なので$x=a$では$f\neq 0$でアウト。定数関数でない一次関数については、
  明らかですね。)磁場も同様。
  なお定数関数な可能性は含めて「三角関数」と書いてあります。
  (後の設問でも$n=0$とか考えてるし。)
  }境界条件$E_z=0$(「導体表面に平行な$\mathbf{E}$成分は0」)より、
  $E_z=E\sin[ (m\pi/a)x ]\sin[ (n\pi/b)y]$となって、答えを得ます。
  \footnote{平行成分は他にもありますが、そちらの境界条件は
  ($\gamma\neq 0$であれば)自動的に満たされます。設問4での$B_z$に対する
  境界条件を使うと、例えば式(30)の右辺は$y=0$において常に$0$になります。}\\

\item
  設問にはなっていませんが、$B_z$を$\cos$で書ける理由にも触れておきます。いま磁場については
  「導体表面に垂直な$\mathbf{B}$成分は0」が成り立ちます。これは$\partial B_z/\partial n=0$と
  表すことができます。($n$は法線方向の微分。)
  それは、例えば$x=0$平面上では$B_x=0$ですが、式(32)右辺の第一項は
  $E_z=0$ on $x=0$という境界条件より$0$($y$で微分していますが、境界条件より$x=0$の平面上で
  $y$方向にどう動かしたところでずっと$0$です。)で、結局第二項の
  $\partial B_z/\partial x$のみ残るからです。よってそれを満たす関数ということで$\cos$になります。\\

  さらに、設問3と同じ整数の組$(m,n)$を使えるのは、設問3で$\gamma^2$を$(m,n)$で表しましたが、
  $B_z$についても同じことが出来るからです。共通の$\gamma^2$を$a\neq b$のときに、このように同じ形で
  書くためには、$(m,n)$も共通でなくてはいけません。\footnote{証明。$(m\pi/a)^2+(n\pi/b)^2=(m'\pi/a)^2
  +(n'\pi/b)^2$とする。変形して$b^2(m-m')(m+m')=a^2(n'-n)(n'+n)$。もしも$b^2$と$a^2$の比が有理数比
  でなければ、これで証明終わり。($m,n$の符合については、$\cos$の対称性に吸収させればOK。)
  ちょうどぴったり都合のいい有理数比(のモデル)だと。。どうなるんでしょう?}\\

  設問自体は、式(30)から(33)に$E_z,B_z$の関数系を代入するだけなのですぐです。
  \begin{align}
    \gamma^2E_x&=i(-\omega\beta B+k\alpha E)\cos \alpha x \sin \beta y,\\
    \gamma^2E_y&=i(\omega\alpha B+k\beta E)\sin \alpha x \cos \beta y,\\
    \gamma^2B_x&=i(-\frac{\omega}{c^2}\beta E-k\alpha B)\sin \alpha x \cos \beta y,\\
    \gamma^2B_y&=i(\frac{\omega}{c^2}\alpha E-k\beta B)\cos \alpha x \sin \beta y.
  \end{align}
  この両辺を$\gamma^2=(m\pi/a)^2+(n\pi/b)^2$で割ったものが答えです。\\

  \textgt{(以下の補足は、おそらく院試の答案には不要なんじゃないかと思います。
  それか、もっと簡単な論理で言えるのかもしれませんが。。いずれにせよ、調べたわけでも
  ないし自信はかなりありません。)}
  ところでもし
  $n=m=0$だとその操作は許されません($\gamma^2=0$より)が、そのときは上の式の両辺が$0$
  なので、そもそも上の式がただのtrivialな等式になってしまいます。脚注8で触れたように
  このとき式(30)から(35)はマクスウェル方程式に等価ではなく、いま$n=m=0$より$E_z=0,B_z=B$
  に留意して(式(30)から(35)が成り立つことには変わりないので、今までの議論を捨てる必要は
  ありません。)、この場合は
  \begin{align}
    E_y &=-cB_x,\\
    E_x &=cB_y,\\
    \frac{\partial E_x}{\partial x}+\frac{\partial E_y}{\partial y}&=0,\\
    \frac{\partial E_y}{\partial x}-\frac{\partial E_x}{\partial y}&=i\omega B
  \end{align}
  が等価な条件式として出てきます。
  (マクスウェル方程式を$E_z=0,B_z=B$として単純に計算すれば出てきます。
  $\gamma^2=0$より$\omega=ck$ともしました。)また脚注10で触れたように、ここには
  さらに$E_x,E_y$に関する境界条件を課す必要があります。\textgt{以下、さらに自信が
  ないので注意してください。}下の2つの式からすぐに、$(\partial^2/\partial x^2+
  \partial^2/\partial y^2)E_x=0$がわかります。これを設問と同じように変数分離法で
  解く(ただし、$E_x=0$ in $y=0,b$)と、$E_x=E_1(c_1exp((n'\pi/b)x)+c_2exp(-(n'\pi/b)x)
  )\sin ((n'\pi/b)y)$という形がわかります。全く同じように$E_y$も(こちらは$x$成分が
  $\sin$で)求まるのですが、
  これを元々の式(42),(43)に入れるとできる恒等式は、明らかに係数が全部$0$でないと
  恒等式でなくなるので、$E_x=E_y=B=0$。また、式(40),(41)から$B_x=B_y=0$になるので
  結局全部消えます。よって伝播する電磁場の解にはなりません。

\item
  上の考察から、$n=m=0$は弾いていいでしょう。残りの$(n,m)$ならどれも$E_z=0$を
  もちろん満たし、かつ他の成分(の一部)はちゃんと(例えばポインティングベクトルが
  ノンゼロなようには)生き残るので伝播する電磁場の解と認められます。
  よって$(m,n)=(1,0)$がよく(理由は後述)、そのとき$\gamma^2=(\pi/a)^2=(\omega/c)^2-k^2$
  を解いて$\omega=c\sqrt{k^2+(\pi/a)^2}$。$(0,1)$を選ばなかったのは、$a>b$だから。(この
  解の形を見ればわかりますよね。)
  (振動数とあるので、角振動数を$2\pi$で割ったほうがいいのかも?)

\item
  $B_z=0$なら$B=0$でなくてはいけません。これで$E=0$だと電磁場が全て$0$になってしまうので$E\neq 0$。
  \footnote{混乱を招くかもしれないので念のため断っておきますが、
  $E$や$B$は(本文中の定義から)あくまで$E_z,B_z$の係数であって、例えば$B=0$になったからといって
  磁場がなくなるわけじゃありません。
  ただ設問4で$E,B$を使って電磁場の各成分を表した式(36)-(39)からわかるように、
  $E=B=0$だと電磁場の全成分が0になります。}
  さらに設問4で得た式をちゃんとチェックすると、$m,n$の少なくとも一方が
  $0$のときも、電磁場の成分が全て$0$になってしまいます。
  \footnote{後で見返したらわかりにくかった
  ので、補足します。例えば式(36)で$B=0$とします。もし$m=0$なら係数の$\alpha=0$なのでゼロ、
  $n=0$なら$\beta=0$よりサインのせいでゼロ。あるいはもっと簡単に考えれば、
  $m,n$のどちらか一方がゼロなら$E_z=0$なので$E=0$を考えているのと同じになるから、
  でもいいでしょう。なお、5番の問題のときはこの問題は起こりません。(チェックしてみてください。)}
  よって$(m,n)=(1,1)$。(このときは大丈夫。)
  このとき$\gamma^2=(\pi/a)^2+(\pi/b)^2=(\omega/c)^2-k^2$を解いて$\omega=c\sqrt{k^2+(\pi/a)^2
  +(\pi/b)^2}$。($2\pi$で割るうんぬんは、設問5と同じ。)
\end{enumerate}
\end{answer}
